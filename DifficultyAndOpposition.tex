\section{Nehézség és ellenállás}

Ha a karakter tettét valamilyen probléma akadályozza, vagy pedig egy másik karakter vagy lény helyett csak a világra próbál hatni, akkor a cselekvésének passzív \definition{nehézséget} kell legyűrnie. Ilyen például a zártörés, ajtók eltorlaszolása vagy felmérni az ellenség táborát. A KM változtathat a nehézségen, ha ezt bizonyos jellemzők (legyenek azok a karakteren, a jeleneten vagy máshol) indokolják.

Más esetekben, az ellenfél aktív \definition{ellenállást} tanúsít a karakter cselekvésével szemben egy védekezés cselekvést használva (\page{21}). Ezekben az esetekben a KM szintén dob, ugyanazokat a lépéseket követve az előző fejezetből, miközben az ellenfél képességeit, fortélyait és jellemzőit használja. Ha megtámadásra vagy az ellenfelet közvetlenül érintő helyzetbehozásra dobsz, az ellenfél védekezésre fog dobni ellene.

Az ellenállásnak számtalan formája létezik. Ha egy szektataggal a rituális áldozótőrért küzdesz, akkor az ellenfél jól definiált. De lehetséges, hogy egy ősi rituálé erejével kell megküzdened, hogy meg tudd menteni a világot. Feltörni egy széfet a First Metropolitan Bank páncéltermében a felfedezés kockázatát hordozza, de az a KM döntése, hogy a járőröző őrség aktív \emph{ellenállása} ellen dobsz, vagy pedig csak a széftől függő passzív \emph{nehézség} ellen.

\section{A dobás módosítása}

Kihasználhatsz egy jellemzőt, hogy +2 bónuszt kapjál a dobásra, vagy pedig újra dobhass. Néhány fortély szintén bónuszokat adhat. A jellemzők kihasználásával egy szövetségesedet is támogathatod (\page{32}), vagy az ellenfél dobásának nehézségét is megnövelheted.

\subsection{Jellemzők kihasználása}

Ha csinálsz valamit, de a kockák eredménye nem elégséges, nem kell feladnod, és elfogadni a vereséget. (Bár nyilvánvalóan megteheted. Az is lehet mókás.) A használható jellemzők opciókat és lehetőségeket biztosítanak ahhoz, hogy elérd a célod.

Ha meg tudod indokolni, hogy egy jellemző hogyan segíthet a fáradozásaidban, fogalmazd meg, hogyan segít, és költs el egy sors pontot a \definition{kihasználásához} (vagy használj el egy ingyen kihasználást). Hogy mi indokolható meg, és mi nem, annak eldöntésére a \definition{fals szabály} szolgál – bárki megvétózhatja, ha kijelenti, hogy „Ez fals!”. Egyszerűen fogalmazva, a fals szabály \textbf{kalibrációs technika}, amit minden jelenlévő használhat, ha úgy érzi, hogy a játék eltávolodott a megcélzott víziótól vagy koncepciótól. Hasonló technikákról, amik a biztonságos játékkörnyezet kialakítását célozzák, a \onpage{44} olvashatsz.
Ha a kihasználásod falsnak bizonyul, két lehetőséged van. Először is, visszavonhatod a kihasználást, és próbálkozhatsz mással, például egy másik jellemzővel. Másodszor, megvitathatod, hogy a jellemző miért is megfelelő. Ha ez nem győzi meg a másikat, akkor vond vissza a kihasználást, és nyugodj bele. Ha magáévá teszi a perspektívádat, akkor viszont csinálhatod a kihasználást, ahogy tervezted. A fals szabály arra szolgál, hogy az asztal mellett töltött idő kellemesen teljen. Akkor használd, ha valami nem hangzik túl jól, nincs sok értelme, vagy nem illik a megcélzott hangulatba. Ha valaki kihasználná a \aspect{Nagyszerű Első Benyomás} képességét, hogy eldobjon egy autót, az valószínűleg fals. De lehetséges, hogy van a karakternek valamilyen természetfeletti jellemzője, ami elképzelhetetlenül erőssé teszi, elég erőssé, hogy eldobjon egy autót, és ez a belépő egy rettentő szörnyeteg elleni harcban. Ez esetben a \aspect{Nagyszerű Első Benyomás} teljes mértékben hihető.

\newpage

Amikor kihasználsz egy jellemzőt, vagy \textbf{+2 bónuszt kapsz} az erőfeszítésre, vagy \textbf{újradobhatod mind a négy kockát}, vagy pedig \textbf{2"~vel emelheted a nehézséget} valaki más dobásánál, ha ez megmagyarázható. Több jellemzőt is kihasználhatsz ugyanahhoz a dobáshoz, de nem lehet kihasználni ugyanazt a jellemzőt többször ugyanahhoz a dobáshoz. Az egyetlen kivétel: tetszőleges számú \emph{ingyen kihasználásod} elhasználhatod ugyanazzal a jellemzővel ugyanahhoz a dobáshoz.

Kihasználni legtöbbször a saját karaktered jellemzőit fogod. De kihasználhatsz helyzet jellemzőket is, sőt, lehetőség van más karakter jellemzőinek ellenséges kihasználására is (\page{24}).

\subsection{Fortélyok használata}

A fortélyok bónuszt adhatnak a dobásra, feltéve, hogy teljesíted a fortély leírásában megfogalmazott feltételt, például a megfelelő körülményeket, a használt cselekvést vagy képességet. Használhatod a helyzetbehozás cselekvést is (\page{19}, hogy olyan jellemzőket hozzál létre, amik megteremtik a fortélyhoz szükséges körülményeket. Végül, vedd figyelembe a fortély szükséges körülményeit a karaktered cselekedeteinek leírásakor is, hogy ennyivel is könnyebb legyen a feltételeket teljesítened.

Normális esetben a fortély +2 bónuszt ad, egy bizonyos, szűken definiált szituációban, bármiféle költség nélkül; bármikor használhatod, amikor csak lehetséges. Ellenben, néhány ritka és kivételesen hatalmas fortélyhoz lehet, hogy el kell költened egy sors pontot a használathoz.

\section{Kimenetelek}

Minden dobás után az erőfeszítés, valamint a passzív nehézség vagy aktív ellenállás közötti különbséget a \definition{sikerességnek} hívjuk. Négyféle lehetséges kimenetel létezik:

\begin{itemize}
    \failureitem Ha az erőfeszítés kisebb, mint a megcélzott nehézség vagy ellenállás, akkor \textbf{kudarc}.
    \tieitem Ha az erőfeszítés egyenlő a céllal, akkor \textbf{döntetlen}.
    \successitem Ha az erőfeszítés egy vagy kettő sikerességgel nagyobb, mint a cél, akkor \textbf{siker}.
    \successwithstyleitem Ha az erőfeszítés legalább három sikerességgel nagyobb, mint a cél, akkor \textbf{átütő siker}.
\end{itemize}

Némelyik kimenetel nyilvánvalóan jobban esik, mint mások, de valamilyen érdekes módon mindegyiküknek előre kell mozdítani a történetet. Az előbb az elképzelés szabállyal kezdted (\page{13}), hát fejezd is be vele, hogy a történet maradjon a játék fókusza, és hogy a kimenetel értelmezése megfeleljen az elképzelésnek.

\example{
Ethan nem valami ügyes mackós (bár megvannak hozzá a szerszámai), de mégis ott találjuk a vészjósló kultusz titkos főhadiszállásán, és már csak egy páncélajtó választja el a keresett rituálé kötettől. Át tud‑e jutni rajta?
}

\newpage

\outcomesection{Kudarc}{Ha az erőfeszítés kisebb, mint a megcélzott nehézség vagy ellenállás, akkor kudarcot vallasz.}

Ezt többféleképpen is ki lehet játszani: egyszerűen kudarc, siker komoly áron, vagy elszenvedni egy találatot.

\subsubsection{Egyszerű kudarc}

Az elsőt a legegyszerűbb megérteni – \definition{egyszerűen kudarc}. Nem éred el a célod, nem haladsz előre, nem kapsz semmit sem. Ez viszont mindenképpen mozdítsa előre a történetet – ha a széfet egyszerűen csak nem tudod kinyitni, az sehová sem vezet, és unalmas is.

\example{
Ethan diadalmasan meghúzta a kart, de a széf határozottan zárva maradt, ellenben a szirénák felharsantak. A kudarc megváltoztatta a szituációt, és előremozdította a történetet – az őrök már a helyszínre tartanak. Ethannek egy újabb dilemmával kell szembenéznie – most, hogy a finom módszerek sikertelennek bizonyultak, próbálja más módon kinyitni a széfet, vagy hagyja a fenébe, és tűnés?
}

\subsubsection{Siker komoly áron}

A második a \definition{siker komoly áron}. Megteszed, amit akartál, de komoly árat kell fizetni érte – a szituáció rosszabbodott, de legalábbis komplikációk léptek fel. Mint KM, egyszerűen kijelentheted, hogy ez lett a kimenetel, vagy pedig felajánlhatod a játékosnak, mint opciót a kudarc helyett. Mindkét variáns jó és hasznos lehet a megfelelő szituációban.

\example{
Ethan elrontja a dobását, és a KM így szól: „Hallod, ahogy az utolsó retesz a helyére kattan. Ez a hang szinte visszhangzik egy felhúzott revolver kattanásában, ahogy az őr megkér, hogy fel a kezekkel.” A komoly ár itt a konfrontáció az őrrel, amit a karaktere megpróbált elkerülni.
}

\subsubsection{Elszenvedni egy találatot}

Végül, \definition{elszenvedhetsz egy találatot}, amit vagy stressz dobozokkal vagy következményekkel kell semlegesítened, vagy valami más hátrányban részesülhetsz. Ilyen kudarc a leggyakrabban megtámadás elleni védekezés közben fordul elő, vagy ha a karakter megoldana veszélyes problémákat. Annyiban különbözik az egyszerű kudarctól, hogy egyedül a karaktert érinti, nem az egész csapatot. A siker komoly áron annyiban más, hogy itt a siker nem feltétlenül lehetséges.

\example{
Ethan kinyitja a széfet, de ahogy megfogja a fogantyút, egy szúrást érez a kézfején. Nem hatástalanította a csapdát! Beírja, hogy Megmérgezett az enyhe következmény rubrikába.
}

Ezeket a lehetőségeket keverni is lehet. Az ártalmas kudarc kicsit durvának tűnhet, de lehet, hogy a legjobb választás abban a pillanatban. Vagy a siker találat árán is egy opció.

\newpage

\outcomesection{Döntetlen}{Ha az erőfeszítés egyenlő a megcélzott nehézséggel vagy ellenállással, az döntetlen.}

A kudarchoz hasonlóan a döntetlennek is előre kell mozdítania a történetet, sosem akadályozhatja az akciót. Muszáj valami érdekesnek történnie. Ugyanúgy, ahogy a kudarc esetén, ezt is többféleképpen ki lehet játszani: siker kisebb áron vagy részleges siker.

\subsubsection{Siker kisebb áron}

Az első a \definition{siker kisebb áron} – pár doboznyi stressz, valamiféle kényelmetlenség vagy hátráltatás a történetben, amik azért magukban nem akadályoznak, netán egy előny jellemző (\page{23}) az ellenfélnek mind‑mind kisebb árnak számítanak.

\example{
Ethan kezdeti próbálkozásai mind hibásnak bizonyulnak. Mire kinyitja a széfet, már hajnal hasadt, így szóba se jöhet, hogy az éj leple alatt tűnjön el. Megszerezte, amiért jött, de a szituáció már kellemetlenebb.
}

\subsubsection{Részleges siker}

A döntetlen kezelésének másik módja a \definition{részleges siker} – ez siker, de csak egy részét tudtad elérni a célodnak.

\example{
Ethan éppen csak résnyire tudja kinyitni a páncélajtót – ha csak egy ujjnyival megmozdítaná, a riasztás megszólalna, és fogalma sincs, hogyan tudná hatástalanítani. Pár oldalnyit a rituáléból ki tud húzni a résen át, de csak találgathat a befejező lépésekről.
}

\outcomesection{Siker}{Ha az erőfeszítés eggyel vagy kettővel nagyobb a célnál, az siker.}

Megszerzed, amit akarsz, bármiféle extra fizetség nélkül.

\example{
Kinyílt! Ethan felmarkolja a rituálé szövegét, és elillan, mielőtt az őrök felfedezhetnék.
}

\subsubsection{Az „előbb az elképzelés” hatása a sikerre}

Az elképzelés \emph{mondja meg}, hogy pontosan hogyan is néz ki a siker. Mi van, ha Ethannek nincsenek szerszámai, vagy nem elég képzett, hogy feltörje a széfet? Ilyenkor lehetséges, hogy a siker inkább „kisebb áron” fog történni. Hasonlóan, ha Ethan azért volt a betörő csapatban, mert ő \emph{építette} a széfet, akkor a siker eléggé „átütő” lesz.

\outcomesection{Átütő siker}{Ha az erőfeszítés legalább hárommal nagyobb a célnál, az átütő siker.}

Megszerzed, amit akarsz, és még egy kicsit többet is.

\example{
Ethan több mint szerencsés; a széf ajtaja szinte rögtön kinyílik. Nem csak, hogy megszerzi a rituálé szövegét, de elég ideje van átfutni a széfben lévő többi dokumentumot is. Mindenféle váltók és szerződések között rábukkan az Akeley kastély tervrajzára.
}
