\subsection{Helyzet jellemzők}

A KM‑nek célszerű érdekes és változékony környezetet teremtenie a jelenet megtervezésekor, ami egyrészt korlátozza a cselekvéseket, másrészt a használata lehetőséget biztosít a játékosoknak a szituáció megváltoztatására. Három és öt közötti részlet bőven elég. Használd az alábbi kategóriákat:

\begin{itemize}
    \item \textbf{Tónus, hangulat vagy időjárás} – sötétség, villámlás és süvöltő orkán.
    \item \textbf{Mozgás korlátozása} – létrákon át elérhető, iszappal borított és füstbe borult.
    \item \textbf{Fedezék és akadályok} – járművek, oszlopok és ládák.
    \item \textbf{Veszélyes tereptárgyak} – TNT ládák, olajhordók és elektromosságtól pattogó, hátborzongató műalkotások.
    \item \textbf{Használható tárgyak} – rögtönzött fegyverek, ledönthető szobrok vagy könyvespolcok és elbarikádozható ajtók.
\end{itemize}

Bárki kihasználhatja vagy késztethet ezekkel a jellemzőkkel, ezért vedd figyelembe őket, amikor lebirkózod a szektatagot, ha \aspect{Maró Iszap Mindenhol}.

Ahogy a jelenet lejátszódik, újabb és újabb jellemzőket találhatsz ki. Ha a játékos rákérdez, hogy vannak‑e a közelben árnyékok, és ha beleillik a szituációba, nyugodtan írd le, hogy \aspect{Koromsötét Árnyékok} vannak a katakomba zugaiban. Más jellemzők akkor kerülnek játékba, ha a karakterek a megold cselekvéssel létrehozzák őket. A karakterek cselekvései nélkül nem keletkezik csak úgy \aspect{Tűzvész} magától. Legalábbis általában.

\subsubsection{Ingyen kihasználások helyzethez?}

A KM dönthet úgy, hogy bizonyos helyzet jellemzőkhöz automatikusan jár egy ingyen kihasználás is a JK‑knak (és néha akár az NJK‑knak is). Az eszesebb játékosoknak talán pont a jelenet néhány jellemzője adhatja meg a kívánt fölényt – és egy ingyen kihasználás elég jól ösztönözhet a környezet megpiszkálására. Az ingyen kihasználások lehetnek az előkészületek eredményei is.

\subsubsection{Zóna jellemzők}

Ahogy azt említettük a \onpage{27}, némely helyzet jellemzők csak bizonyos zónákban használhatók. Ez teljesen rendben van – ez csak egy kis extra struktúrát, lehetőséget és kényelmetlenséget ad a térkép bizonyos részeihez, míg máshol nincsenek jelen.
