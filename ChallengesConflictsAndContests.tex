\chapter{Kihívás, konfliktus és versengés}

Egy jelenetben gyakran előfordul, hogy a cselekvések eredményét egyetlen kockadobással meg tudod határozni – ki tudod"~e nyitni a széfet, elkerülni az őröket, vagy meggyőzni az újságírót, hogy mutassa meg a jegyzeteit? Máskor hosszan tartó ütközetet kell vívnod, ami sok dobást igényel. Ezekhez az esetekhez háromféle döntési mechanizmus létezik: \definition{kihívás}, \definition{konfliktus} és \definition{versengés}. Mindhárom egy kicsit másként működik, mert más"~más a szóban forgó cél, és más akadályokat kell leküzdeni.

\begin{itemize}
    \item \textbf{A kihívás egy komplikált vagy változékony szituáció.} Valaki vagy valami megpróbál megakadályozni, de nincs jól definiált „ellenséges oldal”. Ezzel lehet kezelni, ha egy kutató nyomok után kutatna egy ősi kötetben, míg a csapat nagydumása elvonja a könyvtáros figyelmét, miközben a bunyós elmondhatatlan szörnyűségeket tart vissza attól, hogy belépjenek a könyvtárba.
    \item \textbf{A versengés olyan szituáció, amikor két vagy több fél egymást kizáró célokért küzd, de közvetlenül nem bántják egymást.} A versengés tökéletes üldözésekhez, vitákhoz és mindenféle versenyekhez. (És bár a résztvevők közvetlenül nem bántják egymást, ez nem jelenti, hogy ne szenvedhetnének sérüléseket közben!)
    \item \textbf{A konfliktus, amikor a részvevők nemcsak, hogy képesek rá, de még kárt is akarnak okozni egymásban.} Iszapbirkózás egy szektataggal, miközben pengék lendülnek egymás gyomra felé, vagy kilyuggatni egy csapat ghoul bőrét, míg ők a húsodba vájják karmukat, netán a riválisoddal egymást alázni a királynő éber tekintete előtt – ezek mind"~mind konfliktusok.
\end{itemize}
