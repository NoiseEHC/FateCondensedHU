\section{Méretarány}

A \definition{méretarány} nevű opcionális alrendszerrel természetfeletti lényeket tudsz kezelni, akiknek a lehetőségei messze túlmutatnak a legtöbb karakterénél a játékodban. Általában nem kell törődnöd a méretarány játékbeli hatásával. Néha viszont jól jöhet a játékosokat a szokásosnál sokkal nagyobb veszedelemnek kitenni -- ami jó lehetőség, hogy a karakterek megmutathassák, mire képesek.

Példaképpen -- és ezt a listát testre szabhatod a saját világodhoz -- legyen ötféle szintje a méretaránynak: Evilági, Természetfeletti, Másvilági, Legendás és Isteni.

\begin{itemize}
    \item Az \textsc{\textbf{Evilági}} karaktereknek nincs természetfeletti adottságuk vagy technológiájuk, amik az emberek fölé emelnék őket.
    \item A \textsc{\textbf{Természetfeletti}} karaktereknek van valamilyen természetfeletti adottságuk vagy technológiájuk, amik az emberek fölé emelik őket, de attól még alapvetően emberek maradnak.
    \item A \textsc{\textbf{Másvilági}} karakterek rendkívüliek vagy egyediek, és az adottságaik miatt nem kell többé törődniük az emberek mindennapi problémáival.
    \item A \textsc{\textbf{Legendás}} karakterek hatalmas szellemek, entitások és idegen lények, akik számára az emberiség inkább csak érdekesség, mintsem veszély.
    \item Az \textsc{\textbf{Isteni}} karakterek a világegyetem leghatalmasabb erői: arkangyalok, istenek, tündekirálynők, élő bolygók és így tovább.
\end{itemize}

Ha a méretarányt használod két szemben álló erőre vagy egyénre, akkor hasonlítsd össze a méretarányaikat, hogy melyiké a magasabb, és hány szinttel. A magasabb szintű \emph{egyet} választhat az alábbiakból az alacsonyabb szintű ellen.

\begin{itemize}
    \item +1 bónusz szintkülönbségenként a dobás \emph{előtt}
    \item +2 bónusz szintkülönbségenként a dobás \emph{után}, \emph{ha} a dobás siker
    \item 1 extra ingyen kihasználás szintkülönbségenként, ha a helyzetbehozás cselekvés siker
\end{itemize}

A méretarány szabályok túl gyakori és rugalmatlan használata hátrányosan érinti a játékosok karaktereit. Ezt ellensúlyozandó, osztogasd bőkezűen a különféle lehetőségeket, ahol kis ügyeskedéssel megkerülhetik ezeket a hátrányokat. Ilyen lehet például a célpont gyengeségének kikutatása, olyan helyen küzdeni, ahol a méretaránynak nincs jelentősége, vagy megváltoztatni az elérendő célt, hogy az ellenfél ne tudja a méretarány fölényét felhasználni ellenük.

\newpage

\subsection{Jellemzők és méretarány}

Az életben lévő helyzet jellemzők némelyike természetfeletti eredetű lehet. Ilyenkor a KM"~nek kell eldöntenie, hogy a kihasználásuk extra bónuszokat ad"~e a méretarány miatt. Ezen felül, egy természetfeletti eredetű jellemző kihasználás nélkül is biztosíthat méretarány bónuszokat, például egy varázsköpeny vagy csúcstechnológiás lopakodó álca esetén; nem kell kihasználnod a \aspect{Leplezett} jellemzőt, hogy Természetfeletti bónusszal lopakodhass.

\subsubsection{Számít"~e a természetfeletti helyzetbehozás esetén?}

Ha helyzetbehozásnál \emph{nincs ellenállás}, akkor dobás nélkül létrehozod a helyzet jellemzőt egy ingyen kihasználással. Ez a jellemző természetfeletti bónuszokat ad, ahogy előzőleg írtuk.

Ha a helyzetbehozást \emph{más kárára követed el}, például \aspect{Indákkal Megkötözött} varázslatot alkalmaznál az ellenfeleden, akkor kaphatsz természetfeletti bónuszt rá.

Ha természetfeletti erőkkel érnéd el a helyzetbehozást, és \emph{egy ellenszegülő fél közvetlenül be tud avatkozni fizikai vagy természetfeletti módon}, a természetfeletti bónuszod alkalmazható a védekezés dobása ellen.

\emph{Egyéb esetekben} természetfeletti bónusz nélkül hozod létre a helyzet jellemzőt, de később a jellemző kihasználásakor kaphatsz természetfeletti bónuszt, ha ennek van értelme.

\section{Időtartamok}

Egy cselekvés időtartamának meghatározásakor pontosabb módszert is használhatsz, mint a siker, kudarc és a „bizonyos áron” opciók. Mennyivel tovább vagy gyorsabban? Az alábbi irányelvekkel a sikeresség döntheti el az időtartamot.

Először is döntsd el, hogy egyszerű siker esetén mennyi ideig tart az adott feladat. Használj körülbelüli mennyiségeket egy mértékegységgel együtt: „pár nap”, „fél perc”, „néhány hét” és így tovább. A körülbelüli mennyiségekbe a következők tartoznak: fél, nagyjából egy, pár, néhány az adott mértékegységből.

Ezután nézd meg, hogy a dobás mennyivel haladja meg, vagy múlja alul a célszámot. Minden sikeresség egy lépést jelent a mennyiségek listáján.

Például, ha a kezdő időtartam „pár óra”, akkor egy sikerességnyivel gyorsabb „nagyjából egy óra”, két sikeresség pedig már csak „fél óra”. A „fél” mennyiségnél gyorsabb a mértékegységet váltja kisebbre (órákból percek, és így tovább), míg a mennyiség „néhány” lesz, emiatt három sikerességgel gyorsabb „néhány perc lesz”.

A lassabb esetben az egész fordítva zajlik, egy sikerességgel lassabb „néhány óra”, kettő „fél nap”, míg három már „nagyjából egy nap”.
