\subsubsection{Fortélyok megtervezése}

A fortélyokat te találod ki, a karakter kitalálása közben. Nagy vonalakban a fortélyokat két csoportra lehet osztani.

\textbf{Bónuszt adó fortélyok:} A fortélyok első típusa \textbf{+2 bónuszt ad}, amikor egy meghatározott képességet használsz bizonyos határokon belül, általában csak bizonyos cselekvésekhez (\page{18}), és bizonyos narratív körülmények között.

Az ilyen fortélyokat az alábbi módon célszerű megfogalmazni:

\example{%
Mivel \textbf{[írd le, hogy mennyire menő a karakter vagy a felszerelése]}, ha a \textbf{[válassz egy képességet]} képességet használom \textbf{[válassz a megoldásra, helyzetbehozásra, megtámadásra, védekezésre közül]}, +2 bónuszt kapok, de csak ha \textbf{[definiáld a körülményeket]}.
}

\textbf{Bónuszt adó fortély példa:} Mivel én egy \textbf{katonai mesterlövész} vagyok, ha a \textbf{Célzás} képességet használom \textbf{megtámadásra}, +2 bónuszt kapok, de csak \textbf{ha a \aspect{Célpont Bemérve}}.

\textbf{Szabályszegő fortélyok:} A fortélyok második típusa \textbf{megváltoztatja a játék szabályait}. Ez egy átfogó kategória, amibe az alábbiak mind beletartoznak, de kitalálhatsz mást is:

\begin{itemize}
    \item \textbf{Megváltoztatni, hogy melyik képességet kell használni bizonyos szituációkban.} Például egy kutató használhat Tudományt egy rituálé elvégzéséhez, míg mindenki más Misztikumot.
    \item \textbf{Olyan cselekvést végezni egy képességgel, ami normálisan nem lehetséges.} Például, hogy egy karakter hátba szúrjon egy ellenfelet az árnyékokból a Lopózás képességgel, míg normálisan ez megtámadás cselekvés lenne a Közelharc képességgel.
    \item \textbf{Nagyjából +2 értékű, nem számszerű bónuszt adni egy képesség használatához.} Például, ha egy tehetséges szónok Befolyásolással helyzetbe hoz, kap egy extra ingyen kihasználást.
    \item \textbf{Megengedni, hogy a karakter kinyilváníthasson egy kisebb tényt, ami igazzá válik.} Például egy túlélésben képzett karakternél mindig legyen pár túléléshez szükséges eszköz, mint mondjuk gyufa, még akkor is, ha ez nagyon valószínűtlen is a történetben. Ez a fortély fajta lehetővé teszi, hogy a történet részleteinek kinyilvánítása (\page{24}) jellemző kihasználása nélkül is elérhető legyen.
    \item \textbf{Megengedni, hogy a karakter megszeghessen pár szabályt.} Például a karakter kaphat két extra stressz dobozt vagy egy extra enyhe következmény rubrikát.
\end{itemize}

Az ilyen fortélyokat az alábbi módon célszerű megfogalmazni:

\example{%
Mivel \textbf{[írd le, hogy mennyire menő a karakter vagy a felszerelése]}, képes vagyok \textbf{[írj le egy bámulatos tettet]}, de csak \textbf{[definiáld a körülményeket vagy korlátokat]}.
}

\textbf{Szabályszegő fortély példa:} Mivel \textbf{nem hiszek a mágiában}, képes vagyok \textbf{természetfeletti adottságok hatásait figyelmen kívül hagyni}, de csak \textbf{játékülésenként egyszer}.
