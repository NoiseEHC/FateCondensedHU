\section{A KM sors pontjai}

Minden jelenetet akkora sors pont tartalékkal kezdj, mint a jelenlévő JK‑k száma. Ha jelen van még egy fontos NJK vagy szörny is, amelyik megadást választott (\page{37}) egy előző konfliktusban, vagy ellenséges kihasználás (\page{24}) áldozata lett egy előző jelenetben, akkor megkapod azokat a sors pontokat is. Ha az előző jelenetnek egy késztetés vetett véget, és emiatt nem volt alkalmad a sors pontot elkölteni, azt is hozzáadhatod a tartalékodhoz.

\example{%
Charles, Ruth, Cassandra és Ethan a végső leszámolás helyszínére tartanak, ahol Alice Westforth vár rájuk. Ő előzőleg megadta magát a hősöknek egy konfliktusban, miután egy mérsékelt következményt szenvedett el. Emiatt a KM négy sors pontot kap a JK‑k után, és további kettőt Alice hoz magával a közös sors pont tartalékba.
}

Mint a KM ebből a tartalékból költhetsz sors pontokat, hogy kihasználj egy jellemzőt, ellenállj egy késztetésnek, amit egy játékos ajánlott fel egy NJK‑nak, vagy hogy sors pont elköltését megkövetelő fortélyt használj – pont ugyanúgy, ahogy a játékosok teszik.

\textbf{Ellenben nem \emph{kell} sors pontot költened a tartalékodból jellemzők késztetéséért.} Erre a célra végtelen számú sors pont áll rendelkezésedre.

\section{Biztonsági technikák}

A KM és a játékosok közös felelőssége, hogy mindenki biztonságban érezze magát a játékban és az asztal körül is. A KM elősegítheti ezt egy rendszerrel, ahol bárki megfogalmazhatja az aggodalmait és kifogásait. Ha ez történik, annak prioritást kell élveznie, és foglalkozni kell vele. Alább pár technika található, amivel a játékosok könnyebben jelezhetik fenntartásaikat, és amik könnyebbé tehetik a rendszer használatát.

\begin{itemize}
    \item \textbf{X‑Kártya:} Az X‑Kártya egy opcionális eszköz (John Stavropoulos találmánya), aminek segítségével játék közben bárki (téged is beleértve) kimoderálhat számára kényelmetlen tartalmakat. Többet is megtudhatsz az X‑Kártyáról a \url{http://tinyurl.com/x-card-rpg} weblapon.
    \item \textbf{Script Change RPG Toolbox:} Ha egy kicsit több árnyalatra és felbontásra lenne szükséged, nézd meg Brie Beau Sheldon Script Change rendszerét, amivel megállíthatod, visszatekerheted, átugorhatod a történet részeit, és még egyebekre is jó, mindezt az ismerős média‑lejátszó felhasználói felülettel. Többet is megtudhatsz a Script Change rendszerről a \url{http://tinyurl.com/nphed7m} weblapon.
\end{itemize}

Ezeket a technikákat a fals szabályhoz (\page{14}) hasonlóan kalibrációra is lehet használni. Ez egy kényelmes lehetőséget biztosít a játékosoknak, hogy kifejthessék, mit várnak el a játéktól. Ne kicsinyeld le ezeket a technikákat, és támogasd a használatukat!
