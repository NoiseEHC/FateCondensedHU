\chapter{Jellemzők és sors pontok}

A \definition{jellemző} az egy szó, kifejezés vagy mondás, ami valami egyedit mond egy személyről, helyről, tárgyról, szituációról vagy csoportról. Majdnem minden elképzelhető dolognak lehet jellemzőket adni. Egy embernek lehet a hírneve, hogy ő a \aspect{Legjobb Céllövő a Vadnyugaton} (később többet is megtudhatsz hasonló jellemzőkről). Egy szoba lehet \aspect{Lángokban}, miután fellöktél egy olajlámpást. Egy szörnnyel történt összecsapás után lehetsz \aspect{Rémült}. A jellemzőkkel terelheted a történetet olyan irányba, ahol jobban illeszkedik a karaktered hajlamaihoz, képességeihez vagy problémáihoz.

\subsection{A jellemző mindig igaz}

A jellemzőket kihasználhatod a dobásodhoz bónuszért (\page{24}), és késztethetnek komplikációt generálva (\page{25}). De a jellemzők még akkor is hatnak a narratívára, ha ez a két lehetőség nem játszik. Ha egy hús‑fémvázon szörny \aspect{Beszorult a Hidraulikus Présbe}, akkor az \emph{úgy is van}. A szörny nem túl sok mindent tehet beragadva, és elég bajosan tud csak kimászni onnan.

Leegyszerűsítve, a „jellemző mindig igaz” azt jelenti, hogy a jellemzők \textbf{adhatnak vagy tilthatnak lehetőségeket, hogy mi eshet meg a történetben} (és szintén módosíthatják a nehézséget, lásd a \onpage{42}). Ha az előbb említett szörny \aspect{Beszorult}, akkor KM‑nek (és mindenki másnak is) muszáj ezt figyelembe vennie. A lény elvesztette a jogát a mozgásra, amíg valami olyan nem történik, ami visszaadja azt. Vagy egy sikeres megoldás (amihez bizonyos esetekben akár \aspect{Gigászi Erő} szükséges), vagy valakinek botor módon meg kell nyomnia a kiengedés gombot. Hasonlóképpen, \aspect{Kibernetikusan Erősített Lábak} jellemzővel könnyedén átugorhatsz falakat egy nekirugaszkodásból anélkül, hogy dobni kéne rá.

Ez persze nem jelenti azt, hogy bármilyen jellemzőt létrehozhatsz, és az igazságát használva törtess a történeten keresztül. Az rendben van, hogy a jellemzők nagy hatalmat adnak a történet irányításában, de ezzel együtt jár, hogy a játéknak a történet korlátain belül kell maradnia. A jellemzőknek egybe kell esnie az asztal körül ülők elvárásaival, hogy mi az, ami még elfogadható. \textbf{Ha egy jellemző bűzlik, akkor muszáj átfogalmazni.}

Persze, tök jó \emph{lenne} a \aspect{Feldarabolt} jellemzővel legyőzni a génmanipulált szuper‑katonát, de ez egyrészt feleslegessé tenné a megtámadás cselekvést, másrészt valószínűleg sokkal több munkát igényel levágni a karját (de következményként működhet, lásd a következő oldalon). Szintén kijelentheted, hogy te vagy a \aspect{Világ Legjobb Céllövője}, de ezt azért alá kéne támasztani pár képességgel. És bármennyire is szeretnél \aspect{Golyóálló} lenni, ami megtiltaná mindenkinek, hogy kézifegyverekkel kárt okozzanak benned, de ennek elfogadása elég valószínűtlen, hacsak nincsenek Szuperképesség jellemzők a játékban.
