\section{A jelenetek felépítése}

Függetlenül a jelenet típusától, a KM először is felvázolja a lényeges dolgokat, hogy a játékosok tudják, hogy milyen eszközökhöz férnek hozzá, és milyen bonyodalmakra kell figyelniük.

\subsection{Zónák}

A \definition{zónák} reprezentálják a bejárható teret – egy térképvázlat, ami pár különálló részre van osztva. Egy távoli tanyaházban játszódó konfliktusnak négy zónája lehet: a földszint, az emelet, az elülső udvar és a környező erdő. Kettő‑négy zóna általában elégséges a legtöbb konfliktushoz. Viszont nagyobb vagy komplikáltabb jelenetekhez többre is szükség lehet. Próbáld a zónák térképét egyszerűre venni, legjobb, ha elfér egy jegyzettömb lapján, vagy gyorsan felvázolod egy táblára.

A zónák korlátozzák, hogy mi eshet meg, így segítve a történet irányítását. Hogy kit támadhatsz meg, és hová mozoghatsz, azt mind a zóna határozza meg, amiben tartózkodsz.

\textbf{Az egy zónában tartózkodók hathatnak egymásra, és mindenre, ami az adott zónában van.} Ez azt jelenti, hogy ütheted, vághatod, vagy bármilyen fizikai kontaktust létesíthetsz emberekkel és tárgyakkal a zónádban. Ki akarod nyitni a hálószoba falába épített széfet? Akkor abban a zónában kell tartózkodnod. A zónádon kívül általában minden elérhetetlen számodra – vagy oda kell mozognod, vagy valahogyan távolba hatni (telekinézis, lőfegyverek stb).

Egyik zónából a másikba mozogni könnyű, feltéve, hogy semmi sincs az utadban. \textbf{A cselekvéseden felül egy szomszédos zónába is mozoghatsz egy forduló (\page{31}) alatt, ha semmi sem áll az utadba.} Ha a mozgásod korlátozva van, akkor az felemészti a teljes cselekvésedet. Dobjál megoldásra, hogy felmássz egy falon, elrohanj egy csapat szektatag mellet, vagy átugorj egy másik tetőre. Ha kudarc, akkor a zónádban maradsz, vagy pedig a mozgásnak kisebb vagy komoly ára lesz. A cselekvésedet felhasználva viszont \emph{bármelyik} zónába mozoghatsz – ám a KM joga, hogy magas nehézséget állapítson meg, ha ez egy filmbe illő megmozdulás.

Ha valami se nem rizikós, se nem érdekes eléggé kockadobáshoz, akkor az nem számít korlátozásnak. Például, ha egy ajtó nincs bezárva, akkor nem kell cselekedet a kinyitásához – az a mozgás része lesz.

Célzással távolról is tudsz támadni. A távolsági támadásokkal a szomszédos, vagy messzebb lévő ellenfelek is megtámadhatók, ha semmi sem akadályozza a rálátást a zónájukra. Ha például egy lény az emeleti hálószobában posztol, akkor nem tudsz kanyarban rálőni a lépcső aljáról. Vedd figyelembe a zónák elhelyezkedését, és a helyzet jellemzőket, amikor eldöntöd, hogy mi lehetséges, és mi nem.
