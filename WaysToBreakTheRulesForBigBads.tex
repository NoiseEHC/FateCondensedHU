\section{Szabálymódosítás főellenségekhez}

A csapatmunka közbeni képességek egyesítésével és helyzetbehozással (\page{32}) a JK‑k könnyen bedarálhatnak egy magányos ellenfelet. Ez rendben is van, ha a cél a túlerő demonstrálása, de nem annyira jó ötlet az egész csapattal felérő főellenségekhez.

De ha még emlékszel, teljesen helyén való szörnyek és egyéb nagy veszedelmek esetében megszegni a szabályokat (\page{43}) – így ezt ebben az esetben is megteheted, hogy semlegesíted a csapat normális létszámfölényét, de azért adj esélyt nekik. Ebben a fejezetben pár tanácsot adunk ennek megvalósítására. Ezeket használhatod magukban, vagy többet kombinálva is, ha nagyon nehéz vagy félelmetes főellenségre van szükséged.

\subsection{Kihívás és versengés védettség}

Mindkét technika arra szolgál, hogy elhúzhasd a végső összecsapást, végighajtva a csapatot valamiféle „fogytán az idő” cselekményen, még mielőtt közvetlenül is foglalkozhatnának a főellenséggel.

A \definition{kihívás védettség} azt jelenti, hogy a csapat nem tudja a főellenséget befolyásolni (fizikailag vagy szellemileg, esetleg mindkettő), hacsak le nem győz egy kihívást (elpusztítani a hatalma forrását, kitalálni a rejtett gyengeségét és így tovább). A főellenség eközben szabadon cselekedhet, és meg is támadhatja őket míg ezen dolgoznak, ellenszegülhet a megoldás vagy helyzetbehozás dobásoknak védekezés dobással, véget vethet az ingyen kihasználásoknak a saját megoldás cselekvéseivel, vagy felkészülhet az elkerülhetetlen konfrontációra saját maga helyzetbehozásával.

A \definition{versengés védettséggel} a csapatnak egy versengést kell megnyernie a főellenség megtámadása előtt – míg a főellenség szabadon támadhat rájuk közben. Ha a főellenség nyeri a versengést, akkor megvalósítja gonosz tervét, és sértetlenül távozhat.

\subsection{Feláldozható talpnyaló páncél}

Az egyik módszer, ahogy a főellenség kompenzálhatja a JK‑k túlerejét, ha körülveszi magát talpnyalókkal, de ez nem túl kifizetődő, ha a JK‑k egyszerűen figyelmen kívül hagyhatják a bosszantó talpnyalókat, és közvetlenül a főellenséget veszik célba.

De ha van egy \definition{feláldozható talpnyaló páncél}, akkor a főellenség minden védekezés dobása lehet siker valamilyen áron, mert mindig egy talpnyaló kerül a támadás útjába. Az a bizonyos talpnyaló egyáltalán nem dob védekezést, hanem benyeli a találatot, ami különben a főellenségen csattant volna. Emiatt a JK‑knak muszáj átvágniuk magukat a főellenség seregén a végső konfrontáció előtt.

Vésd észbe, hogy a talpnyalóknak nem muszáj \emph{szó szerint} talpnyalóknak lenniük. Például létrehozhatsz egy vagy több „pajzsgenerátort”, mindegyiket egy külön stresszmérővel, és talán még egy képességet is kaphatnak, hogy helyzetbehozásokkal is védjék a pajzsai mögé bújó főellenséget!

\newpage

\subsection{A végső alak felvétele}

Rendben van, a csapat mindent bevetett, és – \emph{hihetetlen!} – legyőzte a főellenséget. Csak egy probléma maradt: ez csak felszabadította a hús börtönéből, és végre felveheti végső alakját!

A \definition{végső alak felvétele} segítségével a főellenség nem csak egyetlen karakter, hanem legalább \emph{kettő}, akiket egymás után le kell győzni, míg mindketten újabb és újabb tehetségekkel és fortélyokkal, magasabb képesség szintekkel, üres stresszmérőkkel és következmény rubrikákkal, sőt, újabb „szabálymódosításokkal” rendelkeznek.

Ha ezt egy kicsit tompítanád, a főellenségnek maradjanak meg a következményei a két alakja között, eltörölve az enyhe következményt, és leminősítve az enyhébb szintekre a mérsékelt és súlyos következményeket.

\subsection{Növeld meg a méretarányt}

\definition{Növeld meg a méretarányt}, hogy a főellenség a JK‑knál magasabb szinten működjön, ehhez használd a méretarány szabályokat az \onpage{52}. Ezt akkor is bevetheted, ha egyébként a kampányban nincs méretarány – ezek a szabályok csak akkor lépnek érvénybe, ha a főellenség feltűnik a jelenetben.

\subsection{Szóló bónusz}

A játékosok nyilvánvalóan élvezik a csapatmunka bónuszt – de miért ne adhatnál a főellenségnek \definition{szóló bónuszt} akkor, ha egymagában áll ki a hősökkel?

A szóló bónuszt jó néhány módon meg lehet oldani. Egyszerre többet is használhatsz ezekből, de vedd figyelembe, hogy kombinálva hamar elszállhat az erősségük.

\begin{itemize}
    \item A főellenség akkora \textbf{bónuszt kap a képesség dobásaira}, amennyi a csapat maximális lehetséges csapatmunka bónusza (\page{32}) – ez a főellenség ellen dolgozó csapattagok számánál eggyel kevesebb (például +2 ha egy 3 fős csapat ellen van). A JK"~khoz hasonlóan ez a bónusz nem növelheti több, mint duplájára a főellenség képességét (vagy akár megszegheted \emph{ezt} a szabályt is).
    \item A főellenség \textbf{csökkentheti az elszenvedett sérülést} az ellenséges csapat létszámának felével, felfelé kerekítve. Ha úgy gondolod, hogy ez túlságosan elnyújtaná a harcot, akkor ne lehessen a sérüléseket 1 alá csökkenteni ezen a módon.
    \item A főellenség \textbf{kihasználásai sokszorozottak}: ha sors pontot \emph{fizetett} egy jellemző kihasználásáért, a kapott bónusza megegyezik a JK‑k számával. Ugyan ingyen kihasználásoknál nem ilyen szerencsés, de így is minden elköltött sors pontja félelmet lop majd a játékosok szívébe.
    \item A főellenség \textbf{elnyomja a jellemző kihasználásokat}: ha kettő vagy több ellenféllel áll szemben, azok csak +1 bónuszt vagy újradobást kapnak, ha közvetlenül a főellenség ellen használnák a jellemzőt. Opcionálisan a főellenség akár meggátolhatja az ingyen kihasználások halmozását is.
\end{itemize}

\newpage

\subsection{A veszély egy térkép (vagy karakterek hálója)}

A Fate játékban bármit lehet karaktereként kezelni, miért pont egy térképet ne lehetne? Ha \definition{a veszély egy térkép}, akkor a főellenség zónákból áll (\page{29}), amiben navigálni kell a győzelem elérésének érdekében.

A főellenség térkép részletezése közben adhatsz különböző képességeket, jellemzőket és stresszmérőt az egyes zónáknak. Néhány zónán elhelyezhetsz egyszerűbb kihívásokat is, amiket meg kell nyerni a továbbjutáshoz. Minden zóna karakterként tehet egy cselekvést az ott tartózkodó JK‑k ellen, vagy ha valamiféle végtagot vagy hasonlót reprezentál, akkor támadhatja a szomszédos zónákat is. Ha egy zóna kiejtődik a JK‑k támadásaitól, akkor ezután átugorható, és nem képes többé cselekvésekre, de ez még nem öli meg a főellenséget, csak ha a hősök eljutnak a szívéig, és azt véglegesen elpusztítják.

Ez a módszer akkor működik a legjobban, ha a főellenség egy gigantikus szörny, de nem muszáj csak olyankor használni. Kezelheted a főellenséget kapcsolatban álló karakterek csoportjával anélkül, hogy a JK‑knak ténylegesen bele kéne menniük, és ténylegesen egy térképen kéne mászkálniuk. Ez a módszer egy térkép és a feláldozható talpnyaló páncél (\page{54}) ötvözete – valami olyasmi, mint \definition{karakterek hálója}. A főellenség néhány részét le kell győzni, mielőtt a játékosok a tényleg sebezhetőre térhetnek, és ezek a részek bizony kiveszik a részüket a küzdelemből.

Akár részletesen kidolgozod térképként, vagy csak egyszerűen karakterek hálójaként kezeled, biztosítsd, hogy ez egy dinamikus harcot eredményez, ahol a főellenség gyakrabban képes cselekedni, és a játékosoknak elő kell állniuk egy tervvel, ami lépésről lépésre kiiktatja a veszélyeket, mielőtt képesek lennének teljesen megszüntetni.
