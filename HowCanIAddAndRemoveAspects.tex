\section{Jellemzők létrehozása és megszüntetése}

A helyzetbehozás cselekvéssel lehet helyzet jellemzőket létrehozni vagy felfedezni (\page{19}). Az előny jellemzőket ugyanígy lehet létrehozni, valamint döntetlen vagy átütő siker adhat még megold, megtámadás vagy védekezés cselekvés használatánál.

Jellemzőket akkor tudsz megszüntetni, ha kitalálod, hogy a karaktered hogyan tudja megtenni – használj poroltót a \aspect{Tomboló Tűzvész} eloltásához, használj kitérő manővereket, ha az üldöző a \aspect{Nyakadban Liheg}. A szituációtól függően ez igényelhet egy megold cselekvést (\page{18}); ebben az esetben az ellenfeled használhat védekezést, hogy megpróbálja megőrizni a jellemzőt, ha meg tudja indokolni, hogy miért tud közbeavatkozni.

Mindazonáltal, ha a jellemző megszüntetésének nincs történetbeli akadálya, egyszerűen tedd csak meg. Ha \aspect{Gúzsba Kötve} hever a karaktered, és egy barátod megszabadít, a jellemző egyszerűen eltűnik. Ha semmi sem akadályoz meg, nem kell dobni sem.

\section{Egyéb jellemző fajták}

A szokványos jellemző típusokról már volt szó a \onpage{23}. A következő típusok opcionálisak, de emelhetik a játékod színvonalát. Bizonyos szempontból ezek a karakter jellemzők variánsai (ha átfogóbban definiálod, hogy mi lehet karakter) vagy helyzet jellemzők variánsai (ha megváltoztatod, hogy mennyi ideig maradnak aktívak).

\textbf{Szervezet jellemzők:} néha egész szervezetekkel kell foglalkoznod, amik bizonyos elvek alapján viselkednek. Ilyenkor létrehozhatsz olyan jellemzőket, amit a szervezet minden tagja kihasználhat a sajátjaként.

\textbf{Kaland jellemzők:} néha a cselekmény behozhat olyan motívumokat, amik újra és újra feltűnnek a történetben. Ilyenkor létrehozhatsz olyan jellemzőket, amit minden szereplő használhat, amíg az a történetszál véget nem ér.

\textbf{Világ jellemzők:} ahogy a kalandnak, úgy a kampánynak magának is lehetnek ismétlődő motívumai. A kaland jellemzőkkel ellentétben ezek a jellemzők soha nem szűnnek meg.

\textbf{Zóna jellemzők:} a helyzet jellemzőket zónákhoz is kapcsolhatod, amik egy területet reprezentálnak a térképen (\page{29}). Ez felpezsdítheti a játékosaid térképhasználatát. A KM „aki kapja marja” alapon ingyen kihasználásokat helyezhet el bizonyos zónákban, ami odavonzza a karaktereket (mind a JK‑kat és NJK‑kat), hogy a kezdeti stratégiájukban felhasználhassák a zóna jellemzőket.
