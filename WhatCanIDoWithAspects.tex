\section{Mit tehetek jellemzőkkel?}

\subsection{Sors pontok szerzése}

A sors pontok szerzésének egyik módja, ha engeded a karakter jellemzőinek \definition{késztetését} (\page{25}), hogy komplikálja a szituációt, vagy csak nehezebbé tegye az életed. Szintén szerezhetsz sors pontokat, ha valaki ellenséges kihasználást alkalmaz a te jellemződre (lásd lejjebb), vagy ha a megadást választod (\page{37}).

Ne felejtsd el, hogy minden játékülést legalább annyi sors ponttal kezdesz, amennyi az \definition{újratöltésed.} Ha több késztetést éltél át, mint amennyi kihasználást tettél az előző játékülésen, akkor a következőben nyilván több sors ponttal fogsz kezdeni. Erről bővebben a \onpage{10} olvahatsz.

\subsection{Kihasználás}

Ha szeretnéd élvezni a jellemzők által biztosított hatalmat, akkor el kell költened egy \definition{sors pontot}, hogy \definition{kihasználd} őket egy dobáshoz (\page{14}). A sors pontjaidat nyilvántarthatod fémpénzekkel, üveggyöngyökkel, pókerzsetonokkal vagy hasonló jelképekkel.

A jellemzők ingyenesen is kihasználhatók, \emph{ha} van ingyen kihasználásod hozzájuk, amit a te vagy egy szövetséges által kreált helyzetbehozás cselekvés biztosít (\page{19}).

\subsubsection{A szókihagyásos trükk}

Ha szeretnéd a jellemzőket könnyebb beilleszteni a dobásokba, akkor próbáld a cselekedeted szókihagyással („…”) a végén leírni, és aztán befejezheted a kihasználandó jellemzővel. Valahogy így:

\example{%
Ryan így szól: „Szóval megpróbálom megfejteni a rúnákat, és…” (\textit{dob a kockákkal, de nem tetszik neki az eredmény}) „…és \aspect{Ha Nem Is Jártam Ott, Biztosan Olvastam Róla}…” (\textit{elkölt egy sors pontot}) „…így rögtön elkezdem lökni a rizsát az eredetéről.”
}

\subsubsection{Ellenséges kihasználás}

A legtöbbször a kihasznált jellemző vagy karakter jellemző vagy helyzet. De néha kihasználhatod az ellenfeled jellemzőjét (vagy egy hozzá kapcsolódó helyzet jellemzőt) \emph{ellene} is. Ezt \definition{ellenséges kihasználásnak} hívjuk, és pontosan ugyanúgy működik, mint bármilyen más jellemző kihasználása – fizess egy sors pontot egy +2 bónuszért a dobásodra, vagy dobd újra a kockákat. Ám van egy kis különbség \textbf{– minden ellenséges kihasználás esetén az ellenfélnek kell fizetned a sors pontot.} Viszont az ellenfél nem használhatja fel a kapott sors pontot az adott jelenetben. Ezt a fizetséget csak akkor kapják meg, ha a sors pont ténylegesen el lett költve, ingyen kihasználásért nem jár.

\subsubsection{Történet részleteinek kinyilvánítása kihasználással}

Egy jellemzőn alapuló, lényeges vagy valószínűtlen részletet adhatsz a történethez. Ne költs sors pontot, ha a „jellemző mindig igaz” (\page{22}) elégséges. Akkor fizess csak, ha a részlet \emph{határeset}, vagy – ha a társaság beleegyezik – amikor nincs kapcsolódó jellemző.
