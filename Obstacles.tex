\section{Akadályok}

Az ellenségek legfontosabb tulajdonsága, hogy megtámadhatók, és kiejthetők a küzdelemből. Ezzel szemben, az \definition{akadályok} legfontosabb tulajdonsága, hogy ezek egyike sem lehetséges. Az akadályok sokkal nehezebbé teszik a jeleneteket a JK"~k szempontjából, azok viszont képtelenek harcolni ellenük. Az akadályokat meg kell kerülni, elviselni vagy lényegtelenné tenni.

Bár a legtöbb akadály a környezet része, néhányuk lehet olyan karakter, akit normális módszerekkel lehetetlen kiejteni. Egy sárkány lehet egy főellenség, de ugyanígy lehet egy veszélyes akadály is. A gonosz varázslóhoz vezető utat elálló szoborgólem lehet veszély, de ugyanígy lehet torlasz vagy csak zavaró körülmény. Ez mind csak az ellenfél szerepétől függ, és hogy a JK"~k mit tehetnek ellene.

Akadályok nem szerepelnek minden jelenetben. Arra szolgálnak, hogy kihangsúlyozzák az ellenséget a jelenetben, veszélyesebbé vagy emlékezetesebbé téve őket, de az akadályok túlzott használata frusztrálóvá válhat a JK"~k számára, főleg ha harcközpontúak. Ennek ellenére nagyon hasznosak lehetnek, munkát adnak a kevésbé harcközpontú JK"~k számára egy összecsapásban.

Háromféle akadály létezik: veszély, torlasz és zavaró körülmény.

\subsection{Veszély}

Ha egy akadály meg tudja támadni a JK"~t, akkor az egy \definition{veszély}. Lángsugár, legördülő sziklák, netán egy mesterlövész, aki túl távol van, hogy közvetlenül ártani lehessen neki -- mind veszélyek. Minden veszélynek van egy neve, egy képesség szintje, valamint egy Fegyver szintje (\page{58}) 1 és 4 között.

A veszély neve egy képesség és jellemző egyszerre; vagyis a név megadja, hogy a veszély mit tud tenni, a képesség szintje megmondja, hogy mennyire jó ebben, de a nevet ugyanúgy lehet kihasználni vagy késztetni, mint bármelyik jellemzőt.

Általánosságban, a veszély képesség szintje legalább olyan magas legyen, mint a JK"~k legmagasabb képesség szintje, vagy egy kicsit még annál is magasabb. Egy magas képesség \emph{és} Fegyver szinttel rendelkező veszély valószínűleg kiejt egy vagy két JK"~t. Az veszélynek lehet alacsonyabb képesség, viszont magasabb Fegyver szintje is, amitől ritkábban fog találatot szerezni, de akkor sokkal nagyobb sérülést okoz. Ezt megfordítva olyan veszélyt kapunk, ami gyakran talál, de alig okoz sérülést.

A veszélyek ugyanúgy jönnek a cselekvési sorrendben, mint a JK"~k és azok ellenfelei. Ha a szabályok szerint dobni kell a cselekvési sorrend meghatározására, a veszélyek a képesség szintjüket használják. Amikor sorra kerül egy fordulóban, a veszély a neve által megadott cselekedetet végzi a dobáshoz a képesség szintjét használva. Ha megtámadás cselekvést végez, és a kimenetel döntetlen vagy jobb, add a Fegyver szintjét a sikerességhez. A veszélyek képesek megtámadásra és helyzetbehozásra; őket nem lehet megtámadni, és nem alkalmasak a megold cselekvésre.

Ha a játékos megold vagy helyzetbehozás cselekvést tenne egy veszély ellen, akkor passzív nehézségre kell dobnia a veszély képesség szintje ellen.

\newpage

\subsection{Torlasz}

Míg a veszélyek célja, hogy sérülést okozzanak a JK"~knak, addig a \definition{torlaszok} meggátolják őket a cselekedeteikben. Ennek ellenére a torlaszok \emph{képesek} stresszt okozni, de nem mindig teszik ezt. A legnagyobb különbség a veszélyek és torlaszok között, hogy a torlaszok nem csinálnak cselekvéseket, és sokkal nehezebb eltávolítani őket. A torlaszok passzív nehézséget adnak bizonyos körülmények között, és veszélyeztetnek, vagy sérüléseket okozhatnak, ha nem figyelnek oda rájuk.

Ugyanúgy, mint a veszélyeknél, a torlaszoknak van nevük és képesség szintjük, ahol a név egyszerre képesség és jellemző is. A veszélyekkel ellentétben, a torlaszok képesség szintje jó, ha nincs több mint egy szinttel a legmagasabb JK képesség szint fölött; különben nagyon hamar igencsak frusztrálóvá válhatnak. Egy torlasznak lehet akár 4 is a Fegyver szintje, de nem szükségszerű egyáltalán, hogy legyen neki.

A torlaszok csak bizonyos körülmények között kerülnek játékba. Egy \aspect{Kádnyi Sav} csak akkor számít, ha valaki át akar ugrani fölötte, vagy beledobják. Egy \aspect{Drótkerítés} csak arra van hatással, aki át akar mászni rajta. Egy \aspect{Szoborgólem} csak bizonyos szobákba nem enged be.

A torlaszok nem támadnak, és nem kerülnek sorra a cselekvési sorrendben. Ehelyett mindenkinek dobnia kell a torlasz szintje mint nehézség ellen, ha a torlasz akadályozhatja őket a cselekedetükben. Ha a torlasz nem tud sérülést okozni, akkor csak meggátolja a JK"~k a cselekedetében. Ha viszont sérülést tud okozni, és a JK megold cselekvése kudarc, olyankor akkora sérülést szenved el, amennyivel elvétette a célszámot.

A karakterek beletaszíthatnak valaki mást a torlaszba egy megtámadás cselekvéssel. Ha ezt tennéd, akkor dobjál normálisan megtámadásra, de add hozzá a torlasz Fegyver szintjének felét a te Fegyver szintedhez (lefelé kerekítve, de legalább egyet).

Végül, némely torlasz fegyverként vagy páncélként is használható. Ez a szituációtól függ -- bizonyos torlaszoknál például teljesen értelmetlenül hangzik. Nem tudsz elbújni egy \aspect{Kádnyi Sav} mögé, de egy \aspect{Drótkerítés} \emph{teljesen hatékony} baseball"~ütő ellen, valószínűleg meghiúsítja az egész támadást.

Ha valaki fedezéknek használná a torlaszt, döntsd el, hogy ez csökkenti, vagy meghiúsítja a megtámadást. Ha meghiúsítja, akkor a megtámadás egyszerűen nem lehetséges. Ha csak csökkenti, akkor a védekező fél a Páncél szintjéhez adhatja a torlasz képesség szintjének felét (lefelé kerekítve, de legalább egy).

Próbáld ritkán használni a torlaszokat. A torlaszok nehezebbé tesznek néhány dolgot a JK"~k számára -- ami könnyen frusztrálóvá válhat, ha túl sokat használod őket -- viszont a játékosokat meg kreatív gondolkodásra serkentheti. Láthatnak arra lehetőséget, hogy a saját javukra fordítsák a torlaszt. Ha sikerül erre módot találniuk, engedd csak meg nekik!

Néha a játékosok csak meg szeretnék szüntetni a torlaszokat. Ehhez egy megold cselekvésre kell dobniuk, a torlasz szintjénél kettővel magasabb nehézségre.

\newpage

\subsection{Zavaró körülmény}

Míg a veszélyek közvetlenül támadják a JK"~kat, a torlaszok meggátolják őket némely cselekedetben, addig a \definition{zavaró körülmények} rákényszerítik a játékosokat a saját prioritásaik átgondolására. Az akadályok közül a zavaró körülmények vannak legkevésbé játékmechanikailag meghatározva. Nem feltétlenül kell a jelenetet szabályok szerint nehezebbé tenniük. Inkább egy nehéz választás elé kell állítaniuk a JK"~kat. A zavaró körülmények az alábbi dolgokból állnak:

\begin{itemize}
    \item A zavaró körülmény \textbf{neve} egy rövid, ütős leírás. Ez lehet jellemző is, ha feltétlenül szükséges, vagy csak úgy jobbnak látod.
    \item A zavaró körülmény \textbf{választása} egy egyszerű kérdés, ami meghatározza a JK"~k által meghozandó döntést.
    \item A zavaró körülmény \textbf{hatása} az, ami a JK"~kkal történik, ha figyelmen kívül hagyják. Némely zavaró körülménynek több hatása is lehet, beleértve azt is, amikor \emph{sikeresen} kezelik.
    \item A zavaró körülmény \textbf{ellenállása} az a passzív nehézség, ami ellen a JK"~knak dobniuk kell, ha foglalkozni próbálnak vele. Nem minden zavaró körülmény igényel ellenállást.
\end{itemize}

Ha attól tartasz, hogy a JK"~k túl hamar lerendeznek egy hátralévő küzdelmet, egy"~két zavaró körülmény hozzáadásával döntésre kényszerítheted őket, hogy vajon fontosabb"~e a rossz fiúk elnáspángolása, vagy inkább a zavaró körülményekkel foglalkoznak.

A zavaró körülményekkel foglalkozásnak mindig világos haszonnal kell járnia, vagy ha ez nem lehetséges, akkor a zavaró körülmények figyelmen kívül hagyása járjon világos következménnyel.

\subsection{Akadály példák}

\subsubsection{Veszély}

\begin{itemize}
    \item Kimagasló~(+4) \aspect{Géppuska Torony}, Fegyver: 3
    \item Emberfeletti~(+5) \aspect{Távoli Mesterlövész}, Fegyver: 4
\end{itemize}

\subsubsection{Torlasz}

\begin{itemize}
    \item Jó~(+2) \aspect{Drótkerítés}, Kimagasló~(+4) nehézség eltüntetni
    \item Remek~(+3) \aspect{Kádnyi Sav}, Fegyver: 4, Emberfeletti~(+5) nehézség eltüntetni
\end{itemize}

\subsubsection{Zavaró körülmény}

\begin{itemize}
    \item \aspect{Civilekkel tömött busz} -- \textbf{Választás:} \textit{Le fog"~e zuhanni a busz a hídról?}
    \newline
    \textbf{Ellenállás:} Remek~(+3)
    \newline
    \textbf{Hatás (hagyni őket):} Minden civil meghal a buszban.
    \newline
    \textbf{Hatás (megmenteni őket):} A bűnöző elmenekül!
\end{itemize}

\begin{itemize}
    \item \aspect{Csillogó drágakő} -- \textbf{Választás:} \textit{Elemelheted"~e a drágakövet a talapzatról?}
    \newline
    \textbf{Hatás (otthagyni a drágakövet):} Nem kapod meg a (megfizethetetlen) drágakövet.
    \newline
    \textbf{Hatás (elvenni a drágakövet):} Aktiválod a templom csapdáit.
\end{itemize}
