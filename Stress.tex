\subsubsection{Stressz}

Egyszerűen megfogalmazva, a \definition{stressz} célja a főhősök életben tartása. Ez egy erőforrás, ami a küzdelemben tartja a karaktert akkor is, ha az ellenfél eltalálja. Miközben beikszelsz pár stressz dobozt, olyanokat mondasz, hogy „\emph{Éppen csak} mellément,” vagy „Jaj, az ütéstől nem kapok levegőt, de attól még megvagyok.” Ennek ellenére ez véges erőforrás – a legtöbb karakternek csak három fizikai, és három szellemi stressz doboza van, bár ezt a magas Fizikum vagy Akaraterő növelheti.

A karakterlapon két \definition{stresszmérő} található, egy a fizikai sérülésekre, egy pedig a szellemiekre. Ha elszenvedsz egy találatot, a megfelelő típusúból kell beikszelned, hogy a küzdelemben maradhass. Mindegyik stressz doboz egy sikerességet semlegesít. Több stressz dobozt is beikszelhetsz egyszerre, ha szükséges.

A dobozok két állapotúak – vagy üresek, és akkor használhatod őket, vagy beikszeltek, és akkor meg nem. De jól van ez így. Úgyis kitörlöd őket, amint véget ér a jelenet – persze, csak ha a szörnyek föl nem falnak előbb.

\subsubsection{Következmény jellemzők}

A \definition{következmények} új jellemzők, amiket azért írsz a karakterlapra, hogy jelképezzék a karakter által elszenvedett kárt és sebesüléseket.

Ha felveszel egy következményt, hogy semlegesíts egy találatot, írj egy jellemzőt egy üres következmény rubrikába, ami leírja a karakter által elszenvedett sérülést. Használd a következmény súlyosságát iránymutatóként: ha megmar egy szörny, egy enyhe következmény lehetne \aspect{Csúnya Harapásnyom}, a mérsékelt következmény \aspect{Szűnni Nem Akaró Vérzéses Harapás}, míg a súlyos következmény már \aspect{Megbénult Láb}.

Ha stresszel semlegesítesz, akkor éppen csak elkerülsz egy találatot, míg a következmény felvétele azt jelenti, hogy alaposan eltaláltak. Akkor miért is vennél fel következményeket? Mert néha a stressz nem elégséges. Ha emlékszel még, akkor az \emph{egész} sikerességet semlegesíteni kell, hogy a küzdelemben maradj. És nincs túl sok stressz dobozod. A jó hír, hogy a következmények szép nagy találatokat is semlegesíteni tudnak.

Minden karakter három üres következmény rubrikával kezd – enyhe, mérsékelt, és súlyos. Enyhe következmény felvétele két sikerességet semlegesít, a mérsékelt négyet, míg a súlyos hatot.

Például, ha bekapsz egy szép nagy találatot, öt sikerességgel, azt semlegesíteni tudod egyetlen stressz dobozzal, és egy mérsékelt következménnyel. Ez sokkal hatékonyabb, mint öt stressz dobozt elhasználni.

\newpage

A következmények hátránya, hogy jellemzők – és ugye a jellemzők mindig igazak (\page{22}). Tehát, ha \aspect{Hasba Lőtt} vagy, akkor a karaktered hasa át van lőve! Eszerint nem tudsz olyan dolgokat megtenni, amit egy hasba lőtt ember sem tudna megtenni (például rohanni). Ha ez eléggé megbonyolítaná a dolgokat, akkor számíthatsz a következmény késztetésére is. És ahogy az általad helyzetbehozással létrehozott jellemzőkhöz is jár, a következményt okozó karakter – az, aki hasba lőtt – kap egy ingyen kihasználást is. Jaj!

\example{
Charles még mindig a ghoullal hadakozik. Az felé kap a karmaival, a dobása most \dice{00++}, ami hozzáadódik a Jó~(+2) Közelharcához, és kihasználja a Húsra Éhezik jellemzőt további +2 bónuszért, ami így Fantasztikus~(+6) csapás lesz. Charles \dice{-{}-00} dobása, a Remek~(+3) Ügyességével csak egy Átlagos~(+1) védekezés; ez öt sikeresség, amit semlegesítenie kell. Egy mérsékelt következményt felvételét választja. A KM és a játékos úgy dönt, hogy a sebesülés \aspect{Tátongó Mellseb}. Ez a következmény négy sikerességet semlegesít, így csak egy marad, amire Charles elhasználja az utolsó megmaradt stressz dobozát.
}

\subsubsection{Kiejtés}

Ha képtelen vagy az egész sikerességet stresszel és következményekkel semlegesíteni, az a \definition{kiejtés}.

Kiejtődni nem kellemes. Akárki is ejtett ki, eldöntheti, hogy mi történjen. Mivel veszélyes szituációkkal és kemény ellenfelekkel állsz szemben, ez jelentheti, hogy meghaltál, de ez nem az egyetlen lehetőség. A végeredménynek a konfliktus kiterjedésének és mértékének megfelelőnek kell lennie – nem fogsz a szégyenbe belehalni, ha elveszítesz egy vitát – de valószínűleg a karakterlapod (és még más is) megváltozik. A végeredmény nem mehet szembe a csapat által lefektetett elvekkel sem – ha a csapatod úgy gondolja, hogy a karaktereket nem szabad megölni az engedélyed nélkül, az teljesen rendben van.

De ha a halál fel is merül, mint lehetőség (ezt legjobb még a dobás előtt tisztázni), a KM jobb, ha eszébe vési, hogy ez azért elég unalmas. Egy kiejtett JK elveszhet, elrabolhatják, veszélybe sodródhat, következmények felvételére kényszerülhet… a lista végtelen. Ha a karakter meghal, akkor valakinek egy új karaktert kell készítenie és bemesélnie a történetbe, de a halálnál is rosszabb végeredményeknek csak a fantáziád szab határt.

Az elképzelés alapján írd le, hogy valaki – vagy valami – hogyan is ejtődik ki. A szektatag géppuska zárótűz alá került? Vörös fröcskölés tölti be az eget, ahogy teste hangos loccsanással a földre hull. Kilöktek a kamionból, ahogy áthajtott a 26. utca felüljáróján? Eltűnsz a korlát mögött, és hátrahagynak, ahogy a csetepaté folytatódik az autópályán. Vedd lehetőségbe a halált is, a kiejtés végeredményének meghatározásánál, de legtöbbször a végzet kijátszása is legalább olyan érdekes lehet.

\newpage

\example{
A ghoul támadása igencsak szerencsés, Legendás~(+8) megtámadás, Charles Gyenge~(-1) védekezése ellen. A konfliktusnak ebben a szakaszában Charles összes stressz doboza be van ikszelve, ahogy már a mérsékelt következmény rubrikája sem üres. Még ha semlegesítene is nyolc sikerességet, enyhe és súlyos következmények felvételével, az se lenne elég. Emiatt Charles kiejtődött. A ghoul dönt a sorsa felől. A KM‑nek joga lenne ahhoz, hogy a ghoul megölje Charlest… de meghalni nem éppen a legérdekesebb.
\newline
Ehelyett a KM kijelenti, hogy Charles túlélte, mert a ghoul leütötte, és a barlangjába vonszolta, de minden következmény jellemzője megmaradt. Charles elveszve, zúgó fejjel ébred a város alatt húzódó, éjsötét katakombákban. Mivel kiejtődött, nincs más lehetősége, mint elfogadni a történéseket.
}

\subsubsection{Megadás}

Hogy hogyan tudod elkerülni a halált – vagy valami még rosszabbat? Bármilyen cselekvést megszakíthatsz konfliktus közben, ha a dobás még nem történt meg, és bejelentheted a \definition{megadást}. Egyszerűen add fel. Közöld a többiekkel, hogy ennyi volt, nem bírod tovább. A karaktered kiejtődik a konfliktusból, de \textbf{kapsz ezért egy sors pontot}, és egyet‑egyet pluszban minden következményért, amit a karaktered ebben a konfliktusban elszenvedett.

A megadás ezen kívül lehetővé teszi, hogy \emph{te} mondd meg a feltételeit, és hogy hogyan kerülsz ki a konfliktusból. Elmenekülhetsz a szörnyek elől, hogy máskor majd revánsot vehess. Viszont ez mindenképpen vereség. Valamit fel kell ajánlanod az ellenfelednek. Nem adhatod meg magad, hogy te legyél a nap hőse – ez már nem opció.

A megadás nagyon hasznos eszköz. Megadhatod magad, hogy elszökhess, miközben már a következő összecsapást tervezed, vagy szerzel egy követendő nyomot, vagy valami más hasznos dolgot. Az egyetlen, amit nem tehetsz, hogy nem nyerheted meg \emph{ezt} a csatát.

Még azelőtt meg kell adnod magad, hogy az ellenfél elgurítaná a kockákat. Nem várhatod meg a dobás eredményét, hogy csak akkor add meg magad, ha nyilvánvalóan lehetetlen nyerned – az nem lenne túl szép.

Itt számíthatsz egy kis egyezkedésre. Olyan megoldásra kell törekedned, ami minden résztvevőnek elfogadható. Ha az ellenfélnek nem tetszenek a megadásod feltételei, akkor kérhetik, hogy fogalmazd át, vagy áldozz fel valami mást, vagy csak áldozz fel többet. Mivel a megadás veszteség a számodra, emiatt az ellenfélnek legalább részlegesen el kell érnie a célját.

Minél jelentősebb árat fizetsz a megadásért, annál nagyobb jutalmat kell szereznie a te oldaladnak a megadásért – ha biztos halállal kell a csapatnak szembenéznie, de egyikük hősiesen (és öngyilkos módon) utolsó vérig küzd, akkor mindenki más megmenekülhet.
