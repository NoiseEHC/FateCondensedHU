\section{Több célpont kezelése}

Elkerülhetetlen, hogy előbb‑utóbb valamelyik játékos több ellenfélre próbáljon meg hatni. Ha ez engedélyezett, akkor az alábbi módokon lehet kezelni.

Ha meg akarod válogatni a célpontjaidat, akkor \definition{oszd el az erőfeszítést}. Dobj a képességedre, és ha az eredmény pozitív, azt akárhogyan eloszthatod a célpontok között, akik a nekik osztott erőfeszítés ellen védekezés cselekvést tesznek. Legalább egy erőfeszítést kell egy célpontra osztanod, különben egyáltalán nem számít célpontnak.

\example{
Sophie három bunkóval kerül szembe, és szeretné mindhármukat lekaszabolni egy szélvész támadással. Egy kihasználásnak és a jó dobásnak köszönhetően a Közelharc dobása Epikus~(+7). Ebből a Remek~(+3) megtámadást a legtapasztaltabbnak szánja, míg a másik kettőnek a Jó~(+2) megtámadást, ami összesen hét erőfeszítés. Ezután mindhárman védekezésre dobnak.
}

Speciális körülmények között, mint amilyen egy robbanás vagy hasonló, egy \definition{zóna támadást} tehetsz minden adott zónában álló karakter ellen, a saját csapattársaidat is beleértve. Itt nem osztod el az erőfeszítést, hanem minden zónában állónak a teljes dobás ellen kell védekeznie. A körülményeknek és az alkalmazott módszernek viszont meg kell felelnie bizonyos elvárásoknak; és a KM gyakran előírhatja egy jellemző kihasználását vagy egy fortély használatát is.

Ha szeretnéd, hogy a helyzetbehozásod egy teljes zónára vagy csoportra hasson, akkor \definition{hozd létre a jeleneten}: az egyetlen jellemzőt a zónán vagy a jelenten hozd létre, ahelyett, hogy a célpontokon egyesével jellemzőket kreálnál. Ráadásul ez még az adminisztrációt is csökkenti. Ha valaki ragaszkodik a célpontonkénti különálló jellemzőkhöz, akkor ossza el az erőfeszítést.

Ezeknél a módszereknél a célpontoknak mind egy zónában kell tartózkodniuk. A KM kivételt adhat ez alól, ha a módszer vagy a körülmények azt diktálják.

Normálisan egyetlen cselekvést kell használni – például megtámadni több ellenfelet egyetlen csapással, két problémával megküzdeni egyetlen megoldással, vagy több fontos NJK‑t megingatni hitében egyetlen helyzetbehozással. A KM engedélyezheti speciális körülmények között két cselekvés végrehajtását is, de ilyenkor mindkét cselekvést ugyanazzal a képességgel kell végrehajtani.
