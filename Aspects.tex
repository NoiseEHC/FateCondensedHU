\subsection{Jellemzők}

A \definition{jellemzők} rövid leírók, amik megadják, hogy ki is a karaktered, vagy, hogy mi fontos neki. Kapcsolódhatnak a karaktered testi vagy mentális tulajdonságaihoz, történetéhez, hitéhez, kiképzéséhez, ismeretségeihez vagy akár a különösen fontos felszerelési tárgyaihoz.

A legelső dolog, amit tudni kell róluk, hogy a \textbf{jellemzők mindig igazak} (lásd az értekezést a \onpage{22}). Más szóval, amit kitalálsz karakterkészítés közben, az mind valóságos és igaz a történetben. Ha leírod, hogy a karakter egy \aspect{Jövőbelátó Mesterlövész}, akkor ő egy jövőbelátó mesterlövész. Ezzel közhírré tetted, hogy a karakter látja a jövőt, és nagyon jó a távcsöves puskával.

A jellemzőkkel meg tudod változtatni játék közben a történetet. Lehetővé teszik, hogy javíts a dobások eredményén, és tényeket kreálhatsz a világban. Végül, a jellemzőkkel \definition{sors pontokat} szerezhetsz, ha komplikációkat okoznak a karakterednek – így a legsokoldalúbb jellemzők létrehozásához lehetőleg válassz kétélűeket, amik segíthetnek és hátráltathatnak is.

Ha szeretnél többet is megtudni a jellemzőkről, és hogy mi teszi igazán jóvá őket, olvasd el a \textit{„Jellemzők és Sors Pontok”} fejezetet (\page{22}).

\textbf{Kezdetben adj a karakternek öt jellemzőt:} egy koncepciót, egy árnyoldalt, egy kapcsolatot és még kettő szabadon választott jellemzőt. Kezdd az egészet a koncepció jellemzővel.

\subsubsection{Koncepció jellemző (High Concept)}

A \definition{koncepció} a karakter általános leírása, csak a leglényegesebb dolgokat lefedve. Képzeld el, hogy egy ismerősödnek mesélsz a karakterről, és így kezdenéd a történetet.

\subsubsection{Árnyoldal jellemző (Trouble)}

A következő az \definition{árnyoldal}, ami a karakter életének komplikálására szolgál. Ez lehet egy gyengeség, családi kötelék, vagy más kötelezettség. Válassz olyat, amit élveznél eljátszani.

\subsubsection{Kapcsolat jellemző (Relationship)}

A \definition{kapcsolat} összeköti a karakteredet egy másik JK"~val. Lehet, hogy ők már régről ismerték egymást, de az is lehetséges, hogy csak mostanában futottak össze.

Egy kapcsolat jellemző akkor jó, ha magába foglal, vagy utal valami konfliktusra, de legalábbis a kötelék féloldalasságára, ami így dinamikát ad a kapcsolatnak. Ez nem azt jelenti, hogy nyíltan ellenségesek, de az sem a legjobb, ha minden tökéletesen szép és passzol köztük.

Ha neked úgy jobb, várhatsz a kapcsolat jellemzővel, amíg mindenki többé"~kevésbé elkészül a saját karakterével.

\subsubsection{Szabadon választott jellemzők}

A karakter utolsó két jellemzőjének azt választasz, amit csak akarsz – annyi csak a megkötés, hogy szervesen kell illeszkednie a világba. Válassz olyat, ami szerinted érdekesebbé teszi a karaktered, vagy amitől élvezetesebb lesz eljátszani őt, vagy amitől jobban passzol a világba.
