% "openany" kell, különben minden fejezet páros lapon kezdődik.
\documentclass[openany]{book}

\usepackage[magyar]{babel}
\defineshorthand{"~}{\babelhyphen{nobreak}}
\useshorthands{"}

\usepackage{fontspec}
\setmainfont{EB Garamond}
\newfontfamily\FateCoreGlyphs[Mapping=tex-text]{Fate Core Glyphs} % dobókocka, cselekvés ikonok
\newfontfamily\Wingdings[Mapping=tex-text]{Wingdings3} % siker/kudarc ikonok
\newfontfamily\Roboto[Mapping=tex-text]{Roboto Light} % példák (Gotham Condensed helyett)
\newfontfamily\RobotoMedium[Mapping=tex-text]{Roboto Medium} % táblázatok (Gotham Condensed helyett)
\newfontfamily\RobotoCondensed[Mapping=tex-text]{Roboto Condensed Bold} % címek (Gotham Condensed helyett)

\usepackage{geometry}
\geometry{a4paper, outer=20mm, inner=30mm, top=20mm, bottom=20mm}

\usepackage{hyperref}
\hypersetup{
    colorlinks=true,
    linkcolor=black,
    urlcolor=gray,
}

\usepackage{titletoc}
\contentsmargin{0pt}
\titlecontents{chapter}[0pt]
     {\RobotoCondensed\Large\bfseries}{}{}{\titlerule*[0.7pc]{.}\thecontentspage}
\titlecontents{section}[0.5em]
     {}{}{}{\titlerule*[0.7pc]{.}\thecontentspage}
\titlecontents*{subsection}[1em]
     {\filright\small}{}{}{}[;\quad]
\setcounter{tocdepth}{2}

\usepackage{titlesec}
\usepackage{graphicx}
\usepackage[table]{xcolor}
\usepackage{multicol}
\usepackage{amsmath}
\usepackage{lettrine}
\usepackage[toc]{multitoc}
\usepackage{wrapfig}

\usepackage{fancyhdr}
\fancyhf{}
% Ez kikapcsolja a default vonalat a header alatt.
\renewcommand{\headrulewidth}{0pt}
\fancyhead[LE]{\textbf{\thepage\quad Fate Condensed \quad\leftmark}}
\fancyhead[RO]{\textbf{Fate Condensed \quad\leftmark \quad\thepage}}
% Ez kell, különben a \chapter oldalakon (ami fixen plain-t használ) nem lesz header, csak oldalszám footer.
\fancypagestyle{plain}{}

% Ez a szürke csík a \example oldalán.
\usepackage{mdframed}
\newmdenv[
  linecolor=lightgray,
  linewidth=3pt,
  topline=false,
  bottomline=false,
  rightline=false,
  skipabove=\topsep,
  skipbelow=\topsep
]{siderules}

% Ez kikapcsolja a jelentős üres részeket a felsorolásokban és számozásokban.
\usepackage{enumitem}
\setlist[itemize]{parsep=0pt, topsep=0pt, itemsep=0pt}
\setlist[enumerate]{parsep=0pt, topsep=0pt, itemsep=0pt}

% Ezek kellenek, különben a font nem vált vissza a {} végén.
\DeclareTextFontCommand{\dice}{\FateCoreGlyphs}
\DeclareTextFontCommand{\examplefont}{\Roboto}
\DeclareTextFontCommand{\outcome}{\Wingdings}

\newcommand{\bolditalic}[1]{\textbf{\textit{#1}}}
\newcommand{\fate}[1]{\textbf{\textit{#1}}}
\newcommand{\page}[1]{\pageref*{#1}.~oldal}
\newcommand{\onpage}[1]{\pageref*{#1}.~oldalon}
\newcommand{\pg}[1]{(\pageref*{#1}.)}
\newcommand{\aspect}[1]{\textsc{#1}}
\newcommand{\definition}[1]{\textbf{\textsc{#1}}}
\newcommand{\example}[1]{\begin{siderules}\examplefont{#1}\end{siderules}}

% Ez az átrajzolt dobókocka kép szöveg-sorba illesztésére kell.
\newcommand{\vcenteredinclude}[1]{\begingroup\setbox0=\hbox{\includegraphics[height=12pt]{#1}}\parbox{\wd0}{\box0}\endgroup}

% Ezek a számok itt nem UNICODE értékek, hanem a karakter sorszáma a karakterkészletben!!!
\newcommand{\failure}{\outcome{\XeTeXglyph7}}
\newcommand{\tie}{\outcome{\XeTeXglyph39}}
\newcommand{\success}{\outcome{\XeTeXglyph6}}
\newcommand{\successwithstyle}{\outcome{\XeTeXglyph45}}
\newcommand{\failureitem}{\item[\failure]}
\newcommand{\tieitem}{\item[\tie]}
\newcommand{\successitem}{\item[\success]}
\newcommand{\successwithstyleitem}{\item[\successwithstyle]}

\newcommand\outcomesection[3]{\subsection[#1]{#2\vspace*{-18pt}\\{\small#3}}}

\newcommand\actionsection[4]{
{
\setlength{\columnsep}{-14cm}
\begin{multicols}{2}%
{\noindent\fontsize{25pt}{0pt}\dice{#1}}
\columnbreak
\subsection[#2]{#3\vspace*{-18pt}\\{\small#4}}
\end{multicols}
\vspace*{-1em}
}
}

% Itt figyelni kell arra, hogy a \pg ne legyen \MakeUppercase-ben, mert akkor nem fogja megtalálni!!!
\newcommand\cheatsheetsection[2]{\colorbox{black}{\rlap{\textcolor{white}{\RobotoCondensed{\MakeUppercase{\textbf{#1}} \pg{#2}}}}\hspace{\linewidth}}}

\newcommand{\fatetable}[2]{
{\RobotoMedium
\rowcolors{2}{lightgray!40}{white}
\begin{tabular}{ |#1| }
\hline
\rowcolor{black}
#2
\hline
\end{tabular}
}
}

\renewcommand{\baselinestretch}{1.3}
% Ne legyen oldalszám a tartalomjegyzék oldalán.
\AtBeginDocument{\addtocontents{toc}{\protect\thispagestyle{empty}}}
