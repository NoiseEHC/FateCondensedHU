\section{Karakterkészítés játék közben}

Ha egy játékos képes gyors, kreatív gondolkodásra, elkészítheti a karakterét \emph{játék közben} is, nem muszáj előre kidolgoznia. Ez olyan, mint a regényekben, ahol a főhősről menet közben derülnek ki dolgok. Ezt nem kell erőltetni, de olyan társaságokban, akik vevők rá, mindenki kedvére tehet.

Ezen a módon a karaktereknek kezdetben csak nevük, koncepció jellemzőjük és legmagasabb képességük van, vagy még annyi sem. Ahogy a történet halad előre, és dobniuk kell egy fel nem vett képességgel, kiválaszthatnak egy üres képesség rubrikát, és felfedhetik a tudásukat azon nyomban. Ugyanígy, a jellemzőket és fortélyokat be lehet írni, ha a körülmények szükségessé teszik a használatukat, a sors pont elköltésekor vagy a bónusz felhasználásakor.

\section{Visszaszámlálás}

A visszaszámlálás sürgőssé tehet egy szituációt: foglalkozz vele azon nyomban, vagy a dolgok rosszabbra fognak fordulni. Legyen szó egy ketyegő bombáról, egy végkifejletéhez érkező rituáléról, egy függőhíd korlátján egyensúlyozó buszról, vagy egy rádiós katonáról, aki mindjárt erősítést hív, a visszaszámlálás gyors reakciókra kényszeríti a JK‑kat, vagy viselniük kell a következményeket.

A visszaszámlálások három összetevőből állnak: a visszaszámlálás‑mérő, egy vagy két feltétel és végül egy végeredmény.

A \definition{visszaszámlálás‑mérő} nagyon hasonló a stresszmérőhöz: egy sorozat doboz, amiket balról jobbra beikszelsz. Minden egyes beikszelés egy kicsivel közelebb hozza a visszaszámlálás végét. Minél rövidebb a sorozat, annál gyorsabban közeleg a vég.

A \definition{feltétel} egy esemény, ami beikszel egy dobozt a visszaszámlálás‑mérőn. Ez lehet egyszerűen, hogy „egy perc/óra/nap/forduló telte”, vagy nagyon speciális is, mint „a főgonosz elszenved egy következményt vagy kiejtődik”.

Amint beikszeled az utolsó dobozt is, a \definition{végeredmény} megtörténik, akármi legyen is az.

A KM akár be is lengetheti a visszaszámlálás‑mérőt a játékosok előtt, anélkül, hogy elárulná, mire vonatkozik, sejtetésképpen, és hogy megteremtse a kellő feszültséget a történetben.

A visszaszámlálásnak több feltétele is lehet; talán a visszaszámlálás megbízhatóan halad előre, amíg valami váratlan nem történik, ami felgyorsítja. Adhatsz különböző feltételeket is a visszaszámlálás‑mérő különböző dobozainak, ha a végeredményt események egy bizonyos sorrendje váltaná ki.
