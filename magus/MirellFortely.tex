\subsection[Mirell]{Példa: Mirell}

Mivel Mirell ember, neki nem kötelező egy jellemzőt a fajára fordítania, bár megtehetné, hogy a származására használjon fel egyet, például \aspect{Shadoni}. Ellenben a \aspect{Bizonyítanom Kell a Szürkecsulyásoknak!} jellemző tökéletes választás a Tolvaj kaszthoz, így Margit szabadon válogathat annak a fortélyaiból. A KM belátására van bízva, hogy mely kasztok fortélyait engedi szabadon felvenni, de célszerű alapból mindent megengedni ami nem megy szembe a világképpel, hiszen a játékos úgyis csinálhat egy ugyanolyan hatású fortélyt más névvel.

Margitnak nem tetszenek a Tolvaj kaszt fortélyai (\todo{igazándiból még nincs kész az a fejezet}), így inkább a képességek fortélyaiból válogat. Hogy ellensúlyozza Mirell harci haszontalanságát, felveszi neki a Lopózás képesség \stuntref{Hátbaszúrás} fortélyát (\page{Lopózás}), ami egybevág a karakter utcán nevelkedett előtörténetével, így a KM megengedi. Ez egy korlátozott szituációban -- ha a célpont teljesen készületlen -- engedélyezi a Lopózás képességet megtámadásra. Ez a Középszerű~(+0) Fegyverforgatás (hiszen nem vette fel) helyett Remek~(+3) szintet biztosít, de még a Jó~(+2) Célzáshoz képest is +1 bónusznak számít.

Margit második fortélynak a Tolvajlás \stuntref{Mindig van kiút} fortélyát választja, ami korlátozott szituációban -- menekülés közben -- ad +2 bónuszt a megold dobásokra. Ezt használva nagyon látványos manővereket végezhet, vagy pedig észrevehet kerülőutakat.

Harmadiknak saját maga tervez meg egy fortélyt, amihez az \stuntref{Építész} (\page{Ezermester}) szolgál alapul. A célja, hogy a betöréshez előkészülés közben jellemzőket hozhasson létre, amik segítik Mirellt, de ezeket ne kelljen előre meghatározni, hanem visszaemlékezés során történjenek meg. 

\stunttext{Mivel nagyonon alaposan tervezem meg a betöréseket, képes vagyok egy sors pontért Tolvajlás dobást tenni a KM által meghatározott nehézség ellen, és sikeresség számú gyengeséget fedezek fel a védelemben, amiket utólag is bemondhatok visszaemlékezés közben, de csak egyet használhatok ingyen.}

A KM elmagyarázza Margitnak, hogy az utólagos bemondás miatt ez így sajnos túl erős, és emiatt valamilyen megszorítást kell tenni. Javasolja, hogy a dobást még a betörés előtt meg kelljen tenni, és az üresen hagyott jellemzőket lehet később a cselekménynek megfelelően feltölteni. Ezen kívül ezt csak játékülésenként egyszer engedi, különben az egész játék visszaemlékezésekből fog állni.

\stunttext{Mivel nagyonon alaposan tervezem meg a betöréseket, képes vagyok terepszemle közben egy sors pontért Tolvajlás dobást tenni a KM által meghatározott nehézség ellen, és sikeresség számú gyengeséget fedezek fel a védelemben, amiket utólag is bemondhatok visszaemlékezés közben, de csak egyet használhatok ingyen, és csak egyszer tehetem meg játékülésenként.}

Mivel Margit nem kíván több fortélyt felvenni Mirell újratöltésének kárára, így Mirell újratöltése három lesz, és az első játékalmat is ugyanennyi sors ponttal kezdi. Mivel mind a Fizikum, mind az Akaraterő képességek szintje Átlagos~(+1), Mirellnek négy"~négy stressz dobozból áll mind a fizikai, mind a szellemi stresszmérője; ezen felül a következmény rubrikák száma is az alapértelmezett három.

Végül Margit leírja azt is, hogy Mirell hogyan néz ki, de sajnos az ETK példakaland erre nem sok támpontot ad.
