\subsection[Mirell]{Példa: Mirell}

Margit egy tolvajlány karakterrel szeretne játszani, akinek a neve Mirell. Egy ügyes, belevaló lányt képzelt el, aki ugyan nem egy feltűnő szépség, de azért el tudja csavarni a férfiak fejét -- viszont a karakterkészítés előtt összesen csak ennyi elképzelése van.

Koncepció jellemzőnek a \aspect{Tolvajlány} kézenfekvő választásnak tűnik, de a KM elmagyarázza Margitnak, hogy ez eléggé unalmas lenne. Emiatt a KM valószínűleg nem tudná túl gyakran játékba hozni, mert nem jutna eszébe túl sok érdekes dolog róla, így ez egy kicsit elpocsékolná a -- nem túl számos -- jellemzők egyikét. Az egyik legfontosabb dolog, hogy lehetőleg a \textbf{jellemző legyen érdekes}, hogy a KM"~nek ötleteket adhasson a történetbe ágyazásához.

Margit az előbbiek miatt inkább a \aspect{Szürkecsulyások} jellemzővel jelezné, hogy a tolvajklánba tartozik, és így kihasználhatná (\page{Kihasználás}) minden esetben, amikor a szervezet valamit el tudna neki intézni. A KM elmagyarázza Margitnak, hogy mivel a pozitív és negatív jellemzők egyforma erősek -- feltéve ha érdekesek --, a legtöbbet úgy hozhatja ki egy jellemzőből, ha egyszerre pozitív és negatív is: a \textbf{jellemző legyen kétélű} ha lehetséges. Ilyenkor ki lehet használni a pozitív oldalát, és sors pontokat termel a negatív oldala a késztetéseken keresztül (\page{Késztetés}). Margit így végül a \aspect{Bizonyítanom Kell a Szürkecsulyásoknak!} jellemzőnél marad, és megbeszélik a KM"~el, hogy ez pontosan mit is jelent -- ezt a karakterlap hátoldalára jegyzi fel, hogy el ne felejtse miben maradtak. Mirell ugyan nem tagja a szervezetnek, de azért kihasználhatja a jellemzőt bizonyos dolgok elintézésére vagy információszerzésre, viszont a Szürkecsulyások a legalkalmatlanabb pillanatokban fognak szívességeket kérni tőle. Végül, Mirell olthatatlan vágyat fog érezni vakmerő húzások elkövetésére, hogy bizonyítsa nekik rátermettségét.

Margit árnyoldal jellemzőnek sanyarú gyermekkort képzelt el, és a fentiek miatt az \aspect{Utca Gyermeke} jellemzőn gondolkodik. A KM elmagyarázza Margitnak, hogy még könnyebbé teheti a jellemző játékba hozatalát ha nem csak érdekes, de ráadásul még \textbf{több dologról is szól} egyszerre, mert ilyenkor sokkal több szituációban lehet alkalmazni. Emiatt Margit végül a \aspect{Kemény az Árvák Élete Shadvikban} jellemzőnél marad, ami szerint Mirell árva, gyerekként Shadvikban koldult az utcán, emiatt jelentős problémái vannak kapcsolatteremtésben, ami folyamatos késztetéseket -- és így sors pontokat -- jelent majd. De ezt ki is használhatja majd, hiszen mindig tanul az ember ezt"~azt komoly nélkülözések közepette, és ez meg is edzi a kitartást és akaratot. Ezen felül Shadvik tartományban játszódó kalandokban jól jöhet majd egy kis helyismeret.

Margitnak nagyon tetszene a \aspect{Rosszkor Jöttél Kígyó...} az egyik szabadon választott jellemzőnek. A KM elmagyarázza Margitnak, hogy a \textbf{világos megfogalmazás} is nagyon fontos, mert enélkül elég nehéz lesz kihasználni -- hiszen ki tudja mire vonatkozik --, és nagyjából lehetetlen lesz késztetni. Margit végül a \aspect{Lássuk Mit Rejt a Dekoltázs...} jellemzőnél marad, ami ugyan nem hangzik olyan jól mint a kígyós, de Mirell azért így is bármikor elővarázsolhat valami meglepetést, még ha egy sorozatlövő számszeríj meg is haladná a méreteit. Viszont ez arra is lehetőséget ad, hogy kacérkodjon a férfiakkal, vagy elterelje a figyelmüket. Végül -- mivel a jellemző mindig igaz (\page{A jellemző mindig igaz}) -- kimondja, hogy Mirellnek jól kivehető méretű dekoltázsa is van.

Margitnak egyelőre nincs jó ötlete a maradék két jellemzőre, így a \emph{Karakterkészítés játék közben} opcionális szabállyal (\page{Karakterkészítés játék közben}) majd később fogja kitalálni. A példajáték \etk{361} alapján a kapcsolat jellemző valószínűleg Simonel lesz.
