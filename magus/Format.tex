\fancyhead[LE]{\textbf{\thepage\quad Játékosok Törvénykönyve \quad\leftmark}}
\fancyhead[RO]{\textbf{Játékosok Törvénykönyve \quad\leftmark \quad\thepage}}

\newmdenv[
  linecolor=lightgray,
  linewidth=3pt,
  topline=false,
  bottomline=false,
  rightline=false,
  skipabove=0pt,
  skipbelow=0pt
]{siderulesnoskip}

\newcommand{\overcome}{\lettrine[lines=2]{\raisebox{7pt}{\dice{O}}}{} \textbf{Megoldás:} }
\newcommand{\advantage}{\lettrine[lines=2]{\raisebox{7pt}{\dice{C}}}{} \textbf{Helyzetbehozás:} }
\newcommand{\attack}{\lettrine[lines=2]{\raisebox{7pt}{\dice{A}}}{} \textbf{Megtámadás:} }
\newcommand{\defend}{\lettrine[lines=2]{\raisebox{7pt}{\dice{D}}}{} \textbf{Védekezés:} }
\newcommand{\noattackparam}[1]{\noindent \dice{A} #1}
\newcommand{\noattack}{\noattackparam{Ez a képesség a megfelelő fortély nélkül nem alkalmas megtámadásra.}}
\newcommand{\noattackatall}{\noattackparam{Ez a képesség nem alkalmas megtámadásra.}}

\newcommand{\stunt}[3]{\subsubsection{#1}#2\begin{siderulesnoskip}\examplefont{#3}\end{siderulesnoskip}}

\newcommand{\engskill}[2]{\label{#1}\subsection[#1]{#1 (#2)}}
\newcommand{\hunskill}[2]{\label{#1}\subsection[#1]{#1}}
