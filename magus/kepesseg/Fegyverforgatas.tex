\engskill{Fegyverforgatás}{Fight}

A Fegyverforgatás képesség felölel mindenféle testközeli harcot (más szavakkal, aminek a részvevői egyazon zónában tartózkodnak), de csak ha fegyvereket használnak. A rögtönzött fegyverek vagy a puszta ököl használata alapállapotban a Verekedés képesség alá tartozik, míg a távolba ható fegyverek a Célzáshoz.

\overcome Mivel a Fegyverforgatás nem igazán használatos konfliktuson kívül, ezért megoldásra sem gyakran használt. Elképzelhető valamiféle harci tudás demonstráció, netán egy szabályok közé szorított fegyveres sport, ami lehetővé tenné a képesség használatát a versengés szabályaival. Fontos kivétel a lefegyverzés vagy fegyvertörés, de mivel ez nem helyzet jellemző hanem csak egy történetelem, a célpont ilyenkor ugyan Verekedés képesség használatára lesz utalva, ám ez nem ad +2 bónuszt az ellenfelének.

\advantage Fizikai konfliktus esetén a Fegyverforgatás képességgel lehet a legegyszerűbben helyzet jellemzőket létrehozni. Bármiféle speciális harci manőver vagy csel leírható ilyen előnyökként: célzott bénító megtámadás, \aspect{Piszkos Csel} vagy földre döntés. Szintén használható az ellenfél harci stílusának kielemzésére gyengeségek után kutatva, amiket ki lehet használni ellene.

\attack A Fegyverforgatás csak akkor használható megtámadásra, ha a karakter rendelkezik valamilyen nem rögtönzött fegyverrel, de akkor magától értetődő hogyan működik. Fontos még egyszer megjegyezni, hogy ez csak egyazon zónában tartózkodó ellenfelek ellen használható, hacsak a KM máshogy nem dönt valami körülmény miatt.

\defend A Fegyverforgatás használható minden Fegyverforgatás képességgel végzett megtámadás vagy helyzetbehozás ellen, valamint minden olyan helyzetben, ahol az erőszakos közbeavatkozás meggátolja a cselekvést. Célzás ellen csak korlátozottan használható, ennek megállapítása a KM hatásköre: alapállapotban csak dobófegyverek ellen működik, és csak ha a karakter tudatában van a támadónak, míg lőfegyverek ellen csak a megfelelő fortély felvételével lehetséges.

\stunt{Túlütés}{}{Mivel fegyvercsapásaim annyira erőteljesek, képes vagyok átütő sikerű megtámadás esetén egy sikerességet feladva helyzet jellemzőt szerezni ingyen kihasználással, mert maradandó sérülést okozok például a páncélban.}

\stunt{Tartalék fegyver}{}{Mivel mindig gondom van második fegyverre, ha az ellenfelem sikeres megoldás dobással lefegyverezne, akkor nem kell Verekedés képességre váltanom, hanem csak egy átmeneti zavart mutató előny jellemzőt kap a támadó egy ingyen kihasználással.}

\stunt{Pusztítás}{}{Mivel képzett vagyok az ötödkori Kyrek által feltalált harcmodorban, egy sors pontért enyhe helyett mérsékelt, mérsékelt helyett súlyos következményt tudok okozni az ellenfelemnek, de csak konfliktusonként egyszer}

Ha már alapból súlyos következményt kapna, akkor muszáj mellé még egy másik következményt is felvennie, különben kiejtődik a konfliktusból.
