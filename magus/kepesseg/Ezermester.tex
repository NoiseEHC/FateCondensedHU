\engskill{Ezermester}{Crafts}

A Ezermester képesség mutatja, hogy a karakter mennyire boldogul szerkezetekkel, gépekkel és építményekkel, függetlenül attól, hogy a megépítésükről vagy az elpusztításukról van szó.

\overcome A Ezermester képesség lehetőséget ad megalkotni, elrontani vagy megszerelni valamilyen szerkezetet feltéve, hogy a megfelelő szerszámok és kellő idő is rendelkezésére áll. Általában az Ezermestert igénylő cselekedetek csak részét képezik egy nagyobb szituációnak, így igen népszerű képesség kihívásokban. Például megjavítani egy betört ajtót nem annyira érdekfeszítő, hiszen mind a siker mind a kudarc eléggé unalmas. A legjobb ilyenkor ha a karakternek egyszerűen csak sikerül, és a történet akár folytatódhat is. Persze az már egészen más, ha még azelőtt kell az ajtóval elkészülni mielőtt az élőholtak elérnék az erdei kunyhót…

\advantage A Ezermestermel olyan jellemzőket lehet létrehozni, amik valamilyen gép vagy szerkezet hasznos tulajdonságait vagy erősséget reprezentálják, amit aztán a karakter a saját hasznára fordíthat. Esetleg lehet még valami gyengeség is a konstrukcióban, amit kihasználhat. A karakter úgy is létrehozhat helyzet jellemzőket, hogy nem túl elegáns módon szabotálja a berendezést, netán gyorsan összeeszkábál valamit.

\attack Ez a képesség általában nem alkalmas megtámadásra, hacsak a konfliktus nem pont szerkezetek használatáról szól, például várostrom esetén. Normálisan a karakter által készített fegyvereket még mindig a Fegyverforgatás képességgel használják.

\defend A megtámadáshoz hasonlóan, a Ezermester nem használható védekezésre sem, hacsak nem a karakter valamilyen szerkezettel blokkolja azt.

\stunt{Mindig bütyköl valamit}{}{Mivel valami módot mindig találok arra, hogy elüssem az időt szerelgetéssel, képes vagyok sors pont elköltése nélkül kinyilvánítani, hogy a szükséges szerszám pont nálam van.}

Ez még extrém helyzetekben is így van, például ha a karakter egy börtöncellába van zárva a felszerelése nélkül.

\stunt{Jobb, mint újkorában}{}{Mivel a szerkezeteket jobban ismerem mint azok készítői, képes vagyok előny jellemző helyett egy helyzet jellemzőt -- ami a tárgy feljavítását jelképezi -- tehetek rá ingyen kihasználással, de csak ha a szerkezet megjavítása átütő siker.}

\stunt{Sebészi pontosság}{}{Mivel hihetetlen pontosan tudok célozni gépezetekkel, képes vagyok egy egész zóna megtámadásánál -- például katapulttal vagy pokolgránáttal -- meghatározni, hogy kik ne kapjanak a stresszből, és ekkor közöttük nem is kell felosztani a sikerességet.}

\stunt{Építész}{}{Mivel járatos vagyok az építészetben, képes vagyok terepszemle közben egy sors pontért Ezermester dobást tenni a KM által meghatározott nehézség ellen, és sikeresség számú jellemzőt fedezek fel vagy hozok létre az épületen, de csak egyet használhatok ingyen.}

Ezeket a kinyilvánításokat a KM nyilván megvétózhatja, mert például egy \aspect{Elhagyatott Alagút Ami Pont a Titkos Szentélybe Vezet} nem annyira izgalmas, mintha mondjuk a palotán át kéne menni.
