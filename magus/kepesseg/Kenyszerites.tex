\engskill{Kényszerítés}{Provoke}

A Kényszerítés képességgel lehet valakit feldühíteni és negatív érzelmi reakciókat kiváltani -- félelemet, haragot vagy éppen szégyent. Ez a „görénynek lenni” képesség. A használatához a játékosnak muszáj valami indoklással szolgálnia:
\begin{itemize}
    \item Teljesen természetes módon következhet a szituációból.
    \item Lehet a karakter egy jellemzője ami lehetővé teszi.
    \item Lehet egy helyzet jellemző amit egy sikeres helyzetbehozás dobással sikerült létrehozni (például Befolyásolás vagy Megtévesztés használatával).
    \item A karakter rájöhet a célpont egy jellemzőjére (például Empátia dobással).
\end{itemize}

A képesség használatának szintén feltétele, hogy a célpont képes legyen érzelmekre -- gólemek és zombik ellen általában nem vezet túl látványos eredményre.

\overcome A Kényszerítéssel a karakter rávehet valakit, hogy azt tegye amit akar, csak a megfelelő lelkiállapotba kell hajszolni őket. Megfélemlíteni, hogy információkat áruljanak el; feldühíteni, hogy már ne tudják türtőztetni magukat; esetleg annyira rájuk ijeszteni, hogy elmeneküljenek. Ez legtöbbször névtelen NJK"~k ellen működik, vagy ha a KM nem érzi őket elég fontosnak a mozzanatot, hogy lejátszassa a játékosokkal. Egy másik játékos karaktere vagy fontos NJK ellen már egy versengést kell megnyerni, amiben Akaraterővel állnak ellen.

\advantage A Kényszerítéssel a karakter pillanatnyi érzelmi állapotokat kifejező helyzet jellemzőket aggathat a célpontra, mint \aspect{Felbőszült}, \aspect{Megdöbbent} vagy \aspect{Habozó}. A célpont itt is Akaraterővel állhat ellen.

\attack A Kényszerítéssel a karakter érzelmi sérüléseket tud okozni az ellenfélnek. A kettejük kapcsolata és a körülmények jelentős mértékben befolyásolják, hogy a karakter képes"~e használni ezt a képességet megtámadásra.

\defend Nagyobb „görénynek lenni” sajnos nem segít a Kényszerítést jobban elviselni, a védekezéshez Akaraterő szükséges.

\stunt{Rettegett ellenfél}{}{Mivel annyira félelmetesen nézek ki, hogy az ellenfeleim haboznak rám támadni, képes vagyok Fegyverforgatás és Verekedés ellen Kényszerítés képességgel védekezni, de csak fizikai konfliktusban és csak az első stressz elszenvedéséig.}

Bár a karakter tétovaságra késztetheti az ellenfeleit, amint valaki rámutat, hogy ő is csak halandó, ez az előny megszűnik.

\stunt{Kihívás}{}{Olyannyira fel tudom hergelni az ellenfelem, hogy képes vagyok egy ingyen kihasználás feláldozásával magamra vonni az ellenfél következő cselekedetét elvonva a figyelmet az eredeti célpontjáról, de csak ha előzőleg én hoztam létre rajta a jellemzőt helyzetbehozással.}

\stunt{Jól van, de nem tőlem tudod}{}{Mivel a végtelenségig tudok másokat nyaggatni információért, képes vagyok Kényszerítés képességet használni Empátia helyett, hogy rávegyek valakit egy jellemzőjének az elárulására.}

A célpont ennek Akaraterővel állhat ellen, de ha a KM úgy gondolja, hogy a jellemzőt könnyebb megismerni az erőszakos úton, akkor adhat +2 bónuszt a dobásra.
