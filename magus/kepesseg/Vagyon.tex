\engskill{Vagyon}{Resources}

A Vagyon képesség mutatja a karakter által birtokolt anyagiak nagyságát, és hogy mennyire képes használni azt egy adott szituációban. Ez nem feltétlenül jelent konkrétan pénzt a karakter zsebében, hiszen egy nemes birtokainak nagysága, a hűbéreseinek mennyisége, netán a hitelképessége egy dzsad bankháznál ugyanolyan érték lehet.

Mivel a kalandozók általában nagy lábon élnek, feltételezzük, hogy a napi kiadásokra azért mindig van pénzük, így nem is számolgatjuk az aranyakat. Ez a képesség akkor kerül elő, ha valami kivételes kiadásról van szó.

\overcome A Vagyon képesség használható minden olyan szituációban, ahol a pénz megoldhat minden felmerülő problémát, legyen ez valaki megvesztegetése, netán valami ritka és értékes dolog beszerzése. Kihívás vagy versengés jelenetekben is használható, például árverés esetén, vagy üzleti tárgyaláson egymásra licitáláskor.

\advantage A karakter használhat Vagyon képességet szociális szituációkban, hogy egy kicsit megolajozza a kapcsolatot, legyen az tényleges megvesztegetés, netán csak meghívni valakit egy italra.

A Vagyon képességgel a játékos kinyilváníthatja, hogy a karakter számára szükséges dologból éppen van pár a szérűben, netán könnyen beszerezheti, ami egy megfelelő helyzet jellemzővel lesz kifejezve.

\noattackatall
 
\nodefendatall

\stunt{Pénz beszél}{}{Mivel könnyen lenyűgözök embereket a gazdagságommal, képes vagyok Befolyásolás képesség helyett Vagyont használni, de csak ha a célszemélyre lehet ilyen módon hatni.}

\stunt{Befektető}{}{Mivel a befektetéseim gyakran a vártnál is nagyobb hozamot hoznak, az így létrehozott -- befektetéseim hozamát jelképező -- helyzet jellemzőkhöz két ingyen kihasználást kapok, de csak ha a befektetés az előző játékülés folyamán ténylegesen megtörtént, utólag ezt nem lehet kinyilvánítani.}

\stunt{A pénz nem akadály}{}{Mivel gátlások nélkül tudom szórni a pénzt, egy sors pontért a Vagyon dobásom \dice{++++} lesz, de ezután a játékülés végéig -2 minden további Vagyon dobásra, és ezt csak játékülésenként egyszer tehetem meg.}
