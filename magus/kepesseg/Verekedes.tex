\hunskill{Verekedés}

A Verekedés képesség a fegyver nélküli, vagy a rögtönzött fegyverekkel történő testközeli harc, mint egy bokszmérkőzés vagy kocsmai tumultus. Akárcsak a Fegyverforgatásnál, itt is egyazon zónában kell tartózkodni a célponttal, hacsak a KM valamilyen körülmény miatt máshogy nem dönt. Nyilvánvalóan a fegyver nélküli karaktert már nem lehet lefegyverezni, viszont alapállapotban a képesség szintje sem mehet Remek~(+3) fölé ami már bokszbajnok kategória, hacsak nem Harcművész a karakter. Fontos megjegyezni, hogy fizikai konfliktusban a Verekedéssel kiejtett ellenfél narratívájának is ezt kell tükröznie, például puszta ököllel a karakter nem fogja az ellenséget lefejezni.

\overcome Nem nagyon használatos fizikai konfliktuson kívül, kivéve bokszmérkőzéseket vagy erődemonstrációkat, mint például egy tömegen erővel átjutni. Harcművészek esetén ide tartoznak a bemutatók, mint például a deszkák törése.

\advantage A karakter fizikai konfliktusban használhatja olyan helyzet jellemzők létrehozására, amik nem puszta erővel jönnek létre, mert arra a Fizikum való. Például orrba vághatja az ellenfelet, hogy a \aspect{Könnyező Szem} zavarja a látásban, vagy kirúghatja a lábát, amitől az \aspect{Földre Került} lesz. Szintén használható piszkos trükkökre, az ellenfél provokálására, vagy az ellenfél bokszstílusának felmérésére. Lehetséges, hogy a karakter ellenfele \aspect{Elbizakodott} lesz, mert a karakter fegyver nélkül állt ki ellene. Szellemi konfliktusokban megfélemlítés esetén használható, ha a cél az ellenfél meggyőzése, és nem a megölése.

\attack A Verekedés akkor alkalmas megtámadásra, ha az ellenfél szintén Verekedést használ. Ha az ellenfélnek van normális fegyvere, vagyis Fegyverforgatással védekezik, akkor a játékosnak meg kell indokolnia, hogy mi az a szituáció, ami lehetővé teszi a megtámadást. A következmények megfogalmazásánál figyelembe kell venni a megtámadás módját, vagyis nem lehetnek halálosak.

\defend A Verekedés akkor alkalmas védekezésre, ha az ellenfél szintén Verekedést használ megtámadásra vagy helyzetbehozásra, de ha van normális fegyvere, akkor a játékosnak meg kell indokolnia, hogy mi az a szituáció ami lehetővé teszi a védekezést ellene anélkül, hogy a karakter sebet szerezne. Célzás ellen csak a megfelelő fortély felvételével alkalmas védekezésre, de erre csak Harcművészeknek van lehetőségük. A Verekedés képzettséggel a karakter közbeavatkozhat, ha az adott cselekvést egy-két pofon hatásosan szakítaná meg.

\stunt{Övön alul}{}{Mivel nagyon tapasztalt vagyok a piszkos trükkök alkalmazásában, ha kihasználok egy ilyen módon létrehozott helyzet jellemzőt megtámadás dobásnál és a megtámadás siker, +1 bónuszt kapok.}

\stunt{Marcona}{}{Mivel a kinézetem félelmet kelt ellenségeimben, képes vagyok Verekedés képességgel megfélemlíteni Kényszerítés helyett, de csak ha a célpont jogosan tarthat a veréstől, például nem egy udvari bálon történik, ami tömve van testőrökkel.}

\stunt{Letaglózás}{}{Mivel ütéseim csontrepesztők, egy sors pontért megpróbálhatom kiütni az ellenfél hangadóját, és átütő siker esetén a célpont kiejtődik, a többi támadó pedig inkább meghunyászkodik az erődemonstráció láttán, de ezt csak nem vérre menő konfliktusban -- például egy kocsmai verekedésben -- és csak egyszer használhatom.}
