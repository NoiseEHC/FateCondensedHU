\hunskill{Hajózás}

A Hajózás képesség mutatja, hogy a karakter mennyire boldogul a vízi járművekkel való közlekedéssel. Ide tartozik a nyílt tengeren, de a folyókon és tavakon való haladás és tájékozódás is.

\overcome A Hajózás mutatja, hogy a karakter milyen jól tud tájékozódni a vízi közlekedésben. Jellemző akadályok lehetnek például árral szemben haladni, kikerülni a zátonyokat, vagy megtalálni egy rejtett szigetet. A karakter képes egymaga irányítani kisebb vízi járműveket, csónakokat, tutajokat, de matrózok munkáját is összehangolja, ha egy tengerjáró hajóról van szó. Az alapvető tájékozódás, a csillagképek ismerete is ide tartozik, de gyakran használják ezt a képességet a hajókon felmerülő mindennapos problémák megoldásában.

\advantage A karakter halászhat is a Hajózás képességgel. Kockázatos mennyiségű rakomány szállításával is megpróbálkozhat. Ha a csempészek életét éli, ismerheti a rejtett öblök és tengeri barlangok helyét. Eltűnhet egy másik hajó elől, de remek manőverezéssel mögé is kerülhet.

\attack Ez a képesség általában nem alkalmas megtámadásra, kivéve ha a konfliktus hajók között zajló tengeri csata.

\defend A megtámadáshoz hasonlóan, a Hajózás nem használható védekezésre sem, hacsak nem a konfliktus egy tengeri csata.

\stunt{Matrózvér}{}{Mivel matróz karrierem alatt alaposan hozzáedzőttem a megpróbáltatásokhoz, képes vagyok a Fizikum képesség helyett a Hajózást használni, de csak a hajóút során felmerülő és az emberi erőt próbára tevő fizikai kihívások során, mint például éhínség, hiánybetegségek, tengeribetegség vagy fizikai erőt igénylő ellenállások.}

\stunt{Tharr Pörölyétől nyugatra}{}{Mivel az anemónák és szextánsok sem helyettesíthetik az igazi navigátor ösztöneit, ha a Hajózás képességet használom helyzetbehozáshoz, +2 bónuszt kapok, de csak ha sietségről van szó.}

\stunt{Hajóács}{}{ Mivel hajóácsként nem csak vezetem, hanem építem és javítom is a hajókat, képes vagyok Ezermester képesség helyett a Hajózást használni, ha a tengeren kell valami javítást végrehajtani.}

\stunt{Vén tengeri medve}{}{Mivel fél szemem mindig a horizonton, fél kezem mindig a kormánykeréken, ha Hajózás képességet használok megoldásra, +2 bónuszt kapok, de csak ha előtte volt egy sikeres helyzetbehozásom.}
