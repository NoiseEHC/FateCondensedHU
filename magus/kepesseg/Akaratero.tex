\label{Akaraterő}
\section[Akaraterő]{Akaraterő (Will)}

Az Akaraterő Képesség mutatja a Karakter szellemi erejét és kitartását, a lelkierejét, állhatatosságát és bátorságát, ahogy a Fizikum a testi erő és kitartást jelképezi.

Az Akaraterő Képesség extra Szellemi Stressz kockákat vagy Szellemi Következményt adhat. Az (+1 Átlagos) vagy (+2 Tűrhető) szint három Szellemi Stressz kockát jelent, a (+3 Jó) vagy (+4 Kimagasló) pedig négyet. Az (+5 Emberfeletti) egy plusz Enyhe Következményt is ad a kockák mellé, ami csak szellemi sérülésre használható.

OAkadály Leküzdése: Az Akaraterőt a Karakter olyan problémák ellen használhatja, amik mentális erőfeszítést igényelnek. Ide tartoznak a fejtörők és rejtvények, valamint minden olyan cselekedet, ami szellemileg leterheli a Karaktert, például visszafejteni egy titkosított üzenetet. Az Akaraterőt kell használni, ha a feladatot pusztán szellemi munkával le lehet küzdeni, míg ha a gondolkodáson kívül többre is szükség van, akkor a Tudás Képességgel kell Próbát tenni. Sok feladatot, amit Akaraterővel kell leküzdeni, egy nagyobb Kihívás részeként érdemes a játékba illeszteni.

Az Akaraterővel tartott Versengés lehet valami nagyon nehéz játék, mint mondjuk sakk, vagy netán egy verseny, hogy ki végez a legtöbb ponttal egy vizsgán. Ide tartoznak a mágikus vagy pszi energiákkal vívott Szellemi Konfliktusok is.

CSzituáció Kiaknázása: A Karakter az Akaraterő Képességgel saját magára tehet Szituációs Jellemzőket, amik az elmélyült koncentrációt és fókuszálást jelképezik.

ATámadás: Ez a Képesség nem alkalmas Támadásra a megfelelő Specialitás nélkül.
 
DVédekezés: Az Akaraterő a legfontosabb Képesség a Kényszerítés ellen, mert ez adja meg, hogy a Karakter mennyire képes uralkodni a viselkedésén. Szintén ez használatos leginkább a mentális mágikus Támadások ellen.

Az Akaraterő Képességhez az alábbi Specialitások választhatók:

\begin{itemize}
    \item Belső erő: A Karakter olyan erős akarattal rendelkezik, hogy semmi sem állíthatja meg. Akadály Leküzdése Próbáknál használhatja Fizikum helyett az Akaraterőt, ha az puszta erő megnyilvánulásról szól.
    \item Fájdalomtűrés: Ha úgy dönt a Játékos, a Karakter a jelenet végéig figyelmen kívül hagyhatja egy Enyhe vagy Mérsékelt Következmény hatását. Nem lehet se Késztetésre használni, sem az ellenfelek nem képesek Kihasználni. A jelenet elmúltával persze kétszeresen kapja vissza a Karakter, ha Enyhe Következmény volt, akkor Mérsékeltté, ha Mérsékelt volt, akkor Súlyossá változik.
    \item Rettenthetetlen: +2 bónusz Védekezés dobásokra megfélemlítés (Kényszerítés) vagy normális félelem ellen.
\end{itemize}
