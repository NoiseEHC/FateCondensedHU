\engskill{Akaraterő}{Will}

Az Akaraterő képesség mutatja a karakter szellemi erejét és kitartását, lelkierejét, állhatatosságát és bátorságát, ahogy azt a Fizikum a testi erővel és kitartással teszi.

Az Átlagos~(+1) vagy Jó~(+2) szint négy szellemi stressz dobozt jelent, a Remek~(+3) vagy Kimagasló~(+4) pedig hatot. Az Emberfeletti~(+5) ezen kívül egy extra enyhe következmény rubrikát is ad, ami csak szellemi sérülésre használható.

\overcome Az Akaraterőt a karakter olyan problémák ellen használhatja, amik mentális erőfeszítést igényelnek. Ide tartoznak a fejtörők és rejtvények, valamint minden olyan cselekedet, ami szellemileg leterheli a karaktert, például visszafejteni egy titkosított üzenetet. Az Akaraterőt kell használni, ha a feladatot pusztán szellemi munkával le lehet küzdeni, míg ha a gondolkodáson kívül többre is szükség van, akkor a Tudás képességgel kell próbát tenni. Sok feladatot, amit Akaraterővel kell leküzdeni, egy nagyobb kihívás részeként érdemes a játékba illeszteni.

Az Akaraterővel tartott versengés lehet valami nagyon nehéz játék, mint mondjuk sakk, vagy netán egy verseny, hogy ki végez a legtöbb ponttal egy vizsgán. Ide tartoznak a mágikus vagy pszi energiákkal vívott szellemi konfliktusok is.

\advantage A karakter az Akaraterő képességgel saját magára tehet helyzet jellemzőket, amik az elmélyült koncentrációt és fókuszálást jelképezik.

\noattack

\defend Az Akaraterő a legfontosabb képesség a Kényszerítés ellen, mert ez adja meg, hogy a karakter mennyire képes uralkodni a viselkedésén. Szintén ez használatos leginkább a mentális mágikus megtámadások ellen.

\stunt{Belső erő}{}{Mivel olyan erős akarattal rendelkezem, hogy semmi sem állíthat meg, képes vagyok az Akaraterő képességet használni megoldásra a Fizikum helyett, de csak ha az a puszta erő megnyilvánulásáról szól.}

\stunt{Fájdalomtűrés}{Ha úgy dönt a játékos, a karakter a jelenet végéig figyelmen kívül hagyhatja egy fizikai következmény hatását: nem késztet és az ellenfelek sem képesek kihasználni; ám a jelenet elmúltával persze kétszeresen kapja vissza.}{Mivel olyan erős akarattal rendelkezem, hogy teljesen kizárhatom a tudatomból a fájdalmat, a jelenet végéig figyelmen kívül hagyhatom egy enyhe vagy mérsékelt következmény hatását, de a jelenet elmúltával az enyhe következmény mérsékeltté, a mérsékelt pedig súlyossá változik.}

\stunt{Rettenthetetlen}{}{Mivel olyan erős akarattal rendelkezem, hogy semmi sem ijeszhet meg, ha az Akaraterő képességet használom védekezésre, +2 bónuszt kapok, de csak megfélemlítés (Kényszerítés) vagy normális félelem ellen.}
