\engskill{Nyomozás}{Investigate}

A Nyomozás képességet kell használni, hogy kiderítsünk valamit. Ez az Észlelés kiegészítője -- míg az Észlelés pusztán a környezet észleléséről és felszínes benyomásokról szól, addig a Nyomozás tudatos erőfeszítésről és részletekbe menő kutakodásról.

\overcome A Nyomozás akadályai mind-mind információk, amit nehéz megszeretni valami okból. Elemezni egy bűntény helyszínét bizonyítékok után kutatva; átkutatni egy kacatokkal teli szobát az egyetlen holmiért ami szükséges; netán átnyálazni egy ősöreg fóliánst, hogy megtaláljuk az az utalást, amitől minden zavaros részlet értelmet nyerhet.

Az idő ellen dolgozva bizonyítékot gyűjteni, mielőtt még a városőrség megérkezne vagy valami katasztrófa történne, tipikus módja a Nyomozás használatának egy kihívásban.

\advantage A Nyomozás valószínűleg a legsokrétűbben használható képesség, ha helyzet jellemzőkről van szó. Feltéve, hogy a karakter hajlandó rászánni a megfelelő időt, gyakorlatilag bármit kideríthet bárkiről, megfigyelhet minden részletet egy helyszínen vagy egy tárgyon. Sőt, kitalálhat bármilyen jellemzőt bármiről a játékban, ha a legkisebb esély van arra, hogy a karakter erre az információra rábukkanhat.

Ha ez túl széleskörűnek hangozna, akkor pár példa: belehallgatózni mások beszélgetésébe, bűnjelek után kutatni egy a tetthelyen, végignyálazni egy jegyzéket, kideríteni az információ valóságtartalmát, megfigyelést folytatni, előállni egy hihető háttértörténettel.

\noattackatall
 
\nodefendatall

\stunt{Részletekbe menő}{}{Mivel annyira jó vagyok az arcrezdülések megfigyelésében, képes vagyok Megtévesztés ellen Nyomozás képességgel védekezni Empátia helyett.}

\stunt{Csupa fül}{}{Mivel állandóan nyitvatartom a fülem, ha a Nyomozás dobásom siker helyzetbehozásnál, akkor egy extra helyzet jellemzőt is létrehozok, de ehhez már nem kapok ingyen kihasználást.}

\stunt{Dzsenn logika}{}{Mivel kitanítottak a Dzsenn logika használatára, egy sors pontért képes vagyok logikus következtetéssel sikeresség számú jellemzőt felfedezni vagy létrehozni a helyszínen vagy a vizsgálódás tárgyán, de jelenetenként csak egyszer és összesen csak egyetlen ingyen kihasználást kapok.}
