\engskill{Célzás}{Shoot}

A Célzás képesség a Fegyverforgatás kiegészítője, az íj és egyéb dobó valamint lőfegyverek használata. Ellent nem álló célpontok ellen használható -- konfliktusban vagy azon kívül --, mint amilyen például a csűr oldalára festett céltábla. A nyílvesszőket nem számoljuk, ugyanis feltételezzük, hogy a karakter harc közben automatikusan összeszedi őket, így kifogyásuk lehetőleg csak enyhe következmény vagy helyzet jellemző legyen.

\overcome Hacsak nem kell a karakternek harcon kívül demonstrálnia a lövés tudását, nem fogja akadályok ellen használni. Javasoljuk viszont, hogy a játékosok találjanak valami alkalmat az Yneven nagyon népszerű íjász versenyeken való részvételre, ha a karakterük erre specializálódott.

\advantage Fizikai konfliktusban a Célzás többféle manőverre alkalmas, mint például a trükkös lövés, valakire zárótüzet zúdítani és hasonlók. Néhány kalandozó képes még lefegyverezni is ellenfeleit, vagy a ruhaujjuknál fogva az ajtófélfához szegezni őket. Szintén használható helyzet jellemzők létrehozására, ha a karakter lőfegyverekről szóló mély tudásával látja, hogy az ellenfele fegyvere \aspect{Hajlamos Beragadni}.

\attack A Célzás képesség alkalmas megámadásra, a Fegyverforgatással ellentétben egy vagy két zóna távolságig. Bizonyos fegyvereknek lehet nagyobb is a hatótávja, de egyazon zónában álló ellenfél ellen nem használható, hacsak a KM úgy nem dönt, hogy a zóna elég nagy hozzá.

\defend A Célzás annyiban különleges, hogy nem igazán alkalmas védekezésre, általában az Atléta használható ilyen helyzetben. Ennek ellenére néha lehetséges például zárótűzzel fedezni valakit -- ez védekezés cselekvésként használható ha a karakterrel lévő személyt tá,támadják, vagy a karakter dobhat aktív ellenállást így -- de ezek a helyzetek ugyanilyen könnyen kezelhetők helyzet jellemzőkkel is, mint mondjuk \aspect{Fedezz!} vagy \aspect{Nyílfelhő}.

\stunt{Célzott támadás}{}{Mivel hihetelen pontosan célzok, képes vagyok létrehozni egy helyzet jellemzőt is (mint mondjuk \aspect{Átlőtt Kézfej}) a fizikai stresszen felül, de csak ha még a megtámadás előtt kinyilvánítom egy sors pontért és a dobás utána siker.}
\stunt{Fegyverrántás}{}{Mivel hihetelen jó a szemem, képes vagy a Célzást használni Érzékelés helyett a cselekvési sorrend meghatározásához, de csak ha a gyors lövés vagy dobás előnyt jelent a konfliktusban.}
\stunt{Halálos pontosság}{}{Mivel hihetelen pontosan célzok, képes vagyok konfliktusonként egyszer egy előny vagy helyzet jellemzőt kétszer is ingyen kihasználni, de csak ha az annak a következménye, hogy rászántam az időt a tökéletes célzásra.}
