\engskill{Megtévesztés}{Deceive}

A Megtévesztés a hazudozás és félrevezetés képessége.

\overcome A Megtévesztés alkalmas arra, hogy a karakter rászedjen és kihasználjon valakit, vagy elhitessen valakivel egy hazugságot, netán kiszedjen valakiből valamit, ha az hisz az elhintett hazugságoknak. Egy NJK ellen ez csak egy sima megoldás dobás, de játékos karakterek vagy nevesített NJK"~k ellen ez egy versengés lesz amiben az ellenfél Empátiával állhat ellen. Ha a karakter megnyeri a versengést, akkor jogosult lesz egy helyzet jellemzőt tennie a vesztesre, ha a porhintés benyalása segítheti a karaktert egy elkövetkező jelenetben.

Szintén a Megtévesztés használható, ha el kell dönteni, hogy egy álruha működik"~e vagy sem, készítse a karakter saját magára vagy másokra. Az álruha készítéséhez szükség van kellő időre, és megfelelő alapanyagokra.

A Megtévesztés szintén használható kisebb szemfényvesztő trükkökre, vagy a figyelem elterelésére.

\advantage A karakter használhatja a Megtévesztés képességet átmeneti figyelemelterelésre, fedősztorikat vagy hamis benyomásokat hozhat létre. Beijeszthet egy kardpárbajban, az ellenfelet \aspect{Egyensúlyvesztés}"~sel sújtva, ami tökéletes a következő megtámadáshoz. Megpróbálkozhat akár a „Nézd mi van mögötted!” trükkel is, ami egy \aspect{Testhossznyi Előny} ha el akar futni. Kialakíthat egy \aspect{Dúsgazdag Nemes Fedősztori}"~t is, ha egy fogadásra készül. Átverhet valakit, hogy elárulja egyik jellemzőjét vagy valami más információt.

\attack A Megtévesztés egy indirekt képesség, ami rengeteg lehetőséget ad, amit ki lehet használni, de nem alkalmas direkt megtámadásra.
 
\defend A Megtévesztés képesség használható egy Nyomozás félrevezetéséhez, hamis információk csepegtetésével. A karakter Védekezhet Empátia ellen is, ami a valódi szándékai után kutakodna.

\stunt{Hazugság hazugság hátán}{}{Mivel képes vagyok a saját hazugságaimat folyamatosan überelni, ha a Megtévesztés képességet használom helyzetbehozásra, +2 bónuszt kapok, de csak ha a célpont az adott játékülés alatt már elhitte az egyik hazugságom.}

\stunt{Manipulálás}{}{Mivel egy jó hazugsággal teljesen össze tudom zavarni az embereket, képes vagyok Kényszerítés helyett Megtévesztéssel intézni megtámadást, ami szellemi stresszt okoz.}

\stunt{Álnevek}{A karakter egy új szereplővel találkozva egy sors pontotért kinyilváníthatja, hogy már ismeri régebbről de egy másik nevet és személyiséget használva.}{Mivel már annyi álnevet használtam, egy sors pontért létrehozhatok egy fedősztorit egy helyzet jellemzővel valakin, akivel később Befolyásolás helyett Megtévesztés használható, de csak szociális interakciókban és csak új szereplővel működik.}
