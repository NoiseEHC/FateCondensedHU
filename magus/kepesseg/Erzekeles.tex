\engskill{Érzékelés}{Notice}

Az Érzékelés képesség pont arra való, mint amit a neve mond -- hogy észrevegyünk dolgokat. Ez a Nyomozás kiegészítője, ami a karakter általános észlelő képességeit mutatja, például messziről kiszúrni a részleteket. Általában az Érzékelés használata sokkal gyorsabban lezajlik mint a Nyomozásé, emiatt a megszerzett információ is sokkal felületesebb, viszont nem is igényel akkora erőfeszítést.

\overcome Az Érzékelés általában nem igazán használható megoldásra, de ha mégis, az inkább reaktív: észrevenni valamit a helyszínen, meghallani egy erőtlen hangot, kiszúrni az elrejtett tőröket egy tolvaj ruhája alatt.

Ám ez nem jelenti azt, hogy a KM"~nek ezután minden pillanatban Érzékelés próbákat kéne dobatnia, hogy meghatározza a karakterek éberségét, mert ez nagyon unalmas lenne. Ehelyett célszerű csak akkor dobni, ha a siker valami érdekes eseményt idéz elő, és a kudarc is legalább ennyire érdekesbe torkollik.

\advantage A karakter direkt megfigyeléssel tehet szert fölényre -- végignézni a szobán, hátha valami szokatlan feltűnik, megtalálni a menekülőutat egy romos épületben, észrevenni valakit aki kitűnik a tömegből. Ha a karakter valakit megfigyel, csak a külső tulajdonságairól szerez tudomást; hogy megtudja mi történik belülről, arra az Empátia szolgál. A karakter kinyilvánítson valamit az Érzékelés képességgel, például egy jókor felfedezett \aspect{Menekülési útvonal} jól jöhet elhagyni az épületet, netán lehet egy \aspect{Leheletfinom gyengeség} az ellenfél védelmi tervében. Ugyanígy egy kocsmai verekedésben a karakter felfedezhet egy pocsolyát az ellenfél lábánál, amin aztán az el is csúszik majd.

\noattackatall

\defend Az Érzékelés használható védekezésre Lopózás ellen, hogy a karakter elkerülje a meglepetéseket és lesből támadókat. Szintén használható, hogy a karakter észrevegye ha valaki éppen megfigyeli őt.

\stunt{Veszélyérzék}{}{Mivel szinte természetfeletti képességem van a veszély megérzésére, az Érzékelés képességet hátrányok nélkül használhatom teljes takarás, sötétség vagy hasonló érzékszervi korlátok mellett is, de csak ha valaki vagy valami kárt szeretne tenni bennem.}

\stunt{Testbeszéd}{}{Mivel szinte misztikus módon érzékelem mások viselkedését, képes vagyok Érzékeléssel rejtett jellemzőket felfedni Empátia helyett.}

\stunt{Kapáslövés}{}{Mivel villámsebesek a reflexeim, képes vagyok a Célzás helyett Érzékeléssel gyors kapáslövéseket leadni, de ilyenkor nem adhatom meg a célpontot pontosan.}

Mivel ez nem tudatos cselekedet, ha például a karakter mozgást észlel a bozótban, elsőként reagálhat rá, de nem tudja megállapítani a lövés előtt, hogy a célpont barát"~e vagy ellenség. Csak óvatosan!
