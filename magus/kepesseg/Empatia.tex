\engskill{Empátia}{Empathy}

Az Empátia teszi a karaktert képessé arra, hogy észrevegyen leheletfinom változásokat mások érzelmeiben vagy viselkedésében. Gyakorlatilag ez az érzelmi Érzékelés képesség.
Ez a fő képesség a szellemi következmények gyógyítására is.

\overcome Általában az Empátia nem alkalmas megoldásra, hiszen a célja, hogy a karakter észrevegyen valamit amit majd egy másik képességgel tud kihasználni. Néha viszont a KM kérhet Empátia dobást, hogy kiderüljön, hogy a karakternek feltűnt"~e ha valakinek megváltozott a hozzáállása vagy szándéka.

\advantage A karakter használhatja az Empátiát, hogy kiderítse valaki érzelmi állapotát, legyen egy átfogó képe arról, hogy ki is ő valójában, feltéve, hogy személyes kontaktusba kerül vele. Leggyakrabban arra használt, hogy jellemzőket fedezzen fel egy másik karakter karakterlapján, de lehetséges új jellemzőket is kreálni, főleg NJK"~k esetén. Ha a célpontnak a legkisebb gyanúja is felmerül, hogy a karakter megpróbálja kipuhatolni a titkait, védekezhet Megtévesztés vagy Befolyásolás alkalmazásával.

Az Empátia arra is használható, hogy a karakter megtudja, hogy milyen körülmények tehetnek majd lehetővé egy elkövetkező mentális megtámadást, milyen témák képesek a célpont idegeit pattanásig feszíteni.

\noattackatall

\defend Az Empátia képesség használható Megtévesztéssel végzett megtámadás ellen, amivel a karakter keresztül láthat a szitán, és szembesülhet a másik valódi szándékaival. Szintén használható azok ellen, akik szociális előnyök szerzésére használnák a Megtévesztést.

\stunt{Igazlátó}{}{Mivel igazlátónak születtem, ha Empátia képességet használok védekezésre, +2 bónuszt kapok, de csak ha hazugságok észlelésére és kiderítésére használom, és ez függetlenül attól, hogy a rám vagy másvalakire irányul.}

\stunt{Látszott rajtuk}{}{Mivel teljesem kiismertem már a bajkeverők viselkedését, képes vagyok Empátiát használni Érzékelés helyett a konfliktus cselekvési sorrendjének meghatározására, de csak ha lehetőségem volt a résztvevőkkel beszélni vagy megfigyelni őket a megelőző pár percben.}

\stunt{Lélekbúvár}{}{Mivel mások lelkének legmélye is nyitott könyv számomra, képes vagyok egy súlyos szellemi következményt mérsékeltté vagy egy mérsékeltet enyhévé szelidíteni, vagy megszüntethetek egy enyhe szellemi következményt, de csak ha legalább fél órán át beszélgetek a gyógyítottal és az Empátia dobásom siker, és játékülésenként csak egyszer tehetem ezt.}

A nehézség enyhe következménynél Jó~(+2), mérsékeltnél Remek~(+3) és súlyosnál Kimagasló~(+4). Érdemes megjegyezni, hogy a fortély nélkül a karakter szintén az Empátia képességét használná gyógyításra, de nagyobb nehézségekkel, és nem is szüntetné meg teljesen a következményt, csak átnevezné gyógyulást jelentőre.
