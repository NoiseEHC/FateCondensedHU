\engskill{Tudás}{Lore}

A Tudás képesség felölel mindent, ami az ismeretekkel, tanulmányokkal és műveltséggel kapcsolatos. Legyenek akár népi mendemondák, mindennapi bölcsességek, netán a Sigranomói Egyetem legféltettebb tekercseinek tartalma, mind ide tartozik.

\overcome A Tudás képességet kell használni, ha a karakternek a gyakorlatban kell felhasználnia elméleti ismereteit, a célja eléréséhez. Például ilyesmi lehet kibogarászni egy elfeledett nyelven írt feliratot egy ősi sírbolton feltételezve, hogy a karakter a múltban már foglalkozott az adott dialektussal.

Gyakorlatilag bármikor lehet a Tudásra dobni ha kérdéses, hogy a karakter meg tud"~e válaszolni egy nehéz kérdést feltéve, hogy valami érdekes következménye van a tudatlanságnak.

\advantage A Nyomozáshoz hasonlóan a Tudás képesség is szerteágazó módon képes jellemzőket létrehozni, ha a karakternek van lehetősége egy kis kutatásra. Legtöbbször a karakter a Tudást használja, hogy a történet egy újabb részletéről lebbentse le a fátylat, legyen az valami amit a karakter már régebben is tudott, netán valami olyan amit csak most fedezett fel; de ha ez csak egy eljövendő jelenetben lesz a hasznára, akkor célszerű jellemzők csinálni az információból. A karakter szintén fölénybe kerülhet másokkal szemben felhasználva tanulmányait valamely tantárgyban, ami szintén egy érdekes módja, hogy valami részlettel kiegészítse a játék világát.

\noattackatall
 
\nodefendatall

\stunt{Erről olvastam!}{}{Mivel több száz, akár több ezer könyvet is elolvastam már mindenféle témakörökben, képes vagyok egy sors pontért Tudás képességet használni bármelyik nem mágikus képesség helyett, de csak ha meg tudom indokolni, hogy hogyan is olvashattam az adott dologról amivel éppen próbálkoznék, és csak egyetlen dobásra.}

\stunt{A racionalitás talaján}{}{Mivel racionálisan elemezve a helyzetet a fenyegetések többsége csak az elménk szüleménye, képes vagyok a Tudás képességet használni védekezésre Kényszerítés ellen, de csak ha képes vagyok megindokolni hogyan tudom legyűrni a félelmem logikus érveléssel.}

\stunt{Specialista}{}{Mivel specializálódtam egy tudományágra -- mint mondjuk térképészet, heraldika, gyógyfüvek vagy orvoslás --, ha a Tudás képességet használom megoldásra vagy helyzetbehozásra, +2 bónuszt kapok, de csak ha az adott specializációba tartozó tudásra van szükség.}

\stunt{Mágiaismeret}{}{Mivel alaposan tanulmányoztam a mágia működését, képes vagyok a Tudás képességet használni mágikus hatások felismerésére akkor is, ha nincs mágikus képességem.}
