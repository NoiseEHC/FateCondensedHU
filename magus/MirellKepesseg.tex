\subsection[Mirell]{Példa: Mirell}

Margit legmagasabb képességnek a Tolvajlást (\page{Tolvajlás}) veszi fel Mirellnek Kimagasló~(+4) szinten, ami elég nyilvánvaló választás egy tolvaj számára. Másodiknak a Lopózást (\page{Lopózás}) választja Remek~(+3) szinten, mert inkább besurranó tolvajt szeretne, mint egy tolvajcéh verőemberét. Margit harmadiknak még szeretné felvenni a Tudás (\page{Tudás}) képességet szintén Remek~(+3) szinten, hogy Mirell állandóan kioktathassa a többi karaktert. A KM elmagyarázza Margitnak, hogy a karakter magasabb képességeit illik alátámasztani a háttértörténettel, és sajnos mivel Mirell az utcán nevelkedett, és semmiféle képzésben sem részesült, így ezt nem engedélyezi.

Margit végül harmadiknak a Megtévesztés (\page{Megtévesztés}) képességet veszi fel, mert valamiféle simlis bajkeverőt képzelt el. Ezzel már meg is van a piramis csúcsa, ami nagyjából megadja, hogy a karakter miféle szituációkban fog jól teljesíteni, és emiatt a KM is megpróbálja majd a történetet ilyen irányba terelni, hogy lehetősége is legyen a tehetsége megvillantásának.

Mivel kicsit tanácstalan a többi képességgel kapcsolatban, a KM elmagyarázza Margitnak, hogy az Akaraterő (\page{Akaraterő}) és a Fizikum (\page{Fizikum}) szintje akkor is fontos, ha amúgy nem meditáló szerzetes vagy vasgyúró karaktert szeretne: ugyanis az alapból 3 stressz doboz nagyon kevés, és ezek extra dobozokat biztosítanak (\page{Stressz dobozok és következmény rubrikák}). Így Margit felveszi mindkettőt Átlagos~(+1) szinten egy"~egy szellemi és fizikai stressz dobozért, mert a Jó~(+2) úgysem ad több extra dobozt, és inkább valami fontosabbat választana arra a szintre.

Bár Margit nem szeretne harcias amazont, a KM elmagyarázza neki, hogy valamiféle harci képességre azért szüksége lesz, hogy Mirell túlélhesse az első összecsapást, és ne unatkozzon a fizikai konfliktusok során sem. Margit így a Jó~(+2) szintre a Célzást (\page{Célzás}) veszi fel a dobótőrőknek, és az Atlétát (\page{Atléta}) a védekezéshez.

Margit a maradék szintekre a Nyomozás (\page{Nyomozás}), Észlelés (\page{Észlelés}) és Tudás képességeket választja, és ezzel meg is van a tolvajlány, aki nem a szavak embere hanem egy kicsit geek, ami egy kicsit esendővé és szerethetővé teszi a karaktert a tökéletes főhösökkel szemben. A Tudás képesség Átlagos~(+1) szinten inkább azt jelképezi, hogy Mirell sokfelé megfordult és mindenhol hallott ezt"~azt, mintsem tanulmányokat vagy kutatást.

\begin{center}
\fatetable{r l}{
\textcolor{white}{Szint} & \textcolor{white}{Képességek} \\
Kimagasló~(+4) & Tolvajlás \\
Remek~(+3) & Lopózás, Megtévesztés \\
Jó~(+2) & Atléta, Célzás, Nyomozás \\
Átlagos~(+1) & Akaraterő, Észlelés, Fizikum, Tudás \\
}
\end{center}

Ha Margit a \emph{Karakterkészítés játék közben} opcionális szabályt (\page{Karakterkészítés játék közben}) használta volna, akkor elég lett volna a képesség piramis csúcsát kitöltenie, és csak akkor beírni a többi képességet, amikor először dob rájuk. Ilyenkor viszont előfordulhatnak kisebb"~nagyobb inkonzisztenciák: például az egyik jelenetben felveszi a Fizikum képességet magas szinten, és így nem ájult volna el az előző -- harci -- jelenetben. Vagy miért fordultak vissza nemrég a kulcsot keresni, ha most meg hirtelen kiderült, hogy van Tolvajlás képessége amivel feltörhette volna a zárat? Célszerű nagyvonalúan átsiklani ezeken a furcsaságokon, és inkább úgy tekinteni a történetre mint egy regényre, ahol a főszereplő mindig újabb és újabb oldalát ismerhetjük meg.
