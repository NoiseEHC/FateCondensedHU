\chapter{Alapok}

\section{A világ kidolgozása}

Minden Fate játék a világ kidolgozásával kezdődik. Ez lehet a KM saját agyszüleménye, a játékosok által ismert könyv vagy film világa, netán a jelenlévők együtt építhetik fel nulláról. A résztvevők lehet, hogy éppen csak áttekintik a részleteket, amiket muszáj tudni a játékhoz, de az is megeshet, hogy egy egész játékülést ezek kidolgozására áldoznak. A két véglet között bármi elképzelhető.

Az általatok választott világ adja meg, hogy mit tekint a csapat valóságnak, és mi elfogadható a játékban és a karakterkészítés közben. Ha a világotokban az emberek nem tudnak repülni, akkor egy röpképes karakter koncepciója nem állja meg a helyét. Ha a világotokban háttérhatalmak ügyködnek titkos összeesküvések hálójában, akkor a játékosok olyan történetekre számítanak, amiben nem a jó és gonosz erői csapnak össze, és nincsenek nevetséges gyilkos bohócok sem. Ez mind csak tőletek függ!

\section{Karakterkészítés}

\subsection{Ki vagy valójában?}

Amint a résztvevők döntésre jutottak a világgal kapcsolatban, itt az ideje, hogy a játékosok elkészítsék a Játékos Karaktereket (JK). Mindegyik játékos egy hőst irányít a történetben, meghatározva minden tettüket. Neked kell kidolgoznod a karaktert olyanra, amilyenre szeretnéd. Vedd figyelembe, hogy a Fate karakterek kompetensek, drámaiak, és nem haboznak a kalandokba vetni magukat.

A te JK‑d az alábbi dolgokból épül fel:

\begin{itemize}
    \item \textbf{Jellemzők:} kifejezések, amik leírják, hogy ki is a hősöd
    \item \textbf{Képességek:} megadják, hogy a hősöd miben jobb másoknál
    \item \textbf{Fortélyok:} figyelemre méltó dolgok, amit a hősöd meg tud tenni
    \item \textbf{Stressz:} a hős azon adottsága, hogy nyugodt maradjon, és folytassa küldetését
    \item \textbf{Következmények:} fizikai és szellemi sérülések, amit a hős még elvisel
    \item \textbf{Újratöltés:} ez mutatja a hős narratív hatalmát
    \item \textbf{Végső simítások:} a hős leírásának részletei
\end{itemize}
