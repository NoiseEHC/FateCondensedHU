\section{Nehézség és ellenállás meghatározása}

Néha a JK cselekvése -- egy védekezés dobásban megnyilvánuló -- \definition{ellenállással} találja szembe magát. Ilyenkor az ellenséges karakter kockákkal dob, amihez hozzáadja a megfelelő képesség szintjét, pont úgy, ahogy a JK"~k teszik. Ha az ellenséges karakternek van idevágó jellemzője, akkor azt kihasználhatja; a KM az NJK"~k jellemzőit az NJK"~k közös sors pont tartalékából tudja kihasználni (\page{44}).

Ha viszont nincs dinamikus ellenállás, akkor meg kell határoznod a statikus \definition{nehézséget}:

\begin{itemize}
    \item \textbf{Az alacsony nehézségek}, amik a JK idevágó képesség szintje alatt vannak, jól jönnek, ha szeretnél esélyt adni a képesség megvillantására.
    \item \textbf{A közepes nehézségek}, amik a JK idevágó képesség szintje körüliek, akkor hasznosak, ha szeretnél feszültséget teremteni, de nem akarod elnyomni a karaktereket.
    \item \textbf{A magas nehézségek} -- amelyek sokkal magasabbak a JK idevágó képesség szintjénél -- célja kiemelni, hogy mennyire szörnyűek vagy szokatlanok a körülmények, aminek legyőzéséhez mindent be kell vetniük, netán -- ha vesztes a pozíció -- hogy elszenvedjék a kudarcuk következményeit.
\end{itemize}

Ugyancsak használhatod a mellékneveket a képességek szintjeinek skálájából (\page{6}), hogy kiválaszd a megfelelő nehézséget. Ez emberfeletti erőfeszítést igényel? Akkor legyen Emberfeletti~(+5)! Néhány ökölszabály következik segítségképpen.
Ha a feladat egyáltalán nem nehéz, akkor legyen Középszerű~(+0) -- vagy csak mondd azt, hogy a játékos bármiféle dobás nélkül is sikert ért el, feltéve, hogy nincs súlyos időzavarban, netán ha a karakter valamelyik jellemzője szerint nagyon jó az adott dologban.

Ha tudsz legalább egy dologról, ami nehézzé tenné a feladatot, akkor legyen Jó~(+2); minden egyes további ellenük dolgozó tényező +2 nehézséget jelent.

A tényezők számbavételéhez használd a játékban lévő jellemzők listáját. Ha valami elég fontos volt, hogy jellemzőt csináljanak belőle, akkor az megér egy kis figyelmet. Mivel a jellemzők mindig igazak (\page{22}), befolyásolhatják, hogy mennyire könnyű vagy nehéz valamit megtenni. Természetesen ez nem jelenti, hogy kizárólag csak a jellemzőket kell figyelembe venned! Például a sötétség az sötét, függetlenül attól, hogy felvetted"~e helyzet jellemzőnek vagy sem.

Ha a feladat a lehetetlennel határos, olyan magasra veheted a nehézséget, amennyire csak akarod. A JK"~nak el kell használnia pár sors pontot, és jó sok segítséget is igénybe kell vennie a sikerhez, de ez így van jól.

További ötletekhez, hogy miként kreálhatsz a játékosaid számára változatos és érdekes ellenségeket, olvasd el a \fate{Fate Adversary Toolkit} szabálykönyvet. Ez megvásárolható PDF formátumban, vagy a lényege elérhető ingyen is a \url{https://fate-srd.com/} címen.
