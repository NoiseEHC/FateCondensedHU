\subsection{Cselekvési sorrend}

Gyakran nem érdekes, hogy ki pontosan mikor cselekszik, de versengésekben és konfliktusokban nagyon is számít. Ezek a jelenetek \definition{fordulókra} vannak bontva. Minden forduló során, minden résztvevő karakter tehet egy"~egy megoldás, helyzetbehozás vagy megtámadás cselekvést, és ezen felül mozoghat is egyszer (a kihívások kicsit másként működnek, lásd a \onpage{33}). Mivel a védekezés egy reakció valaki más cselekvésére, a karakterek akárhányszor képesek rá, míg más karakterek jönnek, feltéve, hogy meg tudják indokolni a történet alapján, hogyan tudnak közbeavatkozni.

A jelenet elején, a KM és a játékosok közösen eldöntik a szituáció alapján, hogy ki jön először, és azután a cselekvő fél eldönti, hogy ki jön őutána. A KM karakterei is a cselekvési sorrendben kerülnek sorra, ugyanúgy, mint a játékosokéi, és a KM dönti el, hogy ki következik az NJK"~k után. Ha már mindenki sorra került, akkor az utolsó cselekvő dönti el, hogy ki kezdi a következő fordulót.

\example{%
Cassandra és Ruth belebotlottak egy arany maszkos főpap vezette, kisebb csapat szektatagba, akik éppen valamilyen misztikus szertartást folytattak. Mivel a szektatagok nagyon el vannak foglalva a szertartással, a KM kijelenti, hogy a JK"~k jönnek először a konfliktusban. A játékosok eldöntik, hogy Cassandra kezd: a maszkos szektatag felé fut visítva, hogy helyzetbe hozza magukat a \aspect{Zavarodott} jellemzővel. Nem túl kifinomult, de hatásos. Hogy a legtöbbet kihozhassa a helyzet jellemzőből, Cassandra játékosa úgy dönt, hogy következőnek Ruth fog cselekedni. Ruth egy tőrt dob a főpap felé, azon nyomban kihasználva a \aspect{Zavarodott} jellemzőt, hogy javítsa a támadását. Bár ez nem elég, hogy egy ütésből kifektesse a főpapot, de elég jó kombináció, hogy megingassa.
\newline
Sajnos, mivel már mindegyik JK sorra került, Ruthnak nincs más választása, mint valamelyik szektatagot választani következőnek. A maszkosat választja. Ez tetszik a KM"~nek, hiszen nemcsak, hogy már mindig csak szektatagok fognak cselekedni, de az utolsó majd a maszkos főpapot fogja választani, hogy az kezdje a következő fordulót. Lehet, hogy a JK"~knak jó volt a belépője, de most a szektatagokon a sor, hogy visszavágjanak.
}

Ez a módszer a cselekvési sorrend megállapítására, számtalan néven fut online fórumokon: szabadon választható cselekvési sorrend, valamint „popcorn”, „átadásos” vagy „Balsera"~féle” kezdeményezés. Ez utóbbi Leonard Balsera nevét hordozza, aki az egyik szerzője a \fate{Fate Core} rendszernek, és aki elvetette az egész ötlet magjait. Többet is megtudhatsz erről a módszerről, és a kapcsolódó stratégiákról a \url{https://www.deadlyfredly.com/2012/02/marvel/} oldalon.
