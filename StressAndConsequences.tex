\subsection{Stressz dobozok és következmény rubrikák}

A \definition{stressz dobozok} és a \definition{következmény rubrikák} adják meg, hogy a karakter mennyire tudja tolerálni a fizikai és szellemi igénybevételt a kalandok során. Minden karakternek legalább 3, egy‑egy pontot érő fizikai stressz doboza, és legalább 3, egy‑egy pontot érő szellemi stressz doboza van. Szintén van egy‑egy enyhe, mérsékelt és súlyos következmény rubrikája.

A Fizikum képesség szint módosítja, hogy a karakternek mennyi fizikai következmény rubrikája van. Az Akaraterő ugyanígy hat a szellemi következmény rubrikák számára.

\fatetable{l l}{
\textcolor{white}{Fizikum/Akaraterő} & \textcolor{white}{Fizikai/Szellemi stressz} \\
Középszerű~(+0) & \dice{3} \\
Átlagos~(+1) vagy Jó~(+2) & \dice{31} \\
Remek~(+3) vagy Kimagasló~(+4) & \dice{33} \\
Emberfeletti~(+5) vagy magasabb & \begin{tabular}[t]{@{}l@{}}\dice{33}\\és egy extra enyhe következmény rubrika,\\specifikusan csak a fizikai\\vagy csak a szellemi sérülésekre\end{tabular} \\
}

A \emph{„Sérülések elszenvedése”} fejezetben (\page{34}) többet is tanulhatsz a stresszről és következményekről.

\subsubsection{Várjunk csak, nem erre emlékszem!}

A \fate{Fate Condensed} szabályokban minden stressz doboz csak 1 pontot ér. Ellenben a \fate{Fate Core} és a \fate{Fate Accelerated} rendszerekben a stressz dobozok értéke növekvő (1 darab 1 pontos, 1 darab 2 pontos, és így tovább). Használhatod azt a rendszert is, ha úgy tetszik; mi azért váltottunk az 1 pontos dobozokra, mert egyszerűbb – a másik módszer egy kicsit könnyebben összezavarja az embereket.

Ennek a stílusnak viszont van pár folyománya, amit jó lesz észben tartani.

\begin{itemize}
    \item Ahogy azt majd később láthatod, az 1 pontos dobozokból annyit ikszelsz be, amennyit csak akarsz, ha találatot kapsz (míg a növekvő értékű dobozokból mindig csak egyet lehetett).
    \item Ehhez a stílushoz különálló fizikai és szellemi stresszmérők szükségesek, ellentétben a \fate{Fate Accelerated} rendszer közös stresszmérőjével. Ha a közös stresszmérő jobban tetszik, akkor adj hozzá három dobozt kiegyenlítésként, és használd a magasabb szintűt a Fizikum és Akaraterő közül, hogy megnöveld a számukat.
    \item Három pontnyi stressz semlegesítés nem túl sok! Ha a karakterek egy kissé törékenynek tűnnének játék közben, nyugodtan adjál hozzá egy‑két dobozt az alapértelmezetthez mennyiséghez. Ez az egész csak annyiról szól, hogy milyen gyorsan váltunk következményekre. (A régebbi rendszerben egy \boxed{1}~\boxed{2} sorozat 2"~3, egy \boxed{1}~\boxed{2}~\boxed{3} sorozat 3"~6, míg egy \boxed{1}~\boxed{2}~\boxed{3}~\boxed{4} sorozat 4"~10 stresszt semlegesített.)
\end{itemize}

\subsection{Végső simítások}

Nevezd el a karaktert, írd le, hogy hogyan néz ki, és beszéld meg a múltját a többi játékossal. Ha még nem írtad le a kapcsolat jellemzőt, akkor itt az ideje megtenni.
