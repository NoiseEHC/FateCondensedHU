\label{Cselekedni}
\chapter{Cselekedni és kockákkal dobni}

A \fate{Fate Condensed} játékban te irányítod a karaktered cselekedeteit, miközben hozzáteszel a közösen mesélt történethez. Általában a KM meséli, hogy a világban mi történik, és hogy a Nem Játékos Karakterek (\definition{NJK}"~k) mit csinálnak, míg a játékosok a saját karaktereik cselekedeteit mondják el.

Bármit szeretnél csinálni, kövesd az \definition{előbb az elképzelés} szabályt: először mondd ki, hogy a karaktered mire készül, és csak \emph{utána} gondolkodj azon, hogy a rendszerben ezt hogyan lehet modellezni. A karaktered jellemzői megmutatják, hogy mivel próbálkozhat egyáltalán, és segítenek megállapítani az eredmény értelmezésének kereteit. A legtöbb ember meg sem próbálkozhat megműteni a kibelezett társát, de ha van egy jellemződ, ami orvosi háttérről árulkodik, akkor te viszont nekiláthatsz. Enélkül a jellemző nélkül a legtöbb, hogy nyerhetsz pár percet a végső szavakhoz. Ha kétségesnek érzed, kérdezd meg a KM"~et és az asztaltársaságot.

Hogyan állapítod meg, hogy eredményes voltál"~e? Gyakran a karaktered egyszerűen csak megteszi, amit akartál, mert a cselekedet nem túl nehéz, és senki sem próbált meggátolni benne. De problémás és kiszámíthatatlan szituációkban muszáj elővenni a kockákat, hogy eldöntsék, mi történik.

Ha egy karakter cselekedne, az alábbiakat kell figyelembe venni:

\begin{itemize}
    \item Mi gátolja abban, hogy egyszerűen csak megtegye?
    \item Mi mehet félre?
    \item Miért érdekes, ha valami félresikerül?
\end{itemize}

Ha senkinek sincs megfelelő válasza az összes kérdésre, akkor a cselekedet egyszerűen megtörténik. Elvezetni a repülőtérre tényleg nem kíván kockadobálást. Viszont az autópályán versenyezni a ránk várakozó repülőig, miközben másik világról származó, kibernetikával turbózott vadállatok üldöznek, tökéletes alkalom a dobásra.

Ha cselekedni kívánsz, kövesd az alábbi lépéseket:

\begin{enumerate}
    \item Előbb az elképzelés: írd le, mit akarsz megtenni, és utána válassz képességet és cselekvést.
    \item Dobj négy kockával.
    \item Add össze a szimbólumokat a kockákon: a \dice{+} értéke +1, a \dice{-} értéke -1, míg a \dice{0} értéke 0. Ez -4~és~+4 közötti eredményt fog adni.
    \item Add a kockák eredményét a képesség szintjéhez.
    \item Módosíthatod a dobást jellemzők kihasználásával (\page{Jellemzők kihasználása} és \page{Kihasználás}) és fortélyok használatával (\page{Fortélyok használata}).
    \item Állapítsd meg a végső eredményt, aminek a neve \definition{erőfeszítés}.
\end{enumerate}
