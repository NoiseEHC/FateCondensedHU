\subsubsection{Képesség lista}

A következő képességek leírása alább található.

\begin{multicols}{5}
\setlength{\parindent}{0em}
\textbf{%
Akaraterő \\
Atléta \\
Befolyásolás \\
Célzás \\
Empátia \\
Ezermester \\
Észlelés \\
Fizikum \\
Kapcsolatok \\
Kényszerítés \\
Közelharc \\
Lopózás \\
Megtévesztés \\
Misztikum \\
Nyomozás \\
Tolvajlás \\
Tudomány \\
Vagyon \\
Vezetés
}
\end{multicols}

\textbf{Akaraterő (Will):} Lelkierő, adottság a kísértések leküzdésére, valamint a traumák átvészelésére. Az Akaraterő fortélyai lehetőséget adnak a szellemi következmények figyelmen kívül hagyására, a lelki gyötrődéssel vagy misztikus erőkkel szembeni kitartásra, valamint nem engedni a Kényszerítéssel próbálkozó ellenfeleknek. Ezen felül, a magas Akaraterő képesség szint megnöveli a szellemi stressz dobozok vagy a lehetséges következmény rubrikák számát (\page{12}).

\textbf{Atléta (Athletics):} A test teljesítőképességének mértéke. Az Atléta fortélyai a mozgásra -- futás, ugrás és akrobatika -- valamint a támadások elleni kitérésekre fókuszálnak.

\textbf{Befolyásolás (Rapport):} Másokkal kontaktust kiépíteni és együttműködni. Míg a Kényszerítés manipuláció, addig a Befolyásolás őszinteség, bizalom és jóakarat. A Befolyásolás fortélyai lehetővé teszik tömegek irányítását, emberi viszonyok javítását vagy emberi viszonyok kiépítését.

\textbf{Célzás (Shoot):} Mindenféle távolsági fegyver ide tartozik, legyen szó lőfegyverekről, dobótőrről vagy íjakról. A Célzás fortélyai lehetővé teszik a testrészre támadást, gyors fegyverrántást, vagy, hogy mindig legyen kéznél egy pisztoly.

\textbf{Empátia (Empathy):} Ez az, amivel felmérhetjük valaki hangulatát és szándékait. Az Empátia fortélyai közé tartozik nagyobb tömeg megítélése, hazugságok felismerése, vagy segíteni másoknak felépülni szellemi következményekből.

\textbf{Ezermester (Crafts):} Gépezetek építése vagy éppen elpusztítása, szerkentyűk összeeszkábálása, valamint MacGyver"~féle találékonyság megvillantása. Az Ezermester fortélyai segítségével a megfelelő kis bigyó pont a karakternél lesz, esetleg bónuszt kaphat építésre és rombolásra, vagy indoklást adhatnak a játékos kezébe, hogy bizonyos szituációkban miért lehet Ezermestert használni Tolvajlás vagy Tudomány helyett.

\textbf{Észlelés (Notice):} Lehetőséget ad, hogy pillanatnyi benyomásokra figyeljünk fel, még az előtt kiszúrjuk a történéseket, hogy bajba kerülnénk, és általánosságban figyelmesek legyünk. Ellentétes a Nyomozással, ami lassú, szándékos megfigyelés. Az Észlelés fortélyai élesítik az érzékeidet, javítják a reakcióidőd, vagy nehezebbé teszik, hogy becserkésszenek.

\textbf{Fizikum (Physique):} Nyers erő és kitartás. A Fizikum fortélyai lehetővé teszik gigászi erő kifejtését, mások uralását birkózás közben, netán fizikai következmények lerázását. Ezen felül a magas Fizikum képesség szint megnöveli a fizikai stressz dobozok vagy a lehetséges következmény rubrikák számát (\page{12}).

\newpage

\textbf{Kapcsolatok (Contacts):} A megfelelő emberek ismerete, és kapcsolati háló, ami a segítségedre lehet. A Kapcsolatok fortélyai mindenre kész szövetségeseket biztosítanak, valamint besúgó hálózatot, bármerre is vezessen utad a világban.

\textbf{Kényszerítés (Provoke):} Ezzel lehet más embereket rávenni, hogy azt csinálják, amit akarsz. Ez durva és manipulatív, nem egy pozitív interakció. A Kényszerítés fortélyai lehetőséget adnak másokat vakmerő cselekedetekre sarkallni, agressziót magadra vonni vagy elijeszteni az ellenfelet (persze csak, ha tudnak félelmet érezni).

\textbf{Közelharc (Fight):} A Közelharc képesség neve mindent elárul, használható akár fegyverrel, akár puszta  kézzel történik. A Közelharc fortélyai az egyedi fegyverekről és a speciális technikákról szólnak.

\textbf{Lopózás (Stealth):} Láthatatlanul és néma csendben maradni, valamint elszökni a kíváncsi szemek elől, ha rejtőzni kell. A Lopózás fortélyai lehetővé teszik, hogy akkor is el tudj rejtőzni, ha éppen figyelnek, beleolvadj a tömegbe, vagy az árnyékokban osonj észrevétlenül.

\textbf{Megtévesztés (Deceit):} Ez a meggyőzően, hidegvérrel előadott hazugságok és átverések képessége. A Megtévesztés fortélyai bizonyos típusú hazugságokat hihetőbbé tehetnek, netán segíthetnek hamis személyiségek felvételében.

\textbf{Misztikum (Lore):} Speciális, titkos tudás, ami kívül esik a Tudomány hatáskörén, mint például a természetfölötti erők és lények. Ez bizony az őrült történetek tárháza. A Misztikum fortélyai gyakran a titkos tudás gyakorlati alkalmazásáról szólnak, amilyen például a varázslás. Némely világ törölheti ezt a képességet, lecserélheti másikra, vagy összevonhatja a Tudománnyal (Tudás néven).

\textbf{Nyomozás (Investigate):} Szándékos, gondos tanulmányozás és a rejtélyek kibogozása. Ezt használd a nyomok értelmezéséhez vagy egy bűntett rekonstruáláshoz. A Nyomozás fortélyai segítenek briliáns következtetések kigondolásában, netán csak a nyomok értelmezését gyorsítják fel.

\textbf{Tolvajlás (Burglary):} Elméleti és gyakorlati tudás biztonsági rendszerek megkerülésére, zsebmetszésre és általánosságban a bűnözésre. A Tolvajlás fortélyai bónuszokat adnak a bűnözés minden fázisában a tervezéstől a kivitelezésen át a felszívódásig.

\textbf{Tudomány (Academics):} Evilági, mindennapi emberi tudás és tananyag, mint a történelem, fizika vagy orvoslás. A Tudomány fortélyai gyakran szűkebb tudományterületekre vagy gyógyításra fókuszálnak.

\textbf{Vagyon (Resources):} Anyagi javakhoz való hozzáférés, ami nem korlátozódik pénzre vagy birtoklásra.  Ez mutathatja, hogy kölcsön tudsz kérni egy baráttól, netán kipucolhatod egy szervezet fegyvertárát. A Vagyon fortélyaival bizonyos szituációkban használhatod a Vagyon képességet Befolyásolás vagy Kapcsolatok helyett, netán adhatnak néhány ingyen kihasználást, ha a legjobb minőséget veszed meg.

\textbf{Vezetés (Drive):} A járművek feletti uralom megtartása a legkeményebb szituációkban is, eszement manőverek végrehajtása, vagy egyszerűen csak a legtöbbet kihozni a járművedből. A Vezetés fortélyai lehetnek egyedi manőverek, egy speciális jármű birtoklása, vagy indoklást adhatnak a játékos kezébe, hogy bizonyos szituációkban miért lehet Vezetést használni Tolvajlás vagy Tudomány helyett.
