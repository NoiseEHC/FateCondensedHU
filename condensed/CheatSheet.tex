{
\begin{wraptable}{r}{0cm}
\fatetable{r l}{
\multicolumn{2}{|c|}{\textcolor{white}{Skála}} \\
+8 & Legendás \\
+7 & Eposzi \\
+6 & Fantasztikus \\
+5 & Emberfeletti \\
+4 & Kimagasló \\
+3 & Remek \\
+2 & Jó \\
+1 & Átlagos \\
+0 & Középszerű \\
-1 & Gyenge \\
-2 & Borzalmas \\
-3 & Katasztrofális \\
-4 & Horrorisztikus \\
}
\end{wraptable}

\cheatsheetsection{Előbb az elképzelés}{Cselekedni}

Először írd le, hogy mit szeretnél csinálni, és csak \textbf{utána} válassz képességet és cselekvést.

\cheatsheetsection{Kockával dobni}{Cselekedni}

Erőfeszítés = [4dF dobás \pg{Cselekedni}] + [képesség \pg{Képességek}] + [jellemzők \pg{Jellemzők kihasználása}] + [fortélyok \pg{Fortélyok használata}]

\cheatsheetsection{Kimenetelek}{Kimenetelek}

Sikeresség = [Az erőfeszítésed] -- [passzív nehézség vagy aktív ellenállás]

\begin{itemize}
    \failureitem \textbf{Kudarc:} Az erőfeszítésed kisebb, mint a passzív nehézség vagy aktív ellenállás.
    \tieitem \textbf{Döntetlen:} Az erőfeszítésed egyenlő a céllal.
    \successitem \textbf{Siker:} Az erőfeszítésed egy vagy kettő sikerességgel nagyobb, mint a cél.
    \successwithstyleitem \textbf{Átütő siker:} Az erőfeszítésed három vagy több sikerességgel nagyobb, mint a cél.
\end{itemize}
}

\vspace{5pt}

\begin{multicols*}{2}

\cheatsheetsection{Cselekvések}{Cselekvések}

\textbf{\dice{A}Megtámadás \pg{Megtámadás}:} Támadsz, hogy kárt okozz az ellenfélben.

\begin{itemize}
    \failureitem \textbf{Kudarc:} Nem találod el.
    \tieitem \textbf{Döntetlen:} Előnyt kapsz \pg{Előny jellemzők}.
    \successitem \textbf{Siker:} Sikeresség mértékű sérülést okozol.
    \successwithstyleitem \textbf{Átütő siker:} Mint a siker, de csökkentheted a sérülést egy előnyért cserébe.
\end{itemize}

\textbf{\dice{D}Védekezés \pg{Védekezés}:} Kivédesz egy megtámadást, vagy meggátolod az ellenfél cselekvését.

\begin{itemize}
    \failureitem \textbf{Kudarc:} Az ellenfél sikeres. Ha megtámadás volt, sérülést szenvedsz el, amit semlegesítened kell.
    \tieitem \textbf{Döntetlen:} A meggátolt cselekvés döntetlen kimenetele mérvadó.
    \successitem \textbf{Siker:} Meggátolod az ellenfelet a cselekvésében.
    \successwithstyleitem \textbf{Átütő siker:} Mint a siker, de kapsz egy előnyt is.
\end{itemize}

\textbf{\dice{O}Megoldás \pg{Megoldás}:} Akadályok leküzdése.

\begin{itemize}
    \failureitem \textbf{Kudarc:} Kudarc vagy siker komoly áron \pg{Siker komoly áron}.
    \tieitem \textbf{Döntetlen:} Siker kisebb áron \pg{Siker kisebb áron}; kudarc, de kapsz egy előnyt; esetleg részleges siker.
    \successitem \textbf{Siker:} Eléred a célod.
    \successwithstyleitem \textbf{Átütő siker:} Eléred a célod, és kapsz egy előnyt is.
\end{itemize}

\textbf{\dice{C}Helyzetbehozás \pg{Helyzetbehozás}:} Jellemzők felhasználása.

Ha \textbf{új helyzet jellemzőt} hozol létre: 

\begin{itemize}
    \failureitem \textbf{Kudarc:} Vagy nem hozod létre, vagy létrehozod, de az ellenfél kapja az ingyen kihasználást (siker valamilyen áron).
    \tieitem \textbf{Döntetlen:} Nem hozod létre, de kapsz egy előnyt helyette \pg{Előny jellemzők}.
    \successitem \textbf{Siker:} Létrehozod egy ingyen kihasználással.
    \successwithstyleitem \textbf{Átütő siker:} Létrehozod két ingyen kihasználással.
\end{itemize}

Ha \textbf{meglévő -- ismert vagy ismeretlen -- jellemző} a cél:

\begin{itemize}
    \failureitem \textbf{Kudarc:} Az ellenfél kap egy ingyen kihasználást. Ha ismeretlen volt, az ingyen kihasználás helyett továbbra is titokban tarthatja.
    \tieitem \textbf{Döntetlen:} Egy ingyen kihasználást kapsz, ha a jellemző ismert. Egy előnyt kapsz, ha a jellemző ismeretlen.
    \successitem \textbf{Siker:} Egy ingyen kihasználást kapsz a jellemzőhöz.
    \successwithstyleitem \textbf{Átütő siker:} Két ingyen kihasználást kapsz a jellemzőhöz.
\end{itemize}

\cheatsheetsection{Stressz és következmények}{Stressz}

Ha eltalálnak, semlegesítened kell a sikerességeket, különben kiejtődsz.

\textbf{Stressz:} Beikszelheted a még üres stressz dobozaid, mindegyikkel egy"~egy sikerességet semlegesítve.

\textbf{Következmények:} Ezek a jellemzőrubrikák egy ingyen kihasználást adnak a támadónak. Az alábbi számú sikerességet semlegesítik: Enyhe = 2, Mérsékelt = 4, Súlyos = 6.

\textbf{Kiejtés:} Ha nem tudod az összes sikerességet semlegesíteni, akkor kiejtődsz; az ellenfeled mondja meg, hogy mi történik veled, ami miatt kikerülsz a jelenetből \pg{Kiejtés}.

\textbf{Megadás:} Még a dobás előtt megadhatod magad, amiért sors pontot kapsz, és megmondhatod, hogyan kerülsz ki a jelenetből \pg{Megadás}.

\textbf{Felépülés:} A stresszmérő nullázódik a jelenet végén. A következményből való felépülés időtartama a súlyosságától függ \pg{Sérülésekből felépülni}.

\cheatsheetsection{Jellemzők}{Jellemzők és sors pontok}

\textbf{A jellemző mindig igaz \pg{A jellemző mindig igaz}.} A jellemzők biztosíthatnak vagy tilthatnak lehetőségeket, hogy mi eshet meg a történetben.

\textbf{Kihasználhatsz \pg{Kihasználás}} egy jellemzőt +2 bónuszért a dobásodra, egy újradobásra vagy arra, hogy 2"~vel növeld az ellenfél nehézségét. A kihasználás egy sors pontba, vagy egy ingyen kihasználásba kerül \pg{Helyzetbehozás}.

\textbf{Késztethet \pg{Késztetés}} egy jellemző, hogy komplikációkat okozzon a karakternek. A játékos vagy elfogad ezért egy sors pontot, vagy visszautasíthatja a késztetést egy sors pont elköltésével.

\end{multicols*}

\newpage

\begin{multicols*}{2}

\cheatsheetsection{Cselekvési sorrend}{Cselekvési sorrend}

Kezdetben a KM és a játékosok eldöntik, hogy ki kezd. A cselekedete után a cselekvő eldönti, hogy ki a következő. A KM karakterei is ugyanígy kerülnek sorra. Miután mindenki sorra került, az utolsó cselekvő eldönti, hogy ki kezdi a következő kört.

\cheatsheetsection{Csapatmunka}{Csapatmunka}

\textbf{Képességek egyesítése:} A legnagyobb képességgel rendelkező karakter dob. Minden résztvevő, akinek legalább Átlagos (+1) ugyanaz a képessége, +1 bónuszt adhat a dobáshoz a cselekvése elhasználásával. A bónusz legfeljebb a legmagasabb képesség szintje lehet. A résztvevőknek ugyanazokat az árakat és következményeket kell elszenvedniük, mint a cselekvő karakter.

\textbf{Ha te jössz:} A helyzetbehozással megszerzett ingyen kihasználásokat átadhatod egy szövetségesednek, aki felhasználhatja, amikor sorra kerül.

\textbf{Ha nem te jössz:} Kihasználhatsz egy jellemzőt, hogy egy szövetségesed dobásához bónuszt adjál.

\cheatsheetsection{Komoly és kisebb ár}{Siker komoly áron}

\textbf{Komoly ár \pg{Siker komoly áron}:} A szituáció sokkal rosszabbá vagy komplikáltabbá válik. Ez lehet új problémák vagy új ellenfelek megjelenése, a játékosok fenyegetően közeledő határidő elé állítása, egy enyhe vagy mérsékelt következmény elszenvedése, az ellenfélnek előnyös helyzet egy"~két ingyen kihasználással, netán valami más.

\textbf{Kisebb ár \pg{Siker kisebb áron}:} Egy történetbéli csavar vagy komplikáció, ami nem akadály önmagában, pár kockányi stressz vagy egy előny \pg{Előny jellemzők} az ellenfélnek.

\cheatsheetsection{Felépülés}{Sérülésekből felépülni}

\textbf{Dobj a felépülés elkezdéséhez:} Tudomány a fizikai, Empátia a szellemi következményekhez. A nehézség Jó (+2) az enyhe, Kimagasló (+4) a mérsékelt és Fantasztikus (+6) a súlyos esetben. A nehézség kettővel magasabb, ha saját magadat kezeled. Siker esetén úgy fogalmazd át a jellemzőt, hogy mutassa a gyógyulást.

\textbf{Kezelés után:} Az enyhe jellemzőnek egy teljes jelenet kell a gyógyuláshoz, a mérsékeltnek egy teljes játékülés. A súlyos csak egy áttörés elérésekor szűnik meg \pg{Fejlődés}.

\cheatsheetsection{Nehézségek meghatározása}{Nehézség és ellenállás meghatározása}

Alacsony = JK megfelelő képessége alatt; Közepes = JK képessége körül; Magas = JK képessége felett.

Középszerű (+0), ha nem kemény probléma (vagy ne is dobj), +2 ha kemény probléma, +2 minden egyes nehezítő körülmény miatt. A jellemzők ezekhez támpontot adnak.

A képesség szintek skálája jó kiindulópont lehet.

\columnbreak

\cheatsheetsection{Kihívás}{Kihívás}

A KM kiválasztja a kihívás leküzdéséhez szükséges képességeket. A számuk nagyjából a játékosok számával kell, hogy egyezzen. Minden játékos választ egy feladatot, és dob a megoldására. A KM a sikerek és kudarcok alapján meghatározza a végkimenetelt.

\cheatsheetsection{Versengés}{Versengés}

A versengés fordulók sorozata. Minden fél tesz egy"~egy megold cselekvést a célja elérésére. Minden oldalról csak egy"~egy karakter dob.

Minden résztvevő megpróbálkozhat helyzetbehozással is a dobás vagy a képességek egyesítése \pg{Csapatmunka} mellett. Ha a helyzetbehozás kudarc, akkor az az oldal vagy feladja a dobását arra a fordulóra, vagy dob, de az ellenfél kap egy ingyen kihasználást.

Ha egy fenyegetés is jelen van, a kudarc mértéke stresszt okoz. Döntetlen esetén váratlan fordulat történik, amit a KM ír le.

Minden forduló végén a legmagasabb erőfeszítést elérő oldal szerez egy diadalt; az átütő siker két diadalt ér. Amelyik oldal előbb eléri a három diadalt (vagy amennyit a KM meghatároz) nyer.

\cheatsheetsection{Konfliktus}{Konfliktus}

Használj konfliktust, ha a JK"~k harcolhatnak vagy kényszeríthetnek, és minden fél bántalmazhatja a másikat.

A konfliktus fordulók sorozata. Minden karakter a cselekvési sorrendben kerül sorra \pg{Cselekvési sorrend}, és miután megmondják, mit szeretnének tenni, a megfelelő képességgel az adott cselekvésre dobnak. Aki ellenáll, dobhat védekezést, ha ennek van értelme.

Ha az egyik oldalról mindenki vagy megadta magát \pg{Megadás} vagy kiejtődött \pg{Kiejtés}, a konfliktus véget ér. Aki megadást választott, az begyűjtheti a sors pontokat ezért, és a KM ilyenkor fizeti ki az ellenséges kihasználásokat is \pg{Ellenséges kihasználás}.

\cheatsheetsection{Jellemző típusok}{Milyen jellemző típusok léteznek?}

\textbf{Karakter \pg{Milyen jellemző típusok léteznek?}:} Jellemző a karakterlapon.

\textbf{Helyzet \pg{Milyen jellemző típusok léteznek?}:} Jelenetbeli jellemző. Addig érvényes, ameddig a jellemző által reprezentált körülmény.

\textbf{Következmény \pg{Milyen jellemző típusok léteznek?}:} Karakter jellemző, ami tartós sérülést reprezentál.

\textbf{Előny \pg{Előny jellemzők}:} Átmeneti, néha el sem nevezett jellemző. Egy ingyen kihasználást biztosít, ami után megszűnik. Nem késztethet. Nem lehet sors pontért kihasználni.

\textbf{Szervezet, Kaland, Világ és Zóna \pg{Egyéb jellemző fajták}:} Helyzet jellemzők egy csoporton, történetszálon, a világon vagy a térkép egy részén.

\end{multicols*}
