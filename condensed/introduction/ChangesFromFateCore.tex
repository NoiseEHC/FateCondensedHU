\subsection{Veteránoknak: változások a Fate~Core"~hoz képest}

Egy rendszer 300 oldalnyi szövegének kevesebb, mint 60 oldalba tömörítése nyilvánvalóan változtatásokkal jár. Ráadásul, mivel a \fate{Fate Core} már nyolcéves, pár helyen rá is fért a frissítés. Az alábbiakat éreztük fontosnak kiemelni:
\begin{itemize}
    \item A stressz dobozok most már mindig csak egy pontot érnek, ezzel is egyszerűsítve a szabályokat (\page{Stressz dobozok és következmény rubrikák}).
    \item Ahelyett, hogy a képességek döntenék el a cselekvési sorrendet, a „Balsera"~féle” kezdeményezés (ismert még „Válaszható cselekvési sorrend” vagy „Popcorn kezdeményezés” néven is) az alapértelmezett (\page{Cselekvési sorrend}).
    \item A fejlődés egy kicsit másképp működik; nincsenek már döntő mérföldkövek, emiatt a jelentős mérföldkövek (mint áttörések) szabályai ki lettek bővítve ellensúlyozásként (\page{Fejlődés}).
    \item Az aktív ellenállás most már mindig védekezés cselekvésnek minősül (\page{Védekezés}). Ezt a változást átvezettük az összes többi szabályon, a legfontosabb a megold cselekvés eredménye döntetlen esetén (\page{Megoldás}).
    \item A helyzetbehozás világosabban van megfogalmazva, pontosabban leírja az ismeretlen jellemzők felfedezését (\page{Helyzetbehozás}).
    \item A védekező harc opcionális szabály lett, és már magába foglalja a kibővített védekezés cselekvés lehetőségeit (\page{Védekező harc}). Ezen kívül még további opcionális szabályok is belekerültek a könyvbe a \onpage{Opcionális szabályok} kezdődően.
\end{itemize}
