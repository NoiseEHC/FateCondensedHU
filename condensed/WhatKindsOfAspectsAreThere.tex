\section{Milyen jellemző típusok léteznek?}

Számtalan jellemző variáns létezik (lásd a \onpage{27}), de a nevüktől függetlenül nagyjából ugyanúgy működnek. A legnagyobb különbség, hogy mennyi ideig maradnak aktívak, mielőtt eltűnnének.

\subsection{Karakter jellemzők}

Ezek a jellemzők a karakterlapodon vannak, mint például a koncepció vagy árnyoldal. Ezek adják meg a személyiségjegyeket, fontos részleteket a múltadból, másokhoz való viszonyodat, fontos tárgyak vagy címek birtoklását, problémákat, amikkel foglalkoznod kell, vagy célokat, amik eléréséért teszel, valamint az elért hírneved és kötelezettségeidet. Ezek a jellemzők normálisan a mérföldkövek elérésekor változhatnak (\page{39}).

\textbf{Példák:} \aspect{A Túlélők Csapatának Vezetője}; \aspect{Figyelmes a Részletekre}; \aspect{Meg Kell Védenem az Öcsémet}

\subsection{Helyzet jellemzők}

Ezek a jellemzők leírják a környezetet vagy a világot, amiben a cselekmény zajlik. A helyzet jellemzők általában eltűnnek a jelenet végén, vagy ha valaki olyan cselekvést tesz, ami megváltoztatná vagy megszüntetné. Gyakorlatilag csak addig léteznek, amíg az általuk leírt körülmény fennáll.

\textbf{Példák:} \aspect{Lángol}; \aspect{Ragyogó Napsütés}; \aspect{Dühös Tömeg}; \aspect{Földre Döntött}; \aspect{Rendőrség Üldözi}

\subsection{Következmény jellemzők}

Ezek a jellemzők sérüléseket, vagy hosszan tartó megrázkódtatásokat reprezentálnak, amikre gyakran a megtámadásokból származó találatok semlegesítésével lehet szert tenni (\page{35}).

\textbf{Példák:} \aspect{Kificamodott Boka}; \aspect{Agyrázkódás}; \aspect{Bénító Önbizalomhiány}

\subsection{Előny jellemzők}

Az előny egy speciális jellemző, ami egy rendkívül átmeneti, vagy kis hatású körülményt reprezentál. Az előny jellemző nem késztet, és nem lehet sors pont elköltésével kihasználni sem. Egyetlen egyszer ingyen kihasználhatod, ami után megszűnik. A fel nem használt előny jellemző eltűnik, amint a reprezentált fölény már nem áll fenn, ami tarthat pár másodpercig, netán egyetlen cselekvés végéig. Soha nem tart tovább, mint a jelenet vége, és várhatsz az elnevezésével, amíg el nem használod. Ha te birtoklod az előny jellemzőt, azt átadhatod a szövetségesednek, ha ez megindokolható.

\textbf{Példák:} \aspect{Célpont Bemérve}; \aspect{Zavarodott}; \aspect{Egyensúlyvesztés}
