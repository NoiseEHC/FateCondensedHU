\chapter{Opcionális szabályok}

Ez itt jó néhány opcionális vagy alternatív szabály, amit bevezethetsz a játékodban.

\section[Állapotok]{Állapotok (Conditions)}

Az \definition{állapotok} a következményeket helyettesítik, és teljesen lecserélik a játékmechanikát. Az állapotoknak két célja van: egyrészt leveszi a terhet a KM és a játékosok válláról, így nem kell helyben kitalálniuk egy korrekten megfogalmazott jellemzőt minden elszenvedett következményhez, másrészt lehetővé teszik a játék jellegének meghatározását maradandó sérülések előre definiált listájával.

Az állapotok \fate{Fate Condensed} verziója kettébontja minden következmény jellemző rubrika szintjét két, fele akkora értékű beikszelhető dobozra.

\begin{center}
\begin{tabular}{ c l c l }
\boxed{1} & \aspect{Megkarcolt} (Enyhe) & \boxed{1} & \aspect{Rémült} (Enyhe) \\
\boxed{2} & \aspect{Lesérült} (Mérsékelt) & \boxed{2} & \aspect{Megrendült} (Mérsékelt) \\
\boxed{3} & \aspect{Megsebesült} (Súlyos) & \boxed{3} & \aspect{Demoralizált} (Súlyos) \\
\end{tabular}
\end{center}

Ezek fizikai és szellemi állapotoknak felelnek meg -- de ez nem jelenti, hogy fizikai találat esetén ne használhatnál fel egy szellemi állapotot és fordítva, ha van ennek értelme. Elvégre a támadások traumát okozhatnak!

Az állapotok ugyanúgy gyógyulnak, mint a következmények, a súlyosságuktól függően.

Ha egy magas képesség szint miatt egy extra enyhe következmény rubrikát kapnál, akkor vegyél fel helyette kettő extra beikszelhető dobozt a \aspect{Megkarcolt} vagy a \aspect{Rémült} állapothoz, a képességtől függően.

\subsection{Állapotok elszeparálása}

Ha szeretnéd, hogy a fizikai és szellemi állapotok teljesen különállóak legyenek, duplázd meg a dobozok számát mindkettőnél. Ennek is megvan a határa: ha bármelyik állapot szinten két doboz már be van ikszelve, akkor több dobozt már nem lehet beikszelni azon a szinten. Tehát, ha például a kettőből egy doboz beikszelt a \aspect{Megkarcolt} sorban, és a \aspect{Rémült} üres, akkor miután beikszeled vagy a második \aspect{Megkarcolt} dobozt, vagy az első \aspect{Rémült} dobozt, nem ikszelhetsz be többet azon a szinten.

Ha egy extra enyhe következmény rubrikát kapnál (magas Fizikum, Akaraterő vagy egy fortély miatt), akkor vegyél fel helyette kettő extra beikszelhető dobozt a Megkarcolt vagy a Rémült állapothoz, a képességtől függően. Minden ilyen extra doboz eggyel növeli a beikszelhetőség határát azon a szinten.

\subsection{Egyéb állapot szabály verziók}

Jó néhány megjelentetett Fate rendszerű játék használ valamilyen állapotrendszert a következmények helyett. Nyugodtan adaptáld az ő rendszerüket, ha az jobban tetszik, mint ez. Mindegyik ugyanazt éri el a játékban: hogy ne kelljen következmény jellemzőket kitalálni, és lehetővé teszik a játék jellegének meghatározását a maradandó sérülések korlátozásával.
