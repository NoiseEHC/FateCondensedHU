\label{Versengés}
\section[Versengés]{Versengés (Contest)}

A \definition{versengés} az, amikor két vagy több ellenérdekelt fél egymás ellen dolgozik, de nincs konfliktus. Ez persze nem jelenti, hogy egyik fél sem \emph{szeretné} bántani a másikat. Például versengés lehet, ha a csapat szeretne megszökni valamilyen veszedelem elől, mielőtt az teljesen ellehetetlenítené a győzelmet.

A versengés kezdetén mindenki kinyilvánítja, hogy mit szeretne tenni, mit szándékozik elérni. Ha több JK is jelen van, a céljaiktól függően lehetnek ugyanazon vagy ellentétes oldalon is -- például versenyfutás esetén mindenki külön félnek számít. \textbf{Versengés során a JK"~k vagy nem akarnak, vagy képtelenek kárt tenni az ellenfélben.} Tőlük független veszélyek (például kitörő vulkán, vagy egy feldühödött isten) viszont támadhatják az egyik vagy akár az összes felet. Ezek a veszélyek akár megjelenthetnek a versengésben részvevő félként is.

A versengés fordulók sorozata. Minden fordulóban minden fél egy"~egy megold cselekvéssel próbál közelebb kerülni a céljához. Minden oldalon csak egyetlen karakter tesz megold cselekvést, míg a többiek vagy csapatmunkával vagy helyzetek létrehozásával járulnak hozzá a sikerhez (aminek van rizikója, lásd alább). A megold cselekvések mehetnek passzív nehézség ellen -- ha a résztvevőknek valamilyen környezeti kihívást kell leküzdeniük -- vagy közvetlen vetélkedés esetén a másik fél erőfeszítése ellen.

Minden forduló végén hasonlítsd össze a felek erőfeszítés értékét. Amelyik fél a legnagyobb erőfeszítést tette, az szerez egy \definition{diadalt}. Ha átütő sikert ér el -- míg más fél nem -- akkor \textbf{két} diadalt arat. Az nyeri a versengést, aki elsőnek ér el három diadalt. (Bármikor használhatsz hosszabb versengéseket, ahol több diadalt kell elérni, de ötnél többet azért nem ajánlunk.)

Ha több fél egyszerre éri el a legmagasabb erőfeszítést, akkor senki sem szerez diadalt, és egy \definition{váratlan fordulat} történik. A KM behoz egy új helyzet jellemzőt, ami azt szimbolizálja, ahogy a jelenet, terep vagy szituáció megváltozott.

Olyan versengések során, ahol egy fenyegetés bántani próbálja bármelyik részvevőt, a résztvevő oldalának minden tagja sérülést szenved el, ha a versengés dobása alacsonyabb, mint a fenyegetés megtámadás dobása vagy passzív nehézsége. A sérülés mértéke a sikerességgel egyezik meg. Ugyanúgy, ahogy a konfliktusokban, ha a karakter nem tudja a sérülést semlegesíteni, akkor kiejtődik a küzdelemből.

\subsection{Helyzetbehozás versengés során}

A csapatod minden fordulóban próbálkozhat helyzetbehozással a megold cselekvés előtt. A célpont és mindenki más, aki közbeavatkozhat, ellenállhat a szokásos módon. Minden egyes résztvevő megpróbálkozhat a helyzetbehozással a megold cselekvésén vagy a csapatmunka bónuszán felül (\page{Csapatmunka}). Ha nem sikerül a helyzetbehozás, akkor két választásod van: a csapatod lemondhat a megold dobásról, vagy pedig „siker valamilyen áron” (megtartva a megold dobást vagy csapatmunka bónuszt), ahol az ár egy ingyen kihasználás az ellenfélnek. Ha a helyzetbehozás kimenetele legalább döntetlen, akkor jöhet a megold dobás vagy csapatmunka bónusz a szokásos módon.
