\subsection{Konfliktus befejezése}

A konfliktus véget ér, ha az egyik oldalon mindenki megadta magát, vagy kiejtődött. A konfliktus végén, minden játékos, aki megadta magát, megkapja a sors pontokat ezért (\page{Megadás}). A KM szintén ilyenkor osztja a konfliktus közbeni ellenséges kihasználásokért (\page{Ellenséges kihasználás}) járó sors pontokat.

\label{Sérülésekből felépülni}
\subsection[Sérülésekből felépülni]{Sérülésekből felépülni (Recovering)}

Minden jelenet végén, a karakterek lenullázzák a stresszmérőt. Viszont a következmény jellemzőknek több idő kell a megszűnésre.

A \definition{felépülés} kezdetéhez a karaktert kezelő másik karakternek egy sikeres megold cselekvést kell tennie a megfelelő képességgel. A fizikai sérülések általában orvosi tudást igényelnek a Tudomány képességgel, míg a szellemi sérülések gyógyítására az Empátia szolgál. A dobás nehézsége a következmény súlyossága: Jó~(+2) az enyhe következményekhez, Kimagasló~(+4) a mérsékelt következményekhez, és Fantasztikus~(+6) a súlyos következményekhez. Ezek a nehézségek kettővel magasabbak, ha a karakter magát kezeli (mert könnyebb ezt másnak megtennie).

Ha a dobás sikeres, akkor fogalmazd át a következmény jellemzőt, hogy mutassa, hogy a sérülés gyógyulóban van. A \aspect{Törött Kar} következményből például lehet \aspect{Begipszelt Kar}.

A siker csak az első lépcső -- a sérülések teljes gyógyulásához idő is szükséges.

\begin{itemize}
    \item Az \textbf{enyhe} következmények a kezelés után egy teljes jelenetet igényelnek.
    \item A \textbf{mérsékelt} következmények lassabban gyógyulnak, a kezelés után egy teljes játékülést igényelnek.
    \item A \textbf{súlyos} következmények pedig csak a kezelés utáni első áttörés (\page{Fejlődés}) alkalmával tűnnek el.
\end{itemize}
