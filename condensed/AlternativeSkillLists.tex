\subsubsection{Alternatív képesség Listák}

A fenti képesség lista csak egy példa a képességek felépítésére, és egyáltalán nem a „hivatalos” képesség lista, amit minden Fate játékban használni kéne. Az is teljesen elfogadható ha használod, de az is, ha figyelmen kívül hagyod. Bármelyik képesség különböző területeket fedhet le a világtól függően. A testre szabhatóság a lényeg; úgy állítsd össze a listát, hogy passzoljon az elképzelt történetekhez.

Ezt észben tartva, ha egy saját Fate játékon dolgozol, az első dolog, amit el kell döntened, hogy megtartsd"~e az alapértelmezett képesség listát, vagy sem. Általában ez elég jó, és csak pár képességet kell összevonni, megváltoztatni vagy felosztani. De akadnak olyan esetek is, amikor az alapértelmezett lista felbontása nem megfelelő. Ilyenkor az alábbi dolgokon célszerű elgondolkodni.

\begin{itemize}
    \item Az alapértelmezett képesség lista 19 elemből áll, és minden karakternek 10 képesség szintje magasabb, mint Középszerű~(+0). Ha megváltoztatod a képességek számát, célszerű a szinteket is máshogyan elosztani a karakterek között.
    \item Az alapértelmezett képesség lista arra válasz, hogy „mit tudsz megtenni?” -- de a listának nem muszáj ilyennek lennie. Lehet, hogy egy olyan lista jobb a játékodba, ami a „Miben hiszel?” kérdéskörrel foglalkozik, vagy a „Hogyan teszel dolgokat?” (mint a megközelítések a \fate{Fate Accelerated} játékban) kérdésre válaszol, esetleg feladatköri leírások egy csapat szélhámos vagy tolvaj esetében, és így tovább.
    \item A képességek szintjei úgy vannak kitalálva, hogy minden karakternek lehessen egyedi szakterülete. Ez az oka annak, hogy a kezdő karakterek „piramis” struktúrában kapják a képességeiket. Akárhogyan is módosítod a listát, ezt mindenképpen meg kell tartanod.
    \item A legjobb kezdő képesség szintjét Kimagasló~(+4) körül célszerű tartani. Ezt módosíthatod felfelé vagy lefelé is, de ilyenkor figyelned kell arra, hogy ez mennyiben módosítja a JK"~k által megdobandó nehézségeket, vagy az ellenfelek képesség szintjeit.
\end{itemize}

\example{%
Fred úgy dönt, hogy egy űropera Fate játékot szeretne csinálni, amiben a képesség lista rövidebb, és cselekedetek igéire fókuszál. Végül egy 9~elemű listában állapodik meg: Harcolni, Tudni, Mozogni, Észlelni, Vezetni, Lopózni, Gondolkodni és Akarni. Mivel jobban tetszik neki a „rombusz” struktúra, mint a piramis, így a kezdő képesség lista a következő: 1~darab Kimagasló~(+4), 2~darab Remek~(+3), 3~darab Jó~(+2), 2~darab Átlagos~(+1), és végül 1~darab Középszerű~(+0). A JK"~knak így elég sok képessége közös lesz, mivel a rombusz közepe elég széles, míg pár dolog az ő kizárólagos szakterületük marad, a rombusz hegyes „csúcsa” miatt.
}

Ha saját képesség listát szeretnél csinálni a játékodhoz, és szükséged van pár ötletre a fantáziád beindításához, lapozz a \page{46}ra.
