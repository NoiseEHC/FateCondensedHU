\section{Időtartamok}

Egy cselekvés időtartamának meghatározásakor pontosabb módszert is használhatsz, mint a siker, kudarc és a „bizonyos áron” opciók. Mennyivel tovább vagy gyorsabban? Az alábbi irányelvekkel a sikeresség döntheti el az időtartamot.

Először is döntsd el, hogy egyszerű siker esetén mennyi ideig tart az adott feladat. Használj körülbelüli mennyiségeket egy mértékegységgel együtt: „pár nap”, „fél perc”, „néhány hét” és így tovább. A körülbelüli mennyiségekbe a következők tartoznak: fél, nagyjából egy, pár, néhány az adott mértékegységből.

Ezután nézd meg, hogy a dobás mennyivel haladja meg, vagy múlja alul a célszámot. Minden sikeresség egy lépést jelent a mennyiségek listáján.

Például, ha a kezdő időtartam „pár óra”, akkor egy sikerességnyivel gyorsabb „nagyjából egy óra”, két sikeresség pedig már csak „fél óra”. A „fél” mennyiségnél gyorsabb a mértékegységet váltja kisebbre (órákból percek, és így tovább), míg a mennyiség „néhány” lesz, emiatt három sikerességgel gyorsabb „néhány perc lesz”.

A lassabb esetben az egész fordítva zajlik, egy sikerességgel lassabb „néhány óra”, kettő „fél nap”, míg három már „nagyjából egy nap”.
