\chapter{Bevezető}

Ez a \fate{Fate Condensed}, a \fate{Fate Core} rendszer egy verziója, amit olyan kevés oldalba sűrítettünk, amennyire csak tudtuk. Ez egy komplett szerepjáték rendszer; bár egyéb kiadványok kiegészíthetik, nincs más könyvre szükséged a játékhoz.

Ha már itt tartunk, vegyük sorra, hogy mire lesz viszont \emph{tényleg} szükséged!

\section{Mire van szükséged a játékhoz?}

A \fate{Fate Condensed} játékhoz szükséged lesz játékosokra kettő és hat fő között, akikből egy a Kalandmester (KM) szerepét fogja betölteni. Kell még pár kocka, néhány zseton, íróeszközök, papír és még valami a rövid jegyzetekhez (például öntapadó cetlik).

A \fate{Fate Condensed} rendszer \fate{Fate Dobókockákat}™ használ, amikor a karakterek cselekszenek. Ezek hatoldalú kockák, amelyeknek kettő \dice{0}, kettő \dice{+} és kettő \dice{-} oldaluk van. Összesen négy ilyen kocka már elégséges, de a legjobb, ha mindegyik játékos rendelkezik néggyel. Ezen kívül lehet használni standard hat oldalú dobókockát is (1"~2 = \dice{-}, 3"~4 = \dice{0}, 5"~6 = \dice{+}; netán vastagabb filccel át lehet alakítani így:~\vcenteredinclude{d6_to_fudge_v2.png}). Alternatíva még a \fate{Deck of Fate} kártyapakli is, ami kártyalapokat használ dobókockák helyett; ennek ellenére, az egyszerűség kedvéért, a továbbiakban konzekvensen a „dobás” szót használjuk.

\subsection{Veteránoknak: változások a Fate~Core"~hoz képest}

Egy rendszer 300 oldalnyi szövegének kevesebb, mint 60 oldalba tömörítése nyilvánvalóan változtatásokkal jár. Ráadásul, mivel a \fate{Fate Core} már nyolcéves, pár helyen rá is fért a frissítés. Az alábbiakat éreztük fontosnak kiemelni:
\begin{itemize}
    \item A stressz dobozok most már mindig csak egy pontot érnek, ezzel is egyszerűsítve a szabályokat (\page{12}).
    \item Ahelyett, hogy a képességek döntenék el a cselekvési sorrendet, a „Balsera"~féle” kezdeményezés (ismert még „Válaszható cselekvési sorrend” vagy „Popcorn kezdeményezés” néven is) az alapértelmezett (\page{31}).
    \item A fejlődés egy kicsit másképp működik; nincsenek már döntő mérföldkövek, emiatt a jelentős mérföldkövek (mint áttörések) szabályai ki lettek bővítve ellensúlyozásként (\page{39}).
    \item Az aktív ellenállás most már mindig védekezés cselekvésnek minősül (\page{21}). Ezt a változást átvezettük az összes többi szabályon, a legfontosabb a megold cselekvés eredménye döntetlen esetén (\page{18}).
    \item A helyzetbehozás világosabban van megfogalmazva, pontosabban leírja az ismeretlen jellemzők felfedezését (\page{19}).
    \item A védekező harc opcionális szabály lett, és már magába foglalja a kibővített védekezés cselekvés lehetőségeit (\page{48}). Ezen kívül még további opcionális szabályok is belekerültek a könyvbe a \onpage{45} kezdődően.
\end{itemize}
