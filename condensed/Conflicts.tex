\section[Konfliktus]{Konfliktus (Conflict)}

Ha a hősök tisztességes harcba bocsátkoznak -- legyen az a rendőrökkel, szektatagokkal vagy egy megnevezhetetlen szörnnyel -- és esélyük van győzni, az egy \definition{konfliktus}. Másképpen megfogalmazva, akkor használj konfliktust, ha a JK"~k elérhetik a céljukat erőszakkal vagy kényszerítéssel.

A konfliktusok a legegyértelműbbek -- elvégre a legelső szerepjátékok csataszimulációkból nőttek ki. De vésd eszedbe a legfontosabbat: a résztvevő karakterek bántalmazhatják \emph{egymást}. Ha ez egyoldalú -- például ha benyomsz egyet egy élő hegynek -- esélyed sincs sérülést okozni. Ez nem konfliktus. Ez egy versengés, ahol a JK"~k valószínűleg megpróbálnak elmenekülni, vagy valami módot keresnek, hogy hogyan vágjanak vissza.

Egy konfliktus lehet fizikai vagy szellemi. Fizikai konfliktus lehet lövöldözés, kardpárbaj vagy netán más dimenziók szörnyének legázolása egy kamionnal. Szellemi konfliktus lehet egy veszekedés a szeretteinkkel, kihallgatás vagy elmére irányuló mágikus támadás.

Csapatmunka (\page{32}) esetén fontos lehet az időzítés. Bármikor kihasználhatsz egy jellemzőt a társad nevében a dobásának javítására. Segíthetsz egy társadnak \emph{mielőtt} ő jönne helyzetbehozással vagy a cselekvésed +1 bónuszért feláldozásával. Ha előtted jönne a fordulóban, akkor helyzetbehozással nem segítheted, de elhasználhatod a cselekvésed (feladva a cselekvést az adott fordulóban), hogy +1 csapatmunka bónuszt adj neki.

\subsection{Sérülések elszenvedése}

Ha a megtámadás siker, a védekező félnek semlegesíteni kell a találat sikerességét, ami a megtámadás és a védekezés erőfeszítéseinek különbsége.

A sikerességet stressz dobozok beikszelésével vagy következmény jellemzők felvételével tudod semlegesíteni. Ha nem tudod, vagy csak nem akarod az egész sikerességet semlegesíteni, az a \definition{kiejtés} (\page{36}) -- kikerülsz a jelenetből, és a támadó határozza meg, mi történik a továbbiakban.

\example{%
A körülmények szerencsétlen összejátszása Charlest egy nyirkos pincébe vezette, ahol szembe kell szállnia egy ghoullal, ami fel akarja falni. A ghoul kitör éles karmaival kaszálva; ez egy megtámadás Jó~(+2) Közelharccal. A KM dobása \dice{00++}, ami miatt az erőfeszítés Kimagasló~(+4). Charles megpróbál elugrani a Remek~(+3) Atléta képességével, de a dobása \dice{000-}, ami miatt az erőfeszítése csak Jó~(+2). Mivel a ghoul megtámadás erőfeszítése kettővel meghaladja Charles védekezés erőfeszítését, Charlesnak két sikerességet kell semlegesítenie. Beikszeli az első két fizikai stressz dobozát a háromból; a küzdelem nagyon is veszélyesnek bizonyult.
}
