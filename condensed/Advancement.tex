\chapter{Fejlődés}

Ahogy a karaktereitek végighaladnak a történetszálakon, felnőnek, és megváltoznak. Minden játékülés végén elértek egy \definition{mérföldkőhöz} -- ekkor átrendezhetitek a karakterlapot. Amint azonban lezártok egy történetívet, elértek egy \definition{áttöréshez} -- ekkor már hozzá is adhattok valamit a karakterlaphoz. (A \onpage{40} többet is megtudhatsz a játékülésekről és történetívekről.)

\section{Mérföldkövek}

A mérföldkövek a játékülések végén találhatók, a történetívek részeként. Arra szolgálnak, hogy a karaktereket megváltoztassák, ne felfejlesszék. Akár ki is hagyhatsz egy mérföldkövet, ha nincs rá szükséged. Nem kötelező módosítanod a karaktereden. De azért a lehetőség megvan rá.

Mérföldkő esetén egyetlen egyet tehetsz meg az alábbiak közül:

\begin{itemize}
    \item Felcserélheted két képességed szintjét, vagy lecserélhetsz egy Átlagos~(+1) képességet olyanra, ami még nem szerepel a karakterlapon.
    \item Átírhatsz egy fortélyt.
    \item Új fortélyt vehetsz fel egy újratöltésért. (De vésd észbe, hogy nem mehetsz 1 újratöltés alá.)
    \item Átírhatod az egyik jellemződet a koncepció jellemző kivételével.
\end{itemize}

\section{Áttörések}

Az áttörések már jelentősebbek, amikor a karaktered hatalma megnő. Áttörés esetén egyetlen egyet tehetsz meg a mérföldkő listájából. Ezen felül az \emph{összes} alábbit is:

\begin{itemize}
    \item Átírhatod a karaktered koncepció jellemzőjét, ha ahhoz van kedved.
    \item Ha van mérsékelt vagy súlyos következmény jellemződ, ami még nem gyógyul, akkor elindíthatod a gyógyulási folyamatot a jellemző átnevezésével. A gyógyulásban lévő következményeket pedig törölheted.
    \item Megnövelheted egy képességed szintjét eggyel -- akár Középszerűről~(+0) Átlagosra~(+1).
\end{itemize}

Ha a KM úgy gondolja, hogy lezárult a cselekmény egy jelentősebb szakasza, és ideje a karaktereket egy kicsit „felturbózni”, felajánlhat egyet az alábbiakból:

\begin{itemize}
    \item Egy újabb újratöltést kapsz, amit akár rögtön fortélyra is költhetsz.
    \item Még egy képességet növelhetsz eggyel.
\end{itemize}

\newpage

\subsection{Képességek szintjének növelése}

Amikor a képességek szintjét növeled, meg kell tartanod az „oszlop” struktúrát. Egyik szinten sem lehet több képesség, mint a közvetlenül alatta lévő szinten. Ez azt jelenti, hogy először vagy pár Középszerű~(+0) képességet kell megnövelned -- vagy pedig tartalékolnod kell a növeléseket, amiket később egyszerre tudsz egy nagyobb növelésre fordítani.

\example{%
Ruth szeretné Átlagos~(+1) Misztikum képességét Jóra~(+2) növelni, de emiatt négy Jó~(+2) és csak három Átlagos~(+1) képessége lenne, ami nem lehetséges. Szerencséjére, tartalékolt egy képesség szint növelést az előző áttörésből, így megnövelheti Középszerű~(+0) Empátia képességét is Átlagosra~(+1). Így már van egy Kimagasló~(+4), kettő Remek~(+3), négy Jó~(+2) és négy Átlagos~(+1) képessége.
}

\begin{center}
\fatetable{r l}{
\multicolumn{2}{|c|}{\textcolor{white}{Piramis}} \\
+4 & \dice{0} \\
+3 & \dice{00} \\
+2 & \dice{000} \\
+1 & \dice{0000} \\
}
\fatetable{r l}{
\multicolumn{2}{|c|}{\textcolor{white}{Helytelen}} \\
+4 & \dice{0} \\
+3 & \dice{00} \\
+2 & \dice{0000} \\
+1 & \dice{000} \\
}
\fatetable{r l}{
\multicolumn{2}{|c|}{\textcolor{white}{Helyes}} \\
+4 & \dice{0} \\
+3 & \dice{00} \\
+2 & \dice{0000} \\
+1 & \dice{0000} \\
}
\fatetable{r l}{
\multicolumn{2}{|c|}{\textcolor{white}{Helyes}} \\
+4 & \dice{0} \\
+3 & \dice{000} \\
+2 & \dice{000} \\
+1 & \dice{000} \\
}
\end{center}

\section{Játékülések és történetívek}

Van pár feltevésünk, amikor játékülésekről és történetívekről beszélünk. Szeretnénk rávilágítani ezekre a feltevésekre, hogy végrehajthasd a megfelelő változtatásokat, ha a te játékod különbözik ettől.

A \definition{játékülés} egyetlen játékalkalom, ami néhány jelentből áll, és pár óra alatt lezajlik. Fogd fel úgy, mint egy TV sorozat egyetlen epizódját. Valószínűleg három"~négy órán át tart.

A \definition{történetív} játékülések sorozata, amiknek a cselekményei több játékalkalmon is átívelhetnek. Ezeknek a cselekményeknek nem feltétlenül kell az ívben lezáródniuk, de általában jelentős változások állnak elő benne annak folyamán. Fogd fel úgy, mint egy harmad vagy fél TV sorozat évadot. Valószínűleg négy körüli játékülésből állnak.

Ha a te játékod időtartama kívül esik ezeken a „valószínű” hosszakon, megváltoztathatod a fejlődés működését.  Ha egy történetív hosszabb, mint négy"~hat játékülés, akkor megengedheted a súlyos következmények törlését már négy játékülés után, mintsem, hogy várni kelljen a történetív végéig. Ha szeretnéd lassítani a fejlődést, akkor adagolhatod a fejlesztéseket, mint például a képesség növeléseket vagy újratöltéseket, ritkábban is. Ha a csapatod általában rövidebb játéküléseket tart, akkor nem muszáj mérföldkőhöz érni mindegyiknek a végén. Fűszerezd ízlés szerint; a saját játékodat úgy módosíthatod, ahogy csak akarod.
