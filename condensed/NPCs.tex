\section{NJK"~k}

Egy NJK lehet egy szemlélődő, mellékszereplő, szövetséges, ellenfél, szörny vagy bármi más, ami komplikálhatja vagy gátolhatja a JK"~k fáradozásait. Valószínűleg olyan karaktereket is létre szeretnél hozni, amivel a JK"~k interakcióba léphetnek.

\subsection{Fontos NJK"~k}

Ha valaki különösképpen fontos a történet szempontjából, akkor ugyanúgy kidolgozhatod, mint egy JK"~t. Ez különösképpen helyénvaló, ha a JK"~k sokat fognak foglalkozni vele, mint például egy szövetséges, egy rivális, egy hatalmi csoportosulás képviselője, netán a főellenség.

Egy fontos NJK"~nak nem muszáj ugyanazokat a megkötéseket követnie, mint egy kezdő JK"~nak. Ha az NJK egy ismétlődően felbukkanó főellenség szintű fenyegetést jelent, akkor nyugodtan adhatsz neki magasabb szintű képességet is (lásd a \textit{„Nehézség és ellenállás meghatározása”} fejezetet a \onpage{42}), több fortélyt vagy bármi mást is, amitől veszélyesebbé válik.

\subsection{Kisebb NJK"~k}

Azokat az NJK"~kat, akik nem lesznek visszatérő, lényeges karakterek, nem kell annyira részletesen kidolgozni, mint a fontos NJK"~kat. Egy kisebb NJK"~nál csak annyit határozz meg, amennyi feltétlenül szükséges.

A legtöbb kisebb NJK"~nak egyetlen egy jellemzője lesz, ami megadja, hogy ki ő: \aspect{Őrzőkutya}, \aspect{Akadékoskodó Bürokrata} vagy \aspect{Feldühödött Szektatag}, és így tovább.

Ha szükséges, kaphatnak még egy"~két jellemzőt, amivel mondhatsz valami érdekeset róluk, vagy megadhatod a gyengeségüket. Szintén kaphatnak még egy fortélyt is.

Adj nekik egy"~két képességet, amik megmutatják, hogy miben jók. Kiválaszthatod ezeket a képességeket a képesség listáról, vagy kitalálhatsz valami különlegesebbet is, például Jó~(+2) a Tömegverekedés Kierőszakolása képessége, vagy Kimagasló~(+4) a Harapásban.

Végül legyen nulla és három közötti stressz doboza; minél több van neki, annál nagyobb fenyegetést jelent. Általában nincs következmény jellemző rubrikájuk; ha nagyobb sérülést szenvednek el, mint amennyit stresszel semlegesíthetnének, akkor egyszerűen kiejtődnek a küzdelemből. A kisebb NJK"~k nem helytállásra készülnek.

\subsection{Szörnyek, főellenségek és más veszedelmek}

Az NJK"~khoz hasonlóan a szörnyek és más veszedelmek (mint például egy vihar, elharapózó tűzvész vagy egy szakasz felfegyverzett talpnyaló) karakterként vannak létrehozva, de a JK"~khoz képest sokkal egyszerűbb módon. Csak annyit kell kidolgoznod, amennyi feltétlenül szükséges. A kisebb NJK"~kkal szemben ezeket a veszedelmeket igazándiból akárhogyan le lehet írni. Szegd meg a szabályokat (\page{54}). A jellemzők, képességek, fortélyok, stressz és következmény jellemző rubrikák bármilyen kombinációját használhatod, hogy kellőképpen veszélyessé tedd őket, és a szintek meghatározásához döntsd el, hogy mekkora problémát okozzanak a JK"~knak.
