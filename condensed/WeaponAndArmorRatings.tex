\section{Fegyver és páncél szintek}

Szeretnéd átvenni más játékok fegyverzet szabályainak hangulatát? Próbáld ki a fegyver és páncél szinteket! Leegyszerűsítve, ha eltalálnak egy fegyverrel, akkor nagyobb a sebzés, míg a páncél megvéd ettől. (Ezt fortélyokkal is modellezhetnéd, de erre elpazarolni a fortély rubrikákat nem mindenkinek jön be.)

A \definition{fegyver} szint hozzáadódik a sikeres megtámadás sikerességéhez. Ha van Fegyver:2"~d, akkor minden találat 2"~vel nagyobbat sebez. Ez döntetlen esetén is működik; ha a kimenetel döntetlen, akkor előny jellemző \emph{helyett} stressz sebzést érsz el.

A \definition{páncél} szint csökkenti a sikeres megtámadás sikerességét. Így a Páncél:2 esetén minden találat 2"~vel kisebbet sebez. Ha találsz, de a célpont páncél szintje 0"~ra vagy az alá csökkenti a sikerességet, akkor ugyan nem érsz el stressz sebzést, viszont kapsz helyette egy előny jellemzőt.

Nagyon fontos a szintek helyes megválasztása. Szintén érdemes megfontolni, hogy mennyire lesz valószínű a következmény okozása döntetlen esetén. Szerintünk a 0 és 4 közötti szintek megfelelőek.
