\label{Cselekvések}
\section[Cselekvések]{Cselekvések (Actions)}

Négyféle cselekvés van, amire dobhatsz, mindegyik egyedi céllal rendelkezik, és egyedi a hatása is a történetre:

\begin{itemize}
    \item \textbf{Megold} egy problémát, hogy leküzdjön egy akadályt a képességeivel.
    \item \textbf{Helyzetbehozás}, hogy előnyösebbé változtassa a körülményeket.
    \item \textbf{Megtámadás}, ami az ellenfélnek kárt okoz.
    \item \textbf{Védekezés}, hogy megússzon egy megtámadást, megállítsa az ellenfelek helyzetbehozását, vagy szembeszegüljön egy probléma megoldásának.
\end{itemize}

\label{Megoldás}
\actionsection{O}{Megoldás}{Megoldás (Overcome)}{Megold egy problémát, hogy legyőzzön egy akadályt a képességeivel.}

Minden karakter számtalan kihívással szembesül a történetben. A \definition{megold} cselekvéssel tudnak szembeszegülni, és legyőzni ezeket az akadályokat.

Egy karakter, akinek magas az Atléta képessége, fel tud mászni a falakon, vagy a zsúfolt utcákon rohanni. Egy detektív magas Nyomozással olyan nyomokat is értelmezhet, ami felett mások átsiklottak. Ha valaki képzett Befolyásolásban, sokkal könnyebben elkerülheti a verekedést egy ellenséges bárban.

A megoldás kimenetelei:

\begin{itemize}
    \failureitem \textbf{Ha kudarc,} beszéld meg a KM"~el (és a védekező játékossal, ha van ilyen), hogy ez egyszerű kudarc, netán siker komoly áron (\page{Siker komoly áron}).
    \tieitem \textbf{Ha döntetlen,} akkor ez egy siker kisebb áron (\page{Siker kisebb áron}) -- kellemetlen szituációba kerülsz, az ellenfél kap egy előny jellemzőt (\page{Előny jellemzők}), vagy elszenvedsz egy találatot. Alternatíva, ha kudarcot vallasz, de kapsz egy előnyt.
    \successitem \textbf{Ha siker,} akkor eléred a célod, és a történet megy tovább, bármiféle bukkanó nélkül.
    \successwithstyleitem \textbf{Ha átütő siker,} akkor nem csak, hogy siker, de kapsz egy előnyt is.
\end{itemize}

\example{%
Charles eljutott az Antarktiszra a kutató létesítményhez. Az épületek romokban, és a lakók sehol. Át akarja kutatni a törmeléket nyomok reményében. A KM azt mondja, hogy dobjon Nyomozásra Jó~(+2) nehézség ellen. Charles dobása \dice{00++} a kockákon, plusz a Nyomozása Átlagos~(+1), az erőfeszítés így Remek~(+3). Ez siker! A KM leírja a nyomokat: lábnyomok a hóban, olyan lényeké, amik sok vékony, nem emberi lábakon lépkednek.
}

Gyakran használjuk a megold cselekvést, ha el kell dönteni, hogy a karakter hozzáfér"~e, vagy észrevesz"~e egy bizonyos tényt vagy nyomot. Ha így van, tartsd észben, hogy létezik a siker valamilyen áron lehetőség is. Ha megakasztaná a történetet, hogy ha egy részlet felett elsiklanának a játékosok, akkor legjobb ezt a lehetőséget teljesen kizárni, és csak az árra fókuszálni.

\newpage

\label{Helyzetbehozás}
\actionsection{C}{Helyzetbehozás}{Helyzetbehozás (Create An Advantage)}{Létrehoz egy új helyzet jellemzőt, vagy a maga javára fordít egy már meglévő jellemzőt.}

A \definition{helyzetbehozás} cselekvéssel a történet folyását tudod megváltoztatni. A képességeiddel új jellemzőket behozva, vagy meglévő jellemzőkhöz kihasználásokat adva, magad és a csapattársaid felé billentheted a mérleg nyelvét. Megváltoztathatod a körülményeket (eltorlaszolni egy ajtót vagy előkészülni egy tervvel, aminek a végrehajtására jár ingyen kihasználás), új információkat fedezhetsz fel (kikutatni egy szörny gyenge pontját), vagy felhasználhatsz valami köztudottat (például a CEO kedvenc whisky márkáját).

A helyzetbehozással létrehozott jellemző ugyanúgy működik, mint bármelyik másik: lefekteti a történet körülményeit, ami lehetővé teheti, megakadályozhatja, vagy csak gátolhatja bizonyos cselekedetek végrehajtását -- például nem lehet egy varázslatot elolvasni, ha a szoba \aspect{Töksötét}. Ugyanúgy lehet kihasználni (\page{Kihasználás}), és ugyanúgy késztethet (\page{Késztetés}). Ezen felül egy helyzet jellemző létrehozásakor kapsz hozzá egy vagy több \definition{ingyen kihasználást} is. Az ingyen kihasználás, ahogy azt a neve is mutatja, ingyenes, nem kell érte sors pontot fizetni. Akár még a szövetségeseidnek is átengedheted az ingyen kihasználást.

Ha helyzetbehozásra dobsz, el kell döntened, hogy új jellemzőt hozol létre, vagy egy már meglévőnek szeretnéd hasznát venni. Ha az előbbi, akkor ezt a jellemzőt egy szövetségeshez, ellenséghez vagy a környezethez szeretnéd csatolni? Ha egy ellenségre, akkor az védekezés cselekvést tehet ellene. Egyéb esetekben általában egy nehézség ellen kell ilyenkor dobni, de a KM dönthet úgy is, hogy valaki megpróbál meggátolni, és akkor az védekezés dobás lesz.

A kimenetelek új jellemző létrehozásánál:

\begin{itemize}
    \failureitem \textbf{Ha kudarc,} akkor nem tudod létrehozni a jellemzőt (egyszerű kudarc), vagy létrejön, de az ellenfél kapja az ingyen kihasználást (siker komoly áron). Ez utóbbi esetben a jellemzőt lehet, hogy át kell fogalmazni, hogy az ellenfelet segítse. Még így is megérheti, mert a jellemző mindig igaz (\page{A jellemző mindig igaz}).
    \tieitem \textbf{Ha döntetlen,} akkor nem tudod létrehozni a jellemzőt, de helyette kapsz egy előnyt (\page{Előny jellemzők}).
    \successitem \textbf{Ha siker,} akkor létrehozod a jellemzőt, és kapsz egy ingyen kihasználást.
    \successwithstyleitem \textbf{Ha átütő siker,} akkor létrehozod a jellemzőt, és kapsz \emph{két} ingyen kihasználást.
\end{itemize}

A kimenetelek meglévő, ismert vagy ismeretlen jellemző esetén:

\begin{itemize}
    \failureitem \textbf{Ha kudarc,} és a jellemző ismert volt, akkor az ellenfél kap egy ingyen kihasználást a jellemzőhöz. Ha ismeretlen volt, akkor felfedhetik egy ingyen kihasználásért cserébe.
    \tieitem \textbf{Ha döntetlen,} kapsz egy előnyt, ha a jellemző ismeretlen volt; és ismeretlen is marad. Ha a jellemző ismert volt, akkor ehelyett kapsz egy ingyen kihasználást hozzá.
    \successitem \textbf{Ha siker,} akkor kapsz egy ingyen kihasználást a jellemzőhöz, felfedve, ha addig ismeretlen volt.
    \successwithstyleitem \textbf{Ha átütő siker,} két ingyen kihasználást kapsz a jellemzőhöz, felfedve, ha addig ismeretlen volt.
\end{itemize}

\newpage

\example{%
Ethan szembekerül egy shoggothal, ami egy masszív, fáradhatatlan húshegy. Tisztában van vele, hogy ez túl hatalmas szörny egy direkt megtámadáshoz, ezért úgy dönt, hogy inkább elvonja a figyelmét: „Készítek egy Molotov koktélt, és lángba borítom” -- mondja.
\newline
A KM úgy dönt, hogy a shoggothot triviális eltalálni, ezért ez egy Ezermester dobás lesz -- milyen gyorsan tud találni valami gyúlékonyat, amit fegyverré alakíthat. A nehézség Remek~(+3). Ethan Ezermester képessége Átlagos~(+1), a dobása \dice{0+++}, így az erőfeszítés Kimagasló~(+4).
\newline
Ethan összeüt egy Molotov koktélt, és a szörnyre hajítja. A shoggoth így \aspect{Lángol}, és Ethannek van egy ingyen kihasználása ehhez a jellemzőhöz. Ez láthatóan elvonta a shoggoth figyelmét, így ha üldözni kezdi, Ethan kihasználhatja a jellemzőt, hogy egérutat nyerjen.
}

\label{Megtámadás}
\actionsection{A}{Megtámadás}{Megtámadás (Attack)}{Megtámadja az ellenfelet, hogy kárt okozzon neki.}

A \definition{megtámadás} cselekvéssel lehet megpróbálni kiejteni az ellenfelet -- legyen szó egy undorító szörnyeteg levágásáról, vagy csak egy ártatlan őr kupán vágásáról, aki még csak azt sem tudja, hogy mit őriz. Egy megtámadás lehet egy géppuska ellenfélre ürítése, egy kemény jobb egyenes, netán egy káros varázslat.

Vedd figyelembe, hogy egyáltalán lehetséges"~e sérülést okozni a célpontnak. Nem minden megtámadás hasonló nagyságrendű, egy kaiju valószínűleg meg sem érezné, ha behúznál neki egyet. Döntsd el, hogy a megtámadásnak egyáltalán van"~e reális esélye a károkozásra, mielőtt nekiállnál dobni. Vannak olyan hatalmas lények, amiknek valamilyen gyengeségét kell kiaknázni, vagy a védelmüket kell valahogy megkerülni, mielőtt bármiféle kárt tudnál tenni bennük.

A kimenetelek megtámadás esetén:

\begin{itemize}
    \failureitem \textbf{Ha kudarc,} akkor nem tudod eltalálni -- a támadást kivédték, elmozdultak előle, vagy csak felfogta a páncél.
    \tieitem \textbf{Ha döntetlen,} akkor éppen csak súrolod, vagy talán az ellenfél meghátrált. Akárhogyan is, de kapsz egy előnyt (\page{Előny jellemzők}).
    \successitem \textbf{Ha siker,} akkor a sikerességgel megegyező nagyságú sérülést okozol. A védekező félnek ezt kell stressz dobozokkal vagy következmény jellemzőkkel semlegesítenie, vagy kiejtődik (\page{Kiejtés}).
    \successwithstyleitem \textbf{Ha átütő siker,} akkor ugyanúgy sérülést okozol, mint siker esetében, de választhatod, hogy eggyel kisebb sikerességért, egy extra előnyt is kapsz.
\end{itemize}

\example{%
Ruth belebotlik egy csontvázba, amit misztikus erők állítottak valami sötét dolog szolgálatába. Úgy dönt, hogy orrba vágja. A Közelharc képessége Kimagasló~(+4), de a dobása \dice{-{}-00}, így az erőfeszítés csak Jó~(+2).
}

\newpage

\label{Védekezés}
\actionsection{D}{Védekezés}{Védekezés (Defend)}{Védekezik, hogy megússzon egy megtámadást, vagy meggátolja valaki más cselekvését.}

Egy szörny próbál szétmarcangolni? Egy ellenfél próbál félretaszítani, a haragod elől menekülve? Mit teszel, ha egy szektatag próbálja mindkét veséd szétszurkálni? \definition{Védekezés}, védekezés, védekezés.

A védekezés az egyetlen reakció cselekvés a \fate{Fate Condensed} játékban. A védekezéssel meggátolhatsz valamit, még ha nem is te jössz, így passzív nehézség helyett, gyakran aktív ellenállás dobás lesz. Az ellenséged dob, és rögtön ezután lehetőséged van dobni, ha te vagy a célpont, vagy pedig indokolható, hogy miként tudsz közbeavatkozni (ami gyakran téged tesz célponttá). Néhány jellemző vagy fortély adhat is indoklást.

A kimenetelek védekezés esetén:

\begin{itemize}
    \failureitem \textbf{Ha kudarc} megtámadás ellen, akkor elszenveded a találatot, amit köteles vagy stressz dobozokkal vagy következmény jellemzőkkel semlegesíteni (\page{Stressz}). Más esetben az ellenfeled sikeres abban, amit eltervezett.
    \tieitem \textbf{Ha döntetlen,} az ellenfeled cselekvésének a döntetlen kimenetele adja meg, mit kell tenni.
    \successitem \textbf{Ha siker,} akkor nem szenveded el a találatot, vagy meggátolod az ellenfelet a tervében.
    \successwithstyleitem \textbf{Ha átütő siker,} akkor nem szenveded el a találatot, vagy meggátolod az ellenfelet a tervében, és egy előnyt is kapsz, mert pillanatnyilag felülkerekedtél rajta.
\end{itemize}

\example{%
Az előbbi példát folytatva, a csontváz védekezhet Ruth ellen. A KM dobása \dice{-00+}, ami nem módosítja a lény Középszerű~(+0) Atléta képességét.
\newline
Mivel Ruth erőfeszítése a magasabb, a támadása siker. A sikeresség kettő, és a csontváz egy kicsit közelebb került ahhoz, hogy ne keljen fel megint. Ha a csontváz jobbat dobott volna, akkor a védekezése siker lehetett volna, és az élőhalott szörnyűség elkerülhette volna a találatot.
}

\subsubsection{Mely képességekkel lehet megtámadni és védekezni?}

Az alapértelmezett képesség lista az alábbi elveket követi:

\begin{itemize}
    \item Közelharc és Célzás alkalmas fizikai megtámadásra.
    \item Atlétával bármilyen fizikai megtámadás ellen lehet védekezni.
    \item Közelharccal csak Közelharci megtámadás ellen lehet védekezni.
    \item Kényszerítés alkalmas szellemi megtámadásra.
    \item Akaraterővel lehet szellemi megtámadás ellen védekezni.
\end{itemize}

Speciális körülmények között, más képességeket is lehet megtámadásra vagy védekezésre használni, ha a KM engedélyezi, vagy konszenzus alakul ki a játékosok között. Némelyik fortély adhat szélesebb körű, garantált engedélyt akkor is, amikor a körülmények nem lennének alkalmasak. Ha egy képesség nem használható direkt megtámadásra vagy védekezésre, de hasznos lehet a szituációban, akkor használd a helyzetbehozás cselekvést, és használd el az ingyen kihasználásokat a következő megtámadás vagy védekezés dobásodhoz.
