\label{Képességek}
\subsection[Képességek]{Képességek (Skill)}

Míg a jellemzők megadják, hogy ki a karakter, addig a \definition{képességek} azt mutatják, hogy mit képes megtenni. Minden képesség tevékenységek széles körét foglalja magába, amit a karakter képzés vagy gyakorlás útján sajátított el, netán csak veleszületett. A karakter a Tolvajlás képességgel, valamilyen szinten, bármilyen bűntényt elkövethet, ami a lopás művészetéhez kapcsolódik -- terepszemle, biztonsági rendszer megkerülése, zsebmetszés, zárnyitás.

Minden képességnek van egy \definition{szintje}. Minél magasabb a szint, annál jobb a karakter abban a dologban. A karaktered képességeinek összessége mutatja, hogy mi az, amire született, mi az, amivel még megbirkózik, és mi az, amit nem kéne erőltetnie.

A karaktered képességeinek szintjeit az alábbi piramis struktúra alapján választhatod ki, ahol a legmagasabb szint a Kimagasló~(+4):

\begin{itemize}
    \item Egy darab Kimagasló~(+4) képesség
    \item Kettő darab Remek~(+3) képesség
    \item Három darab Jó~(+2) képesség
    \item Négy darab Átlagos~(+1) képesség
    \item Minden más képesség szintje Középszerű~(+0)
\end{itemize}

\label{Képesség szintek skálája}
\subsubsection{Képesség szintek skálája}

A \fate{Fate Condensed} játékban, ahogy a normális Fate játékban is, a képességek szintjei az alábbi skálába vannak rendezve:

\begin{center}
\fatetable{r l}{
\textcolor{white}{Szint} & \textcolor{white}{Melléknév} \\
+8 & Legendás (Legendary) \\
+7 & Eposzi (Epic) \\
+6 & Fantasztikus (Fantastic) \\
+5 & Emberfeletti (Superb) \\
+4 & Kimagasló (Great) \\
+3 & Remek (Good) \\
+2 & Jó (Fair) \\
+1 & Átlagos (Average) \\
+0 & Középszerű (Mediocre) \\
-1 & Gyenge (Poor) \\
-2 & Borzalmas (Terrible) \\
-3 & Katasztrofális (Catastrophic) \\
-4 & Horrorisztikus (Horrifying) \\
}
\end{center}
