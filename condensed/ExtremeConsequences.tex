\section{Extrém következmények}

Az extrém következmények behoznak egy opcionális negyedik következmény súlyosságot a játékba: valamit, ami örökre, visszavonhatatlanul megváltoztatja a karaktert.

Egy extrém következmény felvétele 8"~at hatástalanít az elszenvedett stresszből. Ha felveszed, \textbf{le kell cserélned} a karakter egyik meglévő jellemzőjét (a koncepció jellemző kivételével, ami szóba se jöhet) egy olyanra, ami a karakterben bekövetkező súlyos sérülés elszenvedése miatti változást mutatja.

Alapállapotban nincs lehetőség az extrém következményből felgyógyulásra. Ez már a karakter részévé vált. A következő áttörésnél átnevezheted a jellemzőt, hogy mutassa, ha a karakter megbékélt a sorsával, de nem térhetsz vissza az eredeti jellemzőhöz.

Két áttörés között a karakter csak egyszer élhet ezzel a lehetőséggel.

\section{Gyorsított versengés}

Néhány csapat úgy érezheti, hogy túl sok lehetőség van előnybehozásra fordulónként. Ők kipróbálhatják a következőt: a versengés minden fordulójában, minden résztvevő csak egyetlen egyet tehet az alábbiakból:

\begin{itemize}
    \item Megold cselekvés a saját oldalának (\page{18}).
    \item Dob helyzetbehozásra, de nem adhat ilyenkor csapatmunka bónuszt (\page{32}).
    \item Csapatmunka bónuszt adhat az oldala megold cselekvésére vagy valaki más helyzetbehozására. Ilyenkor viszont nem dobhat egyáltalán.
\end{itemize}

\section{Védekező harc}

Néha egy játékos (vagy a KM) úgy szeretné, hogy a karaktere teljes mértékben csak a védekezésre koncentráljon, amíg újra sorra nem kerül a következő fordulóban, ahelyett, hogy cselekvést tenne. Ezt úgy hívjuk, hogy \definition{védekező harc}.

A védekező harc bejelentésekor meg kell adni annak \definition{célját} is. Alapértelmezésben saját magad véded (megtámadásoktól, valamint ellened irányuló helyzetbehozásoktól), de irányulhat ez mások védelmezésére, védekezésre egy meghatározott agresszor csoport ellen vagy egy meggátolandó tett vagy végkifejlet ellen.

\textbf{A védekező harc alatt +2 bónuszt kapsz minden védekezés dobásra, ami kapcsolódik a védekezés céljához.}

Ha semmi sem sül ki belőle, mert egyetlen egyszer sem kellett védekezésre dobnod mire sorra kerülsz a következő fordulóban, akkor kapsz egy előnyt (\page{23}), mivel lehetőséged nyílt felkészülni a következő cselekvésre. Ezt ellensúlyozza, hogy „elvesztegetsz egy fordulót”, mert olyan dolog ellen védekeztél, ami végül meg sem történt.
