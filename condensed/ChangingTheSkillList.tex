\section{Képesség lista megváltoztatása}

Ahogy azt a \onpage{9} is írtuk, a saját Fate játékod kidolgozásakor elsőként a képesség listán gondolkodj el. Alapértelmezésben 19 képességet kell elosztani egy 10 elemű piramisban. A képesség lista úgy van megalkotva, hogy a tradicionálisan használt cselekedeteket fedje le, lényegében azt megválaszolva, hogy „mit tudsz megtenni”? Más képesség listák nem szükségszerűen ugyanilyen hosszúak, vagy ugyanígy vannak elosztva, vagy ugyanarra a kérdésre válaszolnak. Mindezek után néhány rövid lista, amit figyelembe vehetsz, amiből kölcsönözhetsz ötleteket, és amiket módosíthatsz.

\begin{itemize}
    \item \textbf{Cselekedetek:} Elvisel, Harcol, Tud, Mozog, Észlel, Vezet, Lopózik, Beszél, Bütyköl.
    \item \textbf{Megközelítések:} Óvatos, Eszes, Látványos, Erőteljes, Gyors, Trükkös.
    \item \textbf{Adottságok:} Atléta, Harc, Irányítás, Tudomány, Bűnözés.
    \item \textbf{Tulajdonságok:} Erő, Ügyesség, Keménység, Intelligencia, Karizma.
    \item \textbf{Kapcsolatok:} Irányító, Partneri, Támogató, Egyéni.
    \item \textbf{Szerepek:} Sofőr, Nehézfiú, Számítógépkalóz, Szerelő, Szélhámos, Tolvaj, Lángelme.
    \item \textbf{Elemek:} Levegő, Tűz, Fém, Elme, Kő, Űr, Víz, Szél, Fa.
    \item \textbf{Értékek:} Kötelesség, Dicsőség, Igazságosság, Szeretet, Hatalom, Biztonság, Igazság, Bosszú.
\end{itemize}

Ha hosszabb listára van szükséged, akkor indulj ki az alapértelmezett listából, majd addig kell hozzáadni, összevonni, elvenni képességeket, amíg meg nem kapod, amire vágysz. Ehelyett egybegyúrhatsz két vagy több listát is feljebbről.

\textbf{Fejlődés:} Minél rövidebb a listád az alapértelmezetthez képest, annál ritkábban kell képesség növelést osztanod fejlődés esetén. Akár tartogathatod ezt kizárólag a „felturbózásra” (\page{39}), vagy más módon is korlátozhatod.

\textbf{Alternatívák a piramis helyett:}

\begin{itemize}
    \item \textbf{Rombusz:} Széles közép (körülbelül a képességek harmada), ami elkeskenyedik az alja és a teteje felé, másképpen 1 darab +0, 2 darab +1, 3 darab +2, 2 darab +3, 1 darab +4.
    \item \textbf{Oszlop:} Nagyjából ugyanannyi képesség minden szinten. Ha elég rövid a listád, ez lehet egy vonal is, egy képesség minden szinten.
    \item \textbf{Szabad + Limit:} Minden játékos kapjon annyi képesség pontot, amennyi a piramishoz (vagy más formációhoz) kell, de a forma ne legyen kötelező. Ezután szabadon vásárolhatnak képességeket, de a szinteknek a limit alatt kell maradniuk.
\end{itemize}

\textbf{Lefedés:} Mindenképpen vedd figyelembe, hogy mennyi képességnek lesz szintje a listából. Alapértelmezésben ez 53\% (10 a 19"~ből). Minél magasabb ez a százalék, annál inkább átfedik egymást a játékosok karakterei. Védd meg a karakterek egyedi szakterületeit!

\textbf{Kombinálás:} Lehet két listád is, és a játékosok minden dobásnál egy"~egy képességet használnak mindkét listáról. A legfontosabb, hogy a lehetséges szint összegeket nulla és a limit között tartsd. Például mindkét lista képességei lehetnek +0 és +2 között, vagy -1 és +1 között az egyik, míg +1 és +3 között a másik listán, és így tovább.
