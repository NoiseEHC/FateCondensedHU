\chapter{Kaland Mesterek kézikönyve}

KM"~ként te vagy a rendező a játékülés során. Vésd az eszedbe, hogy ez nem jelenti, hogy te lennél a \emph{főnök}. A \fate{Fate Condensed} játék együttműködésre épül, és a játékosok döntik el, hogy mi történjen a karakterükkel. A te dolgod az események mozgásban tartása az alábbiakkal:

\begin{itemize}
    \item \textbf{Jelenetek levezénylése:} A játékülés jelenetekből áll össze. Döntsd el, hogy mikor kezdődik a jelenet, ki van jelen, és mi történik. Ha szerinted már minden érdekes lehetőség megtörtént a jelenetben, vess véget neki. Ugord át az érdektelen részeket; ahogy nem dobsz akkor se, ha a cselekvés eredménye nem elég érdekes, ugyanúgy hagyd ki a jeleneteket, ahol semmi izgalmas, drámai, használható vagy mókás sem történik.
    \item \textbf{Szabályok meghatározása:} Ha kérdéses, hogy hogyan is kell a szabályokat alkalmazni az adott körülmények között, megbeszélheted a játékosokkal, hogy valami konszenzusra jussatok, de a végső szó mindig a tiéd marad.
    \item \textbf{Nehézségek meghatározása:} Döntsd el, hogy mikor szükséges dobni, és milyen nehézségre.
    \item \textbf{Döntsd el a kudarc árát:} Ha egy karakter dobása kudarc, akkor a te döntésed, hogy siker komoly áron esetében pontosan mekkora is ez az ár. Természetesen elfogadhatod a játékos javaslatát -- lehet, hogy tökéletesen tisztában van, milyen sérülést szeretne, hogy a karaktere elszenvedjen --, de legvégül ez a te döntésed lesz.
    \item \textbf{Játssz az NJK"~kal:} Minden játékos a saját karakterét irányítja, míg te az összes többit, a szektatagoktól kezdve a szörnyeken át a főellenségig mindenkit.
    \item \textbf{Adj lehetőséget a JK"~knak akciózni:} Ha a játékosok nem tudják, hogy mit is kéne tenniük, a te dolgod, hogy megadd nekik a kezdőlökést. Ne hagyd, hogy a cselekmény megakadjon a tétovázás vagy az információhiány miatt -- valamivel rázd fel a dolgokat. Ha kétségeid vannak, gondolkodj el a főellenség  céljain és taktikáján, ami alapján a hősöket zaklatni tudod.
    \item \textbf{Biztosítsd, hogy mindenki megkapja néha a rivaldafényt:} Nem az a dolgod, hogy legyőzd a játékosokat, hanem hogy megfelelő kihívást intézz ellenük. Biztosítsd, hogy mindegyik JK főszereplővé válhasson egy kis időre. A késztetéseket és kihívásokat szabd testre a karakterek tehetségeinek és gyengeségeinek megfelelően.
    \item \textbf{Komplikáld a JK"~k életét:} Azon kívül, hogy szörnyeket zúdítasz a karakterekre, te leszel a legfőbb késztetés kezdeményező is. Természetesen a játékosok is késztethetik magukat vagy másokat, de mindenki számára lehetővé kell tenned, hogy megtapasztalhassák a jellemzőik negatív hatásait.
    \item \textbf{Építkezz a játékosok döntéseire:} Nézd meg, milyen cselekedeteket végeztek a JK"~k játék közben, és gondold tovább, hogy milyen hatással van ez a világra, és az hogyan reagál. Élővé teheted a világot, ha a játékban szembesíted a JK"~kat ezekkel a következményekkel, legyenek azok jók vagy rosszak.
\end{itemize}
