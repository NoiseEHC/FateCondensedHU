\subsection{Késztetés}

A jellemzők \definition{késztethetnek} komplikálva a szituációt, de közben sors pontokat adva. A KM vagy egy játékos felajánlhat egy sors pontot valaki másnak, hogy a karakterének egy jellemzője késztessen, miközben kifejti, hogy a jellemző miért teszi a dolgokat nehezebbé vagy komplikáltabbá. Ha ellenállsz a késztetésnek, akkor a saját sors pontjaidból kell egyet elköltened, és le kell írnod, hogy hogyan kerülöd el a komplikációkat. Ez azt is jelenti, hogy ha kifogytál a sors pontokból, akkor képtelen vagy elkerülni a késztetéseket!

\textbf{Bármelyik jellemző késztethet} – legyen az a karakter jellemzője, helyzet, vagy következmény – de mindenképpen hatással kell lennie a karakterre, akit késztet.

\textbf{Bárki kezdeményezhet késztetést.} A késztetést indítványozó játékosnak viszont a saját sors pontjaiból kell egyet elköltenie. Ezután a KM sajátjaként bonyolítja le a célkaraktert érintő késztetést. A KM soha nem használ el sors pontot késztetés kezdeményezéséért – limitált készletük van kihasználásra, de annyit késztetnek, amennyit csak akarnak.

\textbf{A késztetés lehet utólagos is.} Ha egy játékos beleszerepjátszotta magát egy komplikációba, ami vagy a saját jellemzőjéhez vagy egy helyzethez kapcsolódik, akkor megkérdezheti a KM‑et, hogy ez megfelel‑e \definition{önkésztetésnek}. Ha a csapat beleegyezik, akkor a KM egy sors pontot fizet a játékosnak.

\textbf{Teljesen normális visszavonni egy késztetést, ha rájövünk, hogy nem ütötte meg a mércét.} Ha a csapat úgy dönt, hogy a kezdeményezett késztetés nem megfelelő, az a célkarakternek nem kerül semmibe se.

\subsubsection{A késztetések komplikációk, nem akadályok}

Ha egy késztetést indítványozol, biztosítsd, hogy a komplikáció akciókhoz vezessem, vagy egy jelentős változás a körülményekben, ne csak leszűkítse a játékosok lehetőségeit.

„Aha, szóval homok ment a szemedbe, így mikor rálősz a szörnyre, elhibázod a lövést”, nem egy késztetés. Ez csak megtilt bizonyos cselekedeteket, de nem komplikál semmit sem.

„Átkozhatod a szerencséd, de a homok a szemedben szerintem azt jelenti, hogy nem igazán látsz semmit sem. A shoggothra irányzott lövésed gellert kap, kilukasztva pár hordót, amikből benzin folyik ki a nyitott kemence felé.” Ez egy sokkal jobb késztetés. Megváltoztatja a jelenetet, fokozza a feszültséget, és újabb kezelendő problémát ad a játékosoknak.

Bővebben olvashatsz arról, hogy mi működik, és mi nem, mint késztetés, a késztetések fajtáiról szóló diskurzusban a \fate{Fate Core} szabálykönyv 72. oldalától, vagy az interneten a \url{https://fate-srd.com/fate-core/invoking-compelling-aspects#types-of-compels} oldalon.

\newpage

\subsubsection{Események és döntések}

Nagy általánosságban kétféle késztetés létezik: \definition{események} és \definition{döntések}.

Az esemény késztetés valami külső hatás miatt történik a karakterrel. Ez a külső hatás valahogyan kapcsolódik a jellemzőhöz, és ez valamilyen szerencsétlen komplikációt idéz elő.

A döntés késztetés ellenben belső, ahol a karakter, valamilyen karakterhibája vagy egymással ellentétes belső értékei miatt, a józan ésszel ellentétesen viselkedik. A jellemző irányítja a karaktert egy bizonyos döntés meghozatalára – és ennek a döntésnek az eredménye a komplikáció.

Minkét esetben a létrejövő komplikáció a lényeg! Komplikáció nélkül nincs késztetés sem.

\subsubsection{Ellenséges kihasználás vagy késztetés?}

Ne téveszd össze az ellenséges kihasználást és a késztetést! Bár hasonlóak – mindkettő egy sors pontot fizet, hogy létrehozzon egy közvetlen problémát – másképpen működnek.

A késztetés egy \emph{változást okoz a narratívában}. A késztetés kezdeményezése nem a játék világában lett eldöntve, hanem a KM vagy egy játékos indítványozza a történet megváltoztatását. A hatás lehet, hogy átfogó, de a célpont rögtön megkapja a sors pontot, és lehetősége van ellenállni a késztetésnek.

Az ellenséges kihasználás ezzel szemben \emph{automatikus hatás}. A célpontnak nincs lehetősége visszautasítani a kihasználást – de mint minden kihasználásnál, meg kell indokolnod, hogyan lehetséges az adott jellemzőt kihasználni az adott szituációban. És bár megkapják a sors pontot, azt nem használhatják fel az adott jelenetben. Ezen felül a végeredmény is sokkal szűkebb: egy +2 bónusz, vagy újradobás.

A késztetés segítségével a KM vagy a játékos megváltoztathatja, hogy \emph{miről} is szól az adott jelenet. Akár teljesen tönkrezúzhatják a cselekményt. Egy ellenség ellen használni rizikós vállalkozás – simán ellenállhatnak, netán a komplikációk ellenére is elérhetik a céljukat, felhasználva a vadonatúj sors pontot, amit éppen nekik adtál.

Az ellenséges kihasználások ellenben az adott pillanatban segíthetnek. A saját jellemzőiden kívül immár az ellenfelét is kihasználhatod, ami bővíti a lehetőségeidet, valamint dinamikusabbá és logikusabbá teheti a jeleneteket.
