\subsection{Újratöltés}

Az \definition{újratöltés} megmondja, hogy a karaktered mennyi \definition{sors ponttal} (\page{24}) kezdi a játéküléseket. Alapállapotban a karaktered újratöltése 3.

A sors pontjaid száma minden játékülés elején legalább az újratöltés értéke lesz. Emiatt mindenképpen jegyezd fel, hogy mennyi sors ponttal fejezed be a játékülést – ha ez több, mint az újratöltésed, akkor a következő játékban annyival fogsz kezdeni.

\example{%
Charles jó sok sors pontot szerzett a játék során, a végére 5 maradt nála. Mivel az ő újratöltése 2, így a következő játékot 5"~el is fogja kezdeni. Ellenben Ethannél csak egyetlen sors pont marad. Az ő újratöltése 3, így a következő játékot 3 sors ponttal fogja kezdeni, nem a megmaradt 1"~el.
}

\subsection{Fortélyok}

Bár minden karakter használhat minden képességet – még ha a többségüknek Középszerű~(+0) is a szintje – \definition{fortélyokkal} egyedivé teheted a karakteredet. A fortélyok lehetnek menő vagy titkos technikák, trükkök vagy felszerelések, amik a karaktereket egyedivé és érdekesebbé teszik. Míg egy képesség a karakter hozzáértésének átfogó leírása, addig egy fortély nagyon szűk területen mutatja a kiválóságát; a többségük csak jól meghatározott szituációkban ad bónuszt, vagy tesz lehetővé cselekedeteket, amiket más karakter egyszerűen tud megtenni.

A karakterednek alapértelmezésben három fortély rubrikája lehet. Nem muszáj rögtön karakterkészítéskor kitalálni őket, elég, ha játék közben töltöd ki. Vásárolhatsz több fortélyt is, feláldozva 1 újratöltést mindegyikért, de az újratöltéseid száma nem mehet 1 alá.
