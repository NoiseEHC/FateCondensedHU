\documentclass[oneside]{book}

\usepackage[magyar]{babel}
\defineshorthand{"~}{\babelhyphen{nobreak}}
\useshorthands{"}

\usepackage{fontspec}
\setmainfont{EB Garamond}
\newfontfamily\FateCoreGlyphs[Mapping=tex-text]{Fate Core Glyphs} % dobókocka, cselekvés ikonok
\newfontfamily\Wingdings[Mapping=tex-text]{Wingdings3} % siker/kudarc ikonok
\newfontfamily\Roboto[Mapping=tex-text]{Roboto Light} % példák (Gotham Condensed helyett)
\newfontfamily\RobotoMedium[Mapping=tex-text]{Roboto Medium} % táblázatok (Gotham Condensed helyett)
\newfontfamily\RobotoCondensed[Mapping=tex-text]{Roboto Condensed Bold} % címek (Gotham Condensed helyett)

\usepackage{geometry}
\geometry{a4paper}

\usepackage{hyperref}
\hypersetup{
    colorlinks=true,
    linkcolor=blue,
    filecolor=magenta,      
    urlcolor=cyan,
}

\usepackage{titlesec}
\usepackage{graphicx}
\usepackage[table]{xcolor}
\usepackage{multicol}
\usepackage{amsmath}
\usepackage{lettrine}

\title{Fate Condensed Magyarul}
\author{NoiseEHC}
\date{}

\usepackage{mdframed}
\newmdenv[
  linecolor=lightgray,
  linewidth=3pt,
  topline=false,
  bottomline=false,
  rightline=false,
  skipabove=\topsep,
  skipbelow=\topsep
]{siderules}

\DeclareTextFontCommand{\dice}{\FateCoreGlyphs}
\DeclareTextFontCommand{\examplefont}{\Roboto}
\DeclareTextFontCommand{\outcome}{\Wingdings}

\newcommand{\bolditalic}[1]{\textbf{\textit{#1}}}
\newcommand{\fate}[1]{\textbf{\textit{#1}}}
\newcommand{\page}[1]{#1.~oldal}
\newcommand{\onpage}[1]{#1.~oldalon}
\newcommand{\aspect}[1]{\textsc{#1}}
\newcommand{\definition}[1]{\textbf{\textsc{#1}}}
\newcommand{\example}[1]{\begin{siderules}\examplefont{#1}\end{siderules}}

\newcommand{\vcenteredinclude}[1]{\begingroup\setbox0=\hbox{\includegraphics[height=12pt]{#1}}\parbox{\wd0}{\box0}\endgroup}

% Ezek a számok itt nem UNICODE értékek, hanem a karakter sorszáma a karakterkészletben!!!
\newcommand{\failure}{\outcome{\XeTeXglyph7}}
\newcommand{\tie}{\outcome{\XeTeXglyph39}}
\newcommand{\success}{\outcome{\XeTeXglyph6}}
\newcommand{\successwithstyle}{\outcome{\XeTeXglyph45}}
\newcommand{\failureitem}{\item[\failure]}
\newcommand{\tieitem}{\item[\tie]}
\newcommand{\successitem}{\item[\success]}
\newcommand{\successwithstyleitem}{\item[\successwithstyle]}

\newcommand\outcomesection[2]{\subsection{#1\vspace*{-12pt}\\ {\small#2}}}
\newcommand\actionsection[2]{\subsection{#1\vspace*{-12pt}\\ {\small#2}}}

\newcommand{\fatetable}[2]{
\begin{center}
\RobotoMedium{
\rowcolors{2}{lightgray!40}{white}
\begin{tabular}{ |#1| }
\hline
\rowcolor{black}
#2
\hline
\end{tabular}
}
\end{center}
}

\begin{document}
%\maketitle

% A "\usepackage[magyar]{babel}" egy csomó mindent átállít a "\begin{document}" után, emiatt a "\titleformat" ide kell kerüljön!!!
\titleformat{\chapter}[hang]{}{}{0pt}{\RobotoCondensed\Huge\MakeUppercase}
\titlespacing{\chapter}{0pt}{0pt}{0pt}[0pt]
\titleformat{\section}[hang]{}{}{0pt}{\RobotoCondensed\huge}
\titlespacing{\section}{0pt}{0pt}{0pt}[0pt]
\titleformat{\subsection}[hang]{}{}{0pt}{\RobotoCondensed\LARGE\MakeUppercase}
\titlespacing{\subsection}{0pt}{0pt}{0pt}[0pt]
\titleformat{\subsubsection}[hang]{}{}{0pt}{\RobotoCondensed\large\MakeUppercase}
\titlespacing{\subsubsection}{0pt}{0pt}{0pt}[0pt]

\chapter{Bevezető}

Ez a \fate{Fate Condensed}, a \fate{Fate Core} rendszer egy verziója, amit olyan kevés oldalba sűrítettünk, amennyire csak tudtuk. Ez egy komplett szerepjáték rendszer; bár egyéb kiadványok kiegészíthetik, nincs más könyvre szükséged a játékhoz.

Ha már itt tartunk, vegyük sorra, hogy mire lesz viszont \emph{tényleg} szükséged!

\section{Mire van szükséged a játékhoz?}

A \fate{Fate Condensed} játékhoz szükséged lesz játékosokra kettő és hat fő között, akikből egy a Kalandmester (KM) szerepét fogja betölteni. Kell még pár kocka, néhány zseton, íróeszközök, papír és még valami a rövid jegyzetekhez (például öntapadó cetlik).

A \fate{Fate Condensed} rendszer \fate{Fate Dobókockákat}™ használ, amikor a karakterek cselekszenek. Ezek hatoldalú kockák, amelyeknek kettő \dice{0}, kettő \dice{+} és kettő \dice{-} oldaluk van. Összesen négy ilyen kocka már elégséges, de a legjobb, ha mindegyik játékos rendelkezik néggyel. Ezen kívül lehet használni standard hat oldalú dobókockát is (1"~2 = \dice{-}, 3"~4 = \dice{0}, 5"~6 = \dice{+}; netán vastagabb filccel át lehet alakítani így:~\vcenteredinclude{d6_to_fudge_v2.png}). Alternatíva még a \fate{Deck of Fate} kártyapakli is, ami kártyalapokat használ dobókockák helyett; ennek ellenére, az egyszerűség kedvéért, a továbbiakban konzekvensen a „dobás” szót használjuk.

\subsection{Veteránoknak: változások a Fate~Core"~hoz képest}

Egy rendszer 300 oldalnyi szövegének kevesebb, mint 60 oldalba tömörítése nyilvánvalóan változtatásokkal jár. Ráadásul, mivel a \fate{Fate Core} már nyolcéves, pár helyen rá is fért a frissítés. Az alábbiakat éreztük fontosnak kiemelni:
\begin{itemize}
    \item A stressz dobozok most már mindig csak egy pontot érnek, ezzel is egyszerűsítve a szabályokat (\page{Stressz dobozok és következmény rubrikák}).
    \item Ahelyett, hogy a képességek döntenék el a cselekvési sorrendet, a „Balsera"~féle” kezdeményezés (ismert még „Válaszható cselekvési sorrend” vagy „Popcorn kezdeményezés” néven is) az alapértelmezett (\page{Cselekvési sorrend}).
    \item A fejlődés egy kicsit másképp működik; nincsenek már döntő mérföldkövek, emiatt a jelentős mérföldkövek (mint áttörések) szabályai ki lettek bővítve ellensúlyozásként (\page{Fejlődés}).
    \item Az aktív ellenállás most már mindig védekezés cselekvésnek minősül (\page{Védekezés}). Ezt a változást átvezettük az összes többi szabályon, a legfontosabb a megold cselekvés eredménye döntetlen esetén (\page{Megoldás}).
    \item A helyzetbehozás világosabban van megfogalmazva, pontosabban leírja az ismeretlen jellemzők felfedezését (\page{Helyzetbehozás}).
    \item A védekező harc opcionális szabály lett, és már magába foglalja a kibővített védekezés cselekvés lehetőségeit (\page{Védekező harc}). Ezen kívül még további opcionális szabályok is belekerültek a könyvbe a \onpage{Opcionális szabályok} kezdődően.
\end{itemize}


\chapter{Alapok}

\section{A világ kidolgozása}

Minden Fate játék a világ kidolgozásával kezdődik. Ez lehet a KM saját agyszüleménye, a játékosok által ismert könyv vagy film világa, netán a jelenlévők együtt építhetik fel nulláról. A résztvevők lehet, hogy éppen csak áttekintik a részleteket, amiket muszáj tudni a játékhoz, de az is megeshet, hogy egy egész játékülést ezek kidolgozására áldoznak. A két véglet között bármi elképzelhető.

Az általatok választott világ adja meg, hogy mit tekint a csapat valóságnak, és mi elfogadható a játékban és a karakterkészítés közben. Ha a világotokban az emberek nem tudnak repülni, akkor egy röpképes karakter koncepciója nem állja meg a helyét. Ha a világotokban háttérhatalmak ügyködnek titkos összeesküvések hálójában, akkor a játékosok olyan történetekre számítanak, amiben nem a jó és gonosz erői csapnak össze, és nincsenek nevetséges gyilkos bohócok sem. Ez mind csak tőletek függ!

\section{Karakterkészítés}

\subsection{Ki vagy valójában?}

Amint a résztvevők döntésre jutottak a világgal kapcsolatban, itt az ideje, hogy a játékosok elkészítsék a Játékos Karaktereket (JK). Mindegyik játékos egy hőst irányít a történetben, meghatározva minden tettüket. Neked kell kidolgoznod a karaktert olyanra, amilyenre szeretnéd. Vedd figyelembe, hogy a Fate karakterek kompetensek, drámaiak, és nem haboznak a kalandokba vetni magukat.

A te JK"~d az alábbi dolgokból épül fel:

\begin{itemize}
    \item \textbf{Jellemzők:} kifejezések, amik leírják, hogy ki is a hősöd
    \item \textbf{Képességek:} megadják, hogy a hősöd miben jobb másoknál
    \item \textbf{Fortélyok:} figyelemre méltó dolgok, amit a hősöd meg tud tenni
    \item \textbf{Stressz:} a hős azon adottsága, hogy nyugodt maradjon, és folytassa küldetését
    \item \textbf{Következmények:} fizikai és szellemi sérülések, amit a hős még elvisel
    \item \textbf{Újratöltés:} ez mutatja a hős narratív hatalmát
    \item \textbf{Végső simítások:} a hős leírásának részletei
\end{itemize}

\subsection{Képességek}

Míg a jellemzők megadják, hogy ki a karakter, addig a \definition{képességek} azt mutatják, hogy mit képes megtenni. Minden képesség tevékenységek széles körét foglalja magába, amit a karakter képzés vagy gyakorlás útján sajátított el, netán csak veleszületett. A karakter a Tolvajlás képességgel, valamilyen szinten, bármilyen bűntényt elkövethet, ami a lopás művészetéhez kapcsolódik -- terepszemle, biztonsági rendszer megkerülése, zsebmetszés, zárnyitás.

Minden képességnek van egy \definition{szintje}. Minél magasabb a szint, annál jobb a karakter abban a dologban. A karaktered képességeinek összessége mutatja, hogy mi az, amire született, mi az, amivel még megbirkózik, és mi az, amit nem kéne erőltetnie.

A karaktered képességeinek szintjeit az alábbi piramis struktúra alapján választhatod ki, ahol a legmagasabb szint a Kimagasló~(+4):

\begin{itemize}
    \item Egy darab Kimagasló~(+4) képesség
    \item Kettő darab Remek~(+3) képesség
    \item Három darab Jó~(+2) képesség
    \item Négy darab Átlagos~(+1) képesség
    \item Minden más képesség szintje Középszerű~(+0)
\end{itemize}

\subsubsection{Képesség szintek skálája}

A \fate{Fate Condensed} játékban, ahogy a normális Fate játékban is, a képességek szintjei az alábbi skálába vannak rendezve:

\begin{center}
\fatetable{r l}{
\textcolor{white}{Szint} & \textcolor{white}{Melléknév} \\
+8 & Legendás (Legendary) \\
+7 & Epikus (Epic) \\
+6 & Fantasztikus (Fantastic) \\
+5 & Emberfeletti (Superb) \\
+4 & Kimagasló (Great) \\
+3 & Remek (Good) \\
+2 & Jó (Fair) \\
+1 & Átlagos (Average) \\
+0 & Középszerű (Mediocre) \\
-1 & Gyenge (Poor) \\
-2 & Borzalmas (Terrible) \\
-3 & Katasztrofális (Catastrophic) \\
-4 & Horrorisztikus (Horrifying) \\
}
\end{center}

\subsubsection{Képesség lista}

A következő képességek leírása alább található.

\begin{multicols}{5}
\setlength{\parindent}{0em}
\textbf{%
Akaraterő \\
Atléta \\
Befolyásolás \\
Célzás \\
Empátia \\
Ezermester \\
Észlelés \\
Fizikum \\
Kapcsolatok \\
Kényszerítés \\
Közelharc \\
Lopózás \\
Megtévesztés \\
Misztikum \\
Nyomozás \\
Tolvajlás \\
Tudomány \\
Vagyon \\
Vezetés
}
\end{multicols}

\textbf{Akaraterő (Will):} Lelkierő, adottság a kísértések leküzdésére, valamint a traumák átvészelésére. Az Akaraterő fortélyai lehetőséget adnak a szellemi következmények figyelmen kívül hagyására, a lelki gyötrődéssel vagy misztikus erőkkel szembeni kitartásra, valamint nem engedni a Kényszerítéssel próbálkozó ellenfeleknek. Ezen felül, a magas Akaraterő képesség szint megnöveli a szellemi stressz dobozok vagy a lehetséges következmény rubrikák számát (\page{Stressz dobozok és következmény rubrikák}).

\textbf{Atléta (Athletics):} A test teljesítőképességének mértéke. Az Atléta fortélyai a mozgásra -- futás, ugrás és akrobatika -- valamint a támadások elleni kitérésekre fókuszálnak.

\textbf{Befolyásolás (Rapport):} Másokkal kontaktust kiépíteni és együttműködni. Míg a Kényszerítés manipuláció, addig a Befolyásolás őszinteség, bizalom és jóakarat. A Befolyásolás fortélyai lehetővé teszik tömegek irányítását, emberi viszonyok javítását vagy emberi viszonyok kiépítését.

\textbf{Célzás (Shoot):} Mindenféle távolsági fegyver ide tartozik, legyen szó lőfegyverekről, dobótőrről vagy íjakról. A Célzás fortélyai lehetővé teszik a testrészre támadást, gyors fegyverrántást, vagy, hogy mindig legyen kéznél egy pisztoly.

\textbf{Empátia (Empathy):} Ez az, amivel felmérhetjük valaki hangulatát és szándékait. Az Empátia fortélyai közé tartozik nagyobb tömeg megítélése, hazugságok felismerése, vagy segíteni másoknak felépülni szellemi következményekből.

\textbf{Ezermester (Crafts):} Gépezetek építése vagy éppen elpusztítása, szerkentyűk összeeszkábálása, valamint MacGyver"~féle találékonyság megvillantása. Az Ezermester fortélyai segítségével a megfelelő kis bigyó pont a karakternél lesz, esetleg bónuszt kaphat építésre és rombolásra, vagy indoklást adhatnak a játékos kezébe, hogy bizonyos szituációkban miért lehet Ezermestert használni Tolvajlás vagy Tudomány helyett.

\textbf{Észlelés (Notice):} Lehetőséget ad, hogy pillanatnyi benyomásokra figyeljünk fel, még az előtt kiszúrjuk a történéseket, hogy bajba kerülnénk, és általánosságban figyelmesek legyünk. Ellentétes a Nyomozással, ami lassú, szándékos megfigyelés. Az Észlelés fortélyai élesítik az érzékeidet, javítják a reakcióidőd, vagy nehezebbé teszik, hogy becserkésszenek.

\textbf{Fizikum (Physique):} Nyers erő és kitartás. A Fizikum fortélyai lehetővé teszik gigászi erő kifejtését, mások uralását birkózás közben, netán fizikai következmények lerázását. Ezen felül a magas Fizikum képesség szint megnöveli a fizikai stressz dobozok vagy a lehetséges következmény rubrikák számát (\page{Stressz dobozok és következmény rubrikák}).

\newpage

\textbf{Kapcsolatok (Contacts):} A megfelelő emberek ismerete, és kapcsolati háló, ami a segítségedre lehet. A Kapcsolatok fortélyai mindenre kész szövetségeseket biztosítanak, valamint besúgó hálózatot, bármerre is vezessen utad a világban.

\textbf{Kényszerítés (Provoke):} Ezzel lehet más embereket rávenni, hogy azt csinálják, amit akarsz. Ez durva és manipulatív, nem egy pozitív interakció. A Kényszerítés fortélyai lehetőséget adnak másokat vakmerő cselekedetekre sarkallni, agressziót magadra vonni vagy elijeszteni az ellenfelet (persze csak, ha tudnak félelmet érezni).

\textbf{Közelharc (Fight):} A Közelharc képesség neve mindent elárul, használható akár fegyverrel, akár puszta  kézzel történik. A Közelharc fortélyai az egyedi fegyverekről és a speciális technikákról szólnak.

\textbf{Lopózás (Stealth):} Láthatatlanul és néma csendben maradni, valamint elszökni a kíváncsi szemek elől, ha rejtőzni kell. A Lopózás fortélyai lehetővé teszik, hogy akkor is el tudj rejtőzni, ha éppen figyelnek, beleolvadj a tömegbe, vagy az árnyékokban osonj észrevétlenül.

\textbf{Megtévesztés (Deceit):} Ez a meggyőzően, hidegvérrel előadott hazugságok és átverések képessége. A Megtévesztés fortélyai bizonyos típusú hazugságokat hihetőbbé tehetnek, netán segíthetnek hamis személyiségek felvételében.

\textbf{Misztikum (Lore):} Speciális, titkos tudás, ami kívül esik a Tudomány hatáskörén, mint például a természetfölötti erők és lények. Ez bizony az őrült történetek tárháza. A Misztikum fortélyai gyakran a titkos tudás gyakorlati alkalmazásáról szólnak, amilyen például a varázslás. Némely világ törölheti ezt a képességet, lecserélheti másikra, vagy összevonhatja a Tudománnyal (Tudás néven).

\textbf{Nyomozás (Investigate):} Szándékos, gondos tanulmányozás és a rejtélyek kibogozása. Ezt használd a nyomok értelmezéséhez vagy egy bűntett rekonstruáláshoz. A Nyomozás fortélyai segítenek briliáns következtetések kigondolásában, netán csak a nyomok értelmezését gyorsítják fel.

\textbf{Tolvajlás (Burglary):} Elméleti és gyakorlati tudás biztonsági rendszerek megkerülésére, zsebmetszésre és általánosságban a bűnözésre. A Tolvajlás fortélyai bónuszokat adnak a bűnözés minden fázisában a tervezéstől a kivitelezésen át a felszívódásig.

\textbf{Tudomány (Academics):} Evilági, mindennapi emberi tudás és tananyag, mint a történelem, fizika vagy orvoslás. A Tudomány fortélyai gyakran szűkebb tudományterületekre vagy gyógyításra fókuszálnak.

\textbf{Vagyon (Resources):} Anyagi javakhoz való hozzáférés, ami nem korlátozódik pénzre vagy birtoklásra.  Ez mutathatja, hogy kölcsön tudsz kérni egy baráttól, netán kipucolhatod egy szervezet fegyvertárát. A Vagyon fortélyaival bizonyos szituációkban használhatod a Vagyon képességet Befolyásolás vagy Kapcsolatok helyett, netán adhatnak néhány ingyen kihasználást, ha a legjobb minőséget veszed meg.

\textbf{Vezetés (Drive):} A járművek feletti uralom megtartása a legkeményebb szituációkban is, eszement manőverek végrehajtása, vagy egyszerűen csak a legtöbbet kihozni a járművedből. A Vezetés fortélyai lehetnek egyedi manőverek, egy speciális jármű birtoklása, vagy indoklást adhatnak a játékos kezébe, hogy bizonyos szituációkban miért lehet Vezetést használni Tolvajlás vagy Tudomány helyett.

\label{Alternatív képesség Listák}
\subsubsection{Alternatív képesség Listák}

A fenti képesség lista csak egy példa a képességek felépítésére, és egyáltalán nem a „hivatalos” képesség lista, amit minden Fate játékban használni kéne. Az is teljesen elfogadható ha használod, de az is, ha figyelmen kívül hagyod. Bármelyik képesség különböző területeket fedhet le a világtól függően. A testre szabhatóság a lényeg; úgy állítsd össze a listát, hogy passzoljon az elképzelt történetekhez.

Ezt észben tartva, ha egy saját Fate játékon dolgozol, az első dolog, amit el kell döntened, hogy megtartsd"~e az alapértelmezett képesség listát, vagy sem. Általában ez elég jó, és csak pár képességet kell összevonni, megváltoztatni vagy felosztani. De akadnak olyan esetek is, amikor az alapértelmezett lista felbontása nem megfelelő. Ilyenkor az alábbi dolgokon célszerű elgondolkodni.

\begin{itemize}
    \item Az alapértelmezett képesség lista 19 elemből áll, és minden karakternek 10 képesség szintje magasabb, mint Középszerű~(+0). Ha megváltoztatod a képességek számát, célszerű a szinteket is máshogyan elosztani a karakterek között.
    \item Az alapértelmezett képesség lista arra válasz, hogy „mit tudsz megtenni?” -- de a listának nem muszáj ilyennek lennie. Lehet, hogy egy olyan lista jobb a játékodba, ami a „Miben hiszel?” kérdéskörrel foglalkozik, vagy a „Hogyan teszel dolgokat?” (mint a megközelítések a \fate{Fate Accelerated} játékban) kérdésre válaszol, esetleg feladatköri leírások egy csapat szélhámos vagy tolvaj esetében, és így tovább.
    \item A képességek szintjei úgy vannak kitalálva, hogy minden karakternek lehessen egyedi szakterülete. Ez az oka annak, hogy a kezdő karakterek „piramis” struktúrában kapják a képességeiket. Akárhogyan is módosítod a listát, ezt mindenképpen meg kell tartanod.
    \item A legjobb kezdő képesség szintjét Kimagasló~(+4) körül célszerű tartani. Ezt módosíthatod felfelé vagy lefelé is, de ilyenkor figyelned kell arra, hogy ez mennyiben módosítja a JK"~k által megdobandó nehézségeket, vagy az ellenfelek képesség szintjeit.
\end{itemize}

\example{%
Fred úgy dönt, hogy egy űropera Fate játékot szeretne csinálni, amiben a képesség lista rövidebb, és cselekedetek igéire fókuszál. Végül egy 9~elemű listában állapodik meg: Harcolni, Tudni, Mozogni, Észlelni, Vezetni, Lopózni, Gondolkodni és Akarni. Mivel jobban tetszik neki a „rombusz” struktúra, mint a piramis, így a kezdő képesség lista a következő: 1~darab Kimagasló~(+4), 2~darab Remek~(+3), 3~darab Jó~(+2), 2~darab Átlagos~(+1), és végül 1~darab Középszerű~(+0). A JK"~knak így elég sok képessége közös lesz, mivel a rombusz közepe elég széles, míg pár dolog az ő kizárólagos szakterületük marad, a rombusz hegyes „csúcsa” miatt.
}

Ha saját képesség listát szeretnél csinálni a játékodhoz, és szükséged van pár ötletre a fantáziád beindításához, lapozz a \page{Képesség lista megváltoztatása}ra.

\subsection{Újratöltés}

Az \definition{újratöltés} megmondja, hogy a karaktered mennyi \definition{sors ponttal} (\page{24}) kezdi a játéküléseket. Alapállapotban a karaktered újratöltése 3.

A sors pontjaid száma minden játékülés elején legalább az újratöltés értéke lesz. Emiatt mindenképpen jegyezd fel, hogy mennyi sors ponttal fejezed be a játékülést -- ha ez több, mint az újratöltésed, akkor a következő játékban annyival fogsz kezdeni.

\example{%
Charles jó sok sors pontot szerzett a játék során, a végére 5 maradt nála. Mivel az ő újratöltése 2, így a következő játékot 5"~el is fogja kezdeni. Ellenben Ethannél csak egyetlen sors pont marad. Az ő újratöltése 3, így a következő játékot 3 sors ponttal fogja kezdeni, nem a megmaradt 1"~el.
}

\subsection{Fortélyok}

Bár minden karakter használhat minden képességet -- még ha a többségüknek Középszerű~(+0) is a szintje -- \definition{fortélyokkal} egyedivé teheted a karakteredet. A fortélyok lehetnek menő vagy titkos technikák, trükkök vagy felszerelések, amik a karaktereket egyedivé és érdekesebbé teszik. Míg egy képesség a karakter hozzáértésének átfogó leírása, addig egy fortély nagyon szűk területen mutatja a kiválóságát; a többségük csak jól meghatározott szituációkban ad bónuszt, vagy tesz lehetővé cselekedeteket, amiket más karakter egyszerűen tud megtenni.

A karakterednek alapértelmezésben három fortély rubrikája lehet. Nem muszáj rögtön karakterkészítéskor kitalálni őket, elég, ha játék közben töltöd ki. Vásárolhatsz több fortélyt is, feláldozva 1 újratöltést mindegyikért, de az újratöltéseid száma nem mehet 1 alá.

\subsubsection{Fortélyok megtervezése}

A fortélyokat te találod ki, a karakter kitalálása közben. Nagy vonalakban a fortélyokat két csoportra lehet osztani.

\textbf{Bónuszt adó fortélyok:} A fortélyok első típusa \textbf{+2 bónuszt ad}, amikor egy meghatározott képességet használsz bizonyos határokon belül, általában csak bizonyos cselekvésekhez (\page{Cselekvések}), és bizonyos narratív körülmények között.

Az ilyen fortélyokat az alábbi módon célszerű megfogalmazni:

\example{%
Mivel \textbf{[írd le, hogy mennyire menő a karakter vagy a felszerelése]}, ha a \textbf{[válassz egy képességet]} képességet használom \textbf{[válassz a megoldásra, helyzetbehozásra, megtámadásra, védekezésre közül]}, +2 bónuszt kapok, de csak ha \textbf{[definiáld a körülményeket]}.
}

\textbf{Bónuszt adó fortély példa:} Mivel én egy \textbf{katonai mesterlövész} vagyok, ha a \textbf{Célzás} képességet használom \textbf{megtámadásra}, +2 bónuszt kapok, de csak \textbf{ha a \aspect{Célpont Bemérve}}.

\textbf{Szabályszegő fortélyok:} A fortélyok második típusa \textbf{megváltoztatja a játék szabályait}. Ez egy átfogó kategória, amibe az alábbiak mind beletartoznak, de kitalálhatsz mást is:

\begin{itemize}
    \item \textbf{Megváltoztatni, hogy melyik képességet kell használni bizonyos szituációkban.} Például egy kutató használhat Tudományt egy rituálé elvégzéséhez, míg mindenki más Misztikumot.
    \item \textbf{Olyan cselekvést végezni egy képességgel, ami normálisan nem lehetséges.} Például, hogy egy karakter hátba szúrjon egy ellenfelet az árnyékokból a Lopózás képességgel, míg normálisan ez megtámadás cselekvés lenne a Közelharc képességgel.
    \item \textbf{Nagyjából +2 értékű, nem számszerű bónuszt adni egy képesség használatához.} Például, ha egy tehetséges szónok Befolyásolással helyzetbe hoz, kap egy extra ingyen kihasználást.
    \item \textbf{Megengedni, hogy a karakter kinyilváníthasson egy kisebb tényt, ami igazzá válik.} Például egy túlélésben képzett karakternél mindig legyen pár túléléshez szükséges eszköz, mint mondjuk gyufa, még akkor is, ha ez nagyon valószínűtlen is a történetben. Ez a fortély fajta lehetővé teszi, hogy a történet részleteinek kinyilvánítása (\page{Történet részleteinek kinyilvánítása}) jellemző kihasználása nélkül is elérhető legyen.
    \item \textbf{Megengedni, hogy a karakter megszeghessen pár szabályt.} Például a karakter kaphat két extra stressz dobozt vagy egy extra enyhe következmény rubrikát.
\end{itemize}

Az ilyen fortélyokat az alábbi módon célszerű megfogalmazni:

\example{%
Mivel \textbf{[írd le, hogy mennyire menő a karakter vagy a felszerelése]}, képes vagyok \textbf{[írj le egy bámulatos tettet]}, de csak \textbf{[definiáld a körülményeket vagy korlátokat]}.
}

\textbf{Szabályszegő fortély példa:} Mivel \textbf{nem hiszek a mágiában}, képes vagyok \textbf{természetfeletti adottságok hatásait figyelmen kívül hagyni}, de csak \textbf{játékülésenként egyszer}.

\subsection{Stressz dobozok és következmény rubrikák}

A \definition{stressz dobozok} és a \definition{következmény rubrikák} adják meg, hogy a karakter mennyire tudja tolerálni a fizikai és szellemi igénybevételt a kalandok során. Minden karakternek legalább 3, egy"~egy pontot érő fizikai stressz doboza, és legalább 3, egy"~egy pontot érő szellemi stressz doboza van. Szintén van egy"~egy enyhe, mérsékelt és súlyos következmény rubrikája.

A Fizikum képesség szint módosítja, hogy a karakternek mennyi fizikai következmény rubrikája van. Az Akaraterő ugyanígy hat a szellemi következmény rubrikák számára.

\begin{center}
\fatetable{l l}{
\textcolor{white}{Fizikum/Akaraterő} & \textcolor{white}{Fizikai/Szellemi stressz} \\
Középszerű~(+0) & \dice{3} \\
Átlagos~(+1) vagy Jó~(+2) & \dice{31} \\
Remek~(+3) vagy Kimagasló~(+4) & \dice{33} \\
Emberfeletti~(+5) vagy magasabb & \begin{tabular}[t]{@{}l@{}}\dice{33}\\és egy extra enyhe következmény rubrika,\\specifikusan csak a fizikai\\vagy csak a szellemi sérülésekre\end{tabular} \\
}
\end{center}

A \textit{„Sérülések elszenvedése”} fejezetben (\page{34}) többet is tanulhatsz a stresszről és következményekről.

\subsubsection{Várjunk csak, nem erre emlékszem!}

A \fate{Fate Condensed} szabályokban minden stressz doboz csak 1 pontot ér. Ellenben a \fate{Fate Core} és a \fate{Fate Accelerated} rendszerekben a stressz dobozok értéke növekvő (1 darab 1 pontos, 1 darab 2 pontos, és így tovább). Használhatod azt a rendszert is, ha úgy tetszik; mi azért váltottunk az 1 pontos dobozokra, mert egyszerűbb -- a másik módszer egy kicsit könnyebben összezavarja az embereket.

Ennek a stílusnak viszont van pár folyománya, amit jó lesz észben tartani.

\begin{itemize}
    \item Ahogy azt majd a \onpage{35} láthatod, az 1 pontos dobozokból annyit ikszelsz be, amennyit csak akarsz, ha találatot kapsz (míg a növekvő értékű dobozokból mindig csak egyet lehetett).
    \item Ehhez a stílushoz különálló fizikai és szellemi stresszmérők szükségesek, ellentétben a \fate{Fate Accelerated} rendszer közös stresszmérőjével. Ha a közös stresszmérő jobban tetszik, akkor adj hozzá három dobozt kiegyenlítésként, és használd a magasabb szintűt a Fizikum és Akaraterő közül, hogy megnöveld a számukat.
    \item Három pontnyi stressz semlegesítés nem túl sok! Ha a karakterek egy kissé törékenynek tűnnének játék közben, nyugodtan adjál hozzá egy"~két dobozt az alapértelmezetthez mennyiséghez. Ez az egész csak annyiról szól, hogy milyen gyorsan váltunk következményekre. (A régebbi rendszerben egy \boxed{1}~\boxed{2} sorozat 2"~3, egy \boxed{1}~\boxed{2}~\boxed{3} sorozat 3"~6, míg egy \boxed{1}~\boxed{2}~\boxed{3}~\boxed{4} sorozat 4"~10 stresszt semlegesített.)
\end{itemize}

\subsection{Végső simítások}

Nevezd el a karaktert, írd le, hogy hogyan néz ki, és beszéld meg a múltját a többi játékossal. Ha még nem írtad le a kapcsolat jellemzőt, akkor itt az ideje megtenni.


\label{Cselekedni}
\chapter{Cselekedni és kockákkal dobni}

A \fate{Fate Condensed} játékban te irányítod a karaktered cselekedeteit, miközben hozzáteszel a közösen mesélt történethez. Általában a KM meséli, hogy a világban mi történik, és hogy a Nem Játékos Karakterek (\definition{NJK}"~k) mit csinálnak, míg a játékosok a saját karaktereik cselekedeteit mondják el.

Bármit szeretnél csinálni, kövesd az \definition{előbb az elképzelés} szabályt: először mondd ki, hogy a karaktered mire készül, és csak \emph{utána} gondolkodj azon, hogy a rendszerben ezt hogyan lehet modellezni. A karaktered jellemzői megmutatják, hogy mivel próbálkozhat egyáltalán, és segítenek megállapítani az eredmény értelmezésének kereteit. A legtöbb ember meg sem próbálkozhat megműteni a kibelezett társát, de ha van egy jellemződ, ami orvosi háttérről árulkodik, akkor te viszont nekiláthatsz. Enélkül a jellemző nélkül a legtöbb, hogy nyerhetsz pár percet a végső szavakhoz. Ha kétségesnek érzed, kérdezd meg a KM"~et és az asztaltársaságot.

Hogyan állapítod meg, hogy eredményes voltál"~e? Gyakran a karaktered egyszerűen csak megteszi, amit akartál, mert a cselekedet nem túl nehéz, és senki sem próbált meggátolni benne. De problémás és kiszámíthatatlan szituációkban muszáj elővenni a kockákat, hogy eldöntsék, mi történik.

Ha egy karakter cselekedne, az alábbiakat kell figyelembe venni:

\begin{itemize}
    \item Mi gátolja abban, hogy egyszerűen csak megtegye?
    \item Mi mehet félre?
    \item Miért érdekes, ha valami félresikerül?
\end{itemize}

Ha senkinek sincs megfelelő válasza az összes kérdésre, akkor a cselekedet egyszerűen megtörténik. Elvezetni a repülőtérre tényleg nem kíván kockadobálást. Viszont az autópályán versenyezni a ránk várakozó repülőig, miközben másik világról származó, kibernetikával turbózott vadállatok üldöznek, tökéletes alkalom a dobásra.

Ha cselekedni kívánsz, kövesd az alábbi lépéseket:

\begin{enumerate}
    \item Előbb az elképzelés: írd le, mit akarsz megtenni, és utána válassz képességet és cselekvést.
    \item Dobj négy kockával.
    \item Add össze a szimbólumokat a kockákon: a \dice{+} értéke +1, a \dice{-} értéke -1, míg a \dice{0} értéke 0. Ez -4~és~+4 közötti eredményt fog adni.
    \item Add a kockák eredményét a képesség szintjéhez.
    \item Módosíthatod a dobást jellemzők kihasználásával (\page{Jellemzők kihasználása} és \page{Kihasználás}) és fortélyok használatával (\page{Fortélyok használata}).
    \item Állapítsd meg a végső eredményt, aminek a neve \definition{erőfeszítés}.
\end{enumerate}

\label{Nehézség és ellenállás}
\section{Nehézség és ellenállás}

Ha a karakter tettét valamilyen probléma akadályozza, vagy pedig egy másik karakter vagy lény helyett csak a világra próbál hatni, akkor a cselekvésének passzív \definition{nehézséget} kell legyűrnie. Ilyen például a zártörés, ajtók eltorlaszolása vagy felmérni az ellenség táborát. A KM változtathat a nehézségen, ha ezt bizonyos jellemzők (legyenek azok a karakteren, a jeleneten vagy máshol) indokolják.

Más esetekben, az ellenfél aktív \definition{ellenállást} tanúsít a karakter cselekvésével szemben egy védekezés cselekvést használva (\page{Védekezés}). Ezekben az esetekben a KM szintén dob, ugyanazokat a lépéseket követve az előző fejezetből, miközben az ellenfél képességeit, fortélyait és jellemzőit használja. Ha megtámadásra vagy az ellenfelet közvetlenül érintő helyzetbehozásra dobsz, az ellenfél védekezésre fog dobni ellene.

Az ellenállásnak számtalan formája létezik. Ha egy szektataggal a rituális áldozótőrért küzdesz, akkor az ellenfél jól definiált. De lehetséges, hogy egy ősi rituálé erejével kell megküzdened, hogy meg tudd menteni a világot. Feltörni egy széfet a First Metropolitan Bank páncéltermében a felfedezés kockázatát hordozza, de az a KM döntése, hogy a járőröző őrség aktív \emph{ellenállása} ellen dobsz, vagy pedig csak a széftől függő passzív \emph{nehézség} ellen.

\section{A dobás módosítása}

Kihasználhatsz egy jellemzőt, hogy +2 bónuszt kapjál a dobásra, vagy pedig újra dobhass. Néhány fortély szintén bónuszokat adhat. A jellemzők kihasználásával egy szövetségesedet is támogathatod (\page{Csapatmunka}), vagy az ellenfél dobásának nehézségét is megnövelheted.

\label{Jellemzők kihasználása}
\subsection{Jellemzők kihasználása}

Ha csinálsz valamit, de a kockák eredménye nem elégséges, nem kell feladnod, és elfogadni a vereséget. (Bár nyilvánvalóan megteheted. Az is lehet mókás.) A használható jellemzők opciókat és lehetőségeket biztosítanak ahhoz, hogy elérd a célod.

Ha meg tudod indokolni, hogy egy jellemző hogyan segíthet a fáradozásaidban, fogalmazd meg, hogyan segít, és költs el egy sors pontot a \definition{kihasználásához} (vagy használj el egy ingyen kihasználást). Hogy mi indokolható meg, és mi nem, annak eldöntésére a \definition{fals szabály} szolgál -- bárki megvétózhatja, ha kijelenti, hogy „Ez fals!”. Egyszerűen fogalmazva, a fals szabály \textbf{kalibrációs technika}, amit minden jelenlévő használhat, ha úgy érzi, hogy a játék eltávolodott a megcélzott víziótól vagy koncepciótól. Hasonló technikákról, amik a biztonságos játékkörnyezet kialakítását célozzák, a \onpage{Biztonsági technikák} olvashatsz.
Ha a kihasználásod falsnak bizonyul, két lehetőséged van. Először is, visszavonhatod a kihasználást, és próbálkozhatsz mással, például egy másik jellemzővel. Másodszor, megvitathatod, hogy a jellemző miért is megfelelő. Ha ez nem győzi meg a másikat, akkor vond vissza a kihasználást, és nyugodj bele. Ha magáévá teszi a perspektívádat, akkor viszont csinálhatod a kihasználást, ahogy tervezted. A fals szabály arra szolgál, hogy az asztal mellett töltött idő kellemesen teljen. Akkor használd, ha valami nem hangzik túl jól, nincs sok értelme, vagy nem illik a megcélzott hangulatba. Ha valaki kihasználná a \aspect{Nagyszerű Első Benyomás} képességét, hogy eldobjon egy autót, az valószínűleg fals. De lehetséges, hogy van a karakternek valamilyen természetfeletti jellemzője, ami elképzelhetetlenül erőssé teszi, elég erőssé, hogy eldobjon egy autót, és ez a belépő egy rettentő szörnyeteg elleni harcban. Ez esetben a \aspect{Nagyszerű Első Benyomás} teljes mértékben hihető.

Amikor kihasználsz egy jellemzőt, vagy \textbf{+2 bónuszt kapsz} az erőfeszítésre, vagy \textbf{újradobhatod mind a négy kockát}, vagy pedig \textbf{2"~vel emelheted a nehézséget} valaki más dobásánál, ha ez megmagyarázható. Több jellemzőt is kihasználhatsz ugyanahhoz a dobáshoz, de nem lehet kihasználni ugyanazt a jellemzőt többször ugyanahhoz a dobáshoz. Az egyetlen kivétel: tetszőleges számú \emph{ingyen kihasználásod} elhasználhatod ugyanazzal a jellemzővel ugyanahhoz a dobáshoz.

Kihasználni legtöbbször a saját karaktered jellemzőit fogod. De kihasználhatsz helyzet jellemzőket is, sőt, lehetőség van más karakter jellemzőinek ellenséges kihasználására is (\page{Ellenséges kihasználás}).

\label{Fortélyok használata}
\subsection{Fortélyok használata}

A fortélyok bónuszt adhatnak a dobásra, feltéve, hogy teljesíted a fortély leírásában megfogalmazott feltételt, például a megfelelő körülményeket, a használt cselekvést vagy képességet. Használhatod a helyzetbehozás cselekvést is (\page{Helyzetbehozás}, hogy olyan jellemzőket hozzál létre, amik megteremtik a fortélyhoz szükséges körülményeket. Végül, vedd figyelembe a fortély szükséges körülményeit a karaktered cselekedeteinek leírásakor is, hogy ennyivel is könnyebb legyen a feltételeket teljesítened.

Normális esetben a fortély +2 bónuszt ad, egy bizonyos, szűken definiált szituációban, bármiféle költség nélkül; bármikor használhatod, amikor csak lehetséges. Ellenben, néhány ritka és kivételesen hatalmas fortélyhoz lehet, hogy el kell költened egy sors pontot a használathoz.

\label{Kimenetelek}
\section[Kimenetelek]{Kimenetelek (Outcome)}

Minden dobás után az erőfeszítés, valamint a passzív nehézség vagy aktív ellenállás közötti különbséget a \definition{sikerességnek} hívjuk. Négyféle lehetséges kimenetel létezik:

\begin{itemize}
    \failureitem Ha az erőfeszítés kisebb, mint a megcélzott nehézség vagy ellenállás, akkor \textbf{kudarc}.
    \tieitem Ha az erőfeszítés egyenlő a céllal, akkor \textbf{döntetlen}.
    \successitem Ha az erőfeszítés egy vagy kettő sikerességgel nagyobb, mint a cél, akkor \textbf{siker}.
    \successwithstyleitem Ha az erőfeszítés legalább három sikerességgel nagyobb, mint a cél, akkor \textbf{átütő siker}.
\end{itemize}

Némelyik kimenetel nyilvánvalóan jobban esik, mint mások, de valamilyen érdekes módon mindegyiküknek előre kell mozdítani a történetet. Az előbb az elképzelés szabállyal kezdted (\page{Cselekedni}), hát fejezd is be vele, hogy a történet maradjon a játék fókusza, és hogy a kimenetel értelmezése megfeleljen az elképzelésnek.

\example{%
Ethan nem valami ügyes mackós (bár megvannak hozzá a szerszámai), de mégis ott találjuk a vészjósló kultusz titkos főhadiszállásán, és már csak egy páncélajtó választja el a keresett rituálé kötettől. Át tud"~e jutni rajta?
}

\newpage

\outcomesection{Kudarc}{Kudarc (Failure)}{Ha az erőfeszítés kisebb, mint a megcélzott nehézség vagy ellenállás, akkor kudarcot vallasz.}

Ezt többféleképpen is ki lehet játszani: egyszerűen kudarc, siker komoly áron, vagy elszenvedni egy találatot.

\subsubsection{Egyszerű kudarc}

Az elsőt a legegyszerűbb megérteni -- \definition{egyszerűen kudarc}. Nem éred el a célod, nem haladsz előre, nem kapsz semmit sem. Ez viszont mindenképpen mozdítsa előre a történetet -- ha a széfet egyszerűen csak nem tudod kinyitni, az sehová sem vezet, és unalmas is.

\example{%
Ethan diadalmasan meghúzta a kart, de a széf határozottan zárva maradt, ellenben a szirénák felharsantak. A kudarc megváltoztatta a szituációt, és előremozdította a történetet -- az őrök már a helyszínre tartanak. Ethannek egy újabb dilemmával kell szembenéznie -- most, hogy a finom módszerek sikertelennek bizonyultak, próbálja más módon kinyitni a széfet, vagy hagyja a fenébe, és tűnés?
}

\label{Siker komoly áron}
\subsubsection{Siker komoly áron}

A második a \definition{siker komoly áron}. Megteszed, amit akartál, de komoly árat kell fizetni érte -- a szituáció rosszabbodott, de legalábbis komplikációk léptek fel. Mint KM, egyszerűen kijelentheted, hogy ez lett a kimenetel, vagy pedig felajánlhatod a játékosnak, mint opciót a kudarc helyett. Mindkét variáns jó és hasznos lehet a megfelelő szituációban.

\example{%
Ethan elrontja a dobását, és a KM így szól: „Hallod, ahogy az utolsó retesz a helyére kattan. Ez a hang szinte visszhangzik egy felhúzott revolver kattanásában, ahogy az őr megkér, hogy fel a kezekkel.” A komoly ár itt a konfrontáció az őrrel, amit a karaktere megpróbált elkerülni.
}

\subsubsection{Elszenvedni egy találatot}

Végül, \definition{elszenvedhetsz egy találatot}, amit vagy stressz dobozokkal vagy következményekkel kell semlegesítened, vagy valami más hátrányban részesülhetsz. Ilyen kudarc a leggyakrabban megtámadás elleni védekezés közben fordul elő, vagy ha a karakter megoldana veszélyes problémákat. Annyiban különbözik az egyszerű kudarctól, hogy egyedül a karaktert érinti, nem az egész csapatot. A siker komoly áron annyiban más, hogy itt a siker nem feltétlenül lehetséges.

\example{%
Ethan kinyitja a széfet, de ahogy megfogja a fogantyút, egy szúrást érez a kézfején. Nem hatástalanította a csapdát! Beírja, hogy Megmérgezett az enyhe következmény rubrikába.
}

Ezeket a lehetőségeket keverni is lehet. Az ártalmas kudarc kicsit durvának tűnhet, de lehet, hogy a legjobb választás abban a pillanatban. Vagy a siker találat árán is egy opció.

\newpage

\outcomesection{Döntetlen}{Döntetlen (Tie)}{Ha az erőfeszítés egyenlő a megcélzott nehézséggel vagy ellenállással, az döntetlen.}

A kudarchoz hasonlóan a döntetlennek is előre kell mozdítania a történetet, sosem akadályozhatja az akciót. Muszáj valami érdekesnek történnie. Ugyanúgy, ahogy a kudarc esetén, ezt is többféleképpen ki lehet játszani: siker kisebb áron vagy részleges siker.

\label{Siker kisebb áron}
\subsubsection{Siker kisebb áron}

Az első a \definition{siker kisebb áron} -- pár doboznyi stressz, valamiféle kényelmetlenség vagy hátráltatás a történetben, amik azért magukban nem akadályoznak, netán egy előny jellemző (\page{Előny jellemzők}) az ellenfélnek mind"~mind kisebb árnak számítanak.

\example{%
Ethan kezdeti próbálkozásai mind hibásnak bizonyulnak. Mire kinyitja a széfet, már hajnal hasadt, így szóba se jöhet, hogy az éj leple alatt tűnjön el. Megszerezte, amiért jött, de a szituáció már kellemetlenebb.
}

\subsubsection{Részleges siker}

A döntetlen kezelésének másik módja a \definition{részleges siker} -- ez siker, de csak egy részét tudtad elérni a célodnak.

\example{%
Ethan éppen csak résnyire tudja kinyitni a páncélajtót -- ha csak egy ujjnyival megmozdítaná, a riasztás megszólalna, és fogalma sincs, hogyan tudná hatástalanítani. Pár oldalnyit a rituáléból ki tud húzni a résen át, de csak találgathat a befejező lépésekről.
}

\outcomesection{Siker}{Siker (Success)}{Ha az erőfeszítés eggyel vagy kettővel nagyobb a célnál, az siker.}

Megszerzed, amit akarsz, bármiféle extra fizetség nélkül.

\example{%
Kinyílt! Ethan felmarkolja a rituálé szövegét, és elillan, mielőtt az őrök felfedezhetnék.
}

\subsubsection{Az „előbb az elképzelés” hatása a sikerre}

Az elképzelés \emph{mondja meg}, hogy pontosan hogyan is néz ki a siker. Mi van, ha Ethannek nincsenek szerszámai, vagy nem elég képzett, hogy feltörje a széfet? Ilyenkor lehetséges, hogy a siker inkább „kisebb áron” fog történni. Hasonlóan, ha Ethan azért volt a betörő csapatban, mert ő \emph{építette} a széfet, akkor a siker eléggé „átütő” lesz.

\outcomesection{Átütő siker}{Átütő siker (Success With Style)}{Ha az erőfeszítés legalább hárommal nagyobb a célnál, az átütő siker.}

Megszerzed, amit akarsz, és még egy kicsit többet is.

\example{%
Ethan több mint szerencsés; a széf ajtaja szinte rögtön kinyílik. Nem csak, hogy megszerzi a rituálé szövegét, de elég ideje van átfutni a széfben lévő többi dokumentumot is. Mindenféle váltók és szerződések között rábukkan az Akeley kastély tervrajzára.
}

\label{Cselekvések}
\section[Cselekvések]{Cselekvések (Actions)}

Négyféle cselekvés van, amire dobhatsz, mindegyik egyedi céllal rendelkezik, és egyedi a hatása is a történetre:

\begin{itemize}
    \item \textbf{Megold} egy problémát, hogy leküzdjön egy akadályt a képességeivel.
    \item \textbf{Helyzetbehozás}, hogy előnyösebbé változtassa a körülményeket.
    \item \textbf{Megtámadás}, ami az ellenfélnek kárt okoz.
    \item \textbf{Védekezés}, hogy megússzon egy megtámadást, megállítsa az ellenfelek helyzetbehozását, vagy szembeszegüljön egy probléma megoldásának.
\end{itemize}

\label{Megoldás}
\actionsection{O}{Megoldás}{Megoldás (Overcome)}{Megold egy problémát, hogy legyőzzön egy akadályt a képességeivel.}

Minden karakter számtalan kihívással szembesül a történetben. A \definition{megold} cselekvéssel tudnak szembeszegülni, és legyőzni ezeket az akadályokat.

Egy karakter, akinek magas az Atléta képessége, fel tud mászni a falakon, vagy a zsúfolt utcákon rohanni. Egy detektív magas Nyomozással olyan nyomokat is értelmezhet, ami felett mások átsiklottak. Ha valaki képzett Befolyásolásban, sokkal könnyebben elkerülheti a verekedést egy ellenséges bárban.

A megoldás kimenetelei:

\begin{itemize}
    \failureitem \textbf{Ha kudarc,} beszéld meg a KM"~el (és a védekező játékossal, ha van ilyen), hogy ez egyszerű kudarc, netán siker komoly áron (\page{Siker komoly áron}).
    \tieitem \textbf{Ha döntetlen,} akkor ez egy siker kisebb áron (\page{Siker kisebb áron}) -- kellemetlen szituációba kerülsz, az ellenfél kap egy előny jellemzőt (\page{Előny jellemzők}), vagy elszenvedsz egy találatot. Alternatíva, ha kudarcot vallasz, de kapsz egy előnyt.
    \successitem \textbf{Ha siker,} akkor eléred a célod, és a történet megy tovább, bármiféle bukkanó nélkül.
    \successwithstyleitem \textbf{Ha átütő siker,} akkor nem csak, hogy siker, de kapsz egy előnyt is.
\end{itemize}

\example{%
Charles eljutott az Antarktiszra a kutató létesítményhez. Az épületek romokban, és a lakók sehol. Át akarja kutatni a törmeléket nyomok reményében. A KM azt mondja, hogy dobjon Nyomozásra Jó~(+2) nehézség ellen. Charles dobása \dice{00++} a kockákon, plusz a Nyomozása Átlagos~(+1), az erőfeszítés így Remek~(+3). Ez siker! A KM leírja a nyomokat: lábnyomok a hóban, olyan lényeké, amik sok vékony, nem emberi lábakon lépkednek.
}

Gyakran használjuk a megold cselekvést, ha el kell dönteni, hogy a karakter hozzáfér"~e, vagy észrevesz"~e egy bizonyos tényt vagy nyomot. Ha így van, tartsd észben, hogy létezik a siker valamilyen áron lehetőség is. Ha megakasztaná a történetet, hogy ha egy részlet felett elsiklanának a játékosok, akkor legjobb ezt a lehetőséget teljesen kizárni, és csak az árra fókuszálni.

\newpage

\label{Helyzetbehozás}
\actionsection{C}{Helyzetbehozás}{Helyzetbehozás (Create An Advantage)}{Létrehoz egy új helyzet jellemzőt, vagy a maga javára fordít egy már meglévő jellemzőt.}

A \definition{helyzetbehozás} cselekvéssel a történet folyását tudod megváltoztatni. A képességeiddel új jellemzőket behozva, vagy meglévő jellemzőkhöz kihasználásokat adva, magad és a csapattársaid felé billentheted a mérleg nyelvét. Megváltoztathatod a körülményeket (eltorlaszolni egy ajtót vagy előkészülni egy tervvel, aminek a végrehajtására jár ingyen kihasználás), új információkat fedezhetsz fel (kikutatni egy szörny gyenge pontját), vagy felhasználhatsz valami köztudottat (például a CEO kedvenc whisky márkáját).

A helyzetbehozással létrehozott jellemző ugyanúgy működik, mint bármelyik másik: lefekteti a történet körülményeit, ami lehetővé teheti, megakadályozhatja, vagy csak gátolhatja bizonyos cselekedetek végrehajtását -- például nem lehet egy varázslatot elolvasni, ha a szoba \aspect{Töksötét}. Ugyanúgy lehet kihasználni (\page{Kihasználás}), és ugyanúgy késztethet (\page{Késztetés}). Ezen felül egy helyzet jellemző létrehozásakor kapsz hozzá egy vagy több \definition{ingyen kihasználást} is. Az ingyen kihasználás, ahogy azt a neve is mutatja, ingyenes, nem kell érte sors pontot fizetni. Akár még a szövetségeseidnek is átengedheted az ingyen kihasználást.

Ha helyzetbehozásra dobsz, el kell döntened, hogy új jellemzőt hozol létre, vagy egy már meglévőnek szeretnéd hasznát venni. Ha az előbbi, akkor ezt a jellemzőt egy szövetségeshez, ellenséghez vagy a környezethez szeretnéd csatolni? Ha egy ellenségre, akkor az védekezés cselekvést tehet ellene. Egyéb esetekben általában egy nehézség ellen kell ilyenkor dobni, de a KM dönthet úgy is, hogy valaki megpróbál meggátolni, és akkor az védekezés dobás lesz.

A kimenetelek új jellemző létrehozásánál:

\begin{itemize}
    \failureitem \textbf{Ha kudarc,} akkor nem tudod létrehozni a jellemzőt (egyszerű kudarc), vagy létrejön, de az ellenfél kapja az ingyen kihasználást (siker komoly áron). Ez utóbbi esetben a jellemzőt lehet, hogy át kell fogalmazni, hogy az ellenfelet segítse. Még így is megérheti, mert a jellemző mindig igaz (\page{A jellemző mindig igaz}).
    \tieitem \textbf{Ha döntetlen,} akkor nem tudod létrehozni a jellemzőt, de helyette kapsz egy előnyt (\page{Előny jellemzők}).
    \successitem \textbf{Ha siker,} akkor létrehozod a jellemzőt, és kapsz egy ingyen kihasználást.
    \successwithstyleitem \textbf{Ha átütő siker,} akkor létrehozod a jellemzőt, és kapsz \emph{két} ingyen kihasználást.
\end{itemize}

A kimenetelek meglévő, ismert vagy ismeretlen jellemző esetén:

\begin{itemize}
    \failureitem \textbf{Ha kudarc,} és a jellemző ismert volt, akkor az ellenfél kap egy ingyen kihasználást a jellemzőhöz. Ha ismeretlen volt, akkor felfedhetik egy ingyen kihasználásért cserébe.
    \tieitem \textbf{Ha döntetlen,} kapsz egy előnyt, ha a jellemző ismeretlen volt; és ismeretlen is marad. Ha a jellemző ismert volt, akkor ehelyett kapsz egy ingyen kihasználást hozzá.
    \successitem \textbf{Ha siker,} akkor kapsz egy ingyen kihasználást a jellemzőhöz, felfedve, ha addig ismeretlen volt.
    \successwithstyleitem \textbf{Ha átütő siker,} két ingyen kihasználást kapsz a jellemzőhöz, felfedve, ha addig ismeretlen volt.
\end{itemize}

\newpage

\example{%
Ethan szembekerül egy shoggothal, ami egy masszív, fáradhatatlan húshegy. Tisztában van vele, hogy ez túl hatalmas szörny egy direkt megtámadáshoz, ezért úgy dönt, hogy inkább elvonja a figyelmét: „Készítek egy Molotov koktélt, és lángba borítom” -- mondja.
\newline
A KM úgy dönt, hogy a shoggothot triviális eltalálni, ezért ez egy Ezermester dobás lesz -- milyen gyorsan tud találni valami gyúlékonyat, amit fegyverré alakíthat. A nehézség Remek~(+3). Ethan Ezermester képessége Átlagos~(+1), a dobása \dice{0+++}, így az erőfeszítés Kimagasló~(+4).
\newline
Ethan összeüt egy Molotov koktélt, és a szörnyre hajítja. A shoggoth így \aspect{Lángol}, és Ethannek van egy ingyen kihasználása ehhez a jellemzőhöz. Ez láthatóan elvonta a shoggoth figyelmét, így ha üldözni kezdi, Ethan kihasználhatja a jellemzőt, hogy egérutat nyerjen.
}

\label{Megtámadás}
\actionsection{A}{Megtámadás}{Megtámadás (Attack)}{Megtámadja az ellenfelet, hogy kárt okozzon neki.}

A \definition{megtámadás} cselekvéssel lehet megpróbálni kiejteni az ellenfelet -- legyen szó egy undorító szörnyeteg levágásáról, vagy csak egy ártatlan őr kupán vágásáról, aki még csak azt sem tudja, hogy mit őriz. Egy megtámadás lehet egy géppuska ellenfélre ürítése, egy kemény jobb egyenes, netán egy káros varázslat.

Vedd figyelembe, hogy egyáltalán lehetséges"~e sérülést okozni a célpontnak. Nem minden megtámadás hasonló nagyságrendű, egy kaiju valószínűleg meg sem érezné, ha behúznál neki egyet. Döntsd el, hogy a megtámadásnak egyáltalán van"~e reális esélye a károkozásra, mielőtt nekiállnál dobni. Vannak olyan hatalmas lények, amiknek valamilyen gyengeségét kell kiaknázni, vagy a védelmüket kell valahogy megkerülni, mielőtt bármiféle kárt tudnál tenni bennük.

A kimenetelek megtámadás esetén:

\begin{itemize}
    \failureitem \textbf{Ha kudarc,} akkor nem tudod eltalálni -- a támadást kivédték, elmozdultak előle, vagy csak felfogta a páncél.
    \tieitem \textbf{Ha döntetlen,} akkor éppen csak súrolod, vagy talán az ellenfél meghátrált. Akárhogyan is, de kapsz egy előnyt (\page{Előny jellemzők}).
    \successitem \textbf{Ha siker,} akkor a sikerességgel megegyező nagyságú sérülést okozol. A védekező félnek ezt kell stressz dobozokkal vagy következmény jellemzőkkel semlegesítenie, vagy kiejtődik (\page{Kiejtés}).
    \successwithstyleitem \textbf{Ha átütő siker,} akkor ugyanúgy sérülést okozol, mint siker esetében, de választhatod, hogy eggyel kisebb sikerességért, egy extra előnyt is kapsz.
\end{itemize}

\example{%
Ruth belebotlik egy csontvázba, amit misztikus erők állítottak valami sötét dolog szolgálatába. Úgy dönt, hogy orrba vágja. A Közelharc képessége Kimagasló~(+4), de a dobása \dice{-{}-00}, így az erőfeszítés csak Jó~(+2).
}

\newpage

\label{Védekezés}
\actionsection{D}{Védekezés}{Védekezés (Defend)}{Védekezik, hogy megússzon egy megtámadást, vagy meggátolja valaki más cselekvését.}

Egy szörny próbál szétmarcangolni? Egy ellenfél próbál félretaszítani, a haragod elől menekülve? Mit teszel, ha egy szektatag próbálja mindkét veséd szétszurkálni? \definition{Védekezés}, védekezés, védekezés.

A védekezés az egyetlen reakció cselekvés a \fate{Fate Condensed} játékban. A védekezéssel meggátolhatsz valamit, még ha nem is te jössz, így passzív nehézség helyett, gyakran aktív ellenállás dobás lesz. Az ellenséged dob, és rögtön ezután lehetőséged van dobni, ha te vagy a célpont, vagy pedig indokolható, hogy miként tudsz közbeavatkozni (ami gyakran téged tesz célponttá). Néhány jellemző vagy fortély adhat is indoklást.

A kimenetelek védekezés esetén:

\begin{itemize}
    \failureitem \textbf{Ha kudarc} megtámadás ellen, akkor elszenveded a találatot, amit köteles vagy stressz dobozokkal vagy következmény jellemzőkkel semlegesíteni (\page{Stressz}). Más esetben az ellenfeled sikeres abban, amit eltervezett.
    \tieitem \textbf{Ha döntetlen,} az ellenfeled cselekvésének a döntetlen kimenetele adja meg, mit kell tenni.
    \successitem \textbf{Ha siker,} akkor nem szenveded el a találatot, vagy meggátolod az ellenfelet a tervében.
    \successwithstyleitem \textbf{Ha átütő siker,} akkor nem szenveded el a találatot, vagy meggátolod az ellenfelet a tervében, és egy előnyt is kapsz, mert pillanatnyilag felülkerekedtél rajta.
\end{itemize}

\example{%
Az előbbi példát folytatva, a csontváz védekezhet Ruth ellen. A KM dobása \dice{-00+}, ami nem módosítja a lény Középszerű~(+0) Atléta képességét.
\newline
Mivel Ruth erőfeszítése a magasabb, a támadása siker. A sikeresség kettő, és a csontváz egy kicsit közelebb került ahhoz, hogy ne keljen fel megint. Ha a csontváz jobbat dobott volna, akkor a védekezése siker lehetett volna, és az élőhalott szörnyűség elkerülhette volna a találatot.
}

\subsubsection{Mely képességekkel lehet megtámadni és védekezni?}

Az alapértelmezett képesség lista az alábbi elveket követi:

\begin{itemize}
    \item Közelharc és Célzás alkalmas fizikai megtámadásra.
    \item Atlétával bármilyen fizikai megtámadás ellen lehet védekezni.
    \item Közelharccal csak Közelharci megtámadás ellen lehet védekezni.
    \item Kényszerítés alkalmas szellemi megtámadásra.
    \item Akaraterővel lehet szellemi megtámadás ellen védekezni.
\end{itemize}

Speciális körülmények között, más képességeket is lehet megtámadásra vagy védekezésre használni, ha a KM engedélyezi, vagy konszenzus alakul ki a játékosok között. Némelyik fortély adhat szélesebb körű, garantált engedélyt akkor is, amikor a körülmények nem lennének alkalmasak. Ha egy képesség nem használható direkt megtámadásra vagy védekezésre, de hasznos lehet a szituációban, akkor használd a helyzetbehozás cselekvést, és használd el az ingyen kihasználásokat a következő megtámadás vagy védekezés dobásodhoz.


\chapter{Jellemzők és sors pontok}
A jellemző az egy szó, kifejezés vagy mondás, ami valami egyedit mond egy személyről, helyről, tárgyról, szituációról vagy csoportról. Majdnem minden elképzelhető dolognak lehet jellemzőket adni. Egy embernek lehet a hírneve, hogy ő a Legjobb Céllövő a Vadnyugaton (később többet is megtudhatsz hasonló jellemzőkről). Egy szoba lehet Lángokban, miután fellöktél egy olajlámpást. Egy szörnnyel történt összecsapás után lehetsz Rémült. A jellemzőkkel terelheted a történetet olyan irányba, ahol jobban illeszkedik a karaktered hajlamaihoz, képességeihez vagy problémáihoz.
A jellemző mindig igaz
A jellemzőket kihasználhatod a dobásodhoz bónuszért, és késztethetnek komplikációt generálva. De a jellemzők még akkor is hatnak a narratívára, ha ez a két lehetőség nem játszik. Ha egy hús‑fémvázon szörny Beszorult a Hidraulikus Présbe, akkor az úgy is van. A szörny nem túl sok mindent tehet beragadva, és elég bajosan tud csak kimászni onnan.
Leegyszerűsítve, a „jellemző mindig igaz” azt jelenti, hogy a jellemzők adhatnak vagy tilthatnak lehetőségeket, hogy mi eshet meg a történetben (és szintén módosíthatják a nehézséget, lásd a 42. oldalon). Ha az előbb említett szörny Beszorult, akkor KM‑nek (és mindenki másnak is) muszáj ezt figyelembe vennie. A lény elvesztette a jogát a mozgásra, amíg valami olyan nem történik, ami visszaadja azt. Vagy egy sikeres megoldás (amihez bizonyos esetekben akár Gigászi Erő szükséges), vagy valakinek botor módon meg kell nyomnia a kiengedés gombot. Hasonlóképpen, Kibernetikusan Erősített Lábak jellemzővel könnyedén átugorhatsz falakat egy nekirugaszkodásból anélkül, hogy dobni kéne rá.
Ez persze nem jelenti azt, hogy bármilyen jellemzőt létrehozhatsz, és az igazságát használva törtess a történeten keresztül. Az rendben van, hogy a jellemzők nagy hatalmat adnak a történet irányításában, de ezzel együtt jár, hogy a játéknak a történet korlátain belül kell maradnia. A jellemzőknek egybe kell esnie az asztal körül ülők elvárásaival, hogy mi az, ami még elfogadható. Ha egy jellemző bűzlik, akkor muszáj átfogalmazni.
Persze, tök jó lenne a Feldarabolt jellemzővel legyőzni a génmanipulált szuper‑katonát, de ez egyrészt feleslegessé tenné a megtámadás cselekvést, másrészt valószínűleg sokkal több munkát igényel levágni a karját (de következményként működhet, lásd a következő oldalon). Szintén kijelentheted, hogy te vagy a Világ Legjobb Céllövője, de ezt azért alá kéne támasztani pár képességgel. És bármennyire is szeretnél Golyóálló lenni, ami megtiltaná mindenkinek, hogy kézifegyverekkel kárt okozzanak benned, de ennek elfogadása elég valószínűtlen, hacsak nincsenek Szuperképesség jellemzők a játékban.
Milyen jellemző típusok léteznek?
Számtalan jellemző variáns létezik, de a nevüktől függetlenül nagyjából ugyanúgy működnek. A legnagyobb különbség, hogy mennyi ideig maradnak aktívak, mielőtt eltűnnének.
Karakter jellemzők
Ezek a jellemzők a karakterlapodon vannak, mint például a koncepció vagy árnyoldal. Ezek adják meg a személyiségjegyeket, fontos részleteket a múltadból, másokhoz való viszonyodat, fontos tárgyak vagy címek birtoklását, problémákat, amikkel foglalkoznod kell, vagy célokat, amik eléréséért teszel, valamint az elért hírneved és kötelezettségeidet. Ezek a jellemzők normálisan a mérföldkövek elérésekor változhatnak.
Példák: A Túlélők Csapatának Vezetője; Figyelmes a Részletekre; Meg Kell Védenem az Öcsémet
Helyzet jellemzők
Ezek a jellemzők leírják a környezetet vagy a világot, amiben a cselekmény zajlik. A helyzet jellemzők általában eltűnnek a jelenet végén, vagy ha valaki olyan cselekvést tesz, ami megváltoztatná vagy megszüntetné. Gyakorlatilag csak addig léteznek, amíg az általuk leírt körülmény fennáll.
Példák: Lángol; Ragyogó Napsütés; Dühös Tömeg; Földre Döntött; Rendőrség Üldözi
Következmény jellemzők
Ezek a jellemzők sérüléseket, vagy hosszan tartó megrázkódtatásokat reprezentálnak, amikre gyakran a megtámadásokból származó találatok semlegesítésével lehet szert tenni.
Példák: Kificamodott Boka; Agyrázkódás; Bénító Önbizalomhiány
Előny jellemzők
Az előny egy speciális jellemző, ami egy rendkívül átmeneti, vagy kis hatású körülményt reprezentál. Az előny jellemző nem késztet, és nem lehet sors pont elköltésével kihasználni sem. Egyetlen egyszer ingyen kihasználhatod, ami után megszűnik. A fel nem használt előny jellemző eltűnik, amint a reprezentált fölény már nem áll fenn, ami tarthat pár másodpercig, netán egyetlen cselekvés végéig. Soha nem tart tovább, mint a jelenet vége, és várhatsz az elnevezésével, amíg el nem használod. Ha te birtoklod az előny jellemzőt, azt átadhatod a szövetségesednek, ha ez megindokolható.
Példák: Célpont Bemérve; Zavarodott; Egyensúlyvesztés
Mit tehetek jellemzőkkel?
Sors pontok szerzése
A sors pontok szerzésének egyik módja, ha engeded a karakter jellemzőinek késztetését, hogy komplikálja a szituációt, vagy csak nehezebbé tegye az életed. Szintén szerezhetsz sors pontokat, ha valaki ellenséges kihasználást alkalmaz a te jellemződre, vagy ha a Megadást választod.
Ne felejtsd el, hogy minden játékülést legalább annyi sors ponttal kezdesz, amennyi az újratöltésed. Ha több késztetést éltél át, mint amennyi kihasználást tettél az előző játékülésen, akkor a következőben nyilván több sors ponttal fogsz kezdeni.
Kihasználás
Ha szeretnéd élvezni a jellemzők által biztosított hatalmat, akkor el kell költened egy sors pontot, hogy kihasználd őket egy dobáshoz. A sors pontjaidat nyilvántarthatod fémpénzekkel, üveggyöngyökkel, pókerzsetonokkal vagy hasonló jelképekkel.
A jellemzők ingyenesen is kihasználhatók, ha van ingyen kihasználásod hozzájuk, amit a te vagy egy szövetséges által kreált helyzetbehozás cselekvés biztosít.
A szókihagyásos trükk
Ha szeretnéd a jellemzőket könnyebb beilleszteni a dobásokba, akkor próbáld a cselekedeted szókihagyással („…”) a végén leírni, és aztán befejezheted a kihasználandó jellemzővel. Valahogy így:
Ryan így szól: „Szóval megpróbálom megfejteni a rúnákat, és …” (dob a kockákkal, de nem tetszik neki az eredmény) „… és Ha Nem Is Jártam Ott, Biztosan Olvastam Róla …” (elkölt egy sors pontot) „… így rögtön elkezdem lökni a rizsát az eredetéről.”
Ellenséges kihasználás
A legtöbbször a kihasznált jellemző vagy karakter jellemző vagy helyzet. De néha kihasználhatod az ellenfeled jellemzőjét ellene is. Ezt ellenséges kihasználásnak hívjuk, és pontosan ugyanúgy működik, mint bármilyen más jellemző kihasználása – fizess egy sors pontot egy +2 bónuszért a dobásodra, vagy dobd újra a kockákat. Ám van egy kis különbség – minden ellenséges kihasználás esetén az ellenfélnek kell fizetned a sors pontot. Viszont az ellenfél nem használhatja fel a kapott sors pontot az adott jelenetben. Ezt a fizetséget csak akkor kapják meg, ha a sors pont ténylegesen el lett költve, ingyen kihasználásért nem jár.
Történet részleteinek kinyilvánítása kihasználással
Egy jellemzőn alapuló, lényeges vagy valószínűtlen részletet adhatsz a történethez. Ne költs sors pontot, ha a „jellemző mindig igaz” elégséges. Akkor fizess csak, ha a részlet határeset, vagy – ha a társaság beleegyezik – amikor nincs kapcsolódó jellemző.
Késztetés
A jellemzők késztethetnek komplikálva a szituációt, de közben sors pontokat adva. A KM vagy egy játékos felajánlhat egy sors pontot valaki másnak, hogy a karakterének egy jellemzője késztessen, miközben kifejti, hogy a jellemző miért teszi a dolgokat nehezebbé vagy komplikáltabbá. Ha visszautasítod a késztetést, akkor a saját sors pontjaidból kell egyet elköltened, és le kell írnod, hogy hogyan kerülöd el a komplikációkat. Ez azt is jelenti, hogy ha kifogytál a sors pontokból, akkor képtelen vagy elkerülni a késztetéseket!
Bármelyik jellemző késztethet – legyen az a karakter jellemzője, helyzet, vagy következmény – de mindenképpen hatással kell lennie a karakterre, akit késztet.
Bárki kezdeményezhet késztetést. A késztetést indítványozó játékosnak viszont a saját sors pontjaiból kell egyet elköltenie. Ezután a KM sajátjaként bonyolítja le a célkaraktert érintő késztetést. A KM soha nem használ el sors pontot késztetés kezdeményezéséért – limitált készletük van kihasználásra, de annyit késztetnek, amennyit csak akarnak.
A késztetés lehet utólagos is. Ha egy játékos beleszerepjátszotta magát egy komplikációba, ami vagy a saját jellemzőjéhez vagy egy helyzethez kapcsolódik, akkor megkérdezheti a KM‑et, hogy ez megfelel‑e önkésztetésnek. Ha a csapat beleegyezik, akkor a KM egy sors pontot fizet a játékosnak.
Teljesen normális visszavonni egy késztetést, ha rájövünk, hogy nem ütötte meg a mércét. Ha a csapat úgy dönt, hogy a kezdeményezett késztetés nem megfelelő, az a célkarakternek nem kerül semmibe se.
A késztetések komplikációk, nem akadályok
Ha egy késztetést indítványozol, biztosítsd, hogy a komplikáció akciókhoz vezessem, vagy egy jelentős változás a körülményekben, ne csak leszűkítse a játékosok lehetőségeit.
„Aha, szóval homok ment a szemedbe, így mikor rálősz a szörnyre, elhibázod a lövést”, nem egy késztetés. Ez csak megtilt bizonyos cselekedeteket, de nem komplikál semmit sem.
„Átkozhatod a szerencséd, de a homok a szemedben szerintem azt jelenti, hogy nem igazán látsz semmit sem. A shoggothra irányzott lövésed gellert kap, kilukasztva pár hordót, amikből benzin folyik ki a nyitott kemence felé.” Ez egy sokkal jobb késztetés. Megváltoztatja a jelenetet, fokozza a feszültséget, és újabb kezelendő problémát ad a játékosoknak.
Bővebben olvashatsz arról, hogy mi működik, és mi nem, mint késztetés, a késztetések fajtáiról szóló diskurzusban a \fate{Fate Core} szabálykönyv 72. oldalától, vagy az interneten a \url{https://fate-srd.com/fate-core/invoking-compelling-aspects#types-of-compels} oldalon.
Események és döntések
Nagy általánosságban kétféle késztetés létezik: események és döntések.
Az esemény késztetés valami külső hatás miatt történik a karakterrel. Ez a külső hatás valahogyan kapcsolódik a jellemzőhöz, és ez valamilyen szerencsétlen komplikációt idéz elő.
A döntés késztetés ellenben belső, ahol a karakter, valamilyen karakterhibája vagy egymással ellentétes belső értékei miatt, a józan ésszel ellentétesen viselkedik. A jellemző irányítja a karaktert egy bizonyos döntés meghozatalára – és ennek a döntésnek az eredménye a komplikáció.
Minkét esetben a létrejövő komplikáció a lényeg! Komplikáció nélkül nincs késztetés sem.
Ellenséges kihasználás vagy késztetés?
Ne téveszd össze az ellenséges kihasználást és a késztetést! Bár hasonlóak – mindkettő egy sors pontot fizet, hogy létrehozzon egy közvetlen problémát – másképpen működnek.
A késztetés egy változást okoz a narratívában. A késztetés kezdeményezése nem a játék világában lett eldöntve, hanem a KM vagy egy játékos indítványozza a történet megváltoztatását. A hatás lehet, hogy átfogó, de a célpont rögtön megkapja a sors pontot, és lehetősége van visszautasítani a késztetést.
Az ellenséges kihasználás ezzel szemben automatikus hatás. A célpontnak nincs lehetősége visszautasítani a kihasználást – de mint minden kihasználásnál, meg kell indokolnod, hogyan lehetséges az adott jellemzőt kihasználni az adott szituációban. És bár megkapják a sors pontot, azt nem használhatják fel az adott jelenetben. Ezen felül a végeredmény is sokkal szűkebb: egy +2 bónusz, vagy újradobás.
A késztetés segítségével a KM vagy a játékos megváltoztathatja, hogy miről is szól az adott jelenet. Akár teljesen tönkrezúzhatják a cselekményt. Egy ellenség ellen használni rizikós vállalkozás – simán visszautasíthatják, netán a komplikációk ellenére is elérhetik a céljukat, felhasználva a vadonatúj sors pontot, amit éppen nekik adtál.
Az ellenséges kihasználások ellenben az adott pillanatban segíthetnek. A saját jellemzőiden kívül immár az ellenfelét is kihasználhatod, ami bővíti a lehetőségeidet, valamint dinamikusabbá és logikusabbá teheti a jeleneteket.
Jellemzők létrehozása és megszüntetése
A helyzetbehozás cselekvéssel lehet helyzet jellemzőket létrehozni vagy felfedezni. Az előny jellemzőket ugyanígy lehet létrehozni, valamint döntetlen vagy átütő siker adhat még megold, megtámadás vagy védekezés cselekvés használatánál.
Jellemzőket akkor tudsz megszüntetni, ha kitalálod, hogy a karaktered hogyan tudja megtenni – használj poroltót a Tomboló Tűzvész eloltásához, használj kitérő manővereket, ha az üldöző a Nyakadban Liheg. A szituációtól függően ez igényelhet egy megold cselekvést; ebben az esetben az ellenfeled használhat védekezést, hogy megpróbálja megőrizni a jellemzőt, ha meg tudja indokolni, hogy miért tud közbeavatkozni.
Mindazonáltal, ha a jellemző megszüntetésének nincs történetbeli akadálya, egyszerűen tedd csak meg. Ha Gúzsba Kötve hever a karaktered, és egy barátod megszabadít, a jellemző egyszerűen eltűnik. Ha semmi sem akadályoz meg, nem kell dobni sem.
Egyéb jellemző fajták
A szokványos jellemző típusokról már volt szó a 23. oldalon. A következő típusok opcionálisak, de emelhetik a játékod színvonalát. Bizonyos szempontból ezek a karakter jellemzők variánsai (ha átfogóbban definiálod, hogy mi lehet karakter) vagy helyzet jellemzők variánsai (ha megváltoztatod, hogy mennyi ideig maradnak aktívak).
Szervezet jellemzők: néha egész szervezetekkel kell foglalkoznod, amik bizonyos elvek alapján viselkednek. Ilyenkor létrehozhatsz olyan jellemzőket, amit a szervezet minden tagja kihasználhat a sajátjaként.
Kaland jellemzők: néha a cselekmény behozhat olyan motívumokat, amik újra és újra feltűnnek a történetben. Ilyenkor létrehozhatsz olyan jellemzőket, amit minden szereplő használhat, amíg az a történetszál véget nem ér.
Világ jellemzők: ahogy a kalandnak, úgy a kampánynak magának is lehetnek ismétlődő motívumai. A kaland jellemzőkkel ellentétben ezek a jellemzők soha nem szűnnek meg.
Zóna jellemzők: a helyzet jellemzőket zónákhoz is kapcsolhatod, amik egy területet reprezentálnak a térképen (29. oldal). Ez felpezsdítheti a játékosaid térképhasználatát. A KM „aki kapja marja” alapon ingyen kihasználásokat helyezhet el bizonyos zónákban, ami odavonzza a karaktereket (mind a JK‑kat és NJK‑kat), hogy a kezdeti stratégiájukban felhasználhassák a zóna jellemzőket.

\chapter{Kihívás, konfliktus és versengés}
Egy jelenetben gyakran előfordul, hogy a cselekvések eredményét egyetlen kockadobással meg tudod határozni – ki tudod‑e nyitni a széfet, elkerülni az őröket, vagy meggyőzni az újságírót, hogy mutassa meg a jegyzeteit? Máskor hosszan tartó ütközetet kell vívnod, ami sok dobást igényel. Ezekhez az esetekhez háromféle döntési mechanizmus létezik: kihívás, konfliktus és versengés. Mindhárom egy kicsit másként működik, mert más‑más a szóban forgó cél, és más akadályokat kell leküzdeni.
    • A kihívás egy komplikált vagy változékony szituáció. Valaki vagy valami megpróbál megakadályozni, de nincs jól definiált „ellenséges oldal”. Ezzel lehet kezelni, ha egy kutató nyomok után kutatna egy ősi kötetben, míg a csapat nagydumása elvonja a könyvtáros figyelmét, miközben a bunyós elmondhatatlan szörnyűségeket tart vissza attól, hogy belépjenek a könyvtárba.
    • A versengés olyan szituáció, amikor két vagy több fél egymást kizáró célokért küzd, de közvetlenül nem bántják egymást. A versengés tökéletes üldözésekhez, vitákhoz és mindenféle versenyekhez. (És bár a résztvevők közvetlenül nem bántják egymást, ez nem jelenti, hogy ne szenvedhetnének sérüléseket közben!)
    • A konfliktus, amikor a részvevők nemcsak, hogy képesek rá, de még kárt is akarnak okozni egymásban. Iszapbirkózás egy szektataggal, miközben pengék lendülnek egymás gyomra felé, vagy kilyuggatni egy csapat ghoul bőrét, míg ők a húsodba vájják karmukat, netán a riválisoddal egymást alázni a királynő éber tekintete előtt – ezek mind‑mind konfliktusok.
A jelenetek felépítése
Függetlenül a jelenet típusától, a KM először is felvázolja a lényeges dolgokat, hogy a játékosok tudják, hogy milyen eszközökhöz férnek hozzá, és milyen bonyodalmakra kell figyelniük.
Zónák
A zónák reprezentálják a bejárható teret – egy térképvázlat, ami pár különálló részre van osztva. Egy távoli tanyaházban játszódó konfliktusnak négy zónája lehet: a földszint, az emelet, az elülső udvar és a környező erdő. Kettő‑négy zóna általában elégséges a legtöbb konfliktushoz. Viszont nagyobb vagy komplikáltabb jelenetekhez többre is szükség lehet. Próbáld a zónák térképét egyszerűre venni, legjobb, ha elfér egy jegyzettömb lapján, vagy gyorsan felvázolod egy táblára.
A zónák korlátozzák, hogy mi eshet meg, így segítve a történet irányítását. Hogy kit támadhatsz meg, és hová mozoghatsz, azt mind a zóna határozza meg, amiben tartózkodsz.
Az egy zónában tartózkodók hathatnak egymásra, és mindenre, ami az adott zónában van. Ez azt jelenti, hogy ütheted, vághatod, vagy bármilyen fizikai kontaktust létesíthetsz emberekkel és tárgyakkal a zónádban. Ki akarod nyitni a hálószoba falába épített széfet? Akkor abban a zónában kell tartózkodnod. A zónádon kívül általában minden elérhetetlen számodra – vagy oda kell mozognod, vagy valahogyan távolba hatni (telekinézis, lőfegyverek stb).
Egyik zónából a másikba mozogni könnyű, feltéve, hogy semmi sincs az utadban. A cselekvéseden felül egy szomszédos zónába is mozoghatsz egy forduló (31. oldal) alatt, ha semmi sem áll az utadba. Ha a mozgásod korlátozva van, akkor az felemészti a teljes cselekvésedet. Dobjál megoldásra, hogy felmássz egy falon, elrohanj egy csapat szektatag mellet, vagy átugorj egy másik tetőre. Ha kudarc, akkor a zónádban maradsz, vagy pedig a mozgásnak kisebb vagy komoly ára lesz. A cselekvésedet felhasználva viszont bármelyik zónába mozoghatsz – ám a KM joga, hogy magas nehézséget állapítson meg, ha ez egy filmbe illő megmozdulás.
Ha valami se nem rizikós, se nem érdekes eléggé kockadobáshoz, akkor az nem számít korlátozásnak. Például, ha egy ajtó nincs bezárva, akkor nem kell cselekedet a kinyitásához – az a mozgás része lesz.
Célzással távolról is tudsz támadni. A távolsági támadásokkal a szomszédos, vagy messzebb lévő ellenfelek is megtámadhatók, ha semmi sem akadályozza a rálátást a zónájukra. Ha például egy lény az emeleti hálószobában posztol, akkor nem tudsz kanyarban rálőni a lépcső aljáról. Vedd figyelembe a zónák elhelyezkedését, és a helyzet jellemzőket, amikor eldöntöd, hogy mi lehetséges, és mi nem.
Helyzet jellemzők
A KM‑nek célszerű érdekes és változékony környezetet teremtenie a jelenet megtervezésekor, ami egyrészt korlátozza a cselekvéseket, másrészt a használata lehetőséget biztosít a játékosoknak a szituáció megváltoztatására. Három és öt közötti részlet bőven elég. Használd az alábbi kategóriákat:
    • Tónus, hangulat vagy időjárás – sötétség, villámlás és süvöltő orkán.
    • Mozgás korlátozása – létrákon át elérhető, iszappal borított és füstbe borult.
    • Fedezék és akadályok – járművek, oszlopok és ládák.
    • Veszélyes tereptárgyak – TNT ládák, olajhordók és elektromosságtól pattogó, hátborzongató műalkotások.
    • Használható tárgyak – rögtönzött fegyverek, ledönthető szobrok vagy könyvespolcok és elbarikádozható ajtók.
Bárki kihasználhatja vagy késztethet ezekkel a jellemzőkkel, ezért vedd figyelembe őket, amikor lebirkózod a szektatagot, ha Maró Iszap Mindenhol.
Ahogy a jelenet lejátszódik, újabb és újabb jellemzőket találhatsz ki. Ha a játékos rákérdez, hogy vannak‑e a közelben árnyékok, és ha beleillik a szituációba, nyugodtan írd le, hogy Koromsötét Árnyékok vannak a katakomba zugaiban. Más jellemzők akkor kerülnek játékba, ha a karakterek a megold cselekedettel létrehozzák őket. A karakterek cselekvései nélkül nem keletkezik csak úgy Tűzvész magától. Legalábbis általában.
Ingyen kihasználások helyzethez?
A KM dönthet úgy, hogy bizonyos helyzet jellemzőkhöz automatikusan jár egy ingyen kihasználás is a JK‑knak (és néha akár az NJK‑knak is). Az eszesebb játékosoknak talán pont a jelenet néhány jellemzője adhatja meg a kívánt fölényt – és egy ingyen kihasználás elég jól ösztönözhet a környezet megpiszkálására. Az ingyen kihasználások lehetnek az előkészületek eredményei is.
Zóna jellemzők
Ahogy azt említettük a 27. oldalon, némely helyzet jellemzők csak bizonyos zónákban használhatók. Ez teljesen rendben van – ez csak egy kis extra struktúrát, lehetőséget és kényelmetlenséget ad a térkép bizonyos részeihez, míg máshol nincsenek jelen.
Cselekvési sorrend
Gyakran nem érdekes, hogy ki pontosan mikor cselekszik, de versengésekben és konfliktusokban nagyon is számít. Ezek a jelenetek fordulókra vannak bontva. Minden forduló során, minden résztvevő karakter tehet egy‑egy megoldás, helyzetbehozás vagy megtámadás cselekvést, és ezen felül mozoghat is egyszer (a kihívások kicsit másként működnek, lásd a 33. oldalon). Mivel a védekezés egy reakció valaki más cselekvésére, a karakterek akárhányszor képesek rá, míg más karakterek jönnek, feltéve, hogy meg tudják indokolni a történet alapján, hogyan tudnak közbeavatkozni.
A jelenet elején, a KM és a játékosok közösen eldöntik a szituáció alapján, hogy ki jön először, és azután a cselekvő fél eldönti, hogy ki jön őutána. A KM karakterei is a cselekvési sorrendben kerülnek sorra, ugyanúgy, mint a játékosokéi, és a KM dönti el, hogy ki következik az NJK‑k után. Ha már mindenki sorra került, akkor az utolsó cselekvő dönti el, hogy ki kezdi a következő fordulót.
Cassandra és Ruth belebotlottak egy arany maszkos főpap vezette, kisebb csapat szektatagba, akik éppen valamilyen misztikus szertartást folytattak. Mivel a szektatagok nagyon el vannak foglalva a szertartással, a KM kijelenti, hogy a JK‑k jönnek először a konfliktusban. A játékosok eldöntik, hogy Cassandra kezd: a maszkos szektatag felé fut visítva, hogy helyzetbe hozza magukat a Zavarodott jellemzővel. Nem túl kifinomult, de hatásos. Hogy a legtöbbet kihozhassa a helyzet jellemzőből, Cassandra játékosa úgy dönt, hogy következőnek Ruth fog cselekedni. Ruth egy tőrt dob a főpap felé, azon nyomban kihasználva a Zavarodott jellemzőt, hogy javítsa a támadását. Bár ez nem elég, hogy egy ütésből kifektesse a főpapot, de elég jó kombináció, hogy megingassa.
Sajnos, mivel már mindegyik JK sorra került, Ruthnak nincs más választása, mint valamelyik szektatagot választani következőnek. A maszkosat választja. Ez tetszik a KM‑nek, hiszen nemcsak, hogy már mindig csak szektatagok fognak cselekedni, de az utolsó majd a maszkos főpapot fogja választani, hogy az kezdje a következő fordulót. Lehet, hogy a JK‑knak jó volt a belépője, de most a szektatagokon a sor, hogy visszavágjanak.
Ez a módszer a cselekvési sorrend megállapítására, számtalan néven fut online fórumokon: szabadon választható cselekvési sorrend, valamint „popcorn”, „átadásos” vagy „Balsera‑féle” kezdeményezés. Ez utóbbi Leonard Balsera nevét hordozza, aki az egyik szerzője a \fate{Fate Core} rendszernek, és aki elvetette az egész ötlet magjait. Többet is megtudhatsz erről a módszerről, és a kapcsolódó stratégiákról a https://www.deadlyfredly.com/2012/02/marvel/ oldalon.
Csapatmunka
A Fate rendszer három módszert ismer a csapatmunkához: több karakter ugyanolyan képességét egyesíteni egyetlen dobáshoz, ingyen kihasználásokat felhalmozni helyzetbehozásokkal, hogy egy csapattársat eredményessé tegyünk, valamint egy szövetségesünk nevében jellemzőket kihasználni.
Ha egyesíted a képességeket, akkor határozd meg, hogy melyik karakternek van a legmagasabb szintje. Minden további résztvevő, akinek legalább Átlagos~(+1) ugyanaz a képessége, +1 bónuszt ad a legmagasabb szintű karakter képesség szintjéhez. Ha egy karakter ilyen módon támogat egy másikat, az felhasználja a cselekvését. A támogatóknak ugyanazt az árat kell fizetniük, és ugyanazokat a következményeket kell viselniük, mint aki a dobást teszi. A maximális bónusz, amit ilyen módon lehet szerezni, nem lehet nagyobb, mint a legmagasabb képesség szint.
Egyébként, amikor te jössz, előnybehozás cselekedettel létrehozhatsz helyzet jellemzőket, és a szövetségeseid felhasználhatják az ingyen kihasználásokat, ha ez megmagyarázható a történetben. Ha nem te jössz, akkor kihasználhatsz egy jellemzőt, hogy más dobásához bónuszt adjál.
Kihívás
A legtöbb dolog, amit a karaktereknek tesznek, eldönthető egyetlen kockadobással – hatástalanítani egy bombát, megtalálni egy szörnyű titkot tartalmazó kötetet, netán megfejteni egy rejtjelezést. De néha a dolgok egy kicsit zavarosabbak, komplikáltabbak, és nem is olyan könnyű megtalálni a szörnyű titkot tartalmazó kötetet, mert a keresett hajó éppen Hong Kong kikötőjén száguld keresztül a dühöngő viharban, és a hajó könyvtára meg lángokban áll – amihez persze neked semmi közöd.
Komplikált szituációkban, ha nincs ellenérdekelt fél, valószínűleg kihívást kell használnod: egy sorozat megold cselekvés, amik megbirkóznak a nagyobb problémával. A kihívások alatt az egész csapat együttműködhet a jelenetben, ami lendületessé teheti a történetet.
A kihívás létrehozásához a KM‑nek át kell gondolnia a szituációt, és kiválasztani a képességeket, amikkel a csapat eredményt érhet el. Tekints minden cselekedetet egy különálló megold cselekvésnek. A csapatmunka cselevések engedélyezettek, de ezeknek legyen valamilyen ára, vagy okozzanak komplikációkat, például kifutni az időből vagy más eredménytelenséget.
A legjobb, ha a KM lehetőséget ad a hozzájárulásra minden karakternek a jelenetben – próbálj annyi képességet használni, amennyi karakter jelen van. Használj kevesebb képességet ha néhány karaktert várhatóan elszólít a kötelesség, vagy fontosabb dolgok lekötik a figyelmüket, vagy ha teret akarsz engedni a csapatmunkának. A keményebb kihívásokhoz írj elő több szükséges cselekvést, mint ahány karakter jelen van, és emellett megemelheted a nehézségeket is.
A kockadobások után a KM értékeli a sikereket, kudarcokat és az árat, amit fizetniük kellett, és mindent összevetve eldönti, hogy hogyan játszódott le a jelenet. Lehet, hogy az eredmény újabb kihíváshoz vagy egy versengéshez, netán egy konfliktushoz vezet. Ha mind siker, mind kudarc előfordul, az a karaktereknek részleges győzelmet jelenthet, ami miatt újabb komplikációkba bonyolódnak.
Versengés
A versengés az, amikor két vagy több ellenérdekelt fél egymás ellen dolgozik, de nincs konfliktus. Ez persze nem jelenti, hogy egyik fél sem szeretné bántani a másikat. Például versengés lehet, ha a csapat szeretne megszökni valamilyen veszedelem elől, mielőtt az teljesen ellehetetlenítené a győzelmet.
A versengés kezdetén mindenki kinyilvánítja, hogy mit szeretne tenni, mit szándékozik elérni. Ha több JK is jelen van, a céljaiktól függően lehetnek ugyanazon vagy ellentétes oldalon is – például versenyfutás esetén mindenki külön félnek számít. Versengés során a JK‑k vagy nem akarnak, vagy képtelenek kárt tenni az ellenfélben. Tőlük független veszélyek (például kitörő vulkán, vagy egy feldühödött isten) viszont támadhatják az egyik vagy akár az összes felet. Ezek a veszélyek akár megjelenthetnek a versengésben részvevő félként is.
A versengés fordulók sorozata. Minden fordulóban minden fél egy‑egy megold cselekvéssel próbál közelebb kerülni a céljához. Minden oldalon csak egyetlen karakter tesz megold cselekvést, míg a többiek vagy csapatmunkával vagy helyzetek létrehozásával járulnak hozzá a sikerhez (aminek van rizikója, lásd alább). A megold cselekvések mehetnek passzív nehézség ellen – ha a résztvevőknek valamilyen környezeti kihívást kell leküzdeniük – vagy közvetlen vetélkedés esetén a másik fél erőfeszítése ellen.
Minden forduló végén hasonlítsd össze a felek erőfeszítés értékét. Amelyik fél a legnagyobb erőfeszítést tette, az szerez egy diadalt. Ha átütő sikert ér el – míg más fél nem – akkor két diadalt arat. Az nyeri a versengést, aki elsőnek ér el három diadalt. (Bármikor használhatsz hosszabb versengéseket, ahol több diadalt kell elérni, de ötnél többet azért nem ajánlunk.)
Ha több fél egyszerre éri el a legmagasabb erőfeszítést, akkor senki sem szerez diadalt, és egy váratlan fordulat történik. A KM behoz egy új helyzet jellemzőt, ami azt szimbolizálja, ahogy a jelenet, terep vagy szituáció megváltozott.
Olyan versengések során, ahol egy fenyegetés bántani próbálja bármelyik részvevőt, a résztvevő oldalának minden tagja sérülést szenved el, ha a versengés dobása alacsonyabb, mint a fenyegetés megtámadás dobása vagy passzív nehézsége. A sérülés mértéke a sikerességgel egyezik meg. Ugyanúgy, ahogy a konfliktusokban, ha a karakter nem tudja a sérülést semlegesíteni, akkor kiejtődik a küzdelemből.
Helyzetbehozás versengés során
A csapatod minden fordulóban próbálkozhat helyzetbehozással a megold cselekvés előtt. A célpont és mindenki más, aki közbeavatkozhat, ellenállhat a szokásos módon. Minden egyes résztvevő megpróbálkozhat a helyzetbehozással a megold cselekedetén vagy a csapatmunka bónuszán felül (32. oldal). Ha nem sikerül a helyzetbehozás, akkor két választásod van: a csapatod lemondhat a megold dobásról, vagy pedig „siker valamilyen áron” (megtartva a megold dobást vagy csapatmunka bónuszt), ahol az ár egy ingyen kihasználás az ellenfélnek. Ha a helyzetbehozás kimenetele legalább döntetlen, akkor jöhet a megold dobás vagy csapatmunka bónusz a szokásos módon.
Konfliktus
Ha a hősök tisztességes harcba bocsátkoznak – legyen az a rendőrökkel, szektatagokkal vagy egy megnevezhetetlen szörnnyel – és esélyük van győzni, az egy konfliktus. Másképpen megfogalmazva, akkor használj konfliktust, ha a JK‑k elérhetik a céljukat erőszakkal vagy kényszerítéssel.
A konfliktusok a legegyértelműbbek – elvégre a legelső szerepjátékok csataszimulációkból nőttek ki. De vésd eszedbe a legfontosabbat: a résztvevő karakterek bántalmazhatják egymást. Ha ez egyoldalú – például ha benyomsz egyet egy élő hegynek – esélyed sincs sérülést okozni. Ez nem konfliktus. Ez egy versengés, ahol a JK‑k valószínűleg megpróbálnak elmenekülni, vagy valami módot keresnek, hogy hogyan vágjanak vissza.
Egy konfliktus lehet fizikai vagy szellemi. Fizikai konfliktus lehet lövöldözés, kardpárbaj vagy netán más dimenziók szörnyének legázolása egy kamionnal. Szellemi konfliktus lehet egy veszekedés a szeretteinkkel, kihallgatás vagy elmére irányuló mágikus támadás.
Csapatmunka (32. oldal) esetén fontos lehet az időzítés. Bármikor kihasználhatsz egy jellemzőt a társad nevében a dobásának javítására. Segíthetsz egy társadnak mielőtt ő jönne helyzetbehozással vagy a cselekvésed +1 bónuszért feláldozásával. Ha előtted jönne a fordulóban, akkor helyzetbehozással nem segítheted, de elhasználhatod a cselekvésed (feladva a cselekvést az adott fordulóban), hogy +1 csapatmunka bónuszt adj neki.
Sérülések elszenvedése
Ha a megtámadás siker, a védekező félnek semlegesíteni kell a találat sikerességét, ami a megtámadás és a védekezés erőfeszítéseinek különbsége.
A sikerességet stressz dobozok beikszelésével vagy következmény jellemzők felvételével tudod semlegesíteni. Ha nem tudod, vagy csak nem akarod az egész sikerességet semlegesíteni, az a kiejtés (36. oldal) – kikerülsz a jelenetből, és a támadó határozza meg, mi történik a továbbiakban.
A körülmények szerencsétlen összejátszása Charlest egy nyirkos pincébe vezette, ahol szembe kell szállnia egy ghoullal, ami fel akarja falni. A ghoul kitör éles karmaival kaszálva; ez egy megtámadás Jó~(+2) Közelharccal. A KM dobása 00++, ami miatt az erőfeszítés Kimagasló~(+4). Charles megpróbál elugrani a Remek~(+3) Ügyességével, de a dobása 000-, ami miatt az erőfeszítése csak Jó~(+2). Mivel a ghoul megtámadás erőfeszítése kettővel meghaladja Charles védekezés erőfeszítését, Charlesnak két sikerességet kell semlegesítenie. Beikszeli az első két fizikai stressz dobozát a háromból; a küzdelem nagyon is veszélyesnek bizonyult.
Stressz
Egyszerűen megfogalmazva, a stressz célja a főhősök életben tartása. Ez egy erőforrás, ami a küzdelemben tartja a karaktert akkor is, ha az ellenfél eltalálja. Miközben beikszelsz pár stressz dobozt, olyanokat mondasz, hogy „Éppen csak mellément,” vagy „Jaj, az ütéstől nem kapok levegőt, de attól még megvagyok.” Ennek ellenére ez véges erőforrás – a legtöbb karakternek csak három fizikai, és három szellemi stressz doboza van, bár ezt a magas Fizikum vagy Akaraterő növelheti.
A karakterlapon két stresszmérő található, egy a fizikai sérülésekre, egy pedig a szellemiekre. Ha elszenvedsz egy találatot, a megfelelő típusúból kell beikszelned, hogy a küzdelemben maradhass. Mindegyik stressz doboz egy sikerességet semlegesít. Több stressz dobozt is beikszelhetsz egyszerre, ha szükséges.
A dobozok két állapotúak – vagy üresek, és akkor használhatod őket, vagy beikszeltek, és akkor meg nem. De jól van ez így. Úgyis kitörlöd őket, amint véget ér a jelenet – persze, csak ha a szörnyek föl nem falnak előbb.
Következmény jellemzők
A következmények új jellemzők, amiket azért írsz a karakterlapra, hogy jelképezzék a karakter által elszenvedett kárt és sebesüléseket.
Ha felveszel egy következményt, hogy semlegesíts egy találatot, írj egy jellemzőt egy üres következmény rubrikába, ami leírja a karakter által elszenvedett sérülést. Használd a következmény súlyosságát iránymutatóként: ha megmar egy szörny, egy enyhe következmény lehetne Csúnya Harapásnyom, a mérsékelt következmény Szűnni Nem Akaró Vérzéses Harapás, míg a súlyos következmény már Megbénult Láb.
Ha stresszel semlegesítesz, akkor éppen csak elkerülsz egy találatot, míg a következmény felvétele azt jelenti, hogy alaposan eltaláltak. Akkor miért is vennél fel következményeket? Mert néha a stressz nem elégséges. Ha emlékszel még, akkor az egész sikerességet semlegesíteni kell, hogy a küzdelemben maradj. És nincs túl sok stressz dobozod. A jó hír, hogy a következmények szép nagy találatokat is semlegesíteni tudnak.
Minden karakter három üres következmény rubrikával kezd – enyhe, mérsékelt, és súlyos. Enyhe következmény felvétele két sikerességet semlegesít, a mérsékelt négyet, míg a súlyos hatot.
Például, ha bekapsz egy szép nagy találatot, öt sikerességgel, azt semlegesíteni tudod egyetlen stressz dobozzal, és egy mérsékelt következménnyel. Ez sokkal hatékonyabb, mint öt stressz dobozt elhasználni.
A következmények hátránya, hogy jellemzők – és ugye a jellemzők mindig igazak (22. oldal). Tehát, ha Hasba Lőtt vagy, akkor a karaktered hasa át van lőve! Eszerint nem tudsz olyan dolgokat megtenni, amit egy hasba lőtt ember sem tudna megtenni (például rohanni). Ha ez eléggé megbonyolítaná a dolgokat, akkor számíthatsz a következmény késztetésére is. És ahogy az általad helyzetbehozással létrehozott jellemzőkhöz is jár, a következményt okozó karakter – az, aki hasba lőtt – kap egy ingyen kihasználást is. Jaj!
Charles még mindig a ghoullal hadakozik. Az felé kap a karmaival, a dobása most 00++, ami hozzáadódik a Jó~(+2) Közelharcához, és kihasználja a Húsra Éhezik jellemzőt további +2 bónuszért, ami így Fantasztikus~(+6) csapás lesz. Charles --00 dobása, a Remek~(+3) Ügyességével csak egy Átlagos~(+1) védekezés; ez öt sikeresség, amit semlegesítenie kell. Egy mérsékelt következményt felvételét választja. A KM és a játékos úgy dönt, hogy a sebesülés Tátongó Mellseb. Ez a következmény négy sikerességet semlegesít, így csak egy marad, amire Charles elhasználja az utolsó megmaradt stressz dobozát.
Kiejtés
Ha képtelen vagy az egész sikerességet stresszel és következményekkel semlegesíteni, az a kiejtés.
Kiejtődni nem kellemes. Akárki is ejtett ki, eldöntheti, hogy mi történjen. Mivel veszélyes szituációkkal és kemény ellenfelekkel állsz szemben, ez jelentheti, hogy meghaltál, de ez nem az egyetlen lehetőség. A végeredménynek a konfliktus kiterjedésének és mértékének megfelelőnek kell lennie – nem fogsz a szégyenbe belehalni, ha elveszítesz egy vitát – de valószínűleg a karakterlapod (és még más is) megváltozik. A végeredmény nem mehet szembe a csapat által lefektetett elvekkel sem – ha a csapatod úgy gondolja, hogy a karaktereket nem szabad megölni az engedélyed nélkül, az teljesen rendben van.
De ha a halál fel is merül, mint lehetőség (ezt legjobb még a dobás előtt tisztázni), a KM jobb, ha eszébe vési, hogy ez azért elég unalmas. Egy kiejtett JK elveszhet, elrabolhatják, veszélybe sodródhat, következmények felvételére kényszerülhet… a lista végtelen. Ha a karakter meghal, akkor valakinek egy új karaktert kell készítenie és bemesélnie a történetbe, de a halálnál is rosszabb végeredményeknek csak a fantáziád szab határt.
Az elképzelés alapján írd le, hogy valaki – vagy valami – hogyan is ejtődik ki. A szektatag géppuska zárótűz alá került? Vörös fröcskölés tölti be az eget, ahogy teste hangos loccsanással a földre hull. Kilöktek a kamionból, ahogy áthajtott a 26. utca felüljáróján? Eltűnsz a korlát mögött, és hátrahagynak, ahogy a csetepaté folytatódik az autópályán. Vedd lehetőségbe a halált is, a kiejtés végeredményének meghatározásánál, de legtöbbször a végzet kijátszása is legalább olyan érdekes lehet.
A ghoul támadása igencsak szerencsés, Legendás~(+8) megtámadás, Charles Gyenge~(-1) védekezése ellen. A konfliktusnak ebben a szakaszában Charles összes stressz doboza be van ikszelve, ahogy már a mérsékelt következmény rubrikája sem üres. Még ha semlegesítene is nyolc sikerességet, enyhe és súlyos következmények felvételével, az se lenne elég. Emiatt Charles kiejtődött. A ghoul dönt a sorsa felől. A KM‑nek joga lenne ahhoz, hogy a ghoul megölje Charlest… de meghalni nem éppen a legérdekesebb.
Ehelyett a KM kijelenti, hogy Charles túlélte, mert a ghoul leütötte, és a barlangjába vonszolta, de minden következmény jellemzője megmaradt. Charles elveszve, zúgó fejjel ébred a város alatt húzódó, éjsötét katakombákban. Mivel kiejtődött, nincs más lehetősége, mint elfogadni a történéseket.
Megadás
Hogy hogyan tudod elkerülni a halált – vagy valami még rosszabbat? Bármilyen cselekvést megszakíthatsz konfliktus közben, ha a dobás még nem történt meg, és bejelentheted a megadást. Egyszerűen add fel. Közöld a többiekkel, hogy ennyi volt, nem bírod tovább. A karaktered kiejtődik a konfliktusból, de kapsz ezért egy sors pontot, és egyet‑egyet pluszban minden következményért, amit a karaktered ebben a konfliktusban elszenvedett.
A megadás ezen kívül lehetővé teszi, hogy te mondd meg a feltételeit, és hogy hogyan kerülsz ki a konfliktusból. Elmenekülhetsz a szörnyek elől, hogy máskor majd revánsot vehess. Viszont ez mindenképpen vereség. Valamit fel kell ajánlanod az ellenfelednek. Nem adhatod meg magad, hogy te legyél a nap hőse – ez már nem opció.
A megadás nagyon hasznos eszköz. Megadhatod magad, hogy elszökhess, miközben már a következő összecsapást tervezed, vagy szerzel egy követendő nyomot, vagy valami más hasznos dolgot. Az egyetlen, amit nem tehetsz, hogy nem nyerheted meg ezt a csatát.
Még azelőtt meg kell adnod magad, hogy az ellenfél elgurítaná a kockákat. Nem várhatod meg a dobás eredményét, hogy csak akkor add meg magad, ha nyilvánvalóan lehetetlen nyerned – az nem lenne túl szép.
Itt számíthatsz egy kis egyezkedésre. Olyan megoldásra kell törekedned, ami minden résztvevőnek elfogadható. Ha az ellenfélnek nem tetszenek a megadásod feltételei, akkor kérhetik, hogy fogalmazd át, vagy áldozz fel valami mást, vagy csak áldozz fel többet. Mivel a megadás veszteség a számodra, emiatt az ellenfélnek legalább részlegesen el kell érnie a célját.
Minél jelentősebb árat fizetsz a megadásért, annál nagyobb jutalmat kell szereznie a te oldaladnak a megadásért – ha biztos halállal kell a csapatnak szembenéznie, de egyikük hősiesen (és öngyilkos módon) utolsó vérig küzd, akkor mindenki más megmenekülhet.
Konfliktus befejezése
A konfliktus véget ér, ha az egyik oldalon mindenki megadta magát, vagy kiejtődött. A konfliktus végén, minden játékos, aki megadta magát, megkapja a sors pontokat ezért (37. oldal). A KM szintén ilyenkor osztja a konfliktus közbeni ellenséges kihasználásokért (24. oldal) járó sors pontokat.
Sérülésekből felépülni
Minden jelenet végén, a karakterek lenullázzák a stresszmérőt. Viszont a következmény jellemzőknek több idő kell a megszűnésre.
A felépülés kezdetéhez a karaktert kezelő másik karakternek egy sikeres megold cselekedetet kell tennie a megfelelő képességgel. A fizikai sérülések általában orvosi tudást igényelnek a Tudomány képességgel, míg a szellemi sérülések gyógyítására az Empátia szolgál. A dobás nehézsége a következmény súlyossága: Jó~(+2) az enyhe következményekhez, Kimagasló~(+4) a mérsékelt következményekhez, és Fantasztikus~(+6) a súlyos következményekhez. Ezek a nehézségek kettővel magasabbak, ha a karakter magát kezeli (mert könnyebb ezt másnak megtennie).
Ha a dobás sikeres, akkor fogalmazd át a következmény jellemzőt, hogy mutassa, hogy a sérülés gyógyulóban van. A Törött Kar következményből például lehet Begipszelt Kar.
A siker csak az első lépcső – a sérülések teljes gyógyulásához idő is szükséges.
    • Az enyhe következmények a kezelés után egy teljes jelenetet igényelnek.
    • A mérsékelt következmények lassabban gyógyulnak, a kezelés után egy teljes játékülést igényelnek.
    • A súlyos következmények pedig csak a kezelés utáni első áttörés (39. oldal) alkalmával tűnnek el.

\chapter{Fejlődés}
Ahogy a karaktereitek végighaladnak a történetszálakon, felnőnek, és megváltoznak. Minden játékülés végén elértek egy mérföldkőhöz – ekkor átrendezhetitek a karakterlapot. Amint azonban lezártok egy történetívet, elértek egy áttöréshez – ekkor már hozzá is adhattok valamit a karakterlaphoz. (A 40. oldalon többet is megtudhatsz a játékülésekről és történetívekről.)
Mérföldkövek
A mérföldkövek a játékülések végén találhatók, a történetívek részeként. Arra szolgálnak, hogy a karaktereket megváltoztassák, ne felfejlesszék. Akár ki is hagyhatsz egy mérföldkövet, ha nincs rá szükséged. Nem kötelező módosítanod a karaktereden. De azért a lehetőség megvan rá.
Mérföldkő esetén egyetlen egyet tehetsz meg az alábbiak közül:
    • Felcserélheted két képességed szintjét, vagy lecserélhetsz egy Átlagos~(+1) képességet olyanra, ami még nem szerepel a karakterlapon.
    • Átírhatsz egy fortélyt.
    • Új fortélyt vehetsz fel egy újratöltésért. (De vésd észbe, hogy nem mehetsz 1 újratöltés alá.)
    • Átírhatod az egyik jellemződet a koncepció jellemző kivételével.
Áttörések
Az áttörések már jelentősebbek, amikor a karaktered hatalma megnő. Áttörés esetén egyetlen egyet tehetsz meg a mérföldkő listájából. Ezen felül az összes alábbit is:
    • Átírhatod a karaktered koncepció jellemzőjét, ha ahhoz van kedved.
    • Ha van mérsékelt vagy súlyos következmény jellemződ, ami még nem gyógyul, akkor elindíthatod a gyógyulási folyamatot a jellemző átnevezésével. A gyógyulásban lévő következményeket pedig törölheted.
    • Megnövelheted egy képességed szintjét eggyel – akár Középszerűről~(+0) Átlagosra~(+1).
Ha a KM úgy gondolja, hogy lezárult a cselekmény egy jelentősebb szakasza, és ideje a karaktereket egy kicsit „felturbózni”, felajánlhat egyet az alábbiakból:
    • Egy újabb újratöltést kapsz, amit akár rögtön fortélyra is költhetsz.
    • Még egy képességet növelhetsz eggyel.
Képességek szintjének növelése
Amikor a képességek szintjét növeled, meg kell tartanod az „oszlop” struktúrát. Egyik szinten sem lehet több képesség, mint a közvetlenül alatta lévő szinten. Ez azt jelenti, hogy először vagy pár Középszerű~(+0) képességet kell megnövelned – vagy pedig tartalékolnod kell a növeléseket, amiket később egyszerre tudsz egy nagyobb növelésre fordítani.
Ruth szeretné Átlagos~(+1) Misztikum képességét Jóra~(+2) növelni, de emiatt négy Jó~(+2) és csak három Átlagos~(+1) képessége lenne, ami nem lehetséges. Szerencséjére, tartalékolt egy képesség szint növelést az előző áttörésből, így megnövelheti Középszerű~(+0) Empátia képességét is Átlagosra~(+1). Így már van egy Kimagasló~(+4), kettő Remek~(+3), négy Jó~(+2) és négy Átlagos~(+1) képessége.
Piramis

Helytelen

Helyes

Helyes
+4
0

+4
0

+4
0

+4
0
+3
00

+3
00

+3
00

+3
000
+2
000

+2
0000

+2
0000

+2
000
+1
0000

+1
000

+1
0000

+1
000
Játékülések és történetívek
Van pár feltevésünk, amikor játékülésekről és történetívekről beszélünk. Szeretnénk rávilágítani ezekre a feltevésekre, hogy végrehajthasd a megfelelő változtatásokat, ha a te játékod különbözik ettől.
A játékülés egyetlen játékalkalom, ami néhány jelentből áll, és pár óra alatt lezajlik. Fogd fel úgy, mint egy TV sorozat egyetlen epizódját. Valószínűleg három‑négy órán át tart.
A történetív játékülések sorozata, amiknek a cselekményei több játékalkalmon is átívelhetnek. Ezeknek a cselekményeknek nem feltétlenül kell az ívben lezáródniuk, de általában jelentős változások állnak elő benne annak folyamán. Fogd fel úgy, mint egy harmad vagy fél TV sorozat évadot. Valószínűleg négy körüli játékülésből állnak.
Ha a te játékod időtartama kívül esik ezeken a „valószínű” hosszakon, megváltoztathatod a fejlődés működését.  Ha egy történetív hosszabb, mint négy‑hat játékülés, akkor megengedheted a súlyos következmények törlését már négy játékülés után, mintsem, hogy várni kelljen a történetív végéig. Ha szeretnéd lassítani a fejlődést, akkor adagolhatod a fejlesztéseket, mint például a képesség növeléseket vagy újratöltéseket, ritkábban is. Ha a csapatod általában rövidebb játéküléseket tart, akkor nem muszáj mérföldkőhöz érni mindegyiknek a végén. Fűszerezd ízlés szerint; a saját játékodat úgy módosíthatod, ahogy csak akarod.

\chapter{Kaland Mesterek kézikönyve}
KM‑ként te vagy a rendező a játékülés során. Vésd az eszedbe, hogy ez nem jelenti, hogy te lennél a főnök. A \fate{Fate Condensed} játék együttműködésre épül, és a játékosok döntik el, hogy mi történjen a karakterükkel. A te dolgod az események mozgásban tartása az alábbiakkal:
Jelenetek levezénylése: A játékülés jelenetekből áll össze. Döntsd el, hogy mikor kezdődik a jelenet, ki van jelen, és mi történik. Ha szerinted már minden érdekes lehetőség megtörtént a jelenetben, vess véget neki. Ugord át az érdektelen részeket; ahogy nem dobsz akkor se, ha a cselekvés eredménye nem elég érdekes, ugyanúgy hagyd ki a jeleneteket, ahol semmi izgalmas, drámai, használható vagy mókás sem történik.
Szabályok meghatározása: Ha kérdéses, hogy hogyan is kell a szabályokat alkalmazni az adott körülmények között, megbeszélheted a játékosokkal, hogy valami konszenzusra jussatok, de a végső szó mindig a tiéd marad.
Nehézségek meghatározása: Döntsd el, hogy mikor szükséges dobni, és milyen nehézségre.
Döntsd el a kudarc árát: Ha egy karakter dobása kudarc, akkor a te döntésed, hogy siker komoly áron esetében pontosan mekkora is ez az ár. Természetesen elfogadhatod a játékos javaslatát – lehet, hogy tökéletesen tisztában van, milyen sérülést szeretne, hogy a karaktere elszenvedjen –, de legvégül ez a te döntésed lesz.
Játssz az NJK‑kal: Minden játékos a saját karakterét irányítja, míg te az összes többit, a szektatagoktól kezdve a szörnyeken át a főellenségig mindenkit.
Adj lehetőséget a JK‑knak akciózni: Ha a játékosok nem tudják, hogy mit is kéne tenniük, a te dolgod, hogy megadd nekik a kezdőlökést. Ne hagyd, hogy a cselekmény megakadjon a tétovázás vagy az információhiány miatt – valamivel rázd fel a dolgokat. Ha kétségeid vannak, gondolkodj el a főellenség  céljain és taktikáján, ami alapján a hősöket zaklatni tudod.
Biztosítsd, hogy mindenki megkapja néha a rivaldafényt: Nem az a dolgod, hogy legyőzd a játékosokat, hanem hogy megfelelő kihívást intézz ellenük. Biztosítsd, hogy mindegyik JK főszereplővé válhasson egy kis időre. A késztetéseket és kihívásokat szabd testre a karakterek tehetségeinek és gyengeségeinek megfelelően.
Komplikáld a JK‑k életét: Azon kívül, hogy szörnyeket zúdítasz a karakterekre, te leszel a legfőbb késztetés kezdeményező is. Természetesen a játékosok is késztethetik magukat vagy másokat, de mindenki számára lehetővé kell tenned, hogy megtapasztalhassák a jellemzőik negatív hatásait.
Építkezz a játékosok döntéseire: Nézd meg, milyen cselekedeteket végeztek a JK‑k játék közben, és gondold tovább, hogy milyen hatással van ez a világra, és az hogyan reagál. Élővé teheted a világot, ha a játékban szembesíted a JK‑kat ezekkel a következményekkel, legyenek azok jók vagy rosszak.
Nehézség és ellenállás meghatározása
Néha a JK cselekvése – egy védekezés dobásban megnyilvánuló – ellenállással találja szembe magát. Ilyenkor az ellenséges karakter kockákkal dob, amihez hozzáadja a megfelelő képesség szintjét, pont úgy, ahogy a JK‑k teszik. Ha az ellenséges karakternek van idevágó jellemzője, akkor azt kihasználhatja; a KM az NJK‑k jellemzőit az NJK‑k közös sors pont tartalékából tudja kihasználni (44. oldal).
Ha viszont nincs dinamikus ellenállás, akkor meg kell határoznod a statikus nehézséget:
Az alacsony nehézségek, amik a JK idevágó képesség szintje alatt vannak, jól jönnek, ha szeretnél esélyt adni a képesség megvillantására.
A közepes nehézségek, amik a JK idevágó képesség szintje körüliek, akkor hasznosak, ha szeretnél feszültséget teremteni, de nem akarod elnyomni a karaktereket.
A magas nehézségek – amelyek sokkal magasabbak a JK idevágó képesség szintjénél – célja kiemelni, hogy mennyire szörnyűek vagy szokatlanok a körülmények, aminek legyőzéséhez mindent be kell vetniük, netán – ha vesztes a pozíció – hogy elszenvedjék a kudarcuk következményeit.
Ugyancsak használhatod a mellékneveket a képességek szintjeinek skálájából (6. oldal), hogy kiválaszd a megfelelő nehézséget. Ez emberfeletti erőfeszítést igényel? Akkor legyen Emberfeletti~(+5)! Néhány ökölszabály következik segítségképpen.
Ha a feladat egyáltalán nem nehéz, akkor legyen Középszerű~(+0) – vagy csak mondd azt, hogy a játékos bármiféle dobás nélkül is sikert ért el, feltéve, hogy nincs súlyos időzavarban, netán ha a karakter valamelyik jellemzője szerint nagyon jó az adott dologban.
Ha tudsz legalább egy dologról, ami nehézzé tenné a feladatot, akkor legyen Jó~(+2); minden egyes további ellenük dolgozó tényező +2 nehézséget jelent.
A tényezők számbavételéhez használd a játékban lévő jellemzők listáját. Ha valami elég fontos volt, hogy jellemzőt csináljanak belőle, akkor az megér egy kis figyelmet. Mivel a jellemzők mindig igazak (22. oldal), befolyásolhatják, hogy mennyire könnyű vagy nehéz valamit megtenni. Természetesen ez nem jelenti, hogy kizárólag csak a jellemzőket kell figyelembe venned! Például a sötétség az sötét, függetlenül attól, hogy felvetted‑e helyzet jellemzőnek vagy sem.
Ha a feladat a lehetetlennel határos, olyan magasra veheted a nehézséget, amennyire csak akarod. A JK‑nak el kell használnia pár sors pontot, és jó sok segítséget is igénybe kell vennie a sikerhez, de ez így van jól.
További ötletekhez, hogy miként kreálhatsz a játékosaid számára változatos és érdekes ellenségeket, olvasd el a \fate{Fate Adversary Toolkit} szabálykönyvet. Ez megvásárolható PDF formátumban, vagy a lényege elérhető ingyen is a https://fate-srd.com/ címen.
NJK‑k
Egy NJK lehet egy szemlélődő, mellékszereplő, szövetséges, ellenfél, szörny vagy bármi más, ami komplikálhatja vagy gátolhatja a JK‑k fáradozásait. Valószínűleg olyan karaktereket is létre szeretnél hozni, amivel a JK‑k interakcióba léphetnek.
Fontos NJK‑k
Ha valaki különösképpen fontos a történet szempontjából, akkor ugyanúgy kidolgozhatod, mint egy JK‑t. Ez különösképpen helyénvaló, ha a JK‑k sokat fognak foglalkozni vele, mint például egy szövetséges, egy rivális, egy hatalmi csoportosulás képviselője, netán a főellenség.
Egy fontos NJK‑nak nem muszáj ugyanazokat a megkötéseket követnie, mint egy kezdő JK‑nak. Ha az NJK egy ismétlődően felbukkanó főellenség szintű fenyegetést jelent, akkor nyugodtan adhatsz neki magasabb szintű képességet is, több fortélyt vagy bármi mást is, amitől veszélyesebbé válik.
Kisebb NJK‑k
Azokat az NJK‑kat, akik nem lesznek visszatérő, lényeges karakterek, nem kell annyira részletesen kidolgozni, mint a fontos NJK‑kat. Egy kisebb NJK‑nál csak annyit határozz meg, amennyi feltétlenül szükséges.
A legtöbb kisebb NJK‑nak egyetlen egy jellemzője lesz, ami megadja, hogy ki ő: Őrzőkutya, Akadékoskodó Bürokrata vagy Feldühödött Szektatag, és így tovább.
Ha szükséges, kaphatnak még egy‑két jellemzőt, amivel mondhatsz valami érdekeset róluk, vagy megadhatod a gyengeségüket. Szintén kaphatnak még egy fortélyt is.
Adj nekik egy‑két képességet, amik megmutatják, hogy miben jók. Kiválaszthatod ezeket a képességeket a képesség listáról, vagy kitalálhatsz valami különlegesebbet is, például Jó~(+2) a Tömegverekedés Kierőszakolása képessége, vagy Kimagasló~(+4) a Harapásban.
Végül legyen nulla és három közötti stressz doboza; minél több van neki, annál nagyobb fenyegetést jelent. Általában nincs következmény jellemző rubrikájuk; ha nagyobb sérülést szenvednek el, mint amennyit stresszel semlegesíthetnének, akkor egyszerűen kiejtődnek a küzdelemből. A kisebb NJK‑k nem helytállásra készülnek.
Szörnyek, főellenségek és más veszedelmek
Az NJK‑khoz hasonlóan a szörnyek és más veszedelmek (mint például egy vihar, elharapózó tűzvész vagy egy szakasz felfegyverzett talpnyaló) karakterként vannak létrehozva, de a JK‑khoz képest sokkal egyszerűbb módon. Csak annyit kell kidolgoznod, amennyi feltétlenül szükséges. A kisebb NJK‑kkal szemben ezeket a veszedelmeket igazándiból akárhogyan le lehet írni. Szegd meg a szabályokat (54. oldal). A jellemzők, képességek, fortélyok, stressz és következmény jellemző rubrikák bármilyen kombinációját használhatod, hogy kellőképpen veszélyessé tedd őket, és a szintek meghatározásához döntsd el, hogy mekkora problémát okozzanak a JK‑knak.
A te sors pontjaid
Minden jelenetet akkora sors pont tartalékkal kezdj, mint a jelenlévő JK‑k száma. Ha jelen van még egy fontos NJK vagy szörny is, amelyik megadást választott (37. oldal) egy előző konfliktusban, vagy ellenséges kihasználás (24. oldal) áldozata lett egy előző jelenetben, akkor megkapod azokat a sors pontokat is. Ha az előző jelenetnek egy késztetés vetett véget, és emiatt nem volt alkalmad a sors pontot elkölteni, azt is hozzáadhatod a tartalékodhoz.
Charles, Ruth, Cassandra és Ethan a végső leszámolás helyszínére tartanak, ahol Alice Westforth vár rájuk. Ő előzőleg megadta magát a hősöknek egy konfliktusban, miután egy mérsékelt következményt szenvedett el. Emiatt a KM négy sors pontot kap a JK‑k után, és további kettőt Alice hoz magával a közös sors pont tartalékba.
Mint a KM ebből a tartalékból költhetsz sors pontokat, hogy kihasználj egy jellemzőt, visszautasíts egy késztetést, amit egy játékos ajánlott fel egy NJK‑nak, vagy hogy sors pont elköltését megkövetelő fortélyt használj – pont ugyanúgy, ahogy a játékosok teszik.
Ellenben nem kell sors pontot költened a tartalékodból jellemzők késztetéséért. Erre a célra végtelen számú sors pont áll rendelkezésedre.
Biztonsági technikák
A KM és a játékosok közös felelőssége, hogy mindenki biztonságban érezze magát a játékban és az asztal körül is. A KM elősegítheti ezt egy rendszerrel, ahol bárki megfogalmazhatja az aggodalmait és kifogásait. Ha ez történik, annak prioritást kell élveznie, és foglalkozni kell vele. Alább pár technika található, amivel a játékosok könnyebben jelezhetik fenntartásaikat, és amik könnyebbé tehetik a rendszer használatát.
    • X‑Kártya: Az X‑Kártya egy opcionális eszköz (John Stavropoulos találmánya), aminek segítségével játék közben bárki (téged is beleértve) kimoderálhat számára kényelmetlen tartalmakat. Többet is megtudhatsz az X‑Kártyáról a http://tinyurl.com/x-card-rpg weblapon.
    • Script Change RPG Toolbox: Ha egy kicsit több árnyalatra és felbontásra lenne szükséged, nézd meg Brie Beau Sheldon Script Change rendszerét, amivel megállíthatod, visszatekerheted, átugorhatod a történet részeit, és még egyebekre is jó, mindezt az ismerős média‑lejátszó felhasználói felülettel. Többet is megtudhatsz a Script Change rendszerről a http://tinyurl.com/nphed7m weblapon.
Ezeket a technikákat a fals szabályhoz (14. oldal) hasonlóan kalibrációra is lehet használni. Ez egy kényelmes lehetőséget biztosít a játékosoknak, hogy kifejthessék, mit várnak el a játéktól. Ne kicsinyeld le ezeket a technikákat, és támogasd a használatukat!

\chapter{Opcionális szabályok}
Ez itt jó néhány opcionális vagy alternatív szabály, amit bevezethetsz a játékodban.
Állapotok
Az állapotok a következményeket helyettesítik, és teljesen lecserélik a játékmechanikát. Az állapotoknak két célja van: egyrészt leveszi a terhet a KM és a játékosok válláról, így nem kell helyben kitalálniuk egy korrekten megfogalmazott jellemzőt minden elszenvedett következményhez, másrészt lehetővé teszik a játék jellegének meghatározását maradandó sérülések előre definiált listájával.
Az állapotok \fate{Fate Condensed} verziója kettébontja minden következmény jellemző rubrika szintjét két, fele akkora értékű beikszelhető dobozra.
1 Megkarcolt (Enyhe)
1 Rémült (Enyhe)
2 Lesérült (Mérsékelt)
2 Megrendült (Mérsékelt)
3 Megsebesült (Súlyos)
3 Demoralizált (Súlyos)
Ezek fizikai és szellemi állapotoknak felelnek meg – de ez nem jelenti, hogy fizikai találat esetén ne használhatnál fel egy szellemi állapotot és fordítva, ha van ennek értelme. Elvégre a támadások traumát okozhatnak!
Az állapotok ugyanúgy gyógyulnak, mint a következmények, a súlyosságuktól függően.
Ha egy magas képesség szint miatt egy extra enyhe következmény rubrikát kapnál, akkor vegyél fel helyette kettő extra beikszelhető dobozt a Megkarcolt vagy a Rémült állapothoz, a képességtől függően.
Állapotok elszeparálása
Ha szeretnéd, hogy a fizikai és szellemi állapotok teljesen különállóak legyenek, duplázd meg a dobozok számát mindkettőnél. Ennek is megvan a határa: ha bármelyik állapot szinten két doboz már be van ikszelve, akkor több dobozt már nem lehet beikszelni azon a szinten. Tehát, ha például a kettőből egy doboz beikszelt a Megkarcolt sorban, és a Rémült üres, akkor miután beikszeled vagy a második Megkarcolt dobozt, vagy az első Rémült dobozt, nem ikszelhetsz be többet azon a szinten.
Ha egy extra enyhe következmény rubrikát kapnál (magas Fizikum, Akaraterő vagy egy fortély miatt), akkor vegyél fel helyette kettő extra beikszelhető dobozt a Megkarcolt vagy a Rémült állapothoz, a képességtől függően. Minden ilyen extra doboz eggyel növeli a beikszelhetőség határát azon a szinten.
Egyéb állapot szabály verziók
Jó néhány megjelentetett Fate rendszerű játék használ valamilyen állapotrendszert a következmények helyett. Nyugodtan adaptáld az ő rendszerüket, ha az jobban tetszik, mint ez. Mindegyik ugyanazt éri el a játékban: hogy ne kelljen következmény jellemzőket kitalálni, és lehetővé teszik a játék jellegének meghatározását a maradandó sérülések korlátozásával.
Képesség lista megváltoztatása
Ahogy azt a 9. oldalon is írtuk, a saját Fate játékod kidolgozásakor elsőként a képesség listán gondolkodj el. Alapértelmezésben 19 képességet kell elosztani egy 10 elemű piramisban. A képesség lista úgy van megalkotva, hogy a tradicionálisan használt cselekedeteket fedje le, lényegében azt megválaszolva, hogy „mit tudsz megtenni”? Más képesség listák nem szükségszerűen ugyanilyen hosszúak, vagy ugyanígy vannak elosztva, vagy ugyanarra a kérdésre válaszolnak. Mindezek után néhány rövid lista, amit figyelembe vehetsz, amiből kölcsönözhetsz ötleteket, és amiket módosíthatsz.
    • Cselekedetek: Elvisel, Harcol, Tud, Mozog, Észlel, Vezet, Lopózik, Beszél, Bütyköl.
    • Megközelítések: Óvatos, Eszes, Látványos, Erőteljes, Gyors, Trükkös.
    • Adottságok: Atléta, Harc, Irányítás, Tudomány, Bűnözés.
    • Tulajdonságok: Erő, Ügyesség, Keménység, Intelligencia, Karizma.
    • Kapcsolatok: Irányító, Partneri, Támogató, Egyéni.
    • Szerepek: Sofőr, Nehézfiú, Számítógépkalóz, Szerelő, Szélhámos, Tolvaj, Lángelme.
    • Elemek: Levegő, Tűz, Fém, Elme, Kő, Űr, Víz, Szél, Fa.
    • Értékek: Kötelesség, Dicsőség, Igazságosság, Szeretet, Hatalom, Biztonság, Igazság, Bosszú.
Ha hosszabb listára van szükséged, akkor indulj ki az alapértelmezett listából, majd addig kell hozzáadni, összevonni, elvenni képességeket, amíg meg nem kapod, amire vágysz. Ehelyett egybegyúrhatsz két vagy több listát is feljebbről.
Fejlődés: Minél rövidebb a listád az alapértelmezetthez képest, annál ritkábban kell képesség növelést osztanod fejlődés esetén. Akár tartogathatod ezt kizárólag a „felturbózásra” (39. oldal), vagy más módon is korlátozhatod.
Alternatívák a piramis helyett:
    • Rombusz: Széles közép (körülbelül a képességek harmada), ami elkeskenyedik az alja és a teteje felé, másképpen 1 darab +0, 2 darab +1, 3 darab +2, 2 darab +3, 1 darab +4.
    • Oszlop: Nagyjából ugyanannyi képesség minden szinten. Ha elég rövid a listád, ez lehet egy vonal is, egy képesség minden szinten.
    • Szabad+Limit: Minden játékos kapjon annyi képesség pontot, amennyi a piramishoz (vagy más formációhoz) kell, de a forma ne legyen kötelező. Ezután szabadon vásárolhatnak képességeket, de a szinteknek a limit alatt kell maradniuk.
Lefedés: Mindenképpen vedd figyelembe, hogy mennyi képességnek lesz szintje a listából. Alapértelmezésben ez 53\% (10 a 19‑ből). Minél magasabb ez a százalék, annál inkább átfedik egymást a játékosok karakterei. Védd meg a karakterek egyedi szakterületeit!
Kombinálás: Lehet két listád is, és a játékosok minden dobásnál egy‑egy képességet használnak mindkét listáról. A legfontosabb, hogy a lehetséges szint összegeket nulla és a limit között tartsd. Például mindkét lista képességei lehetnek +0 és +2 között, vagy -1 és +1 között az egyik, míg +1 és +3 között a másik listán, és így tovább.
Karakterkészítés játék közben
Ha egy játékos képes gyors, kreatív gondolkodásra, elkészítheti a karakterét játék közben is, nem muszáj előre kidolgoznia. Ez olyan, mint a regényekben, ahol a főhősről menet közben derülnek ki dolgok. Ezt nem kell erőltetni, de olyan társaságokban, akik vevők rá, mindenki kedvére tehet.
Ezen a módon a karaktereknek kezdetben csak nevük, koncepció jellemzőjük és legmagasabb képességük van, vagy még annyi sem. Ahogy a történet halad előre, és dobniuk kell egy fel nem vett képességgel, kiválaszthatnak egy üres képesség rubrikát, és felfedhetik a tudásukat azon nyomban. Ugyanígy, a jellemzőket és fortélyokat be lehet írni, ha a körülmények szükségessé teszik a használatukat, a sors pont elköltésekor vagy a bónusz felhasználásakor.
Visszaszámlálás
A visszaszámlálás sürgőssé tehet egy szituációt: foglalkozz vele azon nyomban, vagy a dolgok rosszabbra fognak fordulni. Legyen szó egy ketyegő bombáról, egy végkifejletéhez érkező rituáléról, egy függőhíd korlátján egyensúlyozó buszról, vagy egy rádiós katonáról, aki mindjárt erősítést hív, a visszaszámlálás gyors reakciókra kényszeríti a JK‑kat, vagy viselniük kell a következményeket.
A visszaszámlálások három összetevőből állnak: a visszaszámlálás‑mérő, egy vagy két feltétel és végül egy végeredmény.
A visszaszámlálás‑mérő nagyon hasonló a stresszmérőhöz: egy sorozat doboz, amiket balról jobbra beikszelsz. Minden egyes beikszelés egy kicsivel közelebb hozza a visszaszámlálás végét. Minél rövidebb a sorozat, annál gyorsabban közeleg a vég.
A feltétel egy esemény, ami beikszel egy dobozt a visszaszámlálás‑mérőn. Ez lehet egyszerűen, hogy „egy perc/óra/nap/forduló telte”, vagy nagyon speciális is, mint „a főgonosz elszenved egy következményt vagy kiejtődik”.
Amint beikszeled az utolsó dobozt is, a végeredmény megtörténik, akármi legyen is az.
A KM akár be is lengetheti a visszaszámlálás‑mérőt a játékosok előtt, anélkül, hogy elárulná, mire vonatkozik, sejtetésképpen, és hogy megteremtse a kellő feszültséget a történetben.
A visszaszámlálásnak több feltétele is lehet; talán a visszaszámlálás megbízhatóan halad előre, amíg valami váratlan nem történik, ami felgyorsítja. Adhatsz különböző feltételeket is a visszaszámlálás‑mérő különböző dobozainak, ha a végeredményt események egy bizonyos sorrendje váltaná ki.
Extrém következmények
Az extrém következmények behoznak egy opcionális negyedik következmény súlyosságot a játékba: valamit, ami örökre, visszavonhatatlanul megváltoztatja a karaktert.
Egy extrém következmény felvétele 8‑at hatástalanít az elszenvedett stresszből. Ha felveszed, le kell cserélned a karakter egyik meglévő jellemzőjét (a koncepció jellemző kivételével, ami szóba se jöhet) egy olyanra, ami a karakterben bekövetkező súlyos sérülés elszenvedése miatti változást mutatja.
Alapállapotban nincs lehetőség az extrém következményből felgyógyulásra. Ez már a karakter részévé vált. A következő áttörésnél átnevezheted a jellemzőt, hogy mutassa, ha a karakter megbékélt a sorsával, de nem térhetsz vissza az eredeti jellemzőhöz.
Két áttörés között a karakter csak egyszer élhet ezzel a lehetőséggel.
Gyorsított versengés
Néhány csapat úgy érezheti, hogy túl sok lehetőség van előnybehozásra fordulónként. Ők kipróbálhatják a következőt: a versengés minden fordulójában, minden résztvevő csak egyetlen egyet tehet az alábbiakból:
    • Megold cselekvés a saját oldalának (18. oldal).
    • Dob helyzetbehozásra, de nem adhat ilyenkor csapatmunka bónuszt (32. oldal).
    • Csapatmunka bónuszt adhat az oldala megold cselekvésére vagy valaki más helyzetbehozására. Ilyenkor viszont nem dobhat egyáltalán.
Védekező harc
Néha egy játékos (vagy a KM) úgy szeretné, hogy a karaktere teljes mértékben csak a védekezésre koncentráljon, amíg újra sorra nem kerül a következő fordulóban, ahelyett, hogy cselekvést tenne. Ezt úgy hívjuk, hogy védekező harc.
A védekező harc bejelentésekor meg kell adni annak célját is. Alapértelmezésben saját magad véded (megtámadásoktól, valamint ellened irányuló helyzetbehozásoktól), de irányulhat ez mások védelmezésére, védekezésre egy meghatározott agresszor csoport ellen vagy egy meggátolandó tett vagy végkifejlet ellen.
A védekező harc alatt +2 bónuszt kapsz minden védekezés dobásra, ami kapcsolódik a védekezés céljához.
Ha semmi sem sül ki belőle, mert egyetlen egyszer sem kellett védekezésre dobnod mire sorra kerülsz a következő fordulóban, akkor kapsz egy előnyt (23. oldal), mivel lehetőséged nyílt felkészülni a következő cselekvésre. Ezt ellensúlyozza, hogy „elvesztegetsz egy fordulót”, mert olyan dolog ellen védekeztél, ami végül meg sem történt.
Akadályok
Az ellenségek legfontosabb tulajdonsága, hogy megtámadhatók, és kiejthetők a küzdelemből. Ezzel szemben, az akadályok legfontosabb tulajdonsága, hogy ezek egyike sem lehetséges. Az akadályok sokkal nehezebbé teszik a jeleneteket a JK‑k szempontjából, azok viszont képtelenek harcolni ellenük. Az akadályokat meg kell kerülni, elviselni vagy lényegtelenné tenni.
Bár a legtöbb akadály a környezet része, néhányuk lehet olyan karakter, akit normális módszerekkel lehetetlen kiejteni. Egy sárkány lehet egy főellenség, de ugyanígy lehet egy veszélyes akadály is. A gonosz varázslóhoz vezető utat elálló szoborgólem lehet veszély, de ugyanígy lehet torlasz vagy csak zavaró körülmény. Ez mind csak az ellenfél szerepétől függ, és hogy a JK‑k mit tehetnek ellene.
Akadályok nem szerepelnek minden jelenetben. Arra szolgálnak, hogy kihangsúlyozzák az ellenséget a jelenetben, veszélyesebbé vagy emlékezetesebbé téve őket, de az akadályok túlzott használata frusztrálóvá válhat a JK‑k számára, főleg ha harcközpontúak. Ennek ellenére nagyon hasznosak lehetnek, munkát adnak a kevésbé harcközpontú JK‑k számára egy összecsapásban.
Háromféle akadály létezik: veszély, torlasz és zavaró körülmény.
Veszély
Ha egy akadály meg tudja támadni a JK‑t, akkor az egy veszély. Lángsugár, legördülő sziklák, netán egy mesterlövész, aki túl távol van, hogy közvetlenül ártani lehessen neki – mind veszélyek. Minden veszélynek van egy neve, egy képesség szintje, valamint egy Fegyver szintje (58. oldal) 1 és 4 között.
A veszély neve egy képesség és jellemző egyszerre; vagyis a név megadja, hogy a veszély mit tud tenni, a képesség szintje megmondja, hogy mennyire jó ebben, de a nevet ugyanúgy lehet kihasználni vagy késztetni, mint bármelyik jellemzőt.
Általánosságban, a veszély képesség szintje legalább olyan magas legyen, mint a JK‑k legmagasabb képesség szintje, vagy egy kicsit még annál is magasabb. Egy magas képesség és Fegyver szinttel rendelkező veszély valószínűleg kiejt egy vagy két JK‑t. Az veszélynek lehet alacsonyabb képesség, viszont magasabb Fegyver szintje is, amitől ritkábban fog találatot szerezni, de akkor sokkal nagyobb sérülést okoz. Ezt megfordítva olyan veszélyt kapunk, ami gyakran talál, de alig okoz sérülést.
A veszélyek ugyanúgy jönnek a cselekvési sorrendben, mint a JK‑k és azok ellenfelei. Ha a szabályok szerint dobni kell a cselekvési sorrend meghatározására, a veszélyek a képesség szintjüket használják. Amikor sorra kerül egy fordulóban, a veszély a neve által megadott cselekedetet végzi a dobáshoz a képesség szintjét használva. Ha megtámadás cselekvést végez, és a kimenetel döntetlen vagy jobb, add a Fegyver szintjét a sikerességhez. A veszélyek képesek megtámadásra és helyzetbehozásra; őket nem lehet megtámadni, és nem alkalmasak a megold cselekvésre.
Ha a játékos megold vagy helyzetbehozás cselekvést tenne egy veszély ellen, akkor passzív nehézségre kell dobnia a veszély képesség szintje ellen.
Torlasz
Míg a veszélyek célja, hogy sérülést okozzanak a JK‑knak, addig a torlaszok meggátolják őket a cselekedeteikben. Ennek ellenére a torlaszok képesek stresszt okozni, de nem mindig teszik ezt. A legnagyobb különbség a veszélyek és torlaszok között, hogy a torlaszok nem csinálnak cselekvéseket, és sokkal nehezebb eltávolítani őket. A torlaszok passzív nehézséget adnak bizonyos körülmények között, és veszélyeztetnek, vagy sérüléseket okozhatnak, ha nem figyelnek oda rájuk.
Ugyanúgy, mint a veszélyeknél, a torlaszoknak van nevük és képesség szintjük, ahol a név egyszerre képesség és jellemző is. A veszélyekkel ellentétben, a torlaszok képesség szintje jó, ha nincs több mint egy szinttel a legmagasabb JK képesség szint fölött; különben nagyon hamar igencsak frusztrálóvá válhatnak. Egy torlasznak lehet akár 4 is a Fegyver szintje, de nem szükségszerű egyáltalán, hogy legyen neki.
A torlaszok csak bizonyos körülmények között kerülnek játékba. Egy Kádnyi Sav csak akkor számít, ha valaki át akar ugrani fölötte, vagy beledobják. Egy Drótkerítés csak arra van hatással, aki át akar mászni rajta. Egy szoborgólem csak bizonyos szobákba nem enged be.
A torlaszok nem támadnak, és nem kerülnek sorra a cselekvési sorrendben. Ehelyett mindenkinek dobnia kell a torlasz szintje mint nehézség ellen, ha a torlasz akadályozhatja őket a cselekedetükben. Ha a torlasz nem tud sérülést okozni, akkor csak meggátolja a JK‑k a cselekedetében. Ha viszont sérülést tud okozni, és a JK megold cselekvése kudarc, olyankor akkora sérülést szenved el, amennyivel elvétette a célszámot.
A karakterek beletaszíthatnak valaki mást a torlaszba egy megtámadás cselekedettel. Ha ezt tennéd, akkor dobjál normálisan megtámadásra, de add hozzá a torlasz Fegyver szintjének felét a te Fegyver szintedhez (lefelé kerekítve, de legalább egyet).
Végül, némely torlasz fegyverként vagy páncélként is használható. Ez a szituációtól függ – bizonyos torlaszoknál például teljesen értelmetlenül hangzik. Nem tudsz elbújni egy Kádnyi Sav mögé, de egy  Drótkerítés teljesen hatékony baseball‑ütő ellen, valószínűleg meghiúsítja az egész támadást.
Ha valaki fedezéknek használná a torlaszt, döntsd el, hogy ez csökkenti, vagy meghiúsítja a megtámadást. Ha meghiúsítja, akkor a megtámadás egyszerűen nem lehetséges. Ha csak csökkenti, akkor a védekező fél a Páncél szintjéhez adhatja a torlasz képesség szintjének felét (lefelé kerekítve, de legalább egy).
Próbáld ritkán használni a torlaszokat. A torlaszok nehezebbé tesznek néhány dolgot a JK‑k számára – ami könnyen frusztrálóvá válhat, ha túl sokat használod őket – viszont a játékosokat meg kreatív gondolkodásra serkentheti. Láthatnak arra lehetőséget, hogy a saját javukra fordítsák a torlaszt. Ha sikerül erre módot találniuk, engedd csak meg nekik!
Néha a játékosok csak meg szeretnék szüntetni a torlaszokat. Ehhez egy megold cselekvésre kell dobniuk, a torlasz szintjénél kettővel magasabb nehézségre.
Zavaró körülmény
Míg a veszélyek közvetlenül támadják a JK‑kat, a torlaszok meggátolják őket némely cselekedetben, addig a zavaró körülmények rákényszerítik a játékosokat a saját prioritásaik átgondolására. Az akadályok közül a zavaró körülmények vannak legkevésbé játékmechanikailag meghatározva. Nem feltétlenül kell a jelenetet szabályok szerint nehezebbé tenniük. Inkább egy nehéz választás elé kell állítaniuk a JK‑kat. A zavaró körülmények az alábbi dolgokból állnak:
    • A zavaró körülmény neve egy rövid, ütős leírás. Ez lehet jellemző is, ha feltétlenül szükséges, vagy csak úgy jobbnak látod.
    • A zavaró körülmény választása egy egyszerű kérdés, ami meghatározza a JK‑k által meghozandó döntést.
    • A zavaró körülmény hatása az, ami a JK‑kkal történik, ha figyelmen kívül hagyják. Némely zavaró körülménynek több hatása is lehet, beleértve azt is, amikor sikeresen kezelik.
    • A zavaró körülmény ellenállása az a passzív nehézség, ami ellen a JK‑knak dobniuk kell, ha foglalkozni próbálnak vele. Nem minden zavaró körülmény igényel ellenállást.
Ha attól tartasz, hogy a JK‑k túl hamar lerendeznek egy hátralévő küzdelmet, egy‑két zavaró körülmény hozzáadásával döntésre kényszerítheted őket, hogy vajon fontosabb‑e a rossz fiúk elnáspángolása, vagy inkább a zavaró körülményekkel foglalkoznak.
A zavaró körülményekkel foglalkozásnak mindig világos haszonnal kell járnia, vagy ha ez nem lehetséges, akkor a zavaró körülmények figyelmen kívül hagyása járjon világos következménnyel.
Akadály példák
Veszély
    • Kimagasló~(+4) Géppuska Torony, Fegyver: 3
    • Emberfeletti~(+5) Távoli Mesterlövész, Fegyver: 4
Torlasz
    • Jó~(+2) Drótkerítés, Kimagasló~(+4) nehézség eltüntetni
    • Remek~(+3) Kádnyi Sav, Fegyver: 4, Emberfeletti~(+5) nehézség eltüntetni
Zavaró körülmény
    • Civilekkel tömött busz – Választás: Le fog‑e zuhanni a busz a hídról?
Ellenállás: Remek~(+3)
Hatás (hagyni őket): Minden civil meghal a buszban.
Hatás (megmenteni őket): A bűnöző elmenekül!
    • Csillogó drágakő – Választás: Elemelheted‑e a drágakövet a talapzatról?
Hatás (otthagyni a drágakövet): Nem kapod meg a (megfizethetetlen) drágakövet.
Hatás (elvenni a drágakövet): Aktiválod a templom csapdáit.
Méretarány
A méretarány nevű opcionális alrendszerrel természetfeletti lényeket tudsz kezelni, akiknek a lehetőségei messze túlmutatnak a legtöbb karakterénél a játékodban. Általában nem kell törődnöd a méretarány játékbeli hatásával. Néha viszont jól jöhet a játékosokat a szokásosnál sokkal nagyobb veszedelemnek kitenni – ami jó lehetőség, hogy a karakterek megmutathassák, mire képesek.
Példaképpen – és ezt a listát testre szabhatod a saját világodhoz – legyen ötféle szintje a méretaránynak: Evilági, Természetfeletti, Másvilági, Legendás, Isteni.
    • Az Evilági karaktereknek nincs természetfeletti adottságuk vagy technológiájuk, amik az emberek fölé emelnék őket.
    • A Természetfeletti karaktereknek van valamilyen természetfeletti adottságuk vagy technológiájuk, amik az emberek fölé emelik őket, de attól még alapvetően emberek maradnak.
    • A Másvilági karakterek rendkívüliek vagy egyediek, és az adottságaik miatt nem kell többé törődniük az emberek mindennapi problémáival.
    • A Legendás karakterek hatalmas szellemek, entitások és idegen lények, akik számára az emberiség inkább csak érdekesség, mintsem veszély.
    • Az Isteni karakterek a világegyetem leghatalmasabb erői: arkangyalok, istenek, tündekirálynők, élő bolygók és így tovább.
Ha a méretarányt használod két szemben álló erőre vagy egyénre, akkor hasonlítsd össze a méretarányaikat, hogy melyiké a magasabb, és hány szinttel. A magasabb szintű egyet választhat az alábbiakból az alacsonyabb szintű ellen.
    • +1 bónusz szintkülönbségenként a dobás előtt
    • +2 bónusz szintkülönbségenként a dobás után, ha a dobás siker
    • 1 extra ingyen kihasználás szintkülönbségenként, ha a helyzetbehozás cselekvés siker
A méretarány szabályok túl gyakori és rugalmatlan használata hátrányosan érinti a játékosok karaktereit. Ezt ellensúlyozandó, osztogasd bőkezűen a különféle lehetőségeket, ahol kis ügyeskedéssel megkerülhetik ezeket a hátrányokat. Ilyen lehet például a célpont gyengeségének kikutatása, olyan helyen küzdeni, ahol a méretaránynak nincs jelentősége, vagy megváltoztatni az elérendő célt, hogy az ellenfél ne tudja a méretarány fölényét felhasználni ellenük.
Jellemzők és méretarány
Az életben lévő helyzet jellemzők némelyike természetfeletti eredetű lehet. Ilyenkor a KM‑nek kell eldöntenie, hogy a kihasználásuk extra bónuszokat ad‑e a méretarány miatt. Ezen felül, egy természetfeletti eredetű jellemző kihasználás nélkül is biztosíthat méretarány bónuszokat, például egy varázsköpeny vagy csúcstechnológiás lopakodó álca esetén; nem kell kihasználnod a Leplezett jellemzőt, hogy Természetfeletti bónusszal lopakodhass.
Számít‑e a természetfeletti helyzetbehozás esetén?
Ha helyzetbehozásnál nincs ellenállás, akkor dobás nélkül létrehozod a helyzet jellemzőt egy ingyen kihasználással. Ez a jellemző természetfeletti bónuszokat ad, ahogy előzőleg írtuk.
Ha a helyzetbehozást más kárára követed el, például Indákkal Megkötözött varázslatot alkalmaznál az ellenfeleden, akkor kaphatsz természetfeletti bónuszt rá.
Ha természetfeletti erőkkel érnéd el a helyzetbehozást, és egy ellenszegülő fél közvetlenül be tud avatkozni fizikai vagy természetfeletti módon, a természetfeletti bónuszod alkalmazható a védekezés dobása ellen.
Egyéb esetekben természetfeletti bónusz nélkül hozod létre a helyzet jellemzőt, de később a jellemző kihasználásakor kaphatsz természetfeletti bónuszt, ha ennek van értelme.
Időtartamok
Egy cselekedet időtartamának meghatározásakor pontosabb módszert is használhatsz, mint a siker, kudarc és a „bizonyos áron” opciók. Mennyivel tovább vagy gyorsabban? Az alábbi irányelvekkel a sikeresség döntheti el az időtartamot.
Először is döntsd el, hogy egyszerű siker esetén mennyi ideig tart az adott feladat. Használj körülbelüli mennyiségeket egy mértékegységgel együtt: „pár nap”, „fél perc”, „néhány hét” és így tovább. A körülbelüli mennyiségekbe a következők tartoznak: fél, nagyjából egy, pár, néhány az adott mértékegységből.
Ezután nézd meg, hogy a dobás mennyivel haladja meg, vagy múlja alul a célszámot. Minden sikeresség egy lépést jelent a mennyiségek listáján.
Például, ha a kezdő időtartam „pár óra”, akkor egy sikerességnyivel gyorsabb „nagyjából egy óra”, két sikeresség pedig már csak „fél óra”. A „fél” mennyiségnél gyorsabb a mértékegységet váltja kisebbre (órákból percek, és így tovább), míg a mennyiség „néhány” lesz, emiatt három sikerességgel gyorsabb „néhány perc lesz”.
A lassabb esetben az egész fordítva zajlik, egy sikerességgel lassabb „néhány óra”, kettő „fél nap”, míg három már „nagyjából egy nap”.
Szabálymódosítás főellenségekhez
A csapatmunka közbeni képességek egyesítésével és helyzetbehozással (32. oldal) a JK‑k könnyen bedarálhatnak egy magányos ellenfelet. Ez rendben is van, ha a cél a túlerő demonstrálása, de nem annyira jó ötlet az egész csapattal felérő főellenségekhez.
De ha még emlékszel, teljesen helyén való szörnyek és egyéb nagy veszedelmek esetében megszegni a szabályokat (43. oldal) – így ezt ebben az esetben is megteheted, hogy semlegesíted a csapat normális létszámfölényét, de azért adj esélyt nekik. Ebben a fejezetben pár tanácsot adunk ennek megvalósítására. Ezeket használhatod magukban, vagy többet kombinálva is, ha nagyon nehéz vagy félelmetes főellenségre van szükséged.
Kihívás és versengés védettség
Mindkét technika arra szolgál, hogy elhúzhasd a végső összecsapást, végighajtva a csapatot valamiféle „fogytán az idő” cselekményen, még mielőtt közvetlenül is foglalkozhatnának a főellenséggel.
A kihívás védettség azt jelenti, hogy a csapat nem tudja a főellenséget befolyásolni (fizikailag vagy szellemileg, esetleg mindkettő), hacsak le nem győz egy kihívást (elpusztítani a hatalma forrását, kitalálni a rejtett gyengeségét és így tovább). A főellenség eközben szabadon cselekedhet, és meg is támadhatja őket míg ezen dolgoznak, ellenszegülhet a megoldás vagy helyzetbehozás dobásoknak védekezés dobással, véget vethet az ingyen kihasználásoknak a saját megoldás cselekvéseivel, vagy felkészülhet az elkerülhetetlen konfrontációra saját maga helyzetbehozásával.
A versengés védettséggel a csapatnak egy versengést kell megnyernie a főellenség megtámadása előtt – míg a főellenség szabadon támadhat rájuk közben. Ha a főellenség nyeri a versengést, akkor megvalósítja gonosz tervét, és sértetlenül távozhat.
Feláldozható talpnyaló páncél
Az egyik módszer, ahogy a főellenség kompenzálhatja a JK‑k túlerejét, ha körülveszi magát talpnyalókkal, de ez nem túl kifizetődő, ha a JK‑k egyszerűen figyelmen kívül hagyhatják a bosszantó talpnyalókat, és közvetlenül a főellenséget veszik célba.
De ha van egy feláldozható talpnyaló páncél, akkor a főellenség minden védekezés dobása lehet siker valamilyen áron, mert mindig egy talpnyaló kerül a támadás útjába. Az a bizonyos talpnyaló egyáltalán nem dob védekezést, hanem benyeli a találatot, ami különben a főellenségen csattant volna. Emiatt a JK‑knak muszáj átvágniuk magukat a főellenség seregén a végső konfrontáció előtt.
Vésd észbe, hogy a talpnyalóknak nem muszáj szó szerint talpnyalóknak lenniük. Például létrehozhatsz egy vagy több „pajzsgenerátort”, mindegyiket egy külön stresszmérővel, és talán még egy képességet is kaphatnak, hogy helyzetbehozásokkal is védjék a pajzsai mögé bújó főellenséget!
A végső alak felvétele
Rendben van, a csapat mindent bevetett, és – hihetetlen! - legyőzte a főellenséget. Csak egy probléma maradt: ez csak felszabadította a hús börtönéből, és végre felveheti végső alakját!
A végső alak felvétele segítségével a főellenség nem csak egyetlen karakter, hanem legalább kettő, akiket egymás után le kell győzni, míg mindketten újabb és újabb tehetségekkel és fortélyokkal, magasabb képesség szintekkel, üres stresszmérőkkel és következmény rubrikákkal, sőt, újabb „szabálymódosításokkal” rendelkeznek.
Ha ezt egy kicsit tompítanád, a főellenségnek maradjanak meg a következményei a két alakja között, eltörölve az enyhe következményt, és leminősítve az enyhébb szintekre a mérsékelt és súlyos következményeket.
Növeld meg a méretarányt
Növeld meg a méretarányt, hogy a főellenség a JK‑knál magasabb szinten működjön, ehhez használd a méretarány szabályokat az 52. oldalon. Ezt akkor is bevetheted, ha egyébként a kampányban nincs méretarány – ezek a szabályok csak akkor lépnek érvénybe, ha a főellenség feltűnik a jelenetben.
Szóló bónusz
A játékosok nyilvánvalóan élvezik a csapatmunka bónuszt – de miért ne adhatnál a főellenségnek bónuszt akkor, ha egymagában áll ki a hősökkel?
A szóló bónuszt jó néhány módon meg lehet oldani. Egyszerre többet is használhatsz ezekből, de vedd figyelembe, hogy kombinálva hamar elszállhat az erősségük.
    • A főellenség akkora bónuszt kap a képesség dobásaira, amennyi a csapat maximális lehetséges csapatmunka bónusza (32. oldal) – ez a főellenség ellen dolgozó csapattagok számánál eggyel kevesebb (például +2 ha egy 3 fős csapat ellen van).
    • A főellenség csökkentheti az elszenvedett sérülést az ellenséges csapat létszámának felével, felfelé kerekítve. Ha úgy gondolod, hogy ez túlságosan elnyújtaná a harcot, akkor ne lehessen a sérüléseket 1 alá csökkenteni ezen a módon.
    • A főellenség kihasználásai sokszorozottak: ha sors pontot fizetett egy jellemző kihasználásáért, a kapott bónusza megegyezik a JK‑k számával. Ugyan ingyen kihasználásoknál nem ilyen szerencsés, de így is minden elköltött sors pontja félelmet lop majd a játékosok szívébe.
    • A főellenség elnyomja a jellemző kihasználásokat: ha kettő vagy több ellenféllel áll szemben, azok csak +1 bónuszt vagy újradobást kapnak, ha közvetlenül a főellenség ellen használnák a jellemzőt. Opcionálisan a főellenség akár meggátolhatja az ingyen kihasználások halmozását is.
A veszély egy térkép (vagy karakterek hálója)
A Fate játékban bármit lehet karaktereként kezelni, miért pont egy térképet ne lehetne? Ha a veszély egy térkép, akkor a főellenség zónákból áll (29. oldal), amiben navigálni kell a győzelem elérésének érdekében.
A főellenség térkép részletezése közben adhatsz különböző képességeket, jellemzőket és stresszmérőt az egyes zónáknak. Néhány zónán elhelyezhetsz egyszerűbb kihívásokat is, amiket meg kell nyerni a továbbjutáshoz. Minden zóna karakterként tehet egy cselekvést az ott tartózkodó JK‑k ellen, vagy ha valamiféle végtagot vagy hasonlót reprezentál, akkor támadhatja a szomszédos zónákat is. Ha egy zóna kiejtődik a JK‑k támadásaitól, akkor ezután átugorható, és nem képes többé cselekvésekre, de ez még nem öli meg a főellenséget, csak ha a hősök eljutnak a szívéig, és azt véglegesen elpusztítják.
Ez a módszer akkor működik a legjobban, ha a főellenség egy gigantikus szörny, de nem muszáj csak olyankor használni. Kezelheted a főellenséget kapcsolatban álló karakterek csoportjával anélkül, hogy a JK‑knak ténylegesen bele kéne menniük, és ténylegesen egy térképen kéne mászkálniuk. Ez a módszer egy térkép és a feláldozható talpnyaló páncél (54. oldal) ötvözete – valami olyasmi, mint karakterek hálója. A főellenség néhány részét le kell győzni, mielőtt a játékosok a tényleg sebezhetőre térhetnek, és ezek a részek bizony kiveszik a részüket a küzdelemből.
Akár részletesen kidolgozod térképként, vagy csak egyszerűen karakterek hálójaként kezeled, biztosítsd, hogy ez egy dinamikus harcot eredményez, ahol a főellenség gyakrabban képes cselekedni, és a játékosoknak elő kell állniuk egy tervvel, ami lépésről lépésre kiiktatja a veszélyeket, mielőtt képesek lennének teljesen megszüntetni.
Több célpont kezelése
Elkerülhetetlen, hogy előbb‑utóbb valamelyik játékos több ellenfélre próbáljon meg hatni. Ha ez engedélyezett, akkor az alábbi módokon lehet kezelni.
Ha meg akarod válogatni a célpontjaidat, akkor oszd el az erőfeszítést. Dobj a képességedre, és ha az eredmény pozitív, azt akárhogyan eloszthatod a célpontok között, akik a nekik osztott erőfeszítés ellen védekezés cselekvést tesznek. Legalább egy erőfeszítést kell egy célpontra osztanod, különben egyáltalán nem számít célpontnak.
Sophie három bunkóval kerül szembe, és szeretné mindhármukat lekaszabolni egy szélvész támadással. Egy kihasználásnak és a jó dobásnak köszönhetően a Közelharc dobása Epikus~(+7). Ebből a Remek~(+3) megtámadást a legtapasztaltabbnak szánja, míg a másik kettőnek a Jó~(+2) megtámadást, ami összesen hét erőfeszítés. Ezután mindhárman védekezésre dobnak.
Speciális körülmények között, mint amilyen egy robbanás vagy hasonló, egy zóna támadást tehetsz minden adott zónában álló karakter ellen, a saját csapattársaidat is beleértve. Itt nem osztod el az erőfeszítést, hanem minden zónában állónak a teljes dobás ellen kell védekeznie. A körülményeknek és az alkalmazott módszernek viszont meg kell felelnie bizonyos elvárásoknak; és a KM gyakran előírhatja egy jellemző kihasználását vagy egy fortély használatát is.
Ha szeretnéd, hogy a helyzetbehozásod egy teljes zónára vagy csoportra hasson, akkor hozd létre a jeleneten: az egyetlen jellemzőt a zónán vagy a jelenten hozd létre, ahelyett, hogy a célpontokon egyesével jellemzőket kreálnál. Ráadásul ez még az adminisztrációt is csökkenti. Ha valaki ragaszkodik a célpontonkénti különálló jellemzőkhöz, akkor ossza el az erőfeszítést.
Ezeknél a módszereknél a célpontoknak mind egy zónában kell tartózkodniuk. A KM kivételt adhat ez alól, ha a módszer vagy a körülmények azt diktálják.
Normálisan egyetlen cselekvést kell használni – például megtámadni több ellenfelet egyetlen csapással, két problémával megküzdeni egyetlen megoldással, vagy több fontos NJK‑t megingatni hitében egyetlen helyzetbehozással. A KM engedélyezheti speciális körülmények között két cselekvés végrehajtását is, de ilyenkor mindkét cselekvést ugyanazzal a képességgel kell végrehajtani.
Fegyver és páncél szintek
Szeretnéd átvenni más játékok fegyverzet szabályainak hangulatát? Próbáld ki a fegyver és páncél szinteket! Leegyszerűsítve, ha eltalálnak egy fegyverrel, akkor nagyobb a sebzés, míg a páncél megvéd ettől. (Ezt fortélyokkal is modellezhetnéd, de erre elpazarolni a fortély rubrikákat nem mindenkinek jön be.)
A fegyver szint hozzáadódik a sikeres megtámadás sikerességéhez. Ha van Fegyver:2‑d, akkor minden találat 2‑vel nagyobbat sebez. Ez döntetlen esetén is működik; ha a kimenetel döntetlen, akkor előny jellemző helyett stressz sebzést érsz el.
A páncél szint csökkenti a sikeres megtámadás sikerességét. Így a Páncél:2 esetén minden találat 2‑vel kisebbet sebez. Ha találsz, de a célpont páncél szintje 0‑ra vagy az alá csökkenti a sikerességet, akkor ugyan nem érsz el stressz sebzést, viszont kapsz helyette egy előny jellemzőt.
Nagyon fontos a szintek helyes megválasztása. Szintén érdemes megfontolni, hogy mennyire lesz valószínű a következmény okozása döntetlen esetén. Szerintünk a 0 és 4 közötti szintek megfelelőek.

\end{document}
