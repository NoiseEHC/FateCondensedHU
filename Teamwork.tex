\section{Csapatmunka}

A Fate rendszer három módszert ismer a csapatmunkához: több karakter ugyanolyan képességét egyesíteni egyetlen dobáshoz, ingyen kihasználásokat felhalmozni helyzetbehozásokkal, hogy egy csapattársat eredményessé tegyünk, valamint egy szövetségesünk nevében jellemzőket kihasználni.

Ha egyesíted a képességeket, akkor határozd meg, hogy melyik karakternek van a legmagasabb szintje. Minden további résztvevő, akinek legalább Átlagos~(+1) ugyanaz a képessége, +1 bónuszt ad a legmagasabb szintű karakter képesség szintjéhez. Ha egy karakter ilyen módon támogat egy másikat, az felhasználja a cselekvését. A támogatóknak ugyanazt az árat kell fizetniük, és ugyanazokat a következményeket kell viselniük, mint aki a dobást teszi. A maximális bónusz, amit ilyen módon lehet szerezni, nem lehet nagyobb, mint a legmagasabb képesség szint.

Egyébként, amikor te jössz, előnybehozás cselekvéssel létrehozhatsz helyzet jellemzőket, és a szövetségeseid felhasználhatják az ingyen kihasználásokat, ha ez megmagyarázható a történetben. Ha nem te jössz, akkor kihasználhatsz egy jellemzőt, hogy más dobásához bónuszt adjál.

\section{Kihívás}

A legtöbb dolog, amit a karaktereknek tesznek, eldönthető egyetlen kockadobással – hatástalanítani egy bombát, megtalálni egy szörnyű titkot tartalmazó kötetet, netán megfejteni egy rejtjelezést. De néha a dolgok egy kicsit zavarosabbak, komplikáltabbak, és nem is olyan könnyű megtalálni a szörnyű titkot tartalmazó kötetet, mert a keresett hajó éppen Hong Kong kikötőjén száguld keresztül a dühöngő viharban, és a hajó könyvtára meg lángokban áll – amihez persze neked semmi közöd.

Komplikált szituációkban, ha nincs ellenérdekelt fél, valószínűleg \definition{kihívást} kell használnod: egy sorozat megold cselekvés, amik megbirkóznak a nagyobb problémával. A kihívások alatt az egész csapat együttműködhet a jelenetben, ami lendületessé teheti a történetet.

A kihívás létrehozásához a KM"~nek át kell gondolnia a szituációt, és kiválasztani a képességeket, amikkel a csapat eredményt érhet el. Tekints minden cselekedetet egy különálló megold cselekvésnek. A csapatmunka cselekvések engedélyezettek, de ezeknek legyen valamilyen ára, vagy okozzanak komplikációkat, például kifutni az időből vagy más eredménytelenséget.

A legjobb, ha a KM lehetőséget ad a hozzájárulásra minden karakternek a jelenetben – próbálj annyi képességet használni, amennyi karakter jelen van. Használj kevesebb képességet ha néhány karaktert várhatóan elszólít a kötelesség, vagy fontosabb dolgok lekötik a figyelmüket, vagy ha teret akarsz engedni a csapatmunkának. A keményebb kihívásokhoz írj elő több szükséges cselekvést, mint ahány karakter jelen van, és emellett megemelheted a nehézségeket is.

A kockadobások után a KM értékeli a sikereket, kudarcokat és az árat, amit fizetniük kellett, és mindent összevetve eldönti, hogy hogyan játszódott le a jelenet. Lehet, hogy az eredmény újabb kihíváshoz vagy egy versengéshez, netán egy konfliktushoz vezet. Ha mind siker, mind kudarc előfordul, az a karaktereknek részleges győzelmet jelenthet, ami miatt újabb komplikációkba bonyolódnak.
